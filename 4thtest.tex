\documentclass[a4paper,10 pt,addpoints]{exam}
\usepackage[utf8]{inputenc}
\usepackage[banglamainfont=SolaimanLipi, 
            banglattfont=Siyam Rupali
           ]{latexbangla}
\usepackage{amsmath}
\usepackage{amsfonts}
\usepackage{flexisym}
\usepackage{pgfplots}
\usetikzlibrary{datavisualization}
\usetikzlibrary{calc}
\pgfplotsset{width=3cm,compat=1.4}
\usepackage{amssymb}
\usepackage{multicol}
\usepackage{draftwatermark}
\SetWatermarkLightness{0.8}
\SetWatermarkScale{3}
\setlength{\columnsep}{1cm}
\setlength\columnseprule{0.5pt}
%font definition
\SetWatermarkText{Yeshim}
\begin{document}
\begin{center}
\fbox{\fbox{\parbox{5.5in}{\centering  }}}
\end{center}
We are all, of course, shocked and extremely disturbed. Much like many of you who will be reading this, I have a profound sense of unease and sadness that my country is not safe. That the university campus I grew up on is the same place where my father got stabbed while trying to enjoy a robotics competition. A young friend asked me desperately, "WHY Yeshim Apu? Why would you stay in this country, when this is happening?"

It seems, then, that many people are despairing about the state of this country. Asking me if my family will leave, lamenting that we have failed my father.

However.

There's something I'd like to say to all of you. In fact, I might as well just be lazy and steal a few words from my father, because I know exactly what he's going to say as soon as he's up and about again. You see, you are not allowed to give up hope. And you can never, ever stop fighting for all the things that are good and beautiful to you, for the country you so badly wish you were living in and just haven't gotten quite to yet.

Nothing has ever come easily. Every single thing that you enjoy in your world today – a street to walk on as a free person, a meal to eat, a doctor when you are ill, the right to go to school, to vote, to work at a job that pays you enough money to live and maybe take a rickshaw ride and eat some fuchka that will probably mess up your stomach but is worth it because you're eating it with someone that gives you the giggles – every single thing you enjoy now was fought for by someone who came before you. Nothing ever came for free, and nothing ever came easy. Somebody fought for it, little by little, piece by piece, day after day, year after year. By people like my parents, yes, but also by people like you and me. It is our right, and our responsibility, to deeply enjoy every tiny bit of what we have been given. And it is our responsibility to continue to fight for what we don't yet have. Perhaps our children will get it.

When it gets difficult, when you want to scream and cry with the sheer outrage of it –  I know, I know, I am there myself – you do not plan on how you're going to run away. You take a deep, deep breath. You look around, and you gather up the pieces of courage that you might have dropped by accident along the way. You lift your chin up as stubbornly as you possibly can, and you figure out where the next step forward goes.

I know very well that one of the reasons I am able to say this is that my father is alive and well, already talking about how to finish the five courses he's supposed to be teaching this semester. I take a moment now to think of the families and friends of those who have lost their lives in this same fight. There are daughters out there whose fathers have not recovered. I'm thinking about you.

I'm staying in this country because I like it. To me, this place is not the ugliness of the incidents like this. These incidents and the people who cause them are a problem, one there we need to deal with very urgently. They are rather like warts, or maybe fungus. They're gross, and they may have appeared because we haven't yet done a good enough job of keeping clean. We need to remove them.

But they are not what this place is. To me, this country is the gorgeous volunteers I work with at Kaan Pete Roi. It's a trip to Chhayanaut or Shilpakala any day of the week I'm looking for a song in my heart. It's the incredible science and art and literature this tiny country has managed to make despite the battering it has taken in history. It's the groups of university students I often have the privilege to chat with and learn from, and it's my parents young colleagues sitting around our dinner table, planning how to make things better for their next wave of university students. It's the insistence of friends to feed me kababs made of little fish, because apparently that's what makes a healthy baby. It's my mother and father, who – make no mistake – are not going anywhere.

I am so grateful, especially to those who were there in the moment and acted quickly, with no thought to their own safety, to get my father the care he needed. Thank you. To those protesting all over the country, thank you; the sound of you gives me strength. To those who made up the absolute ocean of love we have received in the past few days, thank you; you are exactly what I need.

Make no mistake. We are not going anywhere.
\end{document}