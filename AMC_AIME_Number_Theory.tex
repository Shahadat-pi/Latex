\documentclass[a4paper,11pt]{article}
\usepackage[T1]{fontenc}
\usepackage[utf8]{inputenc}
\usepackage[sexy]{evan}
\usepackage{microtype}
\usepackage{lmodern}
\usepackage{amsfonts}
\usepackage{relsize}
\usepackage{xfrac}
\usepackage{graphicx}
\usepackage[english]{babel}
\usepackage{dirtytalk}
\usepackage{tikz}
\usetikzlibrary{arrows,positioning} 
\usepackage{array}
\usepackage{wrapfig}
\usepackage{multirow}
\usepackage{tabu}
\usetikzlibrary{calc}
\usepackage{changepage}
\usepackage{caption,setspace}
\usepackage{draftwatermark}
\newcommand\perm[2][^n]{\prescript{#1\mkern-2.5mu}{}P_{#2}}
\newcommand\comb[2][^n]{\prescript{#1\mkern-0.5mu}{}C_{#2}}
\SetWatermarkText{$AIME$}
\begin{document}

\begin{center}
\begin{LARGE}
AIME AMC Number Theory
\end{LARGE}
\end{center}

\textbf{Problem 1}

Let $a_n$ equal $6^{n}+8^{n}$. Determine the remainder upon dividing $a_ {83}$ by $49$.

\textbf{Solution 1}
First, we try to find a relationship between the numbers we're provided with and $49$. We realize that $49=7^2$ and both $6$ and $8$ are greater or less than $7$ by $1$.

Expressing the numbers in terms of $7$, we get $(7-1)^{83}+(7+1)^{83}$.

Applying the Binomial Theorem, half of our terms cancel out and we are left with $2(7^{83}+3403\cdot7^{81}+\cdots + 83\cdot7)$. We realize that all of these terms are divisible by $49$ except the final term.

After some quick division, our answer is $\boxed{035}$.

\textbf{Solution 2}

Since $\phi(49) = 42$ (the Euler's totient function), by Euler's Totient Theorem, $a^{42} \equiv 1 \pmod{49}$ where $\text{gcd}(a,49) = 1$. Thus $6^{83} + 8^{83} \equiv 6^{2(42)-1}+8^{2(42)-1}$  $\equiv 6^{-1} + 8^{-1} \equiv \frac{8+6}{48}$ $\equiv \frac{14}{-1}\equiv \boxed{035} \pmod{49}$.

Alternatively, we could have noted that $a^b\equiv a^{b\pmod{\phi{(n)}}}\pmod n$. This way, we have $6^{83}\equiv 6^{83\pmod {42}}\equiv 6^{-1}\pmod {49}$, and can finish the same way.

\textbf{ Problem 2}
What is the largest 2-digit prime factor of the integer $\mathlarger{{200\choose 100}}$?

\textbf{Solution}
Expanding the binomial coefficient, we get $\mathlarger{{200 \choose 100}=\frac{200!}{100!100!}}$. Let the prime be $p$; then $10 \le p < 100$. If $p > 50$, then the factor of $p$ appears twice in the denominator. Thus, we need $p$ to appear as a factor three times in the numerator, or $3p<200$. The largest such prime is $\boxed{061}$, which is our answer.
 
 
 
\end{document}
