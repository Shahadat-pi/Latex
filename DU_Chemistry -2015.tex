\documentclass[addpoints]{exam}
\usepackage[utf8]{inputenc}
\usepackage{amsmath}
\usepackage{tikz}
\usepackage{chemfig}
\usepackage{pgfplots}
\usepackage[american,siunitx]{circuitikz}
\ctikzset{bipoles/resistor/width=.35}
\ctikzset{bipoles/resistor/height=.15}
\pgfplotsset{width=3cm,compat=1.4}
\usepackage{graphicx}%
\usepackage{mathtools}
\usetikzlibrary{arrows,shapes,positioning}
\usetikzlibrary{decorations.markings}
\tikzstyle arrowstyle=[scale=1]
\tikzstyle directed=[postaction={decorate,decoration={markings,
    mark=at position .65 with {\arrow[arrowstyle]{stealth}}}}]
\tikzstyle reverse directed=[postaction={decorate,decoration={markings,
    mark=at position .65 with {\arrowreversed[arrowstyle]{stealth};}}}]
\usepackage{graphicx}%
\usepackage{mathtools}
\usepackage{dirtytalk}
\usepackage{relsize}
\graphicspath{ {images/} }
\DeclarePairedDelimiter{\ceil}{\lceil}{\rceil}
\usepackage[legalpaper]{geometry}
\usepackage{draftwatermark}
\SetWatermarkFontSize{2cm}
\SetWatermarkText{DU-Physics}
\usepackage[banglamainfont=Kalpurush, 
            banglattfont=Siyam Rupali
           ]{latexbangla}
        
\begin{document}
\begin{LARGE}
\begin{center}
রসায়ন (Chemistry - 2015)
\end{center}
\end{LARGE}
\begin{questions}


\question  কাচঁপাত্রের কোন সেটটি সঠিকভাবে আয়তন মাপার উপযুক্ত?

\begin{oneparchoices}
\choice Pipete and beaker       
\choice Burettete and reagent bottle   
\choice Pipette and burette      
\choice Graduated pipette and conical flask
\end{oneparchoices}

\question  CO2(g) + 2H2(g) ⇌ CH3OH(g)  বিক্রিয়ায় KPএর মান হলো – 

(A) Kp = Kc(RT)-1                (B) Kp = Kc(RT)-2             (C)  Kp = Kc            (D) Kp = Kc(RT)2     
\question  SN2   বিক্রিয়ায় অ্যালকাইল হ্যলাইডের সক্রিয়তার ক্রম হলো – 
(A)  CH3X > RCH2X > R2CHX >R3CX          (B) RCH2X > CH3X > R2CHX > R3CX 
(C)  CH3X > RCH2X > R3CX > R2CHX         (D) R3CX > R2CHX > RCH2X >CH3X
\question  প্রতিস্থাপন বিক্রিয়ায় কোন কার্যকরী মূলকটি অর্থপ্যারা নির্দেশ করে ?
(A) –CH3                                (B)  -COOH                   (C)  –CHO                         (D) -Cl
\question  NaCl এর সাথে H2O যুক্ত করলে কি ঘটে?
(A) NaOH (aq) + HCl(aq)     (B)  Na+(aq) + Cl-(aq)    (C)  NaOH (aq) + Cl2(g)   (D) OH- (aq) +Cl- (aq)
\question  নিন্মের বিক্রিয়াগুলো হতে কার্বনের প্রমাণ দহনতাপ নির্ণয় কর –
(i) C(s) + 1/2O2(g) → CO2(g)  ∆H0 = -111.0kJ mol-1
(ii) CO (g) + 1/2O2(g) → CO2(g)  ∆H0 = -283.00kJ mol-1
(A)   173. 0kJ mol-1       (B)  -394. 173. 0kJ mol-1       (C)  373. 173. 0kJ mol-1    (D) 394. 173. 0kJ mol-1    
\question  ইরিথ্রিটল হলো – 
(A)  An enzyme    (B)  A non-caloric sweetener    (C)  An amino acid    (D)  An anti-oxidant
\question অক্সি এসিডসমূহের শক্তির সঠিক সক্রিয়তার ক্রম হলো – 
(A)  HClO4 >HN03 >H2SO3>H2SO4              (B) HN03>H2SO3>H2SO4 >HClO4   
(C)  H2SO3 >H2SO4 >HClO4>HN03             (D) HClO4>H2SO4>HN03>H2SO3
\question বিশুদ্ধ পানির ঘনমাত্রা (মোল/লিটার ) হলো – 
(A)  35.5           (B)   18.0          (C) 1           (D) 55.5



\end{questions}

\end{document}