\documentclass[addpoints]{exam}
\usepackage[utf8]{inputenc}
\usepackage{amsmath}
\usepackage{mathtools}
\usepackage{relsize}
\usepackage{dirtytalk}
\usepackage{graphicx}
\graphicspath{ {./drift/} }
\DeclarePairedDelimiter{\ceil}{\lceil}{\rceil}
\usepackage{geometry}
\usepackage{draftwatermark}
\SetWatermarkFontSize{8cm}
\SetWatermarkText{$e^{i\pi} + 1=0$}
\usepackage[banglamainfont=Kalpurush, 
            banglattfont=Siyam Rupali
           ]{latexbangla}
        
\begin{document}
\begin{LARGE}
\begin{center}
গণিত (Mathematics - 2003)
\end{center}
\end{LARGE}
\begin{questions}

 \question $ 3x+4y-12=0 $ সরলরেখা দ্বারা অক্ষদ্বয়ের মধ্যবর্তী খন্ডিত অংশ (The length of the line segment cut off by the line $ 3x+4y-12=0 $ between the axis is) -

\begin{oneparchoices}
\choice $ 7 $
\choice $ 5 $
\choice $ 9 $
\choice $ 8 $
\end{oneparchoices}

\question বাস্তবসংখ্যায় $ |5-2x|\ge 4 $ অসমতাটির সমাধান (The solution set of the inequality $ |5-2x|\ge 4 $ is) – 

\begin{oneparchoices}
\choice $ \dfrac{1}{2}\le x \le \dfrac{9}{2}$
\choice $ x \le \dfrac{1}{2}$ or $\dfrac{9}{2}\le x$
\choice $ x\ge \dfrac{1}{2}$
\choice $\dfrac{1}{2}\le x \le \dfrac{9}{2}$ or $x\ge \dfrac{27}{2} $
\end{oneparchoices}

\question $ \mathlarger{\int_{0}^{\frac{\pi}{2}}}\sqrt{1+\sin\theta}\, d\theta $ এর মান (The value of $ \mathlarger{\int_{0}^{\frac{\pi}{2}}}\sqrt{1+\sin\theta}\, d\theta $ is ) -  

\begin{oneparchoices}
\choice $ \sqrt{2} $
\choice $ 2 $
\choice $ \pi $
\choice  $ \dfrac{\pi}{2} $
\end{oneparchoices}

\question $ \tan 75^{\circ}-\tan 30^{\circ} -\tan 75^{\circ}\tan 30^{\circ} $ এর মান (The value of $ \tan 75^{\circ}-\tan 30^{\circ} -\tan 75^{\circ}\tan 30^{\circ} $ is )- 

\begin{oneparchoices}
\choice $ 0 $
\choice $ 1 $
\choice $ \dfrac{1}{\sqrt{2}} $
\choice $ \dfrac{1}{\sqrt{3}} $
\end{oneparchoices}

\question $ x^{2}+y^{2}-24x+10y=0 $ বৃত্তের ব্যাসার্ধ (The radius of the circle $ x^{2}+y^{2}-24x+10y=0 $ is  ) 

\begin{oneparchoices}
\choice $ 7 $
\choice $ 5 $
\choice $ 13 $
\choice $ 12 $
\end{oneparchoices}


\question $ \tan^{-1}1+\tan^{-1} 2 +\tan^{-1} 3 $ এর মান- (The value of $ \tan^{-1}1+\tan^{-1} 2 +\tan^{-1} 3 $ is)

\begin{oneparchoices}
\choice $ \dfrac{\pi}{2} $
\choice $ \dfrac{\pi}{4} $
\choice $ \pi $
\choice $2\pi $
\end{oneparchoices}

\question দশমিক সংখ্যা 115 কে দ্বিমিক পদ্ধতিতে প্রকাশ করলে হয়- (The decimal number 115 when expressed in the binary system is )

\begin{oneparchoices}
\choice 1110011
\choice 1110111
\choice 1111011
\choice 1101111
\end{oneparchoices}

\question  $ 4(\sin^{2} \theta + \cos \theta ) = 5 $ সমীকরণের সাধারণ সমাধান (The general solution of the equation  $ 4(\sin^{2} \theta + \cos \theta ) = 5 $ is ) - 

\begin{oneparchoices}
\choice $ \theta =  2n\pi \pm \dfrac{\pi}{2} $
\choice $ \theta =  2n\pi \pm \dfrac{\pi}{3} $
\choice $ \theta =  2n\pi \pm \dfrac{\pi}{4} $
\choice $ \theta = 2n\pi \pm \dfrac{\pi}{5} $
\end{oneparchoices}



\question (The equation of the circle touching the coordinate axes at the points ) $ (5,0) $ এবং (and) $ (0,5) $ বিন্দুতে অক্ষরেখাদ্বয়কে স্পর্শকারী বৃত্তের সমীকরণ (is)- 

\begin{oneparchoices}
\choice $ x^{2}+y^{2}+10x-10y-25=0 $
\choice $ x^{2}+y^{2}+10x+10y+25=0 $
\choice $ x^{2}+y^{2}-10x+10y+25=0 $
\choice $ x^{2}+y^{2}-10x-10y-25=0 $
\end{oneparchoices}

\question  যদি (If) $ A = \begin{pmatrix}
1 & 1 & 1\\
x & a & b\\
x^{2} & a^{2} & b^{2}
\end{pmatrix} =0,\, x=?  $

\begin{oneparchoices}
\choice $ -a $ or $ b $
\choice $ a $ or $ -b $
\choice $-a$ or $ -b $
\choice $ a $ or $b$
\end{oneparchoices}

\question $ \Big(x^{2}+\dfrac{2}{x} \Big)^{6}$ এর সম্প্রসারণে $ x- $মুক্ত পদটি - (The term independent of $ x $ in expansion of $ \Big(x^{2}+\dfrac{2}{x} \Big)^{6}$ is- )

\begin{oneparchoices}
\choice $ 448 $
\choice $ 120 $
\choice $ 240 $
\choice $ 3000 $
\end{oneparchoices}

\question একটি গুনোত্তর প্রগমনের চতুর্থ পদ 9 এবং নবম পদ 2187 হলে সাধারণ অনুপাত- (If the fourth term of a geometric progression is 9 and the ninth term is 2187 then the common ratio is )

\begin{oneparchoices}
\choice $ 7 $
\choice $ 9 $
\choice  $ 3 $
\choice $ 27 $
\end{oneparchoices}

 \question যদি $ A= \begin{pmatrix}
 1 & 0\\
 0 & 5
\end{pmatrix},\, B =\begin{pmatrix}
 5 & 0\\
 2 & 1
\end{pmatrix}   $ হয়, তখন $ AB $ হয়- (If $ A= \begin{pmatrix}
 1 & 0\\
 0 & 5
\end{pmatrix},\, B =\begin{pmatrix}
 5 & 0\\
 2 & 1
\end{pmatrix}   $ then $ AB $ is ) 

\begin{oneparchoices}
\choice $ \begin{pmatrix}
 5 & 0\\
 2 & 5
\end{pmatrix} $
\choice $ \begin{pmatrix}
 5 & 0\\
 10 & 5
\end{pmatrix} $
\choice $ \begin{pmatrix}
 6 & 0\\
 2 & 6
\end{pmatrix} $
\choice $ \begin{pmatrix}
 8 & 1\\
 12 & 5
\end{pmatrix} $
\end{oneparchoices}

\question $ 3,5,7,8,9 $ অঙ্কগুলি এক বা একাধিক বার ব্যবহার করে 7000 থেকে বড় চার অঙ্ক বিশিষ্ট কতগুলি সংখ্যা গঠন করা যায়? (How many four digit numbers greater than 700 an be formed from the digits $ 3,5,7,8,9 $ if repettitions are allowed ? ) 

\begin{oneparchoices}
\choice $ 27 $
\choice $ 81 $
\choice $ 72 $
\choice $ 56 $
\end{oneparchoices}

\question $ \tan 54^{\circ} -\tan 36^{\circ} $ এর মান- (The value of $ \tan 54^{\circ} -\tan 36^{\circ} $ is )

\begin{oneparchoices}
\choice $ 2\tan 18^{\circ} $
\choice $ 2\cot 27^{\circ} $
\choice $ -2\tan 81^{\circ} $
\choice $ -2\tan 72^{\circ} $
\end{oneparchoices}

 \question $ (4,-2) $ বিন্দু থেকে $ 5x+12y=3 $ রেখার উপর অঙ্কিত লম্বের দৈর্ঘ্য- (The length of the perpendicular drawn from the point $ (4,-2) $ to the line $ 5x+12y=3 $ is  )

\begin{oneparchoices}
\choice $ 8  $
\choice $ \dfrac{8}{9}  $
\choice $ \dfrac{3}{7}  $
\choice $ \dfrac{7}{13} $
\end{oneparchoices}

\question $ \mathlarger{\int_{0}^{\frac{\pi}{2}}} (1+\cos x)^{2}\sin x\,dx $ এর মান (The value of $ \mathlarger{\int_{0}^{\frac{\pi}{2}}} (1+\cos x)^{2}\sin x\,dx $ is )

\begin{oneparchoices}
\choice $ \dfrac{8}{3} $
\choice $ \dfrac{5}{8} $
\choice $ \dfrac{2}{7} $
\choice $ \dfrac{7}{3} $
\end{oneparchoices}

\question যদি $ y=\sqrt{\cos 2x} $ হয় $ \dfrac{dy}{dx} $ কত? (If $ y=\sqrt{\cos 2x} $ then $ \dfrac{dy}{dx} = ?$ )

\begin{oneparchoices}
\choice $ -\dfrac{\sin 2x}{\sqrt{\cos 2x}} $ 
\choice $ \dfrac{\cos 2x}{\sqrt{\sin 2x}} $ 
\choice  $ \dfrac{2\sin x}{\sqrt{\tan x}} $
\choice $ \dfrac{\tan 2x}{\sqrt{\sin 2x}} $
\end{oneparchoices}

\question  কোন বিন্দুতে দুইটি বল $ 120^{\circ} $ কোণে ক্রিয়াশীল। বৃহত্তর বলটির মান 10N এবং তাদের লদ্ধি ক্ষুদ্রতর বলের সাথে সমকোণ উৎপন্ন করলে ক্ষুদ্রতর বলের মান - (Two forces acre acting at a points at an angle of $ 120^{\circ} $. If the magnitude of greater force is 10N and the resultant is at right angles with the smaller force, the magnitude of the smaller force is )


\begin{oneparchoices}
\choice 4N
\choice 5N
\choice 6N
\choice 8N
\end{oneparchoices}

\question $ 1+ \dfrac{3}{1!}+\dfrac{5}{2!}+ \dfrac{7}{3!}+ \cdots \cdots $ ধারাটির যোগফল (The sum of the series $ 1+ \dfrac{3}{1!}+\dfrac{5}{2!}+ \dfrac{7}{3!}+ \cdots \cdots $ is )- 


\begin{oneparchoices}
\choice $ e $
\choice $ 2e $
\choice $ 3e $
\choice $ 4e $
\end{oneparchoices}

\question 5 জন বিজ্ঞান জন 3 কলা অনুষদের ছাত্র 4  জনের একটি কমিটি গঠন করতে হবে যাতে অন্তত একজন বিজ্ঞান ও একজন কলার ছাত্র থাকে। কত বিভিন্ন প্রকারে এই কমিটি গঠন করা যেতে পারে? (A committee of 4 is to be formed from 5 science students and 3 arts students. In how many ways can this be done so that the committee contains at least one science and at least one arts students ? ) 

\begin{oneparchoices}
\choice $ 60 $
\choice $ 65 $
\choice $ 70 $
\choice $ 75 $
\end{oneparchoices}

\question $ \cos 675^{\circ} +\sin (-1395^{\circ})$ এর মান (equals) - 


\begin{oneparchoices}
\choice $ \dfrac{1}{2} $
\choice $ \dfrac{1}{\sqrt{2}} $
\choice $ -\sqrt{2} $
\choice $ \sqrt{2} $
\end{oneparchoices}



\question নিচের কোন রাশিমালাটি $ \sin 3A $ কে $ \sin A $ বা $ \cos A $ এর বহুপদী রুপে প্রকাশ করে (Which of the following expression gives  $ \sin 3A $ as a polynomial in $ \sin A $ or $ \cos A $) -

\begin{oneparchoices}
\choice  $ 3\cos A -4\cos^{3} A $
\choice  $ 3\sin A -4\sin^{3} A $
\choice  $ 4\cos^{3} A -3\cos A $
\choice  $ 4\sin^{3} A -3\sin A $
\end{oneparchoices}

\question 40 হতে 50 সংখ্যাগুলি দৈবচয়ন পদ্ধতিতে দেয়া হল। সংখ্যাটি মৌলিক হওয়ার সম্ভাবনা কত? (One of the numbers from 40 to 50 is selected at random. What is the probability that the number is a prime number? )

\begin{oneparchoices}
\choice $ \dfrac{2}{11} $ 
\choice $ \dfrac{3}{11} $ 
\choice $ \dfrac{1}{5} $
\choice $ \dfrac{3}{10} $
\end{oneparchoices}

\question $ \dfrac{\tan^{-1}x}{1+x^{2}} $ এর অনির্দিষ্ট যোগজ- (An indefinite integral of $ \dfrac{\tan^{-1}x}{1+x^{2}} $ is)

\begin{oneparchoices}
\choice $ (\tan^{-1}x)\ln(1+x^{2}) $ 
\choice $ \dfrac{1}{2} (\tan^{-1}x)^{2} $
\choice $ \Big(\dfrac{1}{2}\tan^{-1}x\Big)^{2} $
\choice $ \dfrac{1}{2}\tan^{-1}x $ 
\end{oneparchoices}



\question $ \mathlarger{\lim_{x\to 0}}\dfrac{x(\cos x +\cos 2x)}{\sin x} $ এর মান (the value of $ \mathlarger{\lim_{x\to 0}}\dfrac{x(\cos x +\cos 2x)}{\sin x} $ is)

\begin{oneparchoices}
\choice 0
\choice 1
\choice 2
\choice $ \dfrac{1}{2} $
\end{oneparchoices}

\question   $ 5x-2y+7=0 $ সরলরেখার উপর লম্ব এবং $ (-3,1) $ বিন্দুদিয়ে অতিক্রম করে এমন সরলরেখার সমীকরণ- (The equation of the straight line passing through the point $ (-3,1) $ and perpendicular tot he line $ 5x-2y+7=0 $ is)

\begin{oneparchoices}
\choice $ 2x+5y+1=0 $
\choice $ 2x-5y+1=0 $
\choice $ 2x+5y-1=0 $
\choice $ 2x-5y-1=0 $
\end{oneparchoices}

\question $ (4,5) $ কেন্দ্রবিশষ্ট বৃত্ত যা $ x^{2}+y^{2}+4x+6y-12=0 $ বৃত্তের কেন্দ্রদিয়ে গমন করে তার সমীকরণ (The equation of the circle whose centre id at $ (4,5) $ and which passes through the centre of the circle $ x^{2}+y^{2}+4x+6y-12=0 $ is )


\begin{oneparchoices}
\choice $ x^{2}+y^{2}-8x+10y+59=0 $
\choice $ x^{2}+y^{2}-8x-10y+59=0 $
\choice $ x^{2}+y^{2}+8x+10y-59=0 $
\choice $ x^{2}+y^{2}-8x-10y-59=0 $
\end{oneparchoices}

\question  কোন ত্রিভুজের শীর্ষ বিন্দু সমূহ $ (-1,-2),\,(2,5) $ এবং $ (3,10) $ হলে তার ক্ষেত্রফল- (The area of the triangle whose vertices are at the points $ (-1,-2),\,(2,5) $ and $ (3,10) $ is  )

\begin{oneparchoices}
\choice $ 10 $ sq.units
\choice $ 15 $ sq.units
\choice $ 4 $ sq.units
\choice $ 13 $ sq.units
\end{oneparchoices}

\question $ x^{2}-3x+5 $ এর নূন্যতম মান- (The minimum value of $ x^{2}-3x+5 $ is ) 

\begin{oneparchoices}
\choice $ 3 $
\choice $ 5 $
\choice $ \dfrac{15}{4} $
\choice $ \dfrac{11}{4} $
\end{oneparchoices}

\end{questions}

\end{document}