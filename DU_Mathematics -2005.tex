\documentclass[addpoints]{exam}
\usepackage[utf8]{inputenc}
\usepackage{amsmath}
\usepackage{mathtools}
\usepackage{relsize}
\usepackage{dirtytalk}
\usepackage{graphicx}
\graphicspath{ {./drift/} }
\DeclarePairedDelimiter{\ceil}{\lceil}{\rceil}
\usepackage{geometry}
\usepackage{draftwatermark}
\SetWatermarkFontSize{8cm}
\SetWatermarkText{$e^{i\pi} + 1=0$}
\usepackage[banglamainfont=Kalpurush, 
            banglattfont=Siyam Rupali
           ]{latexbangla}
        
\begin{document}
\begin{LARGE}
\begin{center}
গণিত (Mathematics - 2005)
\end{center}
\end{LARGE}
\begin{questions}

\question  $ i^{2}=-1 $ হলে $ \dfrac{i+i^{-1}}{i-i^{-1}} $ এর মান - 

\begin{oneparchoices}
\choice $ 0 $
\choice $ -2i $
\choice $ 2i $
\choice $ 2 $
\end{oneparchoices}


\question $ (1,4) $ এবং $ (9,12) $ বিন্দুদ্বয়ের সংযোগকারী সরলরেখাকে $ 5:3 $ অনুপাতে অন্ত:স্থভাবে বিভক্তকারী বিন্দুর স্থানাংক –

\begin{oneparchoices}
\choice $ (6,-6) $
\choice $ (3,2) $
\choice $ (5, 5) $
\choice $ (3,4) $
\end{oneparchoices}

\question $ 2x=y^{2}+8y+12 $ পরাবৃত্তটির শীর্ষবিন্দুর স্থানাংক-

\begin{oneparchoices}
\choice $ (3,-4) $
\choice $ (5,5) $
\choice $ (6, -6) $
\choice $ (-1,1) $
\end{oneparchoices}

\question  $ y^{2}=4x $ এবং $ y=x $ দ্বারা আবদ্ধ ক্ষেত্রের ক্ষেত্রফল – 

\begin{oneparchoices}
\choice $ \dfrac{3}{8} $ unit$ ^{2} $
\choice $ \dfrac{8}{3} $ unit$ ^{2} $
\choice $ 3 $ unit$ ^{2} $
\choice $ 8 $ unit$ ^{2} $
\end{oneparchoices}

\question  $ \mathlarger{\int_{0}^{1}\dfrac{\cos^{-1} xdx}{\sqrt{1-x^{2}}}}  $  এর মান

\begin{oneparchoices}
\choice $ \dfrac{\pi^{2}}{8} $
\choice $ \dfrac{\pi}{2} $
\choice $ \dfrac{\pi^{2}}{4} $
\choice $ \dfrac{\pi^{2}}{16} $
\end{oneparchoices}

\question $ \dfrac{(x-4)^{2}}{100} + \dfrac{(y-2)^{2}}{64} = 1 $ উপবৃত্তের উৎকেন্দ্রিকতা – 

\begin{oneparchoices}
\choice $ 1 $
\choice $ \dfrac{3}{5} $
\choice $ \dfrac{5}{3} $
\choice $ \dfrac{4}{5} $
\end{oneparchoices}

\question $ \begin{pmatrix}
\alpha +2 & 2\\
8 & \alpha -4
\end{pmatrix} $ ম্যাট্রিক্সটি ব্যতীক্রমী হবে যদি $ \alpha $ এর মান 

\begin{oneparchoices}
\choice $ -4, 6 $
\choice $-6, 4$
\choice $ 4, 6$
\choice $6, -4$
\end{oneparchoices}

\question $ 3x-7y+2=0 $ সরলরেখার উপর লম্ব এবং $ (1,2) $ বিন্দুগামী সরলরেখার সমীকরণ – 
 
\begin{oneparchoices}
\choice $ 3x+7y-13=0 $
\choice $ 7x+3y-13=0 $
\choice $ 7x+3y+13=0 $
\choice $ 7x-3y-13 = 0 $
\end{oneparchoices}

\question একটি বাক্সে ১০টি নীল ও ১৫টি লাল মার্বেল আছে। একজন বালক যেমন খুশি টেনে প্রতিবারে একটি করে পর পর দুটি মার্বেল উঠালে দুটিই একই রঙের মার্বেল হওয়ার সম্ভাবনা কত? 

\begin{oneparchoices}
\choice $ \dfrac{1}{2} $
\choice $ \dfrac{4}{5} $
\choice $ \dfrac{3}{20} $
\choice $ \dfrac{7}{20} $
\end{oneparchoices}

\question  এককের জটিল ঘনমুল $ \omega $ হলে $ (1-\omega +\omega^{2})(1+\omega -\omega^{2}) $ এর মান

\begin{oneparchoices}
\choice $ -4 $
\choice $ 4 $
\choice  $ -3 $
\choice $ 3 $
\end{oneparchoices}

 \question  কোন স্তম্ভের শীর্ষ হতে $ 19.5\,ms^{-1} $ বেগে খাড়া উপরের দিকে কোন কণা 5 সেকেন্ড পরে স্তম্ভের পাদদেশে পতিত হলে স্তম্ভের উচ্চতা হবে –

\begin{oneparchoices}
\choice $ 20 $ m
\choice $ 25 $ m
\choice $ 30 $ m
\choice $ 50 $ m
\end{oneparchoices}

\question  $ \mathlarger{\int}\dfrac{1}{\cos^{2}x\sqrt{\tan x}}dx $ এর অনির্দিষ্ট যোগজ – 

\begin{oneparchoices}
\choice $ \sqrt{\tan x}\ln (\cos^{2} x)  $
\choice $ \sin x\sqrt{\tan x}  $
\choice $ 2\sqrt{\tan x}  $
\choice $ \dfrac{2}{3}(\tan x)^{\frac{3}{2}} $
\end{oneparchoices}

\question  যখন $ x\to 0  $ তখন  লিমিট  $ \dfrac{\tan^{-1}x}{x} $ কত?

\begin{oneparchoices}
\choice $ 1 $
\choice $ 0 $
\choice $ \dfrac{1}{2}$
\choice  does not exist
\end{oneparchoices}

\question  $ f(x) = x^{2}+4 $ এবং $ g(x) = 2x-1 $ হলে $ g(f(x)) $ হয় – 

\begin{oneparchoices}
\choice $ x^{2}+5 $
\choice $ 2x^{2}+7 $
\choice $ 2x^{2}-3 $
\choice $ x^{2}-5 $
\end{oneparchoices}

\question $ x=-1+i $ হলে $ x^{3}+3x^{2}+4x+7 $ এর মান – 

\begin{oneparchoices}
\choice $ 6+i $
\choice $ 8 $
\choice $ 5 $
\choice $ 9+2i $
\end{oneparchoices}

 \question  $ x^{2}-2x+3=0 $ সমীকরণের  মূলদ্বয় $ \alpha,\, \beta $ হলে $ \alpha+\beta,\, \alpha\beta $ মূল বিশিষ্ট সমীকরণটি হবে- 

\begin{oneparchoices}
\choice $ x^{2}-5x+6=0  $
\choice $ 3x^{2}-2x+1=0  $
\choice $ x^{2}-3x+2=0  $
\choice $ 2x^{2}-3x+1=0  $
\end{oneparchoices}

\question $ \sin (780^{\circ})\cos (390^{\circ}) - \sin (330^{\circ})\cos (-300^{\circ}) $ এর মান- 


\begin{oneparchoices}
\choice $ 0 $
\choice $ -1 $
\choice $ 1 $
\choice $ \dfrac{1}{2} $
\end{oneparchoices}

\question $ \begin{vmatrix}
x+y & x & y\\
x & x+z & z\\
y & z & y+z 
\end{vmatrix} $ নির্ণায়কটির মান- 


\begin{oneparchoices}
\choice $ 4xyz $ 
\choice $ x^{2}yz $ 
\choice  $ xy^{2}z $
\choice $ xyz^{2} $
\end{oneparchoices}

\question  যদি $ A = \begin{pmatrix}
2 & 0\\
0 & -3
\end{pmatrix},\, B=\begin{pmatrix}
3 & 0\\
5 & 1
\end{pmatrix} $ হয় তবে $ AB $ সমান- 

\begin{oneparchoices}
\choice $ \begin{pmatrix}
6 & 0\\
-15 & -3
\end{pmatrix} $
\choice  $ \begin{pmatrix}
3 & -1\\
2 & -5
\end{pmatrix} $
\choice $ \begin{pmatrix}
1 & 0\\
-2 & 15
\end{pmatrix} $
\choice $ \begin{pmatrix}
1 & -2\\
0 & 5
\end{pmatrix} $
\end{oneparchoices}

\question 6 জন ছাত্র এবং 5 জন ছাত্রী থেকে 5 জনের একটি কমিটি গঠন করতে হবে যাতে অন্তত একজন ছাত্র ও একজন ছাত্রী অর্ন্তভুক্ত থাকে । কতপ্রকারে এই কমিটি গঠন করা যেতে পারে?


\begin{oneparchoices}
\choice $ 360 $
\choice $ 160 $
\choice $ 410 $
\choice $ 455 $
\end{oneparchoices}

\question $ (x,y),\,(2,3) $ এবং $ (5,1) $ একই সরলরেখায় অবস্থিত হলে 

\begin{oneparchoices}
\choice $ 4x-3y-17=0 $
\choice $ 4x+3y-17=0 $
\choice $ 3x+4y+17=0 $
\choice $ 3x+4y-17=0 $
\end{oneparchoices}

\question 30 থেকে  40 পর্যন্ত সংখ্যা হতে যে কোন একটিকে ইচ্ছামত নিলে সেই সংখ্যাটি মৌলিক অথবা 5 এর গুণিতক হওয়ার সম্ভাবনা কত?

\begin{oneparchoices}
\choice $ \dfrac{1}{2} $
\choice $ \dfrac{5}{11} $
\choice $ \dfrac{6}{11} $
\choice $ \dfrac{3}{5} $
\end{oneparchoices}

\question  প্রতিবার প্রথম ও শেষে U  রেখে CALCULUS শব্দটির অক্ষরগুলোকে কতভাবে সাজানো যাবে?

\begin{oneparchoices}
\choice 180
\choice 280
\choice 90
\choice 360
\end{oneparchoices}

\question  দশমিক সংখ্যা 69 কে দ্বিমিকে প্রকাশ করলে হয় – 

\begin{oneparchoices}
\choice  $ 1011001 $
\choice  $ 1100101 $
\choice  $ 1000101 $
\choice  $ 1010101 $
\end{oneparchoices}

\question  $ x^{2}+y^{2}-5x=0,\,  x^{2}+y^{2}+3x=0  $ বৃত্তদ্বয়ের কেন্দ্রের দুরত্ব – 

\begin{oneparchoices}
\choice $ 4 $ units
\choice $ 1 $ unit
\choice $ \sqrt{34} $ units
\choice $ 2 $ units
\end{oneparchoices}

\question  $ \cot x -\tan x =2 $ সমীকরনের সাধারণ সমাধান- 

\begin{oneparchoices}
\choice $ \dfrac{n\pi}{4} $
\choice $ \dfrac{n\pi}{2} $
\choice $ \dfrac{(4n+1)\pi}{8} $
\choice $ \dfrac{(4n+1)\pi}{2} $
\end{oneparchoices}



\question $ (-9,9) $ ও $ (5,5) $ বিন্দুদ্বয়ের সংযোজক সরলখাকে ব্যাস ধরে অঙ্কিত বৃত্তের সমীকরণ- 

\begin{oneparchoices}
\choice $ x^{2}+y^{2}+4x+14y=0 $
\choice $ x^{2}+y^{2}+4x-14y=0 $
\choice $ x^{2}+y^{2}-4x+14y=0 $
\choice $ x^{2}+y^{2}-4x-14y=0 $
\end{oneparchoices}

\question  $ a $ এর যে মানের জন্য $ y=ax(1-x) $ বক্ররেখার মুলবিন্দুতে স্পর্শকটি অক্ষের সাথে $ 60^{\circ} $ কোণ উৎপন্ন করে- 

\begin{oneparchoices}
\choice $ \sqrt{3} $
\choice $ \dfrac{1}{\sqrt{3}} $
\choice $ \dfrac{\sqrt{3}}{2} $
\choice $ 1 $
\end{oneparchoices}

\question $ 5x-2y+4=0 $ এবং $ 4x-5y+5=0 $ সররেখার ছেদবিন্দু এবং মুলবিন্দু দিয়ে গমানকারী রেখার সমীকরণ-

\begin{oneparchoices}
\choice $ 2x-3y=0 $
\choice $ 3x-2y=0 $
\choice $ 2x-7y=0 $
\choice $ 9x+2y=0 $

\end{oneparchoices}

\question বাস্তবসংখ্যায় $ |3x-2|\le 1 $ অসমতাটির সমাধান – 

\begin{oneparchoices}
\choice $ \dfrac{1}{3}\le x$ or $x\le 1$
\choice $ x \le 2$ or $\dfrac{1}{2}\le x$
\choice $ x\ge 1$
\choice $x\le 3$ or $x\ge 1 $
\end{oneparchoices}


\end{questions}

\end{document}