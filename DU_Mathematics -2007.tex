\documentclass[addpoints]{exam}
\usepackage[utf8]{inputenc}
\usepackage{amsmath}
\usepackage{mathtools}
\usepackage{relsize}
\usepackage{dirtytalk}
\usepackage{graphicx}
\graphicspath{ {./drift/} }
\DeclarePairedDelimiter{\ceil}{\lceil}{\rceil}
\usepackage{geometry}
\usepackage{draftwatermark}
\SetWatermarkFontSize{6cm}
\SetWatermarkText{$e^{i\pi}+1=0$}
\usepackage[banglamainfont=Kalpurush, 
            banglattfont=Siyam Rupali
           ]{latexbangla}
\lfoot{Sayma Mostafa}
\rhead{\thepage}
\rfoot{প্রমি}
        
\begin{document}
\begin{LARGE}
\begin{center}
গণিত (Mathematics - 2007)
\end{center}
\end{LARGE}
\begin{questions}

\question  $ x^{2}-5x-1=0 $ সমীকরনের মূলদ্বয় হতে 2 কম মূলবিশষ্ট সমীকরণ –

\begin{oneparchoices}
\choice $ x^{2}+x+7=0 $
\choice $ x^{2}-x+7=0 $
\choice $ x^{2}+x-7=0 $
\choice $ x^{2}-x-7=0 $
\end{oneparchoices}

\question $ \begin{pmatrix}
p-4 & 8\\
2 & p+2
\end{pmatrix} $ ম্যাট্রিক্সটি ব্যতীক্রমী হবে যদি p এর মান 

\begin{oneparchoices}
\choice $ -4, 6 $
\choice $-6, 4$
\choice $ 4, 6$
\choice $-6, -4$
\end{oneparchoices}

\question $ \Bigg(2x + \dfrac{1}{6x} \Bigg)^{10} $ এর সম্প্রসারণে $ x- $ বর্জিত পদ হল – 

\begin{oneparchoices}
\choice $ \dfrac{28}{27} $
\choice $ \dfrac{27}{28} $
\choice $ 1 $
\choice  $ 3 $
\end{oneparchoices}

\question $ i^{2}=-1 $ হলে $ \dfrac{i-i^{-1}}{i+2i} $ এর মান –


\begin{oneparchoices}
\choice $ 0 $
\choice $ -2i $
\choice $ 2i $
\choice  $ -2 $
\end{oneparchoices}

\question $ \begin{vmatrix}
\alpha -3 & -1 \\
-8 & \alpha +4
\end{vmatrix} $ নির্ণায়কটির মান শূণ্য হলে এর মান – 

\begin{oneparchoices}
\choice $ 4$ or $-5 $
\choice $ 5$ or $-4 $
\choice $ 3 $
\choice $ 10 $
\end{oneparchoices}

\question SCHOOL শব্দটি হতে তিনটি অক্ষরদিয়ে পৃথকভাবে সাজানোর সংখ্যা – 

\begin{oneparchoices}
\choice 10
\choice 14
\choice 4
\choice 15
\end{oneparchoices}

\question $ y = 1 +\dfrac{1}{2+x} $ বক্ররেখা $ x-$অক্ষকে A বিন্দুতে $ y-$অক্ষকে B বিন্দুতে ছেদ করলে AB সরলরেখার সমীকরণ  হবে–

\begin{oneparchoices}
\choice $ x-2y+3=0 $
\choice $ x+2y+3=0 $
\choice $ 2x-y+3=0 $
\choice $ x-6y-3=0 $
\end{oneparchoices}

\question $ |2x-3|\le 1 $ বাস্তবসংখ্যায় অসমতাটির সমাধান – 

\begin{oneparchoices}
\choice $ 1<x<2 $
\choice $ 1\le x\le 2 $
\choice $ x\le 1 $ or $ x\ge 2 $
\choice $ x\le 2 $ or $x\ge 1$
\end{oneparchoices}

\question মূলবিন্দু হতে $ 3x+4y=10 $ সরলরেখাটির লম্বদুরত্ব - 

\begin{oneparchoices}
\choice 2
\choice 3
\choice 4
\choice  5
\end{oneparchoices}

\question $ 3x+7y-2=0 $ সরলরেখার উপর লম্ব এবং $ (2,-1) $ বিন্দুগামী সরলরেখার সমীকরণ – 
 
\begin{oneparchoices}
\choice $ 3x+7y-13=0 $
\choice $ 7x-3y-11=0 $
\choice $ 7x+3y-17=0 $
\choice $ 7x-3y-2 = 0 $
\end{oneparchoices}

\question সরলরেখা $ y= kx-1 $ বক্ররেখা $ y=x^{2}+3 $ এর স্পর্শক হবে যদি $ k $ এর একটি মান – 

\begin{oneparchoices}
\choice $ 1 $
\choice $ 2\sqrt{2}$
\choice $ 3 $
\choice $ 4 $
\end{oneparchoices}

\question $ (9,-9) $ও $ (-5,5) $ বিন্দুদ্বয়ের সংযোজক সরলরেখাকে ব্যাস ধরে অঙ্কিত বৃত্তের সমীকরণ –

\begin{oneparchoices}
\choice $ x^{2}+y^{2}-4x+4y+90=0 $
\choice $ x^{2}+y^{2}-4x+4y-90=0 $
\choice $ x^{2}+y^{2}+4x-4y-90=0 $
\choice $ x^{2}+y^{2}-4x-4y+90=0 $
\end{oneparchoices}

\question $ \cos 75^{\circ} $ এর সঠিক মান
এর মান – 

\begin{oneparchoices}
\choice $ \dfrac{\sqrt{3}+1}{2\sqrt{2}} $
\choice $ \dfrac{\sqrt{3}}{2\sqrt{2}} $
\choice $ -\dfrac{\sqrt{3}}{2\sqrt{2}} $
\choice $ \dfrac{\sqrt{3}-1}{2\sqrt{2}} $
\end{oneparchoices}

\question $ \tan^{-1}6+ \tan^{-1}\dfrac{7}{5} $  হলে এর মান – 

\begin{oneparchoices}
\choice $ \dfrac{\pi}{2} $
\choice $ \dfrac{3\pi}{2} $
\choice $ \dfrac{3\pi}{4} $
\choice $ \dfrac{\pi}{3} $
\end{oneparchoices}

\question $ 2\cos^{2}\theta + 2\sqrt{2}\sin\theta = 3 $ হলে $ \theta $ এর মান -  

\begin{oneparchoices}
\choice $ 30^{\circ} $
\choice $ 45^{\circ} $
\choice $ 60^{\circ} $
\choice  $ 135^{\circ} $
\end{oneparchoices}

\question $ \dfrac{(x+1)^{2}}{100} + \dfrac{(y-2)^{2}}{64} = 1 $ উপবৃত্তের উৎকেনন্দ্রিকতা – 

\begin{oneparchoices}
\choice $ 1 $
\choice $ \dfrac{3}{5} $
\choice $ \dfrac{5}{3} $
\choice $ \dfrac{4}{5} $
\end{oneparchoices}

\question $ x\to 0 $ হলে $ \dfrac{\tan^{-1}2x}{x} $ এর লিমিট কত?

\begin{oneparchoices}
\choice $ 1 $
\choice $ 0 $
\choice $ 2 $
\choice $ \dfrac{1}{2} $
\end{oneparchoices}

\question $ f(x) = x^{2}+4 $ এবং $ g(x) = 2x-1 $ হলে $ g(f(x)) $ এর মান -  

\begin{oneparchoices}
\choice $ 2x^{2}+7 $
\choice $ 7x^{2}+2 $
\choice $ x^{2}+2x-1 $
\choice $ x^{2}-2x+3 $
\end{oneparchoices}

\question যদি $ y= \ln (x+\sqrt{x^{2}+a^{2}}) $ হয় তবে $ \dfrac{dy}{dx} $ সমান

\begin{oneparchoices}
\choice $ \sqrt{x^{2}+a^{2}} $
\choice  $ \dfrac{1}{1+\sqrt{x^{2}+a^{2}}} $
\choice  $ 1+\sqrt{x^{2}+a^{2}} $
\choice $ \dfrac{1}{\sqrt{x^{2}+a^{2}}} $
\end{oneparchoices}

\question যদি $ x^{2}+3xy+5y^{2} = 1 $ হয় $ \dfrac{dy}{dx} $ সমান

\begin{oneparchoices}
\choice $ \dfrac{2x+3y}{3x+10y} $
\choice $ \dfrac{2x-3y}{3x+10y} $
\choice $ \dfrac{2x+3y}{3x-10y} $
\choice $ -\dfrac{2x+3y}{3x+10y} $
\end{oneparchoices}

\question $ \mathlarger{\int \dfrac{e^{x}(1+x)}{\cos^{2}(xe^{x})}} $ সমান

\begin{oneparchoices}
\choice $ \sin (xe^{x}) +c $
\choice $ \cot (xe^{x}) +c $
\choice $ \tan (xe^{x}) +c $
\choice $ \cos (xe^{x}) +c$
\end{oneparchoices}

\question  $ \mathlarger{\int}\dfrac{1}{\cos^{2}x\sqrt{\tan x}}dx $ সমান

\begin{oneparchoices}
\choice $ \sqrt{\tan x}\ln (\cos^{2} x) + c $
\choice $ \sin x\sqrt{\tan x} + c $
\choice $ 2\sqrt{\tan x} + c $
\choice $ \dfrac{2}{3}(\tan x)^{\frac{3}{2}} + c $
\end{oneparchoices}

\question $ \mathlarger{\int\limits_{0}^{1}}\dfrac{x}{\sqrt{1-x^{2}}}dx $ এর সমান

\begin{oneparchoices}
\choice $ \dfrac{1}{2} $
\choice $ \dfrac{\pi}{\sqrt{2}} $
\choice $ 1 $
\choice $ \dfrac{\pi}{2} $ 
\end{oneparchoices}

\question যদি $ \mathlarger{\int\limits_{0}^{4}}f(x)dx = 5$  হয় তবে $ \mathlarger{\int\limits_{1}^{5}}f(x-1)dx = 5$  এর মান

\begin{oneparchoices}
\choice $ 4 $
\choice $ 6 $
\choice $ 0 $
\choice $ 5 $
\end{oneparchoices}

\question  $ y^{2}=16x $ ও $ y =4x $ দ্বারা আবদ্ধ ক্ষেত্রের ক্ষেত্রফল – 

\begin{oneparchoices}
\choice $ \dfrac{3}{2} $ sq.units
\choice $ -\dfrac{3}{2} $ sq.units
\choice $ -\dfrac{2}{3} $ sq.units
\choice $ \dfrac{2}{3} $ sq.units
\end{oneparchoices}

\question  কোনবিন্দুতে ক্রিয়ারত দুটি বলের একটির মান অপরটির দ্বিগুন হলে এবং তাদের লদ্ধি ক্ষুদ্রতরটির উলম্ব হলে অন্তভুর্ক্ত কোণ হবে –


\begin{oneparchoices}
\choice $ 60^{\circ} $
\choice $ 120^{\circ} $
\choice $ 90^{\circ} $
\choice  $ 210^{\circ} $
\end{oneparchoices}

 \question  কোন স্তম্ভের শীর্ষ হতে $ 19.5\,ms^{-1} $ বেগে খাড়া উপরের দিকে কোন কণা 5 সেকেন্ড পরে স্তম্ভের পাদদেশে পতিত হলে স্তম্ভের উচ্চতা হবে –

\begin{oneparchoices}
\choice $ 20 $ m
\choice $ 25 $ m
\choice $ 30 $ m
\choice $ 50 $ m
\end{oneparchoices}

\question  দশমিক সংখ্যা 181 কে দ্বিমিকে প্রকাশ করলে হয় – 

\begin{oneparchoices}
\choice  $ 10220202 $
\choice  $ 10010111 $
\choice  $ 10101101 $
\choice  $ 11010011 $
\end{oneparchoices}

\question  30 থেকে 40 পর্যন্ত সংখ্যা হতে কোন একটিকে ইচ্ছামত নিলে সেই সংখ্যাটি মৌলিক হওয়ার সম্ভাবনা – 

\begin{oneparchoices}
\choice $ \dfrac{1}{2} $
\choice $ \dfrac{5}{11} $
\choice $ \dfrac{6}{11} $
\choice $ \dfrac{3}{5} $
\end{oneparchoices}

\question  একজন লোক তাঁর কাধে অনুভুমিকভাবে স্থাপিত 6 ফুট দীর্ঘ একটি লাঠির প্রান্তে হাত রেখে অপর প্রান্তে W ওজনের একটি বস্তু বহন করছে। কাধেঁর উপর চাপের পড়িমান বস্তুর ওজনের তিনগুন হলে কাঁধ হতে হাতের দুরত্ব -

\begin{oneparchoices}
\choice $ 3 $ feet
\choice $ 4 $ feet
\choice $ 2 $ feet
\choice $ 1 $ feet
\end{oneparchoices}

\end{questions}

\end{document}