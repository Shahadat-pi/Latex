\documentclass[addpoints]{exam}
\usepackage[utf8]{inputenc}
\usepackage{amsmath}
\usepackage{mathtools}
\usepackage{relsize}
\usepackage{dirtytalk}
\usepackage{graphicx}
\graphicspath{ {./drift/} }
\DeclarePairedDelimiter{\ceil}{\lceil}{\rceil}
\usepackage{geometry}
\usepackage{draftwatermark}
\SetWatermarkFontSize{2cm}
\SetWatermarkText{$2^{77,232,917}-1$}
\usepackage[banglamainfont=Kalpurush, 
            banglattfont=Siyam Rupali
           ]{latexbangla}
        
\begin{document}
\begin{LARGE}
\begin{center}
গণিত (Mathematics - 2009)
\end{center}
\end{LARGE}
\begin{questions}

\question $ x^{2}-7x+12=0 $  সমীকরনের মূলদ্বয় $       \alpha $ এবং $ \beta $ হলে $ \alpha
+\beta $ এবং $ \alpha\beta $ মূলবিশষ্ট সমীকরণ –

\begin{oneparchoices}
\choice $ x^{2}-19x+84=0 $
\choice $ x^{2}+14x-144=0 $
\choice $ x^{2}+19x-84=0 $
\choice $ x^{2}-14x+144=0 $
\end{oneparchoices}

\question $ \omega $ যদি 1 এর একটি জটিল ঘনমূল হয়, তবে প্রদত্ত নির্ণায়কটির মান – $ \begin{vmatrix}
1 & \omega & \omega^{2} \\
\omega^{2} & \omega & 1\\
\omega^{2} & 1 & \omega
\end{vmatrix} $  

\begin{oneparchoices}
\choice $ 0 $
\choice $ 1 $
\choice $ \omega $
\choice $ \omega^{2} $
\end{oneparchoices}

\question  $\begin{pmatrix}
p+4 & 8 \\
2 & p-2
\end{pmatrix} $ ম্যাট্রিক্সটি ব্যতিক্রমী হলে $ p $ এর মান

\begin{oneparchoices}
\choice  -6, 4
\choice  -4,6
\choice -4, 2
\choice  -2, 4
\end{oneparchoices}

\question 6 জন ছাত্র 5 জন ছাত্রী থেকে একটি কমিটি গঠন করতে হবে যাতে অন্তত: একজন ছাত্র এবং একজন ছাত্রী অর্ন্তভূক্ত থাকে। কত প্রকারে এ কমিটি গঠন করা যেতে পারে?

\begin{oneparchoices}
\choice 160
\choice 360
\choice 410
\choice 455
\end{oneparchoices}

\question $ \Big(\dfrac{2}{3} x^{2}-\dfrac{1}{3}x\Big)^{9} $ এর বিস্তৃতিতে বর্জিত পদটি কত?

\begin{oneparchoices}
\choice $ \dfrac{224}{3^{8}} $
\choice $ -\dfrac{224}{3^{8}} $
\choice $ \dfrac{242}{3^{8}} $
\choice $ -\dfrac{242}{3^{8}} $

\end{oneparchoices}

\question  $ n- $তম পদ পর্যন্ত $ 1.2.3+2.3.4+3.4.5+\cdots $ ধারাটির যোগফল-

\begin{oneparchoices}
\choice $ n(n+1)(n+2)(n+3) $
\choice $ (n+1)(n+2)(n+3)(n+4) $
\choice $ \dfrac{1}{2} n(n+1)(n+2)(n+3) $
\choice $ \dfrac{1}{4} n(n+1)(n+2)(n+3) $
\end{oneparchoices}

\question  A, B, C বিন্দুগুলির স্থানাংক $ (a, bc),\,(b,ca),\,(c,ab) $ হলে $ \triangle ABC $ এর ক্ষেফল কত?

\begin{oneparchoices}
\choice $ \dfrac{1}{2}abc $
\choice $ \dfrac{1}{2}(a-b)(b-c)(c-a) $
\choice $ \dfrac{1}{2}(b-a)(b-c)(c-a) $
\choice $ 3abc $
\end{oneparchoices}

 \question  $ 2x-3y+=0 $ সরলরেখার উপর লম্ব এবং $ (1,-1) $ বিন্দুগামী সরলরেখার সমীকরণ -

\begin{oneparchoices}
\choice $ 3x+2y=1 $
\choice $ 3x-2y=5 $
\choice $ 3x+2y=5 $
\choice $ 2x+3y = 1 $
\end{oneparchoices}

\question  $ (2,3) $ কেন্দ্রবিশিষ্ট ও $ x+y-2=0 $ রেখাকে স্পর্শ করে এমন বৃত্তের সমীকরণ -

\begin{oneparchoices}
\choice  $ 2(x^{2}+y^{2})-8x-12y+17=0 $
\choice  $ 2(x^{2}+y^{2})-6x-10y+15=0 $\\
\choice  $ 2(x^{2}+y^{2})-4x-8y+11=0 $
\choice  $ 2(x^{2}+y^{2})-2x-6y+7=0 $
\end{oneparchoices}

\question $ y^{2}=4x+8y $ পরাবৃত্তের শীর্ষবিন্দুর স্থানাংক -

\begin{oneparchoices}
\choice $ (4,4) $
\choice $ (-4,-4) $
\choice $ (4,-4) $
\choice  $ (-4,4) $
\end{oneparchoices}

\question $ \vec{B} = 6\hat{i}-3\hat{j}+2\hat{k} $ ভেক্টরের  $ \vec{A} =2\hat{i}+2\hat{j}+\hat{k} $ উপর ভেক্টরের অভিক্ষেপ –

\begin{oneparchoices}
\choice $ \dfrac{8}{7} $
\choice $ \dfrac{7}{8} $
\choice $ \dfrac{8}{5} $
\choice  $\dfrac{5}{8} $
\end{oneparchoices}

\question $ \cos 198^{\circ} +\sin 432^{\circ} + \tan 168^{\circ} + \tan 12^{\circ}$ এর মান

\begin{oneparchoices}
\choice $ 0 $
\choice $ -1 $
\choice $ 1 $
\choice $ \dfrac{1}{2} $
\end{oneparchoices}

\question $ 4(\cos\theta +\sin^{2}\theta) =5 $ সমীকরণের সাধারণ সমাধান -

\begin{oneparchoices}
\choice $ \theta = 2n\pi \pm \dfrac{\pi}{2} $
\choice $ \theta = 2n\pi \pm \dfrac{\pi}{3} $
\choice $ \theta = 2n\pi \pm \dfrac{\pi}{4} $
\choice $ \theta = 2n\pi \pm \dfrac{\pi}{5} $
\end{oneparchoices}

\question $ i^{2}=-1 $ হলে $ \dfrac{i^{-1}-i}{i+2i^{-1}} $ এর মান

\begin{oneparchoices}
\choice $ 2 $
\choice $ -2i $
\choice $ 2i $
\choice  $ -2 $
\end{oneparchoices}

\question $ \cos\theta = \dfrac{12}{13} $ হলে $ \theta $ এর মান

\begin{oneparchoices}
\choice $ \pm \dfrac{5}{12} $
\choice $  \dfrac{25}{144} $
\choice $ \dfrac{13}{12} $
\choice  $ \pm \dfrac{13}{12} $
\end{oneparchoices}

\question   বাস্তব সংখ্যায় $ \dfrac{1}{|2x-3|}> 5 $ অসমতাটির সমাধান হলো-

\begin{oneparchoices}
\choice $ \Big(\dfrac{7}{5}, \dfrac{8}{5}\Big)$
\choice $ \Big[\dfrac{7}{5}, \dfrac{8}{5}\Big]$
\choice $ \Big(\dfrac{7}{5}, \dfrac{3}{5}\Big)\cup \Big(\dfrac{3}{2}, \dfrac{8}{5}\Big)  $
\choice $ \Big[\dfrac{7}{5}, \dfrac{3}{2}\Big]\cup \Big[\dfrac{3}{2}, \dfrac{8}{5}\Big]  $
\end{oneparchoices}

\question $ f(x) = \sin x,\, g(x)=x^{2} $ হলে $ f\Big(g\Big(\dfrac{\sqrt{\pi}}{2}\Big) \Big) $ এর মান - 

\begin{oneparchoices}
\choice $ \dfrac{\sqrt{2}}{2} $
\choice $  \dfrac{\sqrt{3}}{2} $
\choice $ \dfrac{1}{2} $
\choice  $ 1 $
\end{oneparchoices}

\question  $ \mathlarger{\lim_{x\to 0} \dfrac{\sin x^{2}}{x}}=? $ 

\begin{oneparchoices}
\choice 1
\choice $ -1 $
\choice 0
\choice $ 2 $
\end{oneparchoices}

\question $ x^{2}+xy+ y^{2}=0  $ হলে $ (3,-4) $ বিন্দুতে $ \dfrac{dy}{dx} $ এর মান - 

\begin{oneparchoices}
\choice $ \dfrac{2}{5} $
\choice $ \dfrac{5}{2} $
\choice $ \dfrac{3}{8} $
\choice $ \dfrac{8}{3} $
\end{oneparchoices}

\question যদি $ y=\ln (x+\sqrt{x^{2}+4})  $ হয় তবে   $\dfrac{dy}{dx} $ সমান - 

\begin{oneparchoices}
\choice $ \sqrt{x^{2}+4} $
\choice $ \dfrac{1}{1+\sqrt{x^{2}+4}} $
\choice $ 1+\sqrt{x^{2}+4} $
\choice $ \dfrac{1}{\sqrt{x^{2}+4}} $
\end{oneparchoices}

\question $ \mathlarger{\int \dfrac{dx}{e^{x}+e^{-x}}=?} $

\begin{oneparchoices}
\choice $ \tan^{-1} (e^{x}) +c $
\choice $ \tan (e^{x}+e^{-x}) +c $
\choice $ \tan (e^{x}) +c $
\choice $ \tan (e^{-x}) +c$
\end{oneparchoices}

\question  $ \mathlarger{\int_{1}^{e}\ln x\, dx} $ এর  মান – 

\begin{oneparchoices}
\choice  $ e $
\choice  $ e-1 $
\choice  1
\choice  $ 1-e $
\end{oneparchoices}

\question  $ \mathlarger{\int\dfrac{1}{\cos^{2}x\sqrt{\tan x}} dx} =?$  

\begin{oneparchoices}
\choice  $ \sqrt{\tan x}\ln (\cos^{2}x)+c $
\choice  $ 2\sqrt{\tan x} +c $
\choice  $ 2\sqrt{\tan x +c}  $
\choice  $ \dfrac{2}{3}(\tan x)^{\frac{3}{2}} +c $
\end{oneparchoices}

\question  $ \mathlarger{\int_{0}^{1}\dfrac{\cos^{-1}x}{\sqrt{1-x^{2}}}dx} $এর মান

\begin{oneparchoices}
\choice $ \dfrac{\pi^{2}}{2} $
\choice $ \dfrac{\pi^{2}}{8} $
\choice $ \dfrac{\pi^{2}}{4} $
\choice $ \dfrac{\pi^{2}}{16} $

\end{oneparchoices}

\question  $ u $ বেগে আনুভুমিকের সাথে $ \alpha $ কোণে প্রক্ষিপ্ত বস্তুর সর্বোচ্চ উচ্চতা হবে-

\begin{oneparchoices}
\choice $ \dfrac{u^{2}\sin 2\alpha}{2g} $
\choice $ \dfrac{u^{2}\sin^{2}\alpha}{2g} $
\choice $ \dfrac{u^{2}\sin 2\alpha}{g} $
\choice $ \dfrac{u^{2}\sin^{2}\alpha}{g} $

\end{oneparchoices}

\question  একটি বুলেট কোন দেয়ালের মধ্যে 2 ইঞ্চি ঢুকার পর এর বেগ অর্ধেক হারায়।বুলেটি দেয়ালে আর কতদুর ঢুকবে?

\begin{oneparchoices}
\choice $ 2^{"} $
\choice $ (\dfrac{2}{3})^{"} $
\choice $ 1^{"} $
\choice $ (\dfrac{1}{2})^{"} $

\end{oneparchoices}

\question  3P এবং 2P বলদ্বয়ের লদ্ধি R। প্রথম বলটিদ্বিগুণ করা হলে লদ্ধিও দ্বিগুণ হয়। বলদ্বয়ের অন্তর্গত কোণ হবে-

\begin{oneparchoices}
\choice $ 110^{\circ} $
\choice $ 150^{\circ} $
\choice $ 120^{\circ} $
\choice $ 135^{\circ} $

\end{oneparchoices}

\question দশমিক সংখ্যা 214 কে দ্বিমিক পদ্ধতিতে প্রকাশ করলে হয় - 

\begin{oneparchoices}
\choice 11010110
\choice 10100110
\choice 11001010
\choice 10111011
\end{oneparchoices}



\question নিন্মের লিনিয়ার প্রোগ্রামিং সমস্যার সমাধান কর:
গরিষ্ঠকরণ কর: $ z=3x+2y $\\
শর্ত হচ্ছে-$ x+y\le 7,\, 2x+5y\le 20,\, x\ge 0,\,y\ge 0 $


\begin{oneparchoices}
\choice $ (5,2) $
\choice $ (7,0) $
\choice $ (10,0) $
\choice $ (0,7) $
\end{oneparchoices}

\question  40 থেকে 50 সংখ্যাগুলির মধ্যে দৈবচয়ন পদ্ধতিতে একটি সংখ্যা নেয়া হল। সংখ্যাটি মৌলিক না হওয়ার সম্ভাবনা-

\begin{oneparchoices}
\choice $ \dfrac{8}{11} $
\choice  $ \dfrac{5}{11} $
\choice $ \dfrac{3}{11} $
\choice $ \dfrac{1}{11} $

\end{oneparchoices}

\end{questions}

\end{document}