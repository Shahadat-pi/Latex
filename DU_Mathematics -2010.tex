\documentclass[addpoints]{exam}
\usepackage[utf8]{inputenc}
\usepackage{amsmath}
\usepackage{mathtools}
\usepackage{relsize}
\usepackage{dirtytalk}
\usepackage{graphicx}
\graphicspath{ {./drift/} }
\DeclarePairedDelimiter{\ceil}{\lceil}{\rceil}
\usepackage{geometry}
\usepackage{draftwatermark}
\SetWatermarkFontSize{8cm}
\SetWatermarkText{$e^{i\pi} + 1=0$}
\usepackage[banglamainfont=Kalpurush, 
            banglattfont=Siyam Rupali
           ]{latexbangla}
        
\begin{document}
\begin{LARGE}
\begin{center}
গণিত (Mathematics - 2010)
\end{center}
\end{LARGE}
\begin{questions}

 \question নির্ণায়ক $ \begin{vmatrix}
 1 & bc & bc(b+c)\\
 1 & ca & ca(c+a)\\
 1 & ab & ab(a+b)
\end{vmatrix}   $  এর মান কত?

\begin{oneparchoices}
\choice $ abc(a+b)(b+c)(c+a) $
\choice $ abc(a+b+c) $
\choice $ 1 $
\choice $ 0 $
\end{oneparchoices}

\question $ \Bigg(2x + \dfrac{1}{6x} \Bigg)^{10} $ এর সম্প্রসারণে $ x- $ বর্জিত পদ হল – 

\begin{oneparchoices}
\choice $ \dfrac{28}{27} $
\choice $ \dfrac{27}{28} $
\choice $ \dfrac{540}{243} $
\choice  $ 0 $
\end{oneparchoices}

\question 70 শিক্ষার্থী গণিত, পদার্থবিদ্যা ও রসায়ন অধ্যয়ন করে। তার মধ্যে 40 জন শিক্ষার্থী গণিত, 35 জন পদার্থবিদ্যা এবং 30 জন রসায়ন অধ্যয়ন করে। 15 জন শিক্ষার্থী তিনটি বিষয়ই অধ্যয়ন করে। কত জন শিক্ষার্থী কেবল দুইটি বিষয় অধ্যয়ন করে?


\begin{oneparchoices}
\choice 5
\choice 6
\choice 8
\choice 9

\end{oneparchoices}

\question $ 5-3x-x^{2} $  এর সর্বোচ্চ মান

\begin{oneparchoices}
\choice 3
\choice 5
\choice $ \dfrac{47}{4} $
\choice $ \dfrac{29}{4} $
\end{oneparchoices}

\question  যদি $ A = \begin{bmatrix}
-2 & 1\\
\dfrac{3}{2} & -\dfrac{1}{2} 
\end{bmatrix} $ হয় তবে $ A^{-1} $ সমান

\begin{oneparchoices}
\choice $\begin{bmatrix}
1 & 0\\
0 & 1 
\end{bmatrix} $
\choice $\begin{bmatrix}
1 & 3\\
2 & 4 
\end{bmatrix} $
\choice $\begin{bmatrix}
3 & 4\\
1 & 2 
\end{bmatrix} $
\choice $\begin{bmatrix}
1 & 2\\
3 & 4 
\end{bmatrix} $
\end{oneparchoices}

\question  এককের জটিল ঘনমুল $ \omega $ হলে $ (\omega +\omega^{2}-1)(1+\omega -\omega^{2})(1-\omega +\omega^{2}) $ এর মান

\begin{oneparchoices}
\choice $ -8 $
\choice $ 8 $
\choice  $ 0 $
\choice 1
\end{oneparchoices}

\question  একটি বৃত্ত $ (-1,-1) $ এবং $ (3,2) $ বিন্দুগামী এবং কেন্দ্র $ x+2y+3=0 $ সরলরেখার উপর অবস্থিত। বৃত্তটির সমীকরণ –

\begin{oneparchoices}
\choice  $ x^{2}+y^{2}-4x+5y-15=0 $
\choice  $ x^{2}+y^{2}-8x+7y-3=0 $
\choice  $ x^{2}+y^{2}+8x-7y-3=0 $
\choice  $ x^{2}+y^{2}+4x-5y+15=0 $
\end{oneparchoices}

\question $ 3x+ky-1=0 $ রেখাটি $ x^{2}+y^{2}-8x-2y+4=0 $ বৃত্তকে স্পর্শকরলে $ k $ এর মান নির্ণয় কর-

\begin{oneparchoices}
\choice $ 2,\,\dfrac{1}{6} $
\choice $ -2,\,\dfrac{1}{6} $
\choice $ 2,\,-\dfrac{1}{6} $
\choice  $ -2,\,-\dfrac{1}{6} $
\end{oneparchoices}

\question  $ a $ এর কোন মানের জন্য $ 2\hat{i} +\hat{j}-\hat{k} $, $ 3\hat{i} +2\hat{j}-4\hat{k} $ এবং $ \hat{i} -3\hat{j}-a\hat{k} $ ভেক্টরত্রয় সমতলীয় ?

\begin{oneparchoices}
\choice 5
\choice 4
\choice 3
\choice 2
\end{oneparchoices}

\question  $ \tan^{-1}{\dfrac{1}{7}}+\tan^{-1}{\dfrac{1}{8}} + \tan^{-1}{\dfrac{1}{18}} $ সমান

\begin{oneparchoices}
\choice $ \cot^{-1}{\dfrac{1}{3}} $
\choice $ \cot^{-1}3 $
\choice $ \tan^{-1}{\dfrac{1}{3}} $
\choice $ \sin^{-1}3 $
\end{oneparchoices}

\question  $ \sin^{2}2\theta - 3\cos^{2}\theta = 0 $ সমীকরণের সাধারণ সমাধান - 

\begin{oneparchoices}
\choice $ 2n\pi \pm \dfrac{\pi}{3} $
\choice $ n\pi \pm \dfrac{\pi}{3} $
\choice $ n\pi \pm \dfrac{\pi}{6} $
\choice $ 2n\pi \pm \dfrac{\pi}{6} $
\end{oneparchoices}

\question যদি $ A+B+C = \pi $ হয়, তবে $ \sin^{2}\dfrac{A}{2}+\sin^{2}\dfrac{B}{2}+\sin^{2}\dfrac{C}{2} $ সমান – 

\begin{oneparchoices}
\choice $ 1-2\sin \dfrac{A}{2}\sin \dfrac{B}{2}\sin \dfrac{C}{2} $
\choice $ 1+2\sin \dfrac{A}{2}\sin \dfrac{B}{2}\sin \dfrac{C}{2} $\\
\choice $ 1-\sin \dfrac{A}{2}\sin \dfrac{B}{2}\sin \dfrac{C}{2} $
\choice $ 1-2\sin \dfrac{A}{2}\sin \dfrac{B}{2}\sin \dfrac{C}{2} $
\end{oneparchoices}

\question ENGINEERING শব্দের E গুলো একসঙ্গে রেখে সকল অক্ষরগুলোর বিন্যাস সংখ্যা – 

\begin{oneparchoices}
\choice 1680
\choice 15120
\choice 277200
\choice 1512
\end{oneparchoices}

\question $ 3(9^{x}-4.3^{x-1}) +1=0 $ সমীকরণের সমাধান – 
 
\begin{oneparchoices}
\choice $ x=0, -1 $
\choice $ x=\dfrac{1}{3}, 1 $
\choice $ x=1, 0 $
\choice $ x=1, -1 $
\end{oneparchoices}


\question বাস্তবসংখ্যায় $ |5-2x|\le 2 $ অসমতাটির সমাধান – 

\begin{oneparchoices}
\choice $ -1<x\le 9 $
\choice $ \dfrac{1}{2}\le x \dfrac{9}{2} $
\choice $ x\le -\dfrac{1}{2}$ or $ x\ge \dfrac{9}{2} $
\choice $ -\dfrac{1}{2}<x < \dfrac{9}{2} $
\end{oneparchoices}

\question $ x=a(\theta -\sin \theta),\, y = a(1-\cos\theta);\, \dfrac{dy}{dx}=?  $
 
\begin{oneparchoices}
\choice $ \cot\dfrac{\theta}{2} $
\choice $ \tan\dfrac{\theta}{2} $
\choice $ \cos\dfrac{\theta}{2} $
\choice $ \sin\dfrac{\theta}{2}$
\end{oneparchoices}

\question যে বিন্দু $ (1,4) $ এবং $ (9,-12) $ বিন্দুদ্বয়ের সংযোগকারী সরলরেখাকে $ 3:5 $ অনুপাতে বিভক্ত করে তার স্থানাংক –

\begin{oneparchoices}
\choice $ (4,-2) $
\choice $ (2,-4) $
\choice $ (-4, 2) $
\choice $ (4,2) $
\end{oneparchoices}

\question  $ 5x-7y =15 $ রেখার উপর লম্ব এবং $ (2,-3) $ বিন্দুগামী সরলরেখার সমীকরণটি –

\begin{oneparchoices}
\choice  $ 7x-5y=1 $
\choice  $ 7x+5y=15 $
\choice  $ 5x+7y+15=0 $
\choice  $ 7x+5y+1 =0 $
\end{oneparchoices}

\question  $ y^{2}=4x $ এবং $ y=x $ দ্বারা আবদ্ধ ক্ষেত্রের ক্ষেত্রফল – 

\begin{oneparchoices}
\choice $ \dfrac{8}{3} $
\choice $ 3 $
\choice $ 8 $
\choice $ \dfrac{3}{8} $
\end{oneparchoices}

\question দ্বিমিক সংখ্যা 10011010111 দশমিক সংখ্যাতে  প্রকাশ  –

\begin{oneparchoices}
\choice 1237
\choice 1239
\choice 1241
\choice 1247
\end{oneparchoices}

\question 1 থেকে 520 পর্যন্ত সংখ্যাগুলো থেকে দৈবচয়ন পদ্ধতিতে একটি সংখ্যা নেয়া হলে সংখ্যাটি অযুগ্ম ঘন হওয়ার সম্ভাবনা কত?

\begin{oneparchoices}
\choice $ \dfrac{1}{65} $
\choice $ \dfrac{2}{65} $
\choice $ \dfrac{1}{130} $
\choice $ \dfrac{1}{64} $

\end{oneparchoices}

\question  $ x\ge 0,\, y\ge 0,\, x+y = 5,\, x\ge 2,\, y\le 2 $ শর্তসমূহ সাপেক্ষে $ z=6x+2y $ রাশিটির সবোর্চ্চ মান – 

\begin{oneparchoices}
\choice $ 22 $
\choice  $ 20 $
\choice $ 18 $
\choice $ 30 $
\end{oneparchoices}


\question ভুমি হতে $ u $ আদিবেগে উর্ধ্বমুখী কোন কণার সবোর্চ্চ উচ্চতা – 

\begin{oneparchoices}
\choice  $ \dfrac{u}{2g} $
\choice  $ \dfrac{u^{2}}{g} $
\choice  $ \dfrac{u^{2}}{2g} $
\choice  $ \dfrac{2u}{g} $
\end{oneparchoices}

\question  $ f(x) = \dfrac{5x+3}{4x-5} $ হলে $  f^{-1}(x) $ সমান –

\begin{oneparchoices}
\choice $ \dfrac{5x+3}{4x-5}$
\choice $ \dfrac{4x-5}{5x+3}$
\choice $ \dfrac{5x-3}{4x-5}$
\choice  $ \dfrac{5x+3}{4x+5}$
\end{oneparchoices}

\question $ y= \tan^{-1}\dfrac{2x}{1-x^{2}} $ হলে $ \dfrac{dy}{dx} $ এর মান – 

\begin{oneparchoices}
\choice $ \dfrac{2}{1-x^{2}} $
\choice $ \dfrac{2}{\sqrt{1+x^{2}}} $
\choice $ \dfrac{2}{\sqrt{1-x^{2}}} $
\choice  $ \dfrac{2}{1+x^{2}} $
\end{oneparchoices}

\question $ \mathlarger{\int_{1}^{e^{2}}}\dfrac{dx}{x(1+\ln x)^{2}} $  এর মান – 

\begin{oneparchoices}
\choice $ \dfrac{1}{2} $
\choice $ \dfrac{3}{2} $
\choice $ \dfrac{2}{3} $
\choice $ \dfrac{1}{3} $

\end{oneparchoices}

\question $ \mathlarger{\int}\dfrac{xe^{x}}{(x+1)^{2}}dx $ সমান-

\begin{oneparchoices}
\choice $ \dfrac{x}{(x+1)}+c $
\choice $ \dfrac{x}{(x+1)^{2}}+c $
\choice $ \dfrac{e^{x}}{(x+1)}+c $
\choice $ \dfrac{e^{x}}{(x+1)^{2}}+c $
\end{oneparchoices}

\question $ \mathlarger{\lim_{x\to 0}}\dfrac{e^{x}-e^{-x}-2\ln (1+x)}{x\sin x} $  এর মান – 

\begin{oneparchoices}
\choice $ 0 $
\choice $ -1 $
\choice $ 1 $
\choice $ \infty $

\end{oneparchoices}

\question একটি গাড়ি সমত্বরণে 30km/hour আদিবেগে 100m পথ অতিক্রম করে 50km/hour চূড়ান্ত বেগ অর্জন করে। গাড়িটির ত্বরণ –

\begin{oneparchoices}
\choice $ 8\,kmh^{-2}$
\choice $ 800\,kmh^{-2}$
\choice $ 16\,kmh^{-2}$
\choice $ 80\,kmh^{-2}$

\end{oneparchoices}

\question  20m/sec বেগে উর্ধ্বগামী কোন বেলুন থেকে একটুকরা পাথর 20 সেকেন্ড পরে মাটিতে পড়ল। পাথরের টুকরা পতিত হওয়ার সময় বেলুনের উচ্চতা কত ছিল? 

\begin{oneparchoices}
\choice $ 390\,m $
\choice $ 650\,m $
\choice $ 12580\,m $
\choice $ 1960\,m $

\end{oneparchoices}

\end{questions}

\end{document}