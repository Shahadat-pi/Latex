\documentclass[addpoints]{exam}
\usepackage[utf8]{inputenc}
\usepackage{amsmath}
\usepackage{mathtools}
\usepackage{relsize}
\usepackage{dirtytalk}
\usepackage{graphicx}
\graphicspath{ {./drift/} }
\DeclarePairedDelimiter{\ceil}{\lceil}{\rceil}
\usepackage{geometry}
\usepackage{draftwatermark}
\SetWatermarkFontSize{2cm}
\SetWatermarkText{$2^{77,232,917}-1$}
\usepackage[banglamainfont=Kalpurush, 
            banglattfont=Siyam Rupali
           ]{latexbangla}
        
\begin{document}
\begin{LARGE}
\begin{center}
গণিত (Mathematics - 2011)
\end{center}
\end{LARGE}
\begin{questions}

 \question  $ 3x+7y-2=0 $ সরলরেখার উপর লম্ব এবং $ (2,1) $ বিন্দুগামী সরলরেখার সমীকরণ -

\begin{oneparchoices}
\choice $ 3x+7y-13=0 $
\choice $ 7x-3y-11=0 $
\choice $ 7x+3y-17=0 $
\choice $ 7x-3y-2 =0 $
\end{oneparchoices}

\question  কোন স্তম্ভের শীর্ষ হতে $ 19.5\,m/sec $ বেগে খাড়া উপরের দিকে নিক্ষিপ্ত কোন কণা 5 সেকেন্ড পরে স্তম্ভের পাদদেশে পতিত হলে স্তম্ভের উচ্চতা – 

\begin{oneparchoices}
\choice $ 20\,m$
\choice $ 25\,m$
\choice $ 30\,m $
\choice $ 50\,m $
\end{oneparchoices}

\question 6 জন ছাত্র 5 জন ছাত্রী থেকে একটি কমিটি গঠন করতে হবে যাতে অন্তত: একজন ছাত্র এবং একজন ছাত্রী অর্ন্তভূক্ত থাকে। কত প্রকারে এ কমিটি গঠন করা যেতে পারে?

\begin{oneparchoices}
\choice 160
\choice 360
\choice 410
\choice 455

\end{oneparchoices}

\question  $\begin{pmatrix}
m-2 & 6 \\
2 & m-3
\end{pmatrix} $ ম্যাট্রিক্সটি ব্যতিক্রমী হলে $ m $ এর মান

\begin{oneparchoices}
\choice 6, -1
\choice -4,6
\choice -6, 4
\choice 1, -6
\end{oneparchoices}

\question  $ \mathlarger{\lim_{x\to 0} \dfrac{\sin^{-1}2x}{x}} $ এর মান

\begin{oneparchoices}
\choice 1
\choice 0
\choice 2
\choice $ \dfrac{1}{2} $
\end{oneparchoices}

\question  $ \lambda $ এর যে মানের জন্য $ y=\lambda x(1-x) $ বক্ররেখার স্পর্শকটি মূলবিন্দুতে $ x- $অক্ষের সাথে $ 30^{\circ} $ কোণ উৎপন্ন করে -

\begin{oneparchoices}
\choice $ \sqrt{3} $
\choice $ \dfrac{1}{\sqrt{3}} $
\choice  $ \dfrac{\sqrt{3}}{2} $
\choice 1

\end{oneparchoices}

\question  $ (2,4) $ কেন্দ্রবিশিষ্ট ও $ x- $অক্ষকে স্পর্শ করে এমন বৃত্তের সমীকরণ -

\begin{oneparchoices}
\choice  $ x^{2}+y^{2}-4x-8y+16=0 $
\choice  $ x^{2}+y^{2}-4x-8y+4=0 $
\choice  $ x^{2}+y^{2}-8x+4y+16=0 $
\choice  $ x^{2}+y^{2}-8x-4y+4=0 $
\end{oneparchoices}

\question $ i^{2}=-1 $ হলে $ \dfrac{i-i^{-1}}{i+2i^{-1}} $ এর মান

\begin{oneparchoices}
\choice $ 0 $
\choice $ -2i $
\choice $ 2i $
\choice  $ -2 $

\end{oneparchoices}

\question  $ \mathlarger{\int_{0}^{1}\dfrac{\sin^{-1}x}{\sqrt{1-x^{2}}}dx} $এর মান

\begin{oneparchoices}
\choice $ \dfrac{\pi}{2} $
\choice $ \dfrac{\pi^{2}}{8} $
\choice $ \dfrac{\pi^{2}}{4} $
\choice $ \dfrac{\pi^{2}}{16} $

\end{oneparchoices}

\question  $ \dfrac{(x+4)^{2}}{100} +\dfrac{(y-2)^{2}}{64} =1 $ হলে $ e=? $

\begin{oneparchoices}
\choice 1
\choice $ \dfrac{3}{5} $
\choice $ \dfrac{5}{3} $
\choice $ \dfrac{4}{5} $

\end{oneparchoices}

\question দশমিক সংখ্যা 181 কে দ্বিমিক পদ্ধতিতে প্রকাশ করলে হয় - 

\begin{oneparchoices}
\choice 10110101
\choice 10011011
\choice 11001010
\choice 10111011
\end{oneparchoices}

\question $ \cos\theta +\sqrt{3}\sin\theta =2 $ সমীকরণের সাধারণ সমাধান -

\begin{oneparchoices}
\choice $ \theta = 2n\pi -\dfrac{\pi}{3} $
\choice $ \theta = 2n\pi +\dfrac{\pi}{3} $
\choice $ \theta = 2n\pi +\dfrac{\pi}{6} $
\choice $ \theta = 2n\pi -\dfrac{\pi}{6} $
\end{oneparchoices}

\question যে সমীকরণের মূলগুলো $ x^{2}-5x-1=0 $ সমীকরণের মূল গুলো হতে 2 ছোট তা – 

\begin{oneparchoices}
\choice $ x^{2}+x+7=0 $
\choice $ x^{2}-x+7=0 $
\choice $ x^{2}-x-7=0 $
\choice $ x^{2}+x-7=0 $
\end{oneparchoices}


\question বাস্তব সংখ্যায় $ |3-2x|\le 1 $ অসমতাটির সমাধান-

\begin{oneparchoices}
\choice $ 1<x<2 $
\choice $ 1\le x \le 2 $
\choice $ x\le 1$ or $ x\ge 2 $
\choice $ 1<x\le 2 $
\end{oneparchoices}

\question $ \dfrac{\sin 75^{\circ}+\sin 15^{\circ}}{\sin 75^{\circ}-\sin 15^{\circ}} $ এর মান

\begin{oneparchoices}
\choice $ \sqrt{5} $
\choice $ \sqrt{3} $
\choice $ -\sqrt{5} $
\choice $ -\sqrt{3} $
\end{oneparchoices}

\question $ \mathlarger{\int \dfrac{e^{x}(1+x)}{\cos^{2}(xe^{x})}} $ সমান

\begin{oneparchoices}
\choice $ \sin (xe^{x}) +c $
\choice $ \cot (xe^{x}) +c $
\choice $ \tan (xe^{x}) +c $
\choice $ \cos (xe^{x}) +c$
\end{oneparchoices}

\question  $ x^{2}-2x+5=0 $ এর নূন্যতম মান – 

\begin{oneparchoices}
\choice  1
\choice  2
\choice  3
\choice  4
\end{oneparchoices}

\question  $ 1-\dfrac{1}{2}+\dfrac{1}{2^{2}}-\dfrac{1}{2^{3}}+\dfrac{1}{2^{4}}-\dfrac{1}{2^{5}}+\cdots \infty = $ 

\begin{oneparchoices}
\choice $ \dfrac{2}{3} $
\choice $ \dfrac{4}{3} $
\choice $ 2 $
\choice $ \dfrac{1}{3} $
\end{oneparchoices}

\question $ x^{2}-x+4y-4=0 $ পরাবৃত্তের শীর্ষবিন্দুর স্থানাংক -

\begin{oneparchoices}
\choice $ (-4,2) $
\choice $ (4,-2) $
\choice $ (4,5) $
\choice  $ (5,4) $
\end{oneparchoices}

\question স্রোত না থাকলে একটি ছেলে 5 মিনিটে সাতার কেটে সোজাসুজিভাবে 80 মিটার প্রশস্ত একটি খাল পার হতে পারে এবং স্রোত থাকলে দ্বিগুন সময় লাগে। স্রোতের বেগ -

\begin{oneparchoices}
\choice $ 15\,m/min $
\choice $ 16.5\,m/min $
\choice $ 12\,m/min $
\choice $ 13.86\,m/min $

\end{oneparchoices}

\question  $ \Big(2x^{2}+\dfrac{k}{x^{3}}\Big)^{10} $ এর বিস্তৃতিতে $ x^{5} $ এবং $ x^{15} $ এর সহগদ্বয় সমান হলে $ k $ এর ধনাত্মক মান -

\begin{oneparchoices}
\choice $ \dfrac{1}{\sqrt{2}} $
\choice  $ \dfrac{1}{\sqrt{3}} $
\choice $ \dfrac{1}{2} $
\choice $ \dfrac{1}{\sqrt{5}} $

\end{oneparchoices}


\question প্রতিবার প্রথমে ও শেষে U রেখে CALCULUS শব্দটির অক্ষরগুলোকে কতভাবে সাজানো যাবে?

\begin{oneparchoices}
\choice  90
\choice  180
\choice  280
\choice  360
\end{oneparchoices}

\question  $ y^{2}=16x $ ও $ y=4x $ দ্বারা আবদ্ধ ক্ষেত্রের ক্ষেত্রের ক্ষেত্রফল – 

\begin{oneparchoices}
\choice $ \dfrac{3}{2}\, sq.\, units $
\choice $ \dfrac{3}{4}\, sq.\, units $
\choice $ \dfrac{4}{3}\, sq.\, units $
\choice $ \dfrac{2}{3}\, sq.\, units $
\end{oneparchoices}

\question যদি $ \vec{AB} = 2\hat{i}+\hat{j} $ এবং $ \vec{AC} =3\hat{i}-\hat{j}+5\hat{k} $ হয় তবে $ \vec{AB} $ ও $ \vec{AC} $ কে সন্নিহিত বাহু ধরে অংকিত সামান্তরিকের ক্ষেত্রফল – 

\begin{oneparchoices}
\choice $ 2\sqrt{6} $
\choice $ 3\sqrt{6} $
\choice $ 4\sqrt{6} $
\choice  $ 5\sqrt{6} $
\end{oneparchoices}

\question $ \sqrt{3} $ এককের দুটি সমান বল $ 120^{\circ} $ কোণে একবিন্দুতে কাজ করে। তাদের লদ্ধির মান – 

\begin{oneparchoices}
\choice $ \sqrt{3}\, units $
\choice $ 4\sqrt{3}\, units $
\choice $ 3\, units $
\choice $2 \sqrt{3}\, units $
\end{oneparchoices}

\question   যদি $ y =\dfrac{\tan x-\cot x}{\tan x +\cot x} $ হয় তবে $ \dfrac{dy}{dx} $ সমান-

\begin{oneparchoices}
\choice $ 2\sin 2x $
\choice $ 2\cos 2x $
\choice $ 2\tan 2x $
\choice $ 2\cot 2x $

\end{oneparchoices}

\question একটি নিটল মুদ্রা ও একটি নিটল ছক্কা একত্রে  নিক্ষেপ করা হলো । একই সাথে হেড ও জোড় পাবার সম্ভাবনা কত?

\begin{oneparchoices}
\choice $ \dfrac{1}{2} $
\choice $ \dfrac{1}{3} $
\choice $ \dfrac{1}{4} $
\choice $ \dfrac{1}{5} $
\end{oneparchoices}

\question $ (2,-1),\,(a+1,a+3),\,(a+2, a) $  বিন্দুতিনটি সমরেখ হলে $ a $ এর মান -

\begin{oneparchoices}
\choice $ 4 $
\choice $ 2 $
\choice $ \dfrac{1}{4} $
\choice $ \dfrac{1}{2}$

\end{oneparchoices}

\question $ f(x) = x^{3}+3 $ এবং $ g(x) = \sqrt[3]{\dfrac{x-2}{3}} $ হলে $ (fog)(3) $এর মান


\begin{oneparchoices}
\choice 1
\choice 2
\choice 3
\choice 4

\end{oneparchoices}

\question  $ \cos\tan^{-1}\cot\sin^{-1}x $ সমান

\begin{oneparchoices}
\choice $ x $
\choice $ \dfrac{\pi}{2}-x $
\choice $ -x $
\choice $ x-\dfrac{\pi}{2} $
\end{oneparchoices}

\end{questions}

\end{document}