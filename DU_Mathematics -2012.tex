\documentclass[addpoints]{exam}
\usepackage[utf8]{inputenc}
\usepackage{amsmath}
\usepackage{mathtools}
\usepackage{relsize}
\usepackage{dirtytalk}
\usepackage{graphicx}
\graphicspath{ {./drift/} }
\DeclarePairedDelimiter{\ceil}{\lceil}{\rceil}
\usepackage{geometry}
\usepackage{draftwatermark}
\SetWatermarkFontSize{2cm}
\SetWatermarkText{$2^{77,232,917}-1$}
\usepackage[banglamainfont=Kalpurush, 
            banglattfont=Siyam Rupali
           ]{latexbangla}
        
\begin{document}
\begin{LARGE}
\begin{center}
গণিত (Mathematics - 2012)
\end{center}
\end{LARGE}
\begin{questions}

 \question  $ f(x)= 4-(x-3)^2 $ ফাংশনের ডোমেইন এবং রেঞ্জ যথাক্রমে

\begin{oneparchoices}
\choice $ \mathbb{R},\,\mathbb{R} $

\choice $ \mathbb{R},\,f(x)\le 4 $

\choice $ x \ge 4,\, \mathbb{R}  $

\choice $ \mathbb{R},\, x\ge 3  $

\end{oneparchoices}

\question  $ f(x) = \dfrac{x-3}{2x+1} $ এবং $ x \neq - \dfrac{1}{2} $ হলে $ f^{-1}(2) $ এর মান হবে -

\begin{oneparchoices}
\choice $ \dfrac{1}{2}$
\choice $ \dfrac{1}{5}$
\choice $ 2 $
\choice $ 5 $
\end{oneparchoices}


\question একটি ইলেকট্রিক ফিল্ডে ইলেকট্রনের ত্বরণ এবং শক্তি সমানুপাতিক। $ 10^{-2} N $ শক্তির জন্য ত্বরণ $ 10^{10}\, m/s^{2} $ হলে $ 10^{-25} N $ শক্তির জন্য ত্বরণ হবে

\begin{oneparchoices}
\choice $ 10^{5}\, m/s^{2} $
\choice $ 10^{15}\, m/s^{2} $
\choice $ 10^{-5}\, m/s^{2} $
\choice $ 10^{-15}\, m/s^{2} $

\end{oneparchoices}

\question   বাস্তব সংখ্যায় $ \dfrac{1}{|2x-3|}> 5 $ অসমতাটির সমাধান হলো-

\begin{oneparchoices}
\choice $ \Big(\dfrac{7}{5}, \dfrac{8}{5}\Big)$
\choice $ \Big[\dfrac{7}{5}, \dfrac{8}{5}\Big]$
\choice $ \Big(\dfrac{7}{5}, \dfrac{3}{5}\Big)\cup \Big(\dfrac{3}{2}, \dfrac{8}{5}\Big)  $
\choice $ \Big[\dfrac{7}{5}, \dfrac{3}{2}\Big]\cup \Big[\dfrac{3}{2}, \dfrac{8}{5}\Big]  $
\end{oneparchoices}

\question  $ x^{2}+4x+2y=0 $ পরাবৃত্তের শীর্ষবিন্দু হবে- 

\begin{oneparchoices}
\choice $ (2,-2) $
\choice $ (-2,-2) $
\choice $ (-2, 2) $
\choice $ (2, 2) $

\end{oneparchoices}

\question $ \Big(2x^{2}-\dfrac{1}{2x^{3}}\Big)^{10} $ এর বিস্তৃতিতে বর্জিত পদটি কততম এবং এর মান কত?

\begin{oneparchoices}
\choice পঞ্চম এবং 840 
\choice চতুর্থ এবং 1920
\choice  ষষ্ঠ এবং 252 
\choice সপ্তম এবং 30

\end{oneparchoices}

\question  $A = \begin{bmatrix}
1 & i \\
-i & 1
\end{bmatrix},\; B = \begin{bmatrix}
i & -1 \\
-1 & -i
\end{bmatrix} $ এবং $ i = \sqrt{-1} $ হলে AB এর মান হবে- 

\begin{oneparchoices}
\choice  $\begin{bmatrix}
1 & 0 \\
0 & 1
\end{bmatrix} $
\choice   $\begin{bmatrix}
0 & 0 \\
0 & 0
\end{bmatrix} $
\choice   $\begin{bmatrix}
i & 0 \\
0 & i
\end{bmatrix} $
\choice   $\begin{bmatrix}
i & 1 \\
1 & i
\end{bmatrix} $
\end{oneparchoices}

\question নির্ণয় কর: $ \mathlarger{\lim_{x\to 0} \dfrac{e^{x} -1}{x}} $

\begin{oneparchoices}
\choice $ -1 $
\choice $ -1 $
\choice $ 2 $
\choice  $ 3 $

\end{oneparchoices}

\question  স্বরবর্ণগুলোকে সবসময় একত্রে KACHUA শব্দটির বর্ণগুলোকে সাজানো সংখ্যা হবে-

\begin{oneparchoices}
\choice 24
\choice 72
\choice 144
\choice 8

\end{oneparchoices}

\question  \say{a} এর কোন মানের জন্য $ 2\hat{i}+\hat{j}-\hat{k},\, 3\hat{i}-2\hat{j}+4\hat{k} $ এবং $ \hat{i}-3\hat{j}+a\hat{k} $ ভেক্টরত্রয় সমতলীয়?


\begin{oneparchoices}
\choice 5
\choice 4
\choice 3
\choice 2

\end{oneparchoices}

\question  $ x $ অক্ষকে $ (4,0) $ বিন্দুতে স্পর্শ করে এবং কেন্দ্র $ 5x-7y+1=0 $ সরলখোর উপর অবস্থিত এমন বৃত্তের সমীকরণ হবে-

\begin{oneparchoices}
\choice $ x^{2}+y^{2}-8x-6y+9=0 $
\choice $ x^{2}+y^{2}-8x+6y+16=0 $
\choice $ x^{2}+y^{2}-8x+6y+9=0 $
\choice $ x^{2}+y^{2}-8x-6y+16=0 $

\end{oneparchoices}

\question একজন লোকের 3 জোড়া কালো মোজা এবং 2 জোড়া বাদামী মোজা আছে। একদিন অন্ধকারে তাড়াহুড়া করে লোকটি মোজা পড়ল। সে প্রথমে একটি বাদামী মোজা পরার পর পরবর্তী মোজাটিও বাদামী হওয়ার সম্ভাবনা-

\begin{oneparchoices}
\choice $ \dfrac{1}{3} $
\choice $ \dfrac{2}{15} $
\choice $ \dfrac{1}{10} $
\choice $ \dfrac{3}{10} $
\end{oneparchoices}

\question $ 3x^{2}-5x+1=0 $ সমীকরণের মূলদ্বয় $ \alpha, \beta $ হলে $ \dfrac{1}{\alpha}, \dfrac{1}{\beta} $ মূলবিশিষ্ট সমীকরণ হবে-

\begin{oneparchoices}
\choice $ 3x^{2}-5x+1=0 $
\choice $ x^{2}-5x+3=0 $
\choice $ 5x^{2}-3x-1=0 $
\choice $ 3x^{2}+5x-1=0 $
\end{oneparchoices}


\question $ \mathlarger{\int\sqrt{\dfrac{1+x}{1-x}}dx = f(x)+c} $ হলে $ f(x) $ এর মান-

\begin{oneparchoices}
\choice $ \sin^{-1}x + \sqrt{1-x^{2}} $
\choice $ \sin^{-1}x - \sqrt{1-x^{2}} $
\choice $ \cos^{-1}x + \sqrt{1-x^{2}} $
\choice $ \sin^{-1}x + \sqrt{1+x^{2}} $
\end{oneparchoices}

\question যদি $ y = \sqrt{\cos 2x} $ হয়, তবে $ \dfrac{dy}{dx} = $ 

\begin{oneparchoices}
\choice $ -\dfrac{\sin 2x}{\sqrt{\cos 2x}}  $
\choice $ -\dfrac{\cos 2x}{\sqrt{\sin 2x}}  $
\choice $ -\dfrac{\sin x}{\sqrt{\tan x}}  $
\choice $ \dfrac{\tan 2x}{\sqrt{\sin 2x}}  $
\end{oneparchoices}

\question $ \tan \Big(\tan^{-1}\dfrac{1}{2} + \tan^{-1}\dfrac{1}{3} \Big) $ এর মান হবে-

\begin{oneparchoices}
\choice $ \dfrac{5}{6} $
\choice $ 1 $
\choice $ \dfrac{\pi}{4} $
\choice $ - \dfrac{5}{6}$

\end{oneparchoices}

\question  $ \sin (ax +b) $ এর $ n $ তম অন্তরক হবে-

\begin{oneparchoices}
\choice  $a^{n} \sin\Big(\dfrac{n\pi}{2} + ax+b\Big)$
\choice  $a^{n} \cos\Big(\dfrac{n\pi}{2} + ax+b\Big)$\\
\choice  $(-1)^{n}a^{n} \sin (ax +b) $
\choice  $(-1)^{n}a^{n} \cos (ax +b) $
\end{oneparchoices}

\question $ \dfrac{(i+1)^{2}}{(i-1)^{4}} $  জটিল সংখ্যাটির আর্গুমেন্ট হবে-

\begin{oneparchoices}
\choice $ \pi $
\choice $ -\pi $
\choice $  \dfrac{\pi}{2} $
\choice  $  -\dfrac{\pi}{2} $

\end{oneparchoices}

\question $ 8+4\sqrt{5}i $ এর বর্গমূল হবে-

\begin{oneparchoices}
\choice $ \pm(3-2i) $
\choice $ \pm(\sqrt{10} + \sqrt{2}i) $
\choice $ \pm(\sqrt{10} - \sqrt{2}i) $
\choice $ \pm(3+2i) $

\end{oneparchoices}

\question  $ y=mx,\, y=m_{1}x $ এবং $ y=b  $ সরলরেখাদ্বয় দ্বারা গঠিত ত্রিভুজের বর্গএককে ক্ষেত্রফল হবে-

\begin{oneparchoices}
\choice $ \dfrac{b^{2}(m_{1}-m)}{2mm_{1}} $
\choice  $ \dfrac{b^{2}(m-m_{1})}{2mm_{1}} $
\choice $ \dfrac{b^{2}|m-m_{1}|}{mm_{1}} $
\choice $ \dfrac{b^{2}|m-m_{1}|}{2mm_{1}} $

\end{oneparchoices}


\question $ \dfrac{1}{2} + \dfrac{1}{3^{2}} + \dfrac{1}{2^{3}} + \dfrac{1}{3^{4}} + \dfrac{1}{2^{5}}+ \dfrac{1}{3^{6}}+\cdots $  ধারাটির সমষ্টি হবে – 

\begin{oneparchoices}
\choice  $ \dfrac{24}{19} $
\choice  $ \dfrac{19}{24} $
\choice  $ \dfrac{5}{24} $
\choice $ \dfrac{5}{19} $
\end{oneparchoices}

\question  একজন কৃষক আয়তার বাগানের তিনদিক বেড়াদিয়ে চর্তুদিক দেয়াল দিয়ে ঘেরাও দিলো। যদি তার কাছে $100m$ বেড়া থাকে তবে ঘেরাও দেয়া স্থানের সর্বোচ্চ আয়তন হবে-
 
\begin{oneparchoices}
\choice $ 2500m^{2} $
\choice $ 1250m^{2} $
\choice $ 750m^{2} $
\choice $ 2000m^{2} $
\end{oneparchoices}

\question $\begin{vmatrix}
a & 1 & b+c \\
b & 1 & c+a\\
c & 1 & a+b
\end{vmatrix} $  এর মান হবে-

\begin{oneparchoices}
\choice 0
\choice $ abc(a+b)(b+c)(c+a) $
\choice $ abs $
\choice  $ (a+b)(b+c)(c+a) $
\end{oneparchoices}

\question  $ x^{2}+3xy+5y^{2}=1 $ যদি হয় তাহলে $ \dfrac{dy}{dx} $ সমান হবে-

\begin{oneparchoices}
\choice $ -\dfrac{3x+2y}{3x+10y} $
\choice $ \dfrac{2x+3y}{3x+10y} $
\choice $ \dfrac{2x-3y}{3x+10y} $
\choice $ \dfrac{2x+3y}{3x-10y} $
\end{oneparchoices}

\question   দশমিক সংখ্যা 2013 এর দ্বিমিক প্রকাশ হবে-

\begin{oneparchoices}
\choice 11111011101
\choice 10111011111
\choice 10101110111
\choice 10101110101

\end{oneparchoices}

\question  $ u $ বেগে আনুভুমিকের সাথে $ \alpha $ কোণে প্রক্ষিপ্ত বস্তুর সর্বোচ্চ উচ্চতা হবে-

\begin{oneparchoices}
\choice $ \dfrac{u^{2}\sin 2\alpha}{2g} $
\choice $ \dfrac{u^{2}\sin^{2}\alpha}{2g} $
\choice $ \dfrac{u^{2}\sin 2\alpha}{g} $
\choice $ \dfrac{u^{2}\sin^{2}\alpha}{g} $

\end{oneparchoices}

\question  3P এবং 2P বলদ্বয়ের লদ্ধি R। প্রথম বলটিদ্বিগুণ করা হলে লদ্ধিও দ্বিগুণ হয়। বলদ্বয়ের অন্তর্গত কোণ হবে-

\begin{oneparchoices}
\choice $ 110^{\circ} $
\choice $ 150^{\circ} $
\choice $ 120^{\circ} $
\choice $ 135^{\circ} $

\end{oneparchoices}

\question $ 3x^{2}+5y^{2} =15 $ উপবৃত্তের উৎকেন্দ্রিকতা হবে -

\begin{oneparchoices}
\choice $ \sqrt{\dfrac{3}{5}} $
\choice $ \sqrt{\dfrac{5}{3}} $
\choice $ \sqrt{\dfrac{2}{5}} $
\choice $ \sqrt{\dfrac{5}{2}} $

\end{oneparchoices}

\question  $ x=y^{2} $  এবং $ y=x-2 $ দ্বারা আবদ্ধ ক্ষেত্রের ক্ষেত্রফল হবে -

\begin{oneparchoices}
\choice $ 1\dfrac{1}{3} $
\choice $ 3\dfrac{1}{6} $
\choice $ 4\dfrac{1}{2} $
\choice $ 4\dfrac{3}{4} $
\end{oneparchoices}

\question  একটি বস্তুকণা খাড়া উপরের দিকে প্রক্ষেপ করলে নির্দিষ্ট বিন্দু P তে পৌছাতে $ t_{1} $ সময় লাগে। যদি আরত্ত $ t_{2} $  সময় পর বস্তুটি ভুমিতে পতিত হয় তবে কণাটির সর্বোচ্চ উচ্চতা হবে - 

\begin{oneparchoices}
\choice $ \dfrac{1}{2}g(t_{1}+t_{2})^{2} $
\choice $ \dfrac{1}{8}g(t_{1}+t_{2})^{2} $
\choice $ \dfrac{1}{2}g(t_{1}^{2}+t_{2}^{2}) $
\choice $ \dfrac{1}{8}g(t_{1}^{2}+t_{2}^{2}) $
\end{oneparchoices}

\end{questions}

\end{document}