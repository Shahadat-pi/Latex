\documentclass[addpoints]{exam}
\usepackage[utf8]{inputenc}
\usepackage{amsmath}
\usepackage{mathtools}
\usepackage{relsize}
\usepackage{dirtytalk}
\usepackage{graphicx}
\graphicspath{ {./drift/} }
\newcommand\Mycomb[2][^n]{\prescript{#1\mkern-0.5mu}{}C_{#2}}
\DeclarePairedDelimiter{\ceil}{\lceil}{\rceil}
\usepackage{geometry}
\usepackage{draftwatermark}
\SetWatermarkFontSize{2cm}
\SetWatermarkText{DU - Mathematics}
\usepackage[banglamainfont=Kalpurush, 
            banglattfont=Siyam Rupali
           ]{latexbangla}

\lfoot{Sayma Mostafa}
\rhead{\thepage}
\rfoot{প্রমি}

          
\begin{document}
\begin{LARGE}
\begin{center}
গণিত (Mathematics - 2018)
\end{center}
\end{LARGE}
\begin{questions}

 \question  $ f(x) = 1+x^3 $ বক্ররেখাটির সাথে $ x- $ অক্ষের ছেদবিন্দুর সংখ্যা (The number of the intersection by the curve $ f(x) = 1+x^3 $ with the $ x- $axis is)

\begin{oneparchoices}
\choice 0
\choice 1
\choice 2
\choice 3

\end{oneparchoices}

\question  $ y = \dfrac{1+x}{1-x} $ হলে $ \dfrac{dy}{dx} $ এর মান (If $ y = \dfrac{1+x}{1-x} $ then, $ \dfrac{dy}{dx} $ equals)

\begin{oneparchoices}
\choice $ \dfrac{-2}{(x-1)^2} $
\choice $ \dfrac{2}{(1-x^2)} $
\choice $ \dfrac{2}{(1-x)^2} $
\choice $ \dfrac{2x}{(1-x)^2} $

\end{oneparchoices} 

\question  $ z= \dfrac{-4+3i}{i} $ এর কাল্পনিক অংশ (The imaginary part of $ z= \dfrac{-4+3i}{i} $ is )

\begin{oneparchoices}
\choice 3
\choice 4 
\choice $ -4 $
\choice  $ -3 $
\end{oneparchoices}

\question  $\Mycomb{1} + \Mycomb{2} +\Mycomb{3} + \cdots + \Mycomb{n} = ?$ 

\begin{oneparchoices}
\choice $ 2^{n} + 1 $
\choice $ 2^{n} $
\choice $ 2^{n-1} $
\choice  $ 2^{n} - 1 $
\end{oneparchoices}

\question  দুটি সমান মানের P বল এর সর্বনিন্ম লদ্ধি কত ?(What is the minimum resultant of two equal forces of magnitude P ?) 

\begin{oneparchoices}
\choice 2P
\choice 0
\choice P
\choice $ \dfrac{P}{2} $

\end{oneparchoices}


\question  একটি চলন্ত ট্রেনকে ব্রেক করে 10s এ থামিয়ে দেয়া হলো । ট্রেনটির গড় মন্দন $ 70m/sec^{2} $ হলে এর গতিবেগ কত ছিল? (A running train is stopped in seconds with a break. What was the velocity of train if its retardation was $ 70m/sec^{2} $ ?)

\begin{oneparchoices}
\choice 1000 m/sec
\choice  800 m/sec
\choice  700 m/sec
\choice  500 m/sec

\end{oneparchoices}

\question  $ 3x^{2} + 3y^{2} -5x-6y +4 =0 $ বৃত্তটির কেন্দ্র - (The center of the circle is  $ 3x^{2} + 3y^{2} -5x-6y +4 =0 $) 

\begin{oneparchoices}
\choice $ (1, \dfrac{2}{3})$
\choice $ (\dfrac{5}{6}, 1) $
\choice $ (\dfrac{5}{3}, 1) $
\choice $ (\dfrac{2}{3}, -1) $

\end{oneparchoices}

\question  $ y=ks $ সরলরেখাটি $ y = x^{2}+4 $ বক্ররেখার স্পর্শক হলে $ k $ এর একটি মান- (The straight line $ y=ks $ will be a tangent to the curve $ y = x^{2}+4 $ if one of the values of $ k $ is ) 

\begin{oneparchoices}
\choice 1
\choice $ 2\sqrt{2} $
\choice $ 3 $
\choice $ 4 $

\end{oneparchoices}

\question  $ y=2 $ এবং $ y=|x| $ রেখা দুটি দ্বারা আবদ্ধ ক্ষেত্রের ক্ষেত্রফল (The area bounded by the line $ y=2 $ and $ y=|x| $ is )

\begin{oneparchoices}
\choice 2 sq. units
\choice 4 sq. units
\choice 6 sq. units
\choice 8 sq. units

\end{oneparchoices}

\question  \say{Permutation}  শব্দটির বর্ণগুলোর মধ্যে স্বরবর্ণের অবস্থান পরিবর্তন না করে কতরকমে পুনরায় সাজানো যাবে? (How may ways the word \say{Permutation} can be rearranged without changing the position of the vowels ?)

\begin{oneparchoices}
\choice 360
\choice 460
\choice 459
\choice 359
\end{oneparchoices}


\question   120 জন ছাত্রের মধ্যে 75 জন ক্রিকেট খেলে এবং 65 জন ফুটবল খেলে। কতজন উভয় খেলাই খেলে? (Out of 120 students 75 play cricket and 65 play football. How many of them play both of the games ?)

\begin{oneparchoices}
\choice 10
\choice 20
\choice 30
\choice 45

\end{oneparchoices}

\question   $ \mathlarger{|3-\dfrac{1}{x}|<\dfrac{1}{2}} $ অসমতাটির সমাধান সেট (The solution set of the inequality $ \mathlarger{|3-\dfrac{1}{x}|<\dfrac{1}{2}} $ is )

\begin{oneparchoices}
\choice $ \dfrac{2}{7} < x < \dfrac{2}{5}$
\choice $ -\dfrac{4}{7} < x < -\dfrac{4}{5}$
\choice $ \dfrac{1}{8} < x < \dfrac{1}{7}$
\choice $ \dfrac{1}{5} > x > \dfrac{1}{7}$

\end{oneparchoices}

\question  $ \tan^{-1}\dfrac{2}{3} + \cos^{-1}\dfrac{2}{\sqrt{13}} = ?$

\begin{oneparchoices}
\choice $ \tan^{-1}\dfrac{5}{9} $ 
\choice $ \tan^{-1}\dfrac{3}{7} $ 
\choice $ \dfrac{\pi}{2} $ 
\choice $ \dfrac{\pi}{4} $ 

\end{oneparchoices}

\question  $ (x^{2} + \dfrac{2}{x})^{6} $ এর বিস্তৃতিতে $ x$ মুক্ত পদ- (The $ x $ free term in the expansion of $ (x^{2} + \dfrac{2}{x})^{6} $ is )

\begin{oneparchoices}
\choice 204
\choice 240 
\choice 402 
\choice 420 

\end{oneparchoices}

\question  $ \mathlarger{\int\sqrt{e^{x}}dx} =? $

\begin{oneparchoices}
\choice $ \dfrac{2}{3}(e^x)^{\frac{3}{2}} + c$ 
\choice $ \dfrac{1}{2}\sqrt{e^x} + c $
\choice $ 2e^{\frac{x}{2}} + c $
\choice $ e^{\frac{x}{2}} + c $

\end{oneparchoices}

\question  $ \mathlarger{\int\dfrac{\tan(\sin^{-1}x)}{\sqrt{1-x^{2}}} dx} =? $

\begin{oneparchoices}
\choice $ \sec^{2}(\sin^{-1}x) + c$ 
\choice $ \sec(\sin^{-1}x) + c $ 
\choice $ ln|\sec(\sin^{-1}x)| + c $
\choice $ ln|\tan(\sin^{-1}x)| + c $

\end{oneparchoices}

\question    4 থেকে 15 পর্যন্ত সংখ্যা হতে যেকোনো একটিকে দৈবচয়নের মাধ্যমে নিলে সেই সংখ্যাটি মৌলিক অথবা এর গুণিতক হওয়ার সম্ভাবনা কত? (What is the probability that a number chosen randomly from 4 to 15 is prime number or multiple of 3?)

\begin{oneparchoices}
\choice $ \dfrac{6}{7}$ 
\choice $ \dfrac{5}{12}$ 
\choice $ \dfrac{2}{3} $ 
\choice $ \dfrac{7}{12} $ 
\end{oneparchoices}

\question   $ f(x) = \dfrac{-1}{|1-x|} $  ফাংশনের রেঞ্জ (The range of the function $ f(x) = \dfrac{-1}{|1-x|} $ is)
 
\begin{oneparchoices}
\choice $ \mathbb{R}-{1} $
\choice $ \mathbb{R}-{0} $
\choice $ \mathbb{R}-{0,1} $
\choice  $ (-\infty, 0) $

\end{oneparchoices}

\question $ y=b $ এবং $ \sqrt{3}x-y+1=0 $ রেখাদ্বয়ের অর্ন্তভুক্ত সুক্ষকোণের মান (The acute angle between the straight line $ y=b $ and $ \sqrt{3}x-y+1=0 $ is ) 

\begin{oneparchoices}
\choice $ 30^{\circ} $
\choice $ 45^{\circ} $
\choice $ 60^{\circ} $
\choice $ 90^{\circ} $

\end{oneparchoices}


\question  ভেক্টর $ \vec{u} = \hat{i} + \hat{j} $ ও $ \vec{v} = \hat{j} + \hat{k} $ এর অন্তর্ভুক্ত কোণ (The angle between vector $ \vec{u} = \hat{i} + \hat{j} $ and  $ \vec{v} = \hat{j} + \hat{k} $ is)

\begin{oneparchoices}
\choice $ \cos^{-1}(\dfrac{1}{\sqrt{3}}) $
\choice $ \cos^{-1}(\dfrac{1}{3}) $
\choice $ \cos^{-1}(\dfrac{1}{2}) $
\choice $ \cos^{-1}(\dfrac{1}{\sqrt{2}}) $

\end{oneparchoices}

\question  $ x^{2} + y^{2}+2x-4y+4=0 $ বৃত্তের একটি স্পর্শক (One of the tangents to circle $ x^{2} + y^{2}+2x-4y+4=0 $ is ) 

\begin{oneparchoices}
\choice $ x =0 $
\choice $ x=2 $
\choice $ y=0 $
\choice  $ y=4 $

\end{oneparchoices}

\question  $ \cos^{2}(60^{\circ} + A) + \cos^{2}(60^{\circ} -A) $ এর মান (The value of  $ \cos^{2}(60^{\circ} + A) + \cos^{2}(60^{\circ} -A) $ is ) 

\begin{oneparchoices}
\choice $ 1-\dfrac{1}{2}\cos 2A $
\choice $ 1+\sin 2A $
\choice $ 1+3\cos 2A $ 
\choice $ 1-\dfrac{1}{2}\cos 2A $

\end{oneparchoices}

\question  $ 2r\sin^{2}(\dfrac{\theta}{2}) = 1 $ এর কার্তেসীয় সমীকরণ (The Cartesian equation of $ 2r\sin^{2}(\dfrac{\theta}{2}) $ is)

\begin{oneparchoices}
\choice $ y^{2} = 1+2x $
\choice $ y^{2} = 4(1-x) $
\choice $ y^{2} = 4(1+x) $
\choice $ x^{2} = 4(1+y) $

\end{oneparchoices}

\question   $ \cot\theta \cot 3\theta =1 $ সমীকরণের সাধারণ সমাধান  (The general solution of the equation $ \cot\theta \cot 3\theta =1 $ is)

\begin{oneparchoices}
\choice $ \dfrac{(2n+1)\pi}{4} $
\choice $ \dfrac{(2n+1)\pi}{8} $
\choice $ \dfrac{n\pi}{4} $
\choice  $ \dfrac{(2n-1)\pi}{2} $

\end{oneparchoices}

\question $ y=x+4 $  এবং $ y=x $  রেখাদ্বয়ের লম্বদুরত্ব (The perpendicular distance between the line $ y=x+4 $ and $ y=x $ is ) 

\begin{oneparchoices}
\choice 4 একক
\choice $2\sqrt{2}$ একক
\choice 2 একক
\choice  $ 4\sqrt{2} $ একক
\end{oneparchoices}

\question $ x = \dfrac{1}{2}(-1+\sqrt{-3}) $  এবং $ y = \dfrac{1}{2}(-1-\sqrt{-3})  $ হলে $ x^{2} + xy + y^{2} $ এর মান (If $ x = \dfrac{1}{2}(-1+\sqrt{-3}) $ and $ y = \dfrac{1}{2}(-1-\sqrt{-3})  $ then the value of  $ x^{2} + xy + y^{2} $ is )

\begin{oneparchoices}
\choice $ 0 $
\choice $ 2 $
\choice $ 1 + \sqrt{3} $
\choice  $ 1 $

\end{oneparchoices}

\question  $ y^{2}-4y-x^{2}+6x =12 $  সমীকরণটি কোন ধরনের কণিক? (The type of the conic $ y^{2}-4y-x^{2}+6x =12 $ is)

\begin{oneparchoices}
\choice বৃত্ত (Circle)
\choice উপবৃত্ত (Ellipse)
\choice পরাবৃত্ত (Parabola)
\choice  অধিবৃত্ত (Hyperbola)

\end{oneparchoices}

\question  $ 2x^{2}-8y^{2} = 2 $ অধিবৃত্তের উৎকেন্দ্রিকতার মান (The value of the eccentricity of the hyperbola $ 2x^{2}-8y^{2} = 2 $ is )

\begin{oneparchoices}
\choice $ \dfrac{3}{2\sqrt{2}} $
\choice $ \dfrac{3}{2} $
\choice $ \dfrac{\sqrt{5}}{2} $
\choice  $ \sqrt{\dfrac{5}{2}} $

\end{oneparchoices}

\question  $ \mathlarger{\lim_{x \to 0} \dfrac{\sin x}{\tan^{-1}(3x)}} $

\begin{oneparchoices}
\choice $ 0 $
\choice $ \dfrac{1}{3} $
\choice $ 1 $
\choice  $ 3 $

\end{oneparchoices}

\question   $ x^{2}-7x+2=0 $ সমীকরণের মূলদ্বয় হতে 2 কমমূল বিশিষ্ট সমীকরণটি (The equation whose roots are less by 2 than those of the equation $ x^{2}-7x+2=0 $ is)

\begin{oneparchoices}
\choice $ x^{2}-4x+6=0 $
\choice $ x^{2}-3x-8=0 $
\choice $ x^{2}-11x+8=0 $
\choice $ x^{2}-3x+8=0 $

\end{oneparchoices}

\end{questions}


\begin{LARGE}
\begin{center}
গণিত (Mathematics - 2017 )
\end{center}
\end{LARGE}
\begin{questions}

\question $ A=\begin{vmatrix}
a & 2 & 5 \\
-2 & b & -3 \\
-5 & 3 & c
\end{vmatrix} $  একটি বক্র প্রতিসম ম্যাট্রিক্স হলে $ a,\,b $ ও $ c $ এর মানগুলো (If $ A=\begin{vmatrix}
a & 2 & 5 \\
-2 & b & -3 \\
-5 & 3 & c
\end{vmatrix} $ is a skew symmetric matrix then the values of $ a,\,b $ and $ c $ are )

\begin{oneparchoices}
\choice  $-2,\, -5,\, 3$
\choice  $0,\, 0,\, 0$
\choice  $1,\, 1,\, 1$
\choice  $2,\, 5,\, 3$

\end{oneparchoices}

\question  $ k $ এর কোন মানের জন্য $ \begin{vmatrix}
1 & 1 & 1 \\
1 & k & k^2 \\
1 & k^2 & k^3
\end{vmatrix}  $ নির্ণায়কটির মান শূণ্য হবে না? (For which value of $ k $, the value of the  determinant $ \begin{vmatrix}
1 & 1 & 1 \\
1 & k & k^2 \\
1 & k^2 & k^3
\end{vmatrix}  $ is not zero ? )

\begin{oneparchoices}
\choice $ k=1 $
\choice $ k=-1 $
\choice $ k=3 $
\choice  $ k=0 $

\end{oneparchoices}
    

\question   অসমতা $ |5-2x|\ge 4 $ এর সমাধান সেট (The solution set of the  inequality $ |5-2x|\ge 4 $ is )

\begin{oneparchoices}
\choice $ \Big[ \dfrac{1}{2}, \dfrac{9}{2} \Big] $
\choice $ \Big(-\infty, \dfrac{1}{2} \Big] \cup \Big[\dfrac{9}{2}, \infty \Big)$ 
\choice $ \Big[ -\infty, \dfrac{1}{2} \Big] $
\choice  $ \Big[ \dfrac{1}{2}, \dfrac{9}{2} \Big]\cup \Big[ \dfrac{27}{2}, \infty \Big) $

\end{oneparchoices}

\question   যদি $z_{1}=1-i, \; z_{2} = \sqrt{3} +i $ হয় তবে $ \dfrac{z_{2}}{z_{1}} $ এর নতি ( If $z_{1}=1-i, \; z_{2} = \sqrt{3} +i $, then the argument of $ \dfrac{z_{2}}{z_{1}} $ is )

\begin{oneparchoices}
\choice $ \dfrac{5\pi}{12} $
\choice $ \dfrac{\pi}{6} $
\choice $ -\dfrac{\pi}{4} $
\choice $ -\dfrac{5\pi}{12} $

\end{oneparchoices}


\question  কোন দ্বিঘাত সমীকরণের একটি মূল $ \dfrac{1}{1+i} $ হলে সমীকরণটি হবে (If one root of a quadratic equation is $\dfrac{1}{1+i}$, then the equation is )

\begin{oneparchoices}
\choice $ x^2 -x +1=0 $
\choice $ 2x^2 -2x +1=0 $
\choice $ x^2 +x +1=0 $
\choice $ 2x^2 +2x +1=0 $

\end{oneparchoices}

\question  RAJSHAHI শব্দটির অক্ষরগুলির একত্রে বিন্যাস সংখ্যা BARISHAL শব্দটির অক্ষরগুলির বিন্যাস সংখ্যার $ k $ গুণ হলে $ k $ এর মান (If the permutation by taking all the letters of RAJSHAHI is $ k $ times of the permutation by taking all the letters of BARISHAL, then the value of $ k $ is)

\begin{oneparchoices}
\choice $ 2$
\choice $ 3$
\choice $ 4$
\choice $ 5$

\end{oneparchoices}

\question  $ \Big(x-\dfrac{1}{x}\Big)^{16} $  এর  বিস্তৃতিতে মধ্যপদটি হবে (The middle term of the expansion of $ \Big(x-\dfrac{1}{x}\Big)^{16} $ is )

\begin{oneparchoices}
\choice $ 12780 $
\choice $ 42708 $
\choice $ 12870 $
\choice $ 12807 $

\end{oneparchoices}

\question $ 1+\dfrac{1}{3} + \Big( \dfrac{1}{3}\Big)^2+ \Big( \dfrac{1}{3}\Big)^3 +\cdots $  অসীম পর্যন্ত এর মান (The value of $ 1+\dfrac{1}{3} + \Big( \dfrac{1}{3}\Big)^2+ \Big( \dfrac{1}{3}\Big)^3 +\cdots $ up to infinity is )

\begin{oneparchoices}
\choice $ \dfrac{2}{3} $
\choice $ \dfrac{3}{2} $
\choice $ \dfrac{1}{3} $
\choice $ \dfrac{1}{2} $

\end{oneparchoices}

\question  $ \vec{a} = 4\hat{i}-3\hat{j} +2\hat{k} $ ও $ \vec{b} = 2\hat{i}-3\hat{j} +4\hat{k} $ ভেক্টর দুইটি যে সামান্তরিকের সন্নিহিত বাহু তার ক্ষেত্রফল হবে (The area of the parallelogram having $ \vec{a} = 4\hat{i}-3\hat{j} +2\hat{k} $ and  $ \vec{b} = 2\hat{i}-3\hat{j} +4\hat{k} $ as the adjacent side is)

\begin{oneparchoices}
\choice $3\sqrt{3}$ sqrt units
\choice $6\sqrt{3}$ sqrt units
\choice $ 6\sqrt{6} $ sqrt units
\choice $ 3\sqrt{6} $ sqrt units

\end{oneparchoices}


\question   ভেক্টর  $ \vec{u} = 2\hat{i}+\hat{j} -3\hat{k} $ ও  $ \vec{v} = 3\hat{i}-2\hat{j} -\hat{k} $ এর অন্তর্ভুক্ত কোণ (The angle between vector  $ \vec{u} = 2\hat{i}+\hat{j} -3\hat{k} $ and $ \vec{v} = 3\hat{i}-2\hat{j} -\hat{k} $ is)

\begin{oneparchoices}
\choice $ 60^{\circ} $
\choice $ 45^{\circ} $
\choice $ 30^{\circ} $
\choice $ 120^{\circ} $

\end{oneparchoices}

\question  P(2,5), Q(5,9) এবং S(6,8) বিন্দুত্রয় PQRS রম্বসের শীর্ষবিন্দু হলে R এর স্থানাংক ( P(2,5), Q(5,9) and S(6,8) are three vertices of a rhombus PQRS, then the coordinates of R is )

\begin{oneparchoices}
\choice $ (12,\,9) $
\choice $ (\dfrac{7}{2},\, 7) $
\choice $ (4,\, \dfrac{13}{2}) $
\choice $ (9,12) $

\end{oneparchoices}


\question  মূলবিন্দুগামী একটি বৃত্ত ধনাত্নক $ x- $অক্ষহতে 4 একক এবং ধনাত্নক $ y- $অক্ষহতে 2 একক ছেদক কর্তন করলে, এর সমীকরণ হবে(The equation of the circle which passes through the origin and cuts off intercepts 4 and 2 units from the positive sides of $ x $ and $ y $ axis respectively is)

\begin{oneparchoices}
\choice $ x^2 + y^2 -4x-2y=0 $
\hspace*{1cm}\choice $ x^2 + y^2 +4x+2y=0 $\\
\hspace*{-.3cm}\choice $ x^2 + y^2 +2x+4y=0 $
\hspace*{1cm}\choice $ x^2 + y^2 -2x-4y=0 $

\end{oneparchoices}

\question  $ 25x^2 + 16y^2 = 400 $ এর উৎকেন্দ্রিকতা হবে (The eccentricity of $ 25x^2 + 16y^2 = 400 $ is)

\begin{oneparchoices}
\choice $ \dfrac{3}{5} $
\choice $ \dfrac{3}{4} $
\choice $ \dfrac{4}{5} $
\choice $ \dfrac{2}{3} $

\end{oneparchoices}

\question  $ y- $অক্ষের সমান্তরাল এবং $ 2x-7y+11=0 $ ও $ x+3y=8 $ রেখাদ্বয়ের ছেদবিন্দুদিয়ে অতিক্রমকারী সরলরেখার সমীকরণ(The equation of the straight line parallel to $y-$axis and passing through the point of intersection of the lines $ 2x-7y+11=0 $ and $ x+3y=8 $ is)

\begin{oneparchoices}
\choice $ 13x-23=0 $
\choice $ 3x-7=0 $
\choice $ 7x-3=0 $
\choice $ 23x-13=0 $

\end{oneparchoices}

\question $ A+B = \dfrac{\pi}{2} $  হলে $ \cos^{2}A - \cos^{2}B $ এর মান (If $ A+B = \dfrac{\pi}{2} $, then the value of $ \cos^{2}A - \cos^{2}B $ is)

\begin{oneparchoices}
\choice $ \sin (A-B) $
\choice $ \sin (B-A) $
\choice $ \cos (B-A) $
\choice $ -\cos (A-B) $

\end{oneparchoices}

\question  $ 0\le x\le 90^{\circ} $ হলে $ \sin 3x = \cos x $ সমীকরণের সমাধান হবে (If $ 0\le x\le 90^{\circ} $ then the solution of the equation $ \sin 3x = \cos x $ is)

\begin{oneparchoices}
\choice $ 0^{\circ},\, 45^{\circ} $
\choice $ 0^{\circ},\, 22.5^{\circ} $
\choice $ 45^{\circ},\,45^{\circ} $
\choice $ 22.5^{\circ},\, 45^{\circ}  $

\end{oneparchoices}

\question  $ \sin^{-1}x + \sin^{-1}y = \dfrac{\pi}{2} $ হলে কোনটি সঠিক? (If $ \sin^{-1}x + \sin^{-1}y = \dfrac{\pi}{2} $, then which one is correct?)

\begin{oneparchoices}
\choice $ x^2 + y^2 =1 $
\choice $ x^2 - y^2 =1 $
\choice $ x + y =1 $
\choice $ x-y=1  $

\end{oneparchoices}

\question   $ f(x) = \dfrac{1}{\sqrt{|x|}} $ এর ডোমেন (The domain of  $ f(x) = \dfrac{1}{\sqrt{|x|}} $ is)

\begin{oneparchoices}
\choice $ [0,+\infty) $
\choice $ (0,+\infty) $
\choice $ (-\infty, +\infty) $
\choice $ (-\infty, 0)\cup (0, +\infty)  $

\end{oneparchoices}

\question   $ \lim\limits_{x\to \infty} \dfrac{2x^2 +3x+5}{3x^2 +5x-6} $ এর মান (The value of $ \lim\limits_{x\to \infty} \dfrac{2x^2 +3x+5}{3x^2 +5x-6} $ is)

\begin{oneparchoices}
\choice $ \dfrac{3}{5} $
\choice $ -\dfrac{5}{6} $
\choice $ \dfrac{2}{3} $
\choice $ -\dfrac{2}{3}  $

\end{oneparchoices}

\question   $ f(x) = \sqrt{x-1} $ হলে $ f^{-1}(2) $ এর মান (If $ f(x) = \sqrt{x-1} $, then the value of $ f^{-1}(2) $ is)

\begin{oneparchoices}
\choice $ -1 $
\choice $ 3 $
\choice $ 1 $
\choice $ 5  $

\end{oneparchoices}

\question    $ (4,3) $ বিন্দুতে $ 3x^2 -4y^2 =12 $ অধিবৃত্তের স্পর্শকের ঢালের মান (The value of the slope of the tangent at the point $ (4,3) $ of the hyperbola $ 3x^2 -4y^2 =12 $ is)

\begin{oneparchoices}
\choice $ -1 $
\choice $ 1 $
\choice $ \dfrac{3}{4} $
\choice $ \dfrac{4}{3} $

\end{oneparchoices}

\question  $ x^{4}-4x^{3}+4x^{2} + 5  $  এর লঘিষ্ঠ মান (The minimum value of $ x^{4}-4x^{3}+4x^{2} + 5  $ is )

\begin{oneparchoices}
\choice $ 4 $
\choice $ 5 $
\choice $ 6 $
\choice $ 8 $

\end{oneparchoices}

\question  যদি $ \mathlarger{\int\limits_{0}^{6}}f(t)dt =8 $ হয় তবে $ \mathlarger{\int\limits_{0}^{3}}f(2x)dx $ এর মান (If $ \mathlarger{\int\limits_{0}^{6}}f(t)dt =8 $, then the value of $ \mathlarger{\int\limits_{0}^{3}}f(2x)dx $ is)

\begin{oneparchoices}
\choice $ 0 $
\choice $ 6 $
\choice $ 10 $
\choice $ 4 $

\end{oneparchoices}

\question  $ \mathlarger{\int \dfrac{dx}{x\sqrt{x^2 -1}} = f(x)+c} $  হলে $ f(x) $ সমান (If $ \mathlarger{\int \dfrac{dx}{x\sqrt{x^2 -1}} = f(x)+c} $, then $f(x)$ is equal ) 

\begin{oneparchoices}
\choice $ \sin x $
\choice $ \sin^{-1}x $
\choice $ \cos x $
\choice $ \sec^{-1} x $

\end{oneparchoices}


\question $ \mathlarger{\int\limits_{-1}^{1}|x|dx} $ এর মান (The value of $ \mathlarger{\int\limits_{-1}^{1}|x|dx} $ ) 

\begin{oneparchoices}
\choice $ 2 $
\choice $ -1 $
\choice $ 1 $
\choice $ 0 $

\end{oneparchoices}

\question $ y=x^2,\,x=1,\,x=3 $ এবং $ x- $অক্ষ দ্বারা সীমাবদ্ধ ক্ষেত্রের ক্ষেত্রফল (The area bounded by $ y=x^2,\,x=1,\,x=3 $ and $ x- $axis is)

\begin{oneparchoices}
\choice $ \dfrac{26}{3} $ sq units
\choice $ \dfrac{80}{3} $ sq units
\choice $ \dfrac{8}{3} $ sq units
\choice $ \dfrac{35}{3} $ sq units

\end{oneparchoices}

\question   $ x+2y\le 10,\, x+y\le 6,\, x\le 4,\, x,y\ge 0 $ শর্তাধীনে $ z= 2x+3y $ এর সর্বোচ্চ মান (The maximum value of $ z= 2x+3y $ subjected to $ x+2y\le 10,\, x+y\le 6,\, x\le 4,\, x,y\ge 0 $ is )

\begin{oneparchoices}
\choice 14
\choice 15
\choice 16
\choice 18

\end{oneparchoices}

\question   2N এবং 5N  মানের দুটি বল একই দিকে ক্রিয়ারত। উহাদের সর্বাধিক  লদ্ধি হবে (Two forces of magnitude 2N and 5N act on a same line in the same direction, then the maximum magnitude of the resultant is )

\begin{oneparchoices}
\choice 7N
\choice 3N
\choice $\sqrt{29}$N
\choice 5N

\end{oneparchoices}

\question   যদি $ u $ বেগে অনুভূমিকের সাথে $ \alpha $ কোণে প্রক্ষিপ্ত বস্তু $ T $ সময়ে তার গতিপথের সর্বোচ্চ উচ্চতা $ H $ এ পৌঁছায় তবে $ \dfrac{H}{T^2} $ হবে (If the greatest height $ H $ is attained in time $ T $ by a body projected with  a velocity $ u $ at an angle $ \alpha $, then $ \dfrac{H}{T^2} is $  )

\begin{oneparchoices}
\choice $ \dfrac{2}{g} $
\choice $ \dfrac{g}{2} $
\choice $ g $
\choice $ \dfrac{1}{g} $

\end{oneparchoices}

\question  1  হতে 99 পর্যন্ত সংখ্যাগুলো থেকে দৈবচয়ন পদ্ধতিতে একটি সংখ্যা নেয়া হলে সেটি বর্গ হওয়ার সম্ভাবনা হবে (If a number randomly from 1 to 99, then the probability that it would be a square is) 

\begin{oneparchoices}
\choice $ \dfrac{1}{9} $
\choice $ \dfrac{2}{9} $
\choice $ \dfrac{1}{11} $
\choice $ \dfrac{2}{11} $

\end{oneparchoices}

\end{questions}
\begin{LARGE}
\begin{center}
গণিত (Mathematics - 2016)
\end{center}
\end{LARGE}
\begin{questions}

 \question $ |x^2 +1|<10 $  এর সমাধান (Solution of $ |x^2 +1|<10 $ is)

\begin{oneparchoices}
\choice $ -3<x<3 $
\choice $ -3\le x < 3 $
\choice $ -3<x \le 3 $
\choice $ -3 \le x \le 3 $

\end{oneparchoices}

\question  $ 5x -7y -15=0 $ সরলরেখার উপর লম্ব এবং $ (2,-3) $ বিন্দুগামী সরলরেখার সমীকরণ হবে (The equation of the straight line passing through the point $ (2,-3) $ and perpendicular to the straight line $ 5x-7y-15=0 $ is)

\begin{oneparchoices}
\choice $ 7x-5y-29=0 $
\choice $ 5x-7y-31=0 $
\choice $ 5x+7y+11=0 $
\choice  $ 7x+5y+1=0 $

\end{oneparchoices}

\question  $ y- $ অক্ষকে $ (0,4) $ বিন্দুতে স্পর্শ করে এবং কেন্দ্র $ 5x-7y-2=0 $ রেখার উপর অবস্থিত বৃত্তের সমীকরণ হবে (The equation of the circle touching $ y- $ axis at $ (0,4) $ and centre lying on the line $ 5x-7y-2=0 $ is)

\begin{oneparchoices}
\choice $ x^2 +y^2 + 12x-8y+16=0 $
\choice $ x^2 +y^2 -8x-6y+8=0 $ \\
\hspace*{-.45cm} \choice $ x^2 +y^2 - 12x-8y+16=0 $
\choice  $ x^2 +y^2 + 8x +6y-40=0 $

\end{oneparchoices}

\question  $ 2x +3y -4=0 $ এবং $ x\cos\alpha +y\sin\alpha= p$ একই সরলরেখা নির্দেশ করলে $ p $ এর মান ($ 2x +3y -4=0 $ and $ x\cos\alpha +y\sin\alpha= p$ represents the same line then the value of $ p $ is )

\begin{oneparchoices}
\choice $ \dfrac{1}{\sqrt{13}} $
\choice $ \dfrac{2}{\sqrt{13}} $
\choice $ \dfrac{3}{\sqrt{13}} $
\choice $ \dfrac{4}{\sqrt{13}} $

\end{oneparchoices}


\question  $ x=a $ এবং $ \sqrt{3}x-y+1=0 $ রেখাদ্বয়ের মধ্যবর্তী সূক্ষ্মকোণ (The acute angle between the straight lines $ x=a $ and $ \sqrt{3}x-y+1=0 $ )

\begin{oneparchoices}
\choice $ 30^{\circ} $
\choice $ 45^{\circ} $
\choice $ 60^{\circ} $
\choice $ 75^{\circ} $

\end{oneparchoices}

\question  সমাধান কর(Solve:) $ \sec^{2}\theta + \tan^{2}\theta = \dfrac{5}{3}, 0<\theta<\pi $

\begin{oneparchoices}
\choice $ -\dfrac{\pi}{6}, -\dfrac{5\pi}{6} $
\choice $ -\dfrac{\pi}{6}, \dfrac{5\pi}{6} $
\choice $ \dfrac{\pi}{6}, -\dfrac{5\pi}{6} $
\choice $ \dfrac{\pi}{6}, \dfrac{5\pi}{6} $

\end{oneparchoices}

\question   $ \sin (A-30^{\circ}) + \sin (150^{\circ} +A) $ এর মান (The value of $ \sin (A-30^{\circ}) + \sin (150^{\circ} +A) $ is)

\begin{oneparchoices}
\choice $ -\dfrac{1}{2}\cos A$
\choice $ 0 $
\choice $ \cos A $
\choice $ \sin A $

\end{oneparchoices}

\question   $ 4x^{2} + y^{2} = 2 $ উপবৃত্তটির বৃহৎ ও ক্ষুদ্র অক্ষের দৈর্ঘ্য যথাক্রমে (The length of major and minor axes of the ellipse $ 4x^{2} + y^{2} = 2 $ are respectively)

\begin{oneparchoices}
\choice 4 and 2
\choice 2 and 4
\choice $ \sqrt{2}$ and $ 2\sqrt{2} $
\choice $ 2\sqrt{2}$ and $ \sqrt{2} $

\end{oneparchoices}

\question   $ \mathlarger{\lim_{x \to 0}} \dfrac{e^{\cos x}}{\cos x} $ এর মান (The value of $ \mathlarger{\lim_{x \to 0}} \dfrac{e^{\cos x}}{\cos x} $ is)

\begin{oneparchoices}
\choice $e$
\choice $1 $
\choice $ \dfrac{1}{e} $
\choice $ 0 $

\end{oneparchoices}


\question   $ \mathlarger{\int\limits_{1}^{4}} f(x)dx=5$ হলে $ \mathlarger{\int\limits_{1}^{4}} f(3x+1)dx$ এর মান (If $ \mathlarger{\int\limits_{1}^{4}} f(x)dx=5$ then the value of $ \mathlarger{\int\limits_{1}^{4}} f(3x+1)dx$ is )

\begin{oneparchoices}
\choice $\dfrac{5}{4}$
\choice $ \dfrac{4}{3} $
\choice $ \dfrac{5}{3} $
\choice $ 5 $

\end{oneparchoices}

\question   $ y=x,\, y=0 $ রেখাদ্বয় এবং $ x^2 + y^2 =16 $ বৃত্তদ্বারা প্রথম চর্তুভাগে আবদ্ধ ক্ষেত্রের ক্ষেত্রফল (The area bounded in the first quadrant by the straight lines $ y=x,\, y=0 $ and the circle $ x^2 + y^2 =16 $ is)

\begin{oneparchoices}
\choice $ 2\pi$ sq. units
\choice $ 3\pi $ sq.units
\choice $ 4\pi $ sq.units
\choice $ 5\pi $ sq.units

\end{oneparchoices}

\question  $ f(x) = \sin x $ এবং $ g(x) = x^2 $ হলে $ (f\circ g)(\dfrac{\pi}{2}) $ এর মান হবে (If $ f(x) = \sin x $ and $ g(x) = x^2 $, then the value of $ (f\circ g)(\dfrac{\pi}{2}) $ is)

\begin{oneparchoices}
\choice $ \dfrac{1}{\sqrt{2}}$ 
\choice $ \dfrac{1}{2} $ 
\choice $ \dfrac{\sqrt{3}}{2} $ 
\choice $ \sqrt{2} $ 

\end{oneparchoices}

\question    যদি $ y=\sin^{-1}(\sin x) $ হয়,  তবে $ \dfrac{dy}{dx} $ হবে (If $ y=\sin^{-1}(\sin x) $, then $ \dfrac{dy}{dx} $ is )

\begin{oneparchoices}
\choice $ \sin x$ 
\choice $ \cos x $ 
\choice $ x $ 
\choice $ 1 $ 

\end{oneparchoices}

\question    $ x $ এর কোন মানের জন্য $ y= x+ \dfrac{1}{x} $ বক্ররেখাটির ঢাল শূণ্য হবে? (For what values of $ x $, the slope of the curve $ y= x+ \dfrac{1}{x} $ is zero?)

\begin{oneparchoices}
\choice $ x=\pm 2 $
\choice $ 1 $
\choice $ \pm 1 $
\choice  $ \pm \dfrac{3}{2} $

\end{oneparchoices}

\question $ y^2 +4x+2y-8=0 $  পরাবৃত্তের শীর্ষবিন্দু হবে (The vertex of the parabola $ y^2 +4x+2y-8=0 $ is)

\begin{oneparchoices}
\choice $ \Big( \dfrac{9}{4}, -1\Big) $
\choice $ \Big(- \dfrac{9}{4}, 1\Big) $
\choice $ (0,2) $
\choice  $ (2,0) $

\end{oneparchoices}


\question  $ \vec{P} = 5\hat{i}-3\hat{j}+2\hat{k} $  ভেক্টরের উপর $ \vec{Q} = 2\hat{i}+\hat{j}-2\hat{k} $ ভেক্টরের অভিক্ষেপ (The projection of $ \vec{Q} = 2\hat{i}+\hat{j}-2\hat{k} $ on $ \vec{P} = 5\hat{i}-3\hat{j}+2\hat{k} $ is)

\begin{oneparchoices}
\choice $ \dfrac{5}{\sqrt{38}} $
\choice $ \dfrac{3}{\sqrt{38}} $
\choice $ \dfrac{2}{\sqrt{38}} $
\choice  $ \dfrac{1}{\sqrt{38}} $

\end{oneparchoices}


\question  32 ft/s  আদিবেগ এবং ভূমি সাথে $ 30^{\circ} $ কোণে একটি বস্তু নিক্ষেপ করা হলো। ইহার ভ্রমণকাল (A particle is projected with an initial velocity 32 ft/s at an angle $ 30^{\circ} $ with the horizon. Tis time of flight is)

\begin{oneparchoices}
\choice 0.5 s
\choice 1 s
\choice 1.5 s
\choice  2 s

\end{oneparchoices}

\question   একটি সমবাহু ত্রিভুজের বাহুত্রয়ের সমান্তরালে একইক্রমে সমবিন্দুতে কার্যরত 6, 10, 14 একক মানের তিনটি বেগের লদ্ধির মান হবে (The magnitude of the resultant of the three velocities of 6, 10, 14 units acting at a point along three sides of an equilateral triangle in the same sense is)

\begin{oneparchoices}
\choice $ 4\sqrt{3} $ units
\choice $ 7\sqrt{3} $ units
\choice $ 10\sqrt{3} $ units
\choice  $ 15\sqrt{3} $ units

\end{oneparchoices}

\question   $ z=x+iy $ হলে $ |z-5| + |z+5| = 16 $ নির্দেশ করে (If $ z=x+iy $, then $ |z-5| + |z+5| = 16 $ represents)

\begin{oneparchoices}
\choice Circle
\choice Parabola
\choice Hyperbola
\choice Ellipse

\end{oneparchoices}

\question  $ \dfrac{1}{a+i} = \dfrac{i}{a-i} $  হলে $ a $ এর মান (If $ \dfrac{1}{a+i} = \dfrac{i}{a-i} $, then the value of $ a $ is)

\begin{oneparchoices}
\choice $ 1 $
\choice $ \dfrac{i}{2} $
\choice $ -1 $
\choice  $ -\dfrac{i}{2} $

\end{oneparchoices}

\question     1, 2, 0  দ্বারা গঠিত তিন অঙ্কবিশিষ্ট সংখ্যাগুলির মধ্যে কয়টি সংখ্যা 2 দ্বারা বিভাজ্য? (How many three digit numbers formed by 1, 2, 0 are divisible by 2?)

\begin{oneparchoices}
\choice 6
\choice 18
\choice 4
\choice  12

\end{oneparchoices}

\question   $ 3x^3 -1=0 $  এর মূলগুলি $ \alpha, \beta, \gamma $ হলে $ \alpha^{3} + \beta^{3} + \gamma^{3} $ এর মান (If $ \alpha, \beta, \gamma $ are the roots of $ 3x^3 -1=0 $, then the value of $ \alpha^{3} + \beta^{3} + \gamma^{3} $ is)

\begin{oneparchoices}
\choice $ -1 $
\choice $ 0 $
\choice $ \dfrac{1}{3} $
\choice  $ 1 $

\end{oneparchoices}

\question   $ \Bigg( 2x^2 -\dfrac{1}{2x^3}\Bigg)^{10} $   এর বিস্তৃতিতে বর্জিত পদের মান (The term independent of $ x $ in the expansion of $ \Bigg( 2x^2 -\dfrac{1}{2x^3}\Bigg)^{10} $ is )

\begin{oneparchoices}
\choice 540
\choice 640
\choice 740
\choice  840

\end{oneparchoices}

\question    'Mathematics' শব্দটির বর্ণগুলিকে কতরকমে সাজানো যাবে যেখানে প্রথমে ও শেষ স্থানে 'T' থাকবে (In how many ways the letters of the word 'Mathematics' can be arranged when first and last place will be fixed for 'T'?)

\begin{oneparchoices}
\choice 10080
\choice 9680
\choice 50720
\choice  90720

\end{oneparchoices}

\question  $ \dfrac{3x-1}{(x+1)(x^2 +1)} \equiv \dfrac{A}{x+1} + \dfrac{Bx+1}{x^2 +1} $  অভেদে (A,B) এর মান হবে (In the identity (A, B) equals)

\begin{oneparchoices}
\choice $ (-2,-2) $
\choice $ (-2, 2) $
\choice $ (2,-2) $
\choice  $ (2,2) $

\end{oneparchoices}

\question    $ A= \{1,2,3,5,9\} $ এবং $ B= \{1,2,9, 10\} $ হলে $ (A\setminus B)\cup (B\setminus A) $ এর সমান হবে (If $ A= \{1,2,3,5,9\} $ and $ B= \{1,2,9, 10\} $, then $ (A\setminus B)\cup (B\setminus A) $ equals)

\begin{oneparchoices}
\choice $ \{3,5\} $
\choice $ \{1,2,9\} $
\choice $ \{3,5, 10\} $
\choice  $ \{1,2,3,5,9,10\} $

\end{oneparchoices}

\question  $ \dfrac{1}{2}(e^{x}-e^{-x}) $   ধারাটির বিস্তৃতি কি? (What is the expansion of the series  $ \dfrac{1}{2}(e^{x}-e^{-x}) $ ? )

\begin{oneparchoices}
\choice $ x-\dfrac{x^{3}}{3!} +\dfrac{x^{5}}{5!}-\cdots$
\choice $ x+\dfrac{x^{3}}{3!} +\dfrac{x^{5}}{5!}+\cdots$
\choice $ 1+x+\dfrac{x^{3}}{3!} +\dfrac{x^{5}}{5!}+\cdots$
\choice  $ -x-\dfrac{x^{3}}{3!} -\dfrac{x^{5}}{5!}-\cdots$

\end{oneparchoices}

\question   যদি $ 9\theta = \pi $ হয়, তবে $ \cos\theta\cos 2\theta \cos 4\theta $ এর মান (If $ 9\theta = \pi $, then the value of $ \cos\theta\cos 2\theta \cos 4\theta $ is)

\begin{oneparchoices}
\choice $ \dfrac{1}{9} $
\choice $ \dfrac{1}{8} $
\choice $ 8 $
\choice  $ 9 $

\end{oneparchoices}

\question  $ \tan^{-1}\Bigg(x+\dfrac{1}{3}\Bigg) \tan^{-1}\Bigg(x-\dfrac{1}{3}\Bigg) =\tan^{-1}2 $  হলে $ x $ এর মান (If $ \tan^{-1}\Bigg(x+\dfrac{1}{3}\Bigg) \tan^{-1}\Bigg(x-\dfrac{1}{3}\Bigg) =\tan^{-1}2 $, then the value of $ x $ is)

\begin{oneparchoices}
\choice $ -\dfrac{5}{6} $
\choice $ -\dfrac{1}{3} $
\choice $ \dfrac{1}{3} $
\choice  $ \dfrac{2}{3} $

\end{oneparchoices}

\question   একটি বাক্সে 3টি লাল, 3টি সবুজ 2টি নীল বল আছে। দৈবভাবে 3টি বল তোলা হলে 2টি বল সবুজ হওয়ার সম্ভাবনা কত? (There are 3 red balls, 3 green balls and 2 blue balls in a box. If 3 balls are taken randomly, then what is the probability of 2 balls to be green ?)

\begin{oneparchoices}
\choice $ \dfrac{15}{56} $
\choice $ \dfrac{3}{7} $
\choice $ \dfrac{28}{65} $
\choice  $  \dfrac{13}{22} $

\end{oneparchoices}

\end{questions}
\begin{LARGE}
\begin{center}
গণিত (Mathematics - 2015)
\end{center}
\end{LARGE}
\begin{questions}

 \question $ \mathlarger{|}5-\dfrac{2}{3x}\mathlarger{|}<1 $  অসমতাটির সমাধান সেট (The solution set of the inequality $ |5-\dfrac{2}{3x}|<1 $ is)

\begin{oneparchoices}
\choice $ 3<x<4 $
\choice $ \dfrac{1}{9}> x > \dfrac{1}{10} $
\choice $ \dfrac{1}{9}< x < \dfrac{1}{6} $
\choice $ \dfrac{1}{3}< x < \dfrac{1}{2} $

\end{oneparchoices}

\question  $ \sin A + \cos A = \sin B + \cos B $, $ A+B =  $

\begin{oneparchoices}
\choice $ \pi $
\choice $ 2\pi $
\choice $ \dfrac{\pi}{2} $
\choice  $ \dfrac{\pi}{4} $

\end{oneparchoices}

\question  $ \cot^{2}\theta -(\sqrt{3}+1)\cot \theta +\sqrt{3} =0,\; 0<\theta <\dfrac{\pi}{2} $ হলে $ \theta = $\\ (If $ \cot^{2}\theta -(\sqrt{3}+1)\cot \theta +\sqrt{3} =0,\; 0<\theta <\dfrac{\pi}{2} $, then $ \theta = $)

\begin{oneparchoices}
\choice $ \dfrac{\pi}{6}, \dfrac{\pi}{3} $
\choice $ \dfrac{\pi}{4}, \dfrac{5\pi}{3} $
\choice $ \dfrac{\pi}{3}, \dfrac{\pi}{5} $
\choice $ \dfrac{\pi}{6}, \dfrac{\pi}{4} $

\end{oneparchoices}

\question  $ f:\mathbb{R}\to\mathbb{R} $ কে $ f(x) = e^{x-3} $ দ্বারা সংজ্ঞায়িত করা হলে $ f^{-1}(e) $ এর মান (If $ f:\mathbb{R}\to\mathbb{R} $ is defined by $ f(x) = e^{x-3} $, then the value of $ f^{-1}(e) $ is)

\begin{oneparchoices}
\choice 4
\choice 3
\choice 2
\choice 0

\end{oneparchoices}


\question  দ্বিমিক সংখ্যা 1111111 কে দ্বিমিক সংখ্যা 101 সংখ্যা দ্বারা ভাগ করলে ভাগশেষ= (If the binary number 1111111 is divided by the binary number 101 , the remainder is)

\begin{oneparchoices}
\choice 0
\choice 10
\choice 11
\choice 100

\end{oneparchoices}

\question  $ x\ge 0,\,y\ge 0,\, x+y\le 5,\,x+2y\ge 8 $ শর্তানুসারে $ z=2x-y $ এর সর্বনিন্ম মান (The minimum value of $ z=2x-y $, subjected to the constrains $ x\ge 0,\,y\ge 0,\, x+y\le 5,\,x+2y\ge 8 $, is) 

\begin{oneparchoices}
\choice 1
\choice $ -1 $
\choice $ -4 $
\choice $ -5 $

\end{oneparchoices}

\question  $ \Bigg(2x-\dfrac{1}{4x^2}\Bigg)^{12} $ এর বিস্তৃতিতে $ x^3 $ এর সহগ (The coefficient of $x^3$ in the expansion of $ \Bigg(2x-\dfrac{1}{4x^2}\Bigg)^{12} $ is) 
\begin{oneparchoices}
\choice $ 495 $
\choice $ 4223 $
\choice $ -1760 $
\choice $ 1760 $

\end{oneparchoices}

\question  $ \vec{a}  = \hat{i}+2\hat{j}-3\hat{k} $ এবং $ \vec{b} = 3\hat{i}-\hat{j}+2\hat{k} $ হলে নিন্মের কোনটি সত্য? (If $ \vec{a}  = \hat{i}+2\hat{j}-3\hat{k} $ and $ \vec{b} = 3\hat{i}-\hat{j}+2\hat{k} $, then which of the following is true?)

\begin{oneparchoices}
\choice $ a\cdot b =0  $
\choice $ a\wedge b =0 $
\choice $ (a+b)\cdot (a-b) = 0 $
\choice $ (a+b)\wedge (a-b) = 0 $

\end{oneparchoices}

\question   কোন বিন্দুর পোলার স্থানাংক $ 3,150^{\circ} $ হলে ঐ বিন্দুর কার্তেসীয় স্থানাংক (If the polar coordinates of a point is $ 3,150^{\circ} $, then its Cartesian coordinates are) 

\begin{oneparchoices}
\choice $\Big(\dfrac{3\sqrt{3}}{2}, \dfrac{3}{2} \Big)$
\choice $\Big(\dfrac{3\sqrt{3}}{2}, -\dfrac{3}{2} \Big) $
\choice $\Big(-\dfrac{3\sqrt{3}}{2}, \dfrac{3}{2} \Big) $
\choice $\Big(-\dfrac{3\sqrt{3}}{2}, -\dfrac{3}{2} \Big) $

\end{oneparchoices}


\question $ y=kx-1 $ সরলরেখাটি $ y=x^{2}+3 $ বক্ররেখার একটি স্পর্শক হলে $ k $ এর একটি মান (The straight line $ y=kx-1 $ will be a tangent to the curve $ y=x^{2}+3 $ if one of the values of $k$ is)

\begin{oneparchoices}
\choice $ 1 $
\choice $ 2\sqrt{2} $
\choice $ 3 $
\choice $ 4 $

\end{oneparchoices}

\question  $ (-4,3) $ এবং $ (12,-1) $ বিন্দুদ্বয়ের সংযোযক রেখাকে ব্যাস ধরে অংকিত বৃত্তের সমীকরণ (The equation of the circle whose diameter is the line segment joining the points $ (-4,3) $ and $ (12,-1) $, is)

\begin{oneparchoices}
\choice $ x^{2} + y^{2}+8x-2y+51=0 $
\hspace*{.5cm} \choice $ x^{2} + y^{2}-8x-2y+51=0 $\\
\hspace*{-.45cm} \choice $ x^{2} + y^{2}+8x+2y-51=0 $
\hspace*{.5cm} \choice $ x^{2} + y^{2}-8x-2y-51=0 $
\end{oneparchoices}

\question  6 জন বাল 5 জন বালিকার একটি দল থেকে কত উপায়ে 3 জন বালক 2 জন বালিকার একটি দল গঠন করা যেতে পারে? (In how many ways a team of 3 boys and 2 girls can be formed from a group of 6 boys and 5 girls?)

\begin{oneparchoices}
\choice 10
\choice 20 
\choice 50 
\choice 200 

\end{oneparchoices}

\question    এককের একটি কাল্পনিক ঘনমূল $ \omega $ হলে $ (1-\omega)(1-\omega^{2})(1-\omega^{4})(1-\omega^{8}) $ এর মান (If $ \omega $ is a complex cube root of unity, then the value of $ (1-\omega)(1-\omega^{2})(1-\omega^{4})(1-\omega^{8}) $ is) 

\begin{oneparchoices}
\choice 18
\choice 6 
\choice -9 
\choice 9 

\end{oneparchoices}

\question  $ \sec^{2}(\cot^{-1}3) +\text{cosec}^{2}(\tan^{-1}2) = $

\begin{oneparchoices}
\choice $ \dfrac{85}{26} $
\choice $ \dfrac{36}{85} $
\choice $ \dfrac{10}{9} $
\choice  $ \dfrac{9}{10} $

\end{oneparchoices}

\question  $ y=\dfrac{\sin x + \cos x}{\sqrt{1+\sin 2x}} $ হলে $ \dfrac{dy}{dx} = $ (If $ y=\dfrac{\sin x + \cos x}{\sqrt{1+\sin 2x}} $, then  $ \dfrac{dy}{dx} = $)

\begin{oneparchoices}
\choice $ 2\sin 2x $
\choice $ 0 $
\choice $ 1 $
\choice  $ \cos 2x $

\end{oneparchoices}

\question $ \mathlarger{\int\limits_{0}^{10}|x-5|dx=} $  

\begin{oneparchoices}
\choice $ \dfrac{25}{2} $
\choice $ 25 $
\choice $ 50 $
\choice $ 5 $

\end{oneparchoices}

\question  $ \mathlarger{\int\dfrac{e^{x}(1+x)}{\cos^{2}(xe^{x})}dx = f(x) + c;\; f(x) = }$

\begin{oneparchoices}
\choice $ \sin (xe^{x}) $
\choice $ \tan(xe^{x}) $
\choice $ \cot(xe^{x}) $
\choice  $ \sec(xe^{x}) $

\end{oneparchoices}


\question  $ \mathlarger{\int\limits_{0}^{x}f(p)f^{\prime}(p)dp = } $

\begin{oneparchoices}
\choice $ \dfrac{1}{2}f^{2}(x) $
\choice $ \dfrac{1}{2}x^{2} $
\choice $ \dfrac{1}{2}[\{f(x)\}^{2} - \{f(0)\}^{2}] $
\choice  $ f(x)-f(0) $

\end{oneparchoices}

\question  $ y= \dfrac{1}{\sqrt{4-x}} $ ফাংশনের ডোমেইন ও রেঞ্জ (The domain and range of the function $ y= \dfrac{1}{\sqrt{4-x}} $)
\begin{oneparchoices}
\choice $ -\infty< x\le 4;\;  0\le y<\infty $
\hspace*{.7cm} \choice $ -\infty< x< 4;\;  0< y<\infty $\\
\hspace*{-.45cm} \choice $ -\infty< x< 4;\;  0\le y<\infty $
\hspace*{.7cm} \choice $ -\infty< x\le 4;\;  0< y<\infty $

\end{oneparchoices}

\question   $ \mathlarger{\lim\limits_{x\to 0}\dfrac{\sin 7x -\sin x}{\sin 6x} = } $

\begin{oneparchoices}
\choice $ \dfrac{7}{6} $
\choice $ -\dfrac{7}{6} $
\choice $ 1 $
\choice $ -1 $

\end{oneparchoices}

\question ABC ত্রিভুজে $ a:b:c = 3:7:5 $ হলে $ \angle B = $ (In the triangle ABC , if $ a:b:c = 3:7:5 $, then $ \angle B = $)

\begin{oneparchoices}
\choice $ 60^{\circ} $
\choice $ 30^{\circ} $
\choice $ 90^{\circ} $
\choice  $ 120^{\circ} $

\end{oneparchoices}

\question  $ 2x^{2}-7x+5 =0 $ সমীকরণের মূলদ্বয় $ \alpha $ এবং $ \beta $, এবং $ x^{2}-4x+3 =0 $ সমীকরণের মূলদ্বয় $ \beta $ এবং $ \gamma $, হলে $ (\gamma+\alpha):(\gamma -\alpha)= $ (If $ \alpha $ and $ \beta $ are the roots of the equation  $ 2x^{2}-7x+5 =0 $ and, $ \beta $ and $ \gamma $ are the roots of equation  $ x^{2}-4x+3 =0 $, then )


\begin{oneparchoices}
\choice $ 6:5 $
\choice $ 5:6 $
\choice $ 11:1 $
\choice  $ 1:6 $

\end{oneparchoices}

\question $ (x-2)^{2} + (y-3)^{2} = 16 $  এবং $ (x-2)^{2} + (y-10)^{2} = 9 $ বৃত্তদ্বয়ের স্পর্শবিন্দুর স্থানাংক (The coordinates of the point of contact of the circle $ (x-2)^{2} + (y-3)^{2} = 16 $ and $ (x-2)^{2} + (y-10)^{2} = 9 $ )

\begin{oneparchoices}
\choice $ (2,3) $
\choice $ (16,9) $
\choice $ (2,10) $
\choice  $ (2,7) $

\end{oneparchoices}

\question $ \mathlarger{z = 1-\dfrac{1}{1-\dfrac{1}{1+i}}} $ জটিল সংখ্যাটির মডুলাস ও আর্গুমেন্ট (The modulus and the argument of the complex number $ \mathlarger{z = 1-\dfrac{1}{1-\dfrac{1}{1+i}}} $ are)

\begin{oneparchoices}
\choice $1,0$
\choice $ 1, \dfrac{\pi}{2} $
\choice $ 1,\pi $
\choice  $ 1, \dfrac{3\pi}{2} $

\end{oneparchoices}

\question  $ k $  এর কোন মানের জন্য $ y =kx(1-x) $ বক্ররেখার মূলবিন্দুতে স্পর্শকটি $ x- $অক্ষের সাথে $ 30^{\circ} $ কোণ উৎপন্ন করে (For what value of $ k $ the tangent to the curve $ y =kx(1-x) $ at the origin makes an angle $ 30^{\circ} $ with the $x-$axis)

\begin{oneparchoices}
\choice $ \sqrt{3} $
\choice $ \dfrac{1}{\sqrt{3}} $
\choice $ \dfrac{\sqrt{3}}{2} $
\choice  $ 1 $

\end{oneparchoices}

\question  $ -7<x<-1 $ কে পরমমানের সাহায্যে লিখলে দাঁড়ায়(expressed in terms of absolute value, $ -7<x<-1 $ becomes)
  
\begin{oneparchoices}
\choice $ |x+3|<4 $
\choice $ |x+1|<3 $
\choice $ |x+4|<3 $
\choice  $ |x-4|<1 $

\end{oneparchoices}

\question  ABC একটি সমকোণী ত্রিভুজ হলে(If ABC is a right angled triangle then) $ \cos^{2}A +\cos^{2}B + \cos^{2}C = $
\begin{oneparchoices}
\choice $ \dfrac{1}{2} $
\choice $ 1 $
\choice $ 0 $
\choice  $ -1 $

\end{oneparchoices}

\question $ y^{2} = 16x $ এবং $ y=4x $ দ্বারা আবদ্ধ ক্ষেত্রের ক্ষেত্রফল (The area of the region bounded by $ y^{2} = 16x $ and $ y=4x $ is)

\begin{oneparchoices}
\choice $ \dfrac{2}{3}\, unit^{2}$
\choice $  -\dfrac{2}{3}\, unit^{2} $
\choice $  \dfrac{3}{2}\, unit^{2} $
\choice  $  \dfrac{1}{3}\, unit^{2}$

\end{oneparchoices}

\question   8N এবং 3N দুইটি বল একটি বিন্দুতে $ 60^{\circ} $ কোণে একটি বস্তুতে ক্রিয়ারত। বলদ্বয়ের লদ্ধির মান(Two forces 8N and 3N are acting on an object at a point with $ 60^{\circ} $ angle. The magnitude of the resultant of two force is) 


\begin{oneparchoices}
\choice $ \sqrt{73}N $
\choice $ \sqrt{93}N $
\choice $ \sqrt{55}N $
\choice  $ 11N $

\end{oneparchoices}

\question  $ 1+(1+2)+(1+2+3)+\cdots +n $ তম পদ পর্যন্ত(to $ n $ terms) =

\begin{oneparchoices}
\choice $ \dfrac{1}{6}n(n+1)(2n+1) $
\choice $ \dfrac{1}{6}n(n+1)(n+2) $
\choice $ \dfrac{1}{2}n(n+1)(n+2) $
\choice $ \dfrac{1}{6}n(n+1)(n+2)(n+3) $

\end{oneparchoices}

\end{questions}
\begin{LARGE}
\begin{center}
গণিত (Mathematics - 2014)
\end{center}
\end{LARGE}
\begin{questions}

\question  $ |x|<1 $ শর্তে $ \dfrac{1+2x}{1-x} $ এর বিস্তৃতিতে $ x^9 $ এর সহগ (The coefficient of $x^3$ in the expansion of $ \dfrac{1+2x}{1-x} $ is) 

\begin{oneparchoices}
\choice 1
\choice 5
\choice 2
\choice 3
\end{oneparchoices}

 \question  $ x $ এর বাস্তব মানের জন্য $ |4x-3|>1 $ অসমতাটির সমাধান  ( For real $ x $ solution of the inequality $ |4x-3|>1 $ is)

\begin{oneparchoices}
\choice $ \Big(-\infty, \dfrac{1}{2}\Big) $
\choice $ \Big(1,-\infty) $
\choice $ \Big(-\infty, \dfrac{1}{2}\Big) \cup \Big(1,\infty\Big)  $
\choice $ \Big(-\infty, \dfrac{1}{2}\Big] \cup \Big(1,\infty\Big)  $

\end{oneparchoices}

\question  $ \dfrac{1}{3.4} + \dfrac{1}{4.5} + \dfrac{1}{5.6} + \cdots +n$ তম পদ পর্যন্ত (to $ n $ terms) = ?

\begin{oneparchoices}
\choice $ \dfrac{n+1}{3(n+2)}$
\choice $ \dfrac{n}{3(n+3)} $
\choice $ \dfrac{n}{2(n+3)} $
\choice  $ \dfrac{n+2}{3(n+3)} $

\end{oneparchoices}

\question  $ \begin{vmatrix}
\alpha & \alpha & x \\
\beta & \beta & \beta \\
\theta & x & \theta
\end{vmatrix} = 0$ , $ x=? $

\begin{oneparchoices}
\choice $ \alpha, \beta, \theta $
\choice $ \alpha, \theta $
\choice $ \beta, \theta $
\choice $ \alpha, \beta $

\end{oneparchoices}

\question  $ 3x^{2}-kx+4=0 $ সমীকরণটির একটি মূল অপরটির 3 গুণ হলে $ k $ এর মান   (If one root is 3-times of the other root of the equation $ 3x^{2}-kx+4=0 $, then the value of $ k $ is)

\begin{oneparchoices}
\choice 8
\choice $ -8 $
\choice $ \sqrt{8} $
\choice $ \pm 8 $

\end{oneparchoices}


\question  COURAGE শব্দটির বর্ণগুলি নিয়ে কতগুলি বিন্যাস সংখ্যা নির্ণয় করা যায় যেন প্রত্যেক বিন্যাসের প্রথমে একটি স্বরবর্ণ থাকে (The number of the arrangement that can be made using the letters of the word COURAGE with the one vowel at first is )

\begin{oneparchoices}
\choice 720
\choice 2880
\choice 180
\choice 5040

\end{oneparchoices}

\question  নিন্মের কোন বৃত্তটি অক্ষকে স্পর্শ করে ? (Which one of the following circles touches the $ x $ axis ? )

\begin{oneparchoices}
\choice $ x^{2} + y^{2}-2x+6y+4=0 $
\choice $ x^{2} + y^{2}-4x+6y+5=0 $\\
\hspace*{-.35cm}\choice $ x^{2} + y^{2}-2x+6y+1=0 $
\choice $ 2x^{2} + 2y^{2}-2x+6y+3=0 $

\end{oneparchoices}

\question 1  থেকে 21 পর্যন্ত সংখ্যা হতে যেকোন একটিকে দৈবচয়নের মাধ্যমে নিলে সেই সংখ্যাটি 3 বা 7 এর গুণিতক হওয়ার সম্ভাবনা কত? (What is the probability that one number chosen randomly from 1 to 21 is a multiple of 3 or 7 ? )

\begin{oneparchoices}
\choice $ \dfrac{8}{21} $
\choice $ \dfrac{3}{7} $
\choice $ \dfrac{10}{21} $
\choice $ \dfrac{11}{21} $

\end{oneparchoices}

\question  $ y = -5x+9 $ রেখার সাথে লম্বরেখার ‍নতি (The slope of the line perpendicular to the line $ y = -5x+9 $ is)

\begin{oneparchoices}
\choice $ 5 $
\choice $ -5 $
\choice $ \dfrac{1}{5} $
\choice $- \dfrac{1}{5}$

\end{oneparchoices}


\question  (1,4) এবং (9,12) বিন্দুদ্বয়ের সংযোজক রেখা যে বিন্দুতে 3:5 অনুপাতে অর্ন্তবিভক্ত হয়, তার স্থানাংক (The coordinates of the points divide the line joining the points (1,4) and (9,12) internally in the ratio 3:5 are )

\begin{oneparchoices}
\choice (7,4)
\choice (4,7)
\choice (5,8)
\choice (5,8)
\end{oneparchoices}

\question P(6,8), Q(4,0) এবং R(0,0)  শীর্ষবিন্দু বিশিষ্ট ত্রিভুজের ক্ষেত্রফল (The area of the triangle having vertices  P(6,8), Q(4,0) and R(0,0) is)

\begin{oneparchoices}
\choice 32 Sq. units
\choice 16 Sq. units
\choice 12 Sq. units
\choice 24 Sq. units
\end{oneparchoices}

\question  যদি $ a*b = \dfrac{ab}{a+b} $ দ্বারা $ a $ এবং $ b $ বাস্তব সংখ্যার মধ্যে সম্পর্ক $ * $ দ্বারা সংজ্ঞায়িত করা হয়, তবে $ 10*2 = $ (If is defined by all positive real numbers $ a $ and $ b $ by then $ 10*2 = $ )

\begin{oneparchoices}
\choice $ \dfrac{5}{3} $
\choice $ \dfrac{5}{2} $
\choice $ 5 $
\choice $ 2 $

\end{oneparchoices}

\question   $ \dfrac{i-i^{-1}}{i+2i^{-1}} $ এর মান এবং নতি হবে যথাক্রমে (The value and argument are respectively ) 

\begin{oneparchoices}
\choice 0,0
\choice $ -2i,\, -\dfrac{\pi}{2} $
\choice $ 2i,\, \dfrac{\pi}{2} $
\choice $ -2, \pi $

\end{oneparchoices}

\question  $ A = \begin{pmatrix} 2 & -2\\ -2 & 2 \end{pmatrix}$ এবং (and) $ A = \begin{pmatrix} 2 & 2\\ 3 & 3 \end{pmatrix},\; AB=?$

\begin{oneparchoices}
\choice  $\begin{pmatrix} -2 & 2\\ -2 & 2 \end{pmatrix}$
\choice  $\begin{pmatrix} 0 & 0\\ 0 & 0 \end{pmatrix}$
\choice  $\begin{pmatrix} 0 & 0\\ 3 & 0 \end{pmatrix}$
\choice  $\begin{pmatrix} -2 & -2\\ 2 & 2 \end{pmatrix}$
\end{oneparchoices}

\question $ a $ এর মান কত হলে $ \dfrac{1}{2}\hat{i} + \dfrac{1}{3}\hat{j}+a\hat{k} $ ভেক্টরটি একটি একক ভেক্টর হবে? (For what values of a the vector $ \dfrac{1}{2}\hat{i} + \dfrac{1}{3}\hat{j}+a\hat{k} $ will represent a unit vector ?)

\begin{oneparchoices}
\choice $ \pm \dfrac{2}{3} $
\choice $ \pm \dfrac{\sqrt{15}}{6} $
\choice $ \pm \dfrac{7}{6} $
\choice  $ \pm \dfrac{\sqrt{23}}{6} $

\end{oneparchoices}

\question ABC  ত্রিভুজের BC, CA এবং AB বাহুর মধ্যবিন্দু যথাক্রমে D, E এবং F হলে (If D, E and F are the middle points of the sides BC, CA and AB respectively of the triangle ABC, then )

\begin{oneparchoices}
\choice $ \vec{AD} = \vec{AB} + \vec{BC} $
\choice $ \vec{DA} = \vec{DF} + \vec{DE} $
\choice $ \vec{AD} = \vec{AB} + \vec{AC} $
\choice $ \vec{AD} = \vec{BE} + \vec{CF} $

\end{oneparchoices}

\question  $ 3x+5y=2,\; 2x+3y=0,\; ax+by+1=0 $ সমবিন্দু হলে $ a $ এবং $ b $ এর সম্পর্ক (If $ 3x+5y=2,\; 2x+3y=0,\; ax+by+1=0 $ are concurrent then the relation between $ a $ and $ b $ is )

\begin{oneparchoices}
\choice $ 4a-6b=1 $
\choice $ 4a-6b=2 $
\choice $ 6a-4b=1 $
\choice $ 6a-4b=2 $

\end{oneparchoices}


\question   $ \sin^{-1}\dfrac{4}{3} + \cos^{-1}\dfrac{2}{5} $ সমান (equals)

\begin{oneparchoices}
\choice $ \tan^{-1}\dfrac{2}{11} $
\choice $ \sin^{-1}\dfrac{11}{2} $
\choice $ \tan^{-1}\dfrac{11}{2} $
\choice $ \cos^{-1}\dfrac{11}{2} $

\end{oneparchoices}

\question  $ \csc\theta + \cot\theta = \sqrt{3} $, $ (0<\theta<2\pi) $ হলে $ \theta $ এর মান (The value of $ \theta $ is) 

\begin{oneparchoices}
\choice $ \dfrac{\pi}{6} $
\choice $ \dfrac{\pi}{4} $
\choice $ \dfrac{\pi}{3} $
\choice $ \dfrac{2\pi}{3} $

\end{oneparchoices}

\question  $ f(x) = \dfrac{1}{\sqrt{4-x^{2}}} $ বাস্তব ফাংশনটির ডোমেন ও রেঞ্জ (Domain and range of the real valued function ) 

\begin{oneparchoices}
\choice $ x<-2,\; y>\dfrac{1}{2} $\\
\choice $ -2<x<2,\; y\ge \dfrac{1}{2} $\\
\choice $ -2\le x\le 2,\; y<\dfrac{1}{2} $\\
\choice $ -x<-2\; \&\; x>2,\;  -2<y<2$
\end{oneparchoices}

\question  $ x=0 $ বিন্দুতে $ y = x + e^{x} $ এর লেখচিত্রে স্পর্শকের সমীকরণ হবে (The equation of the tangent to the graph of $ y = x + e^{x} $, at $x=0$ is )

\begin{oneparchoices}
\choice $ y=x $
\choice $ y=x+1 $
\choice $ y=2x+1 $
\choice $ y=2x $

\end{oneparchoices}

\question  $ 5x^2 +15x -10y -4 =0 $ পরাবৃত্তের নিয়ামকের সমীকরণ (The equation of the directrix of the parabola $ 5x^2 +15x -10y -4 =0 $ is ) 

\begin{oneparchoices}
\choice $ 40x+81 = 0 $
\choice $ 2x+3 = 0 $
\choice $ 40y+81= 0 $
\choice $ 40y +41 =0 $

\end{oneparchoices}

\question $ \sin 65^{\circ} + \cos 65^{\circ} $ এর মান (The value of $ \sin 65^{\circ} + \cos 65^{\circ} $ is )

\begin{oneparchoices}
\choice $ 2\cos 20^{\circ} $
\choice $ \sqrt{2}\cos 20^{\circ} $
\choice $ \sqrt{2}\sin 20^{\circ} $
\choice  $ 2\sin 20^{\circ} $

\end{oneparchoices}

\question  ABC ত্রিভুজের $ \cos A + \cos C = \sin B $ হলে $ \angle C $ এর মান (If $ \cos A + \cos C = \sin B $ of the triangle ABC, then $ \angle C $ is)

\begin{oneparchoices}
\choice $ \dfrac{\pi}{4} $
\choice $ \dfrac{\pi}{2} $
\choice $ \dfrac{\pi}{6} $
\choice $ \dfrac{\pi}{3} $

\end{oneparchoices}

\question  $ \mathlarger{\int_{0}^{1}\dfrac{ln(x+1)}{x+1} dx =} $

\begin{oneparchoices}
\choice $ \dfrac{1}{2}(ln2)^{2} $
\choice $ \ln 2 $
\choice $ \infty $
\choice  $ 0 $
\end{oneparchoices}

\question $ y = x $ এবং $ y=x^2 $ দ্বারা আবদ্ধ ক্ষেত্রের ক্ষেত্রফল (The area of the region bounded by $ y = x $ and $ y = x^2 $ is)

\begin{oneparchoices}
\choice $ \dfrac{5}{6}$
\choice $  -\dfrac{1}{6} $
\choice $  \dfrac{1}{6} $
\choice  $  \dfrac{1}{3}$
\end{oneparchoices}

\question  3P এবং 2P বল দুটির লদ্ধির মান R যদি প্রথম বলের পরিমান দ্বিগুণ করা হয় তবে লদ্ধির মানও দ্বিগুন হয়। বলদ্বয়ের মধ্যবর্তী কোণ হবে (Two forces of magnitude 2P and 3P have the resultant R. If the first force is doubled. The angle between the force is)

\begin{oneparchoices}
\choice $ 60^{\circ} $
\choice $ 90^{\circ} $
\choice $ 120^{\circ} $
\choice  $ 50^{\circ} $
\end{oneparchoices}

\question  $ \mathlarger{\lim_{x\to\infty} \dfrac{x^{2}+6x}{2x^{2}+5}} $

\begin{oneparchoices}
\choice $ 0 $
\choice $ \dfrac{3}{2} $
\choice $ \dfrac{1}{2} $
\choice  $ 1 $

\end{oneparchoices}

\question   $\mathlarger{\int \dfrac{e^{x}(1+x)}{\cos^{2}(xe^{x})}dx =?}$


\begin{oneparchoices}
\choice $ xe^{x} + c $
\choice $ \tan (xe^{x}) + c $
\choice $ \cot(xe^{x}) + c $
\choice $ \cos(xe^{x}) + c $

\end{oneparchoices}

\question  $ e^{xy + 1} =5 $ হলে (if) $ \dfrac{dy}{dx} = ? $

\begin{oneparchoices}
\choice $ \dfrac{ln 5}{xy} $
\choice $ \dfrac{ln 5}{x^{2}} $
\choice $ -\dfrac{y}{x} $
\choice $ \dfrac{ln 5}{y} $

\end{oneparchoices}

\end{questions}

\begin{LARGE}
\begin{center}
গণিত (Mathematics - 2013)
\end{center}
\end{LARGE}
\begin{questions}

\question  (3, -1) এবং (5,2) বিন্দুদ্বয়ের সংযোগকারী সরলরেখাকে 3:4 অনুপাতে বহি:স্থভাবে বিভক্তকারী বিন্দুর স্থানাংক -

\begin{oneparchoices}
\choice $ \dfrac{17}{3}, 3 $
\choice $ \dfrac{2}{7}, \dfrac{2}{7} $
\choice $ \dfrac{27}{4}, \dfrac{4}{3} $
\choice None
\end{oneparchoices}

 \question  $ f(x)= \sqrt{x^{2}-5x+6} $ ফাংশনের ডোমেইন এবং রেইঞ্জ যথাক্রমে

\begin{oneparchoices}
\choice $ x \le 2,\, 3 \le x \,and\, y \ge 0 $\\

\choice $ 2 \le x \le 3\, and\, y ≥ 0) $\\

\choice $ x \ge 3\, and\, y > 0  $\\

\choice $ x \le 2,\, x \ge 3\, and\, y > 0  $

\end{oneparchoices}

\question 32 ft/sec আদিবেগে এবং ভূমির সাথে $ 30^{\circ} $ কোণে একটি বস্তু নিক্ষেপ করা হলো। ইহার আনুভূমিক পাল্লা - 

\begin{oneparchoices}
\choice 16 ft
\choice $ 3\sqrt{2} $ ft
\choice $ 32 $ ft
\choice  $ 16\sqrt{3} $ ft

\end{oneparchoices}

\question  If  $x^{n} +y^{n} = a^{n} $ then $ \dfrac{dy}{dx}=? $

\begin{oneparchoices}
\choice $ \Big( \dfrac{x}{y}\Big)^{n}$
\choice $ \Big( -\dfrac{x}{y}\Big)^{n}$
\choice $ -\Big( \dfrac{x}{y}\Big)^{n-1}$
\choice $ \Big(- \dfrac{x}{y}\Big)^{n-1}$

\end{oneparchoices}

\question  $\cot\theta + \sqrt{3} = 2\csc\theta $ সমীকরণের সমাধান

\begin{oneparchoices}
\choice $ \theta = 2n\pi -\dfrac{\pi}{3} $
\choice $ \theta = 2n\pi +\dfrac{\pi}{3} $
\choice $ \theta = 2n\pi +\dfrac{\pi}{6} $
\choice $ \theta = 2n\pi -\dfrac{\pi}{6} $

\end{oneparchoices}

\question  $\begin{pmatrix}
\cos\theta & \sin\theta \\
-\sin\theta & \cos\theta
\end{pmatrix} $ এর বিপরীত ম্যাট্রিক্স

\begin{oneparchoices}
\choice  $\begin{pmatrix}
\cos\theta & -\sin\theta \\
-\sin\theta & \cos\theta
\end{pmatrix} $
\choice  $\begin{pmatrix}
\cos\theta & -\sin\theta \\
\sin\theta & -\cos\theta
\end{pmatrix} $
\choice  $\begin{pmatrix}
\cos\theta & -\sin\theta \\
\sin\theta & \cos\theta
\end{pmatrix} $
\choice  $\begin{pmatrix}
\cos\theta & \sin\theta \\
\sin\theta & \cos\theta
\end{pmatrix} $

\end{oneparchoices}


\question  4 জন মহিলাসহ 10 ব্যক্তির মধ্য থেকে 5 জনের একটি কমিটি গঠন করতে হবে যাতে অন্তত একজন মহিলা অন্তর্ভুক্ত থাকবে। কত বিভিন্ন প্রকারে এ কমিটি গঠন করা যেতে পারে?

\begin{oneparchoices}
\choice 1440
\choice 246
\choice 120
\choice 60

\end{oneparchoices}

\question  $ A=\begin{vmatrix}
0 & 3 & 2x+7 \\
2 & 7x & 9+5x \\
0 & 0 & 2x+5
\end{vmatrix} $  হলে, $  x $ এর মান

\begin{oneparchoices}
\choice $ -\dfrac{9}{5} $
\choice  $ -\dfrac{7}{5} $
\choice  $ -\dfrac{5}{2} $
\choice 0

\end{oneparchoices}

\question $ \tan^{-1}(\sin(\cos^{-1}\sqrt{\dfrac{2}{3}})) $ সমান

\begin{oneparchoices}
\choice $ \dfrac{\pi}{2} $
\choice $ \dfrac{\pi}{3} $
\choice $ \dfrac{\pi}{4} $
\choice $ \dfrac{\pi}{6} $

\end{oneparchoices}

\question কোনো বিন্দুতে P এবং 2P মানের দুইটি বল ক্রিয়াশীল। প্রথম বলটিকে দ্বিগুণ করে দ্বিতীয়টিইর মান 8 একক বৃদ্ধি করা হলে তাদের লব্ধির দিক অপরিবর্তিত থাকে। P এর মান

\begin{oneparchoices}
\choice 1
\choice 2
\choice 4
\choice 8
\end{oneparchoices}

\question  101101 এর সাথে কোন ন্যূনতম দ্বিমিক সংখ্যা যোগ করলে যোগফল 16 দ্বারা বিভাজ্য হবে?

\begin{oneparchoices}
\choice 10011
\choice 11
\choice 110
\choice 11

\end{oneparchoices}


\question $ y=-\sqrt{a^2 -x^2} $  ও $ y = 0$ দ্বারা আবদ্ধ ক্ষেত্রের ক্ষেত্রফল

\begin{oneparchoices}
\choice $ \dfrac{1}{4}\pi a^2 $
\choice $ \dfrac{1}{2}\pi a^2 $
\choice $ \pi a^2 $
\choice $ \dfrac{1}{2}a^2 $
\end{oneparchoices}

\question $ -\dfrac{1}{2} - \dfrac{1}{2.2^{2}} -\dfrac{1}{3.2^3} - \dfrac{1}{4.2^{4}} $ ধারাটির সমষ্টি -

\begin{oneparchoices}
\choice $ -2 ln2  $
\choice $ -ln 2 $
\choice $ -2e $
\choice $ -e $
\end{oneparchoices}

\question  $ \dfrac{1}{|3x+1|}\ge 5 $ বাস্তব সংখ্যায় অসমতাটির সমাধান-

\begin{oneparchoices}
\choice $ \Big(-\dfrac{2}{3}, -\dfrac{1}{3}\Big)\cup \Big(\dfrac{1}{3}, \dfrac{4}{5}\Big)  $
\choice $ \Big[-\dfrac{2}{3}, -\dfrac{1}{3}\Big)\cup \Big(\dfrac{1}{3}, \dfrac{4}{5}\Big)  $
\choice $ \Big(-\dfrac{2}{3}, -\dfrac{4}{15}\Big)$
\choice None

\end{oneparchoices}

\question  $ \mathlarger{\lim_{x\to 0} \dfrac{\cos x -1}{x^{2}} = ?} $

\begin{oneparchoices}
\choice $ -1 $
\choice $ -\dfrac{1}{2} $
\choice $ \dfrac{1}{2} $
\choice  $ 1 $

\end{oneparchoices}

\question   $(3, -1)$ বিন্দুগামী এবং  $x^2 + y^2 – 6x + 8y = 0$ বৃত্তের সাথে এককেন্দ্রিক বৃত্তের সমীকরণ  -

\begin{oneparchoices}
\choice $x^2 + y^2 + 6x - 8y +16 = 0$
\choice $x^2 + y^2 – 6x - 8y -16= 0$
\choice $x^2 + y^2 – 6x + 8y +16 = 0$
\choice $x^2 + y^2 – 6x -8y + 16 = 0$

\end{oneparchoices}

\question  $ \sin A + \cos A = \sin B + \cos B $ হলে $ A+B =? $

\begin{oneparchoices}
\choice  $\pi$
\choice  $\dfrac{\pi}{2}$
\choice  $2\pi$
\choice  $\dfrac{\pi}{4}$
\end{oneparchoices}

\question  $ \Big(2x^2 - \dfrac{1}{x} \Big)^{11} $ এর বিস্তৃতিতে $ x^7 $ এর সহগ-

\begin{oneparchoices}
\choice $ - \dfrac{231}{8} $
\choice $ 231 $
\choice $  \dfrac{231}{4} $
\choice  $  \dfrac{231}{8} $

\end{oneparchoices}

\question  $ z_{1}= 2+i $ এবং $ z_{2} = 3+i $ হলে $ z_{1}\bar{z_{2}} $ এর মডুলাস -

\begin{oneparchoices}
\choice $ 6 $
\choice $ 5\sqrt{2} $
\choice $ 7 $
\choice $ 5\sqrt{3} $

\end{oneparchoices}

\question  পূর্ণসংখ্যা সহগসহ দ্বিমাত্রিক সমীকরণ, যার একটি মূল $ \sqrt{-5}-1 $

\begin{oneparchoices}
\choice $ x^2 + 2x + 6 = 0 $
\choice $  x^2 + x + 3 = 0 $
\choice $ x^2 + 2x - 6 = 0 $
\choice $ x^2 + x - 3 = 0 $

\end{oneparchoices}


\question   একটি আয়তক্ষেত্রের দৈর্ঘ্য $20\%$ বৃদ্ধি এবং প্রস্থ $20\%$ হ্রাস করলে এর ক্ষেত্রফলের শতকরা পরিবর্তন

\begin{oneparchoices}
\choice  decreases by $4\% $
\choice   increases by $4\% $
\choice  increases by $5\% $
\choice remains unchanged

\end{oneparchoices}

\question  $x + y = 3$ এবং $y – x = 1$ সরলরেখাদ্বয়ের ছেদবিন্দুগামী $x-$অক্ষের সমান্তরাল সরলরেখার সমীকরণ

\begin{oneparchoices}
\choice $ y=2 $
\choice $ 2y =3 $
\choice $ x=1 $
\choice $ x+3=0 $

\end{oneparchoices}

\question  একক ব্যাসার্ধের বৃত্তে অন্তর্লিখিত একটি সমবাহু ত্রিভুজের বাহুর দৈর্ঘ্য

\begin{oneparchoices}
\choice $ \dfrac{3}{2} $
\choice $ \dfrac{\sqrt{3}}{2} $
\choice $ \sqrt{3} $
\choice $ 1 $
\end{oneparchoices}

\question   ধনাত্মক $ x $ এর জন্য $ F(x) = \mathlarger{\int_{1}^{x}\ln t dt} $ হলে, $ F^{\prime} (x) = ?$

\begin{oneparchoices}
\choice $ \dfrac{1}{x} $
\choice $ lnx $
\choice $ xlnx $
\choice $ xlnx -x $

\end{oneparchoices}

\question 1, 2, 3, 4, 5, 6 ও 7 থেকে পুনরাবৃত্তি ছাড়া তিন অঙ্কের সংখ্যা গঠন করা হলে, কয়টি সংখ্যার মান 100 থেকে 500 এর মধ্যে?

\begin{oneparchoices}
\choice 240
\choice 60
\choice 120
\choice 480

\end{oneparchoices}

\question  ABC ত্রিভুজের BC, CA ও AB বাহুর মধ্যবিন্দুগুলো যথাক্রমে D, E ও F হলে

\begin{oneparchoices}
\choice $ \vec{AD} = \vec{AB} + \vec{BC} $
\choice $ \vec{AD} = \vec{AF} + \vec{AE} $
\choice $ \vec{AD} = \vec{AB} + \vec{AC} $
\choice $ \vec{AD} = \vec{BE} + \vec{CF} $

\end{oneparchoices}

\question যদি  $  f(x) = (x – 2)(1 – x) $ হয়, তবে $ f(f(3)) $ এর মান

\begin{oneparchoices}
\choice $ 9 $
\choice $ -12 $
\choice $ 12 $
\choice $ 8 $

\end{oneparchoices}

\question   1, 0, 2 দ্বারা গঠিত তিন অঙ্কবিশিষ্ট সংখ্যাগুলো হতে দৈবচয়ন পদ্ধতিতে একটি সংখ্যা নেয়া হলে সংখ্যাটি 10 দ্বারা বিভাজ্য হওয়ার সম্ভাবনা

\begin{oneparchoices}
\choice $ \dfrac{1}{2} $
\choice $ \dfrac{1}{3} $
\choice $ \dfrac{2}{9} $
\choice $ \dfrac{1}{6} $
\end{oneparchoices}

\question  $ x^{2}- 4x+12y-40 = 0 $ পরাবৃত্তের উপকেন্দ্রিক লম্বের দৈর্ঘ্য 

\begin{oneparchoices}
\choice 12
\choice 8
\choice 6
\choice 4
\end{oneparchoices}

\question  $ x $ এর কোন মানের জন্য $ y = x + \dfrac{1}{x} $ বক্ররেখাটির ঢাল শূন্য হবে?

\begin{oneparchoices}
\choice $ \pm \dfrac{3}{2} $
\choice $ \pm 2 $
\choice $ 1 $
\choice  $ \pm 1 $
\end{oneparchoices}

\end{questions}
\begin{LARGE}
\begin{center}
গণিত (Mathematics - 2012)
\end{center}
\end{LARGE}
\begin{questions}

 \question  $ f(x)= 4-(x-3)^2 $ ফাংশনের ডোমেইন এবং রেঞ্জ যথাক্রমে

\begin{oneparchoices}
\choice $ \mathbb{R},\,\mathbb{R} $

\choice $ \mathbb{R},\,f(x)\le 4 $

\choice $ x \ge 4,\, \mathbb{R}  $

\choice $ \mathbb{R},\, x\ge 3  $

\end{oneparchoices}

\question  $ f(x) = \dfrac{x-3}{2x+1} $ এবং $ x \neq - \dfrac{1}{2} $ হলে $ f^{-1}(2) $ এর মান হবে -

\begin{oneparchoices}
\choice $ \dfrac{1}{2}$
\choice $ \dfrac{1}{5}$
\choice $ 2 $
\choice $ 5 $
\end{oneparchoices}


\question একটি ইলেকট্রিক ফিল্ডে ইলেকট্রনের ত্বরণ এবং শক্তি সমানুপাতিক। $ 10^{-2} N $ শক্তির জন্য ত্বরণ $ 10^{10}\, m/s^{2} $ হলে $ 10^{-25} N $ শক্তির জন্য ত্বরণ হবে

\begin{oneparchoices}
\choice $ 10^{5}\, m/s^{2} $
\choice $ 10^{15}\, m/s^{2} $
\choice $ 10^{-5}\, m/s^{2} $
\choice $ 10^{-15}\, m/s^{2} $

\end{oneparchoices}

\question   বাস্তব সংখ্যায় $ \dfrac{1}{|2x-3|}> 5 $ অসমতাটির সমাধান হলো-

\begin{oneparchoices}
\choice $ \Big(\dfrac{7}{5}, \dfrac{8}{5}\Big)$
\choice $ \Big[\dfrac{7}{5}, \dfrac{8}{5}\Big]$
\choice $ \Big(\dfrac{7}{5}, \dfrac{3}{5}\Big)\cup \Big(\dfrac{3}{2}, \dfrac{8}{5}\Big)  $
\choice $ \Big[\dfrac{7}{5}, \dfrac{3}{2}\Big]\cup \Big[\dfrac{3}{2}, \dfrac{8}{5}\Big]  $
\end{oneparchoices}

\question  $ x^{2}+4x+2y=0 $ পরাবৃত্তের শীর্ষবিন্দু হবে- 

\begin{oneparchoices}
\choice $ (2,-2) $
\choice $ (-2,-2) $
\choice $ (-2, 2) $
\choice $ (2, 2) $

\end{oneparchoices}

\question $ \Big(2x^{2}-\dfrac{1}{2x^{3}}\Big)^{10} $ এর বিস্তৃতিতে বর্জিত পদটি কততম এবং এর মান কত?

\begin{oneparchoices}
\choice পঞ্চম এবং 840 
\choice চতুর্থ এবং 1920
\choice  ষষ্ঠ এবং 252 
\choice সপ্তম এবং 30

\end{oneparchoices}

\question  $A = \begin{bmatrix}
1 & i \\
-i & 1
\end{bmatrix},\; B = \begin{bmatrix}
i & -1 \\
-1 & -i
\end{bmatrix} $ এবং $ i = \sqrt{-1} $ হলে AB এর মান হবে- 

\begin{oneparchoices}
\choice  $\begin{bmatrix}
1 & 0 \\
0 & 1
\end{bmatrix} $
\choice   $\begin{bmatrix}
0 & 0 \\
0 & 0
\end{bmatrix} $
\choice   $\begin{bmatrix}
i & 0 \\
0 & i
\end{bmatrix} $
\choice   $\begin{bmatrix}
i & 1 \\
1 & i
\end{bmatrix} $
\end{oneparchoices}

\question নির্ণয় কর: $ \mathlarger{\lim_{x\to 0} \dfrac{e^{x} -1}{x}} $

\begin{oneparchoices}
\choice $ -1 $
\choice $ -1 $
\choice $ 2 $
\choice  $ 3 $

\end{oneparchoices}

\question  স্বরবর্ণগুলোকে সবসময় একত্রে KACHUA শব্দটির বর্ণগুলোকে সাজানো সংখ্যা হবে-

\begin{oneparchoices}
\choice 24
\choice 72
\choice 144
\choice 8

\end{oneparchoices}

\question  \say{a} এর কোন মানের জন্য $ 2\hat{i}+\hat{j}-\hat{k},\, 3\hat{i}-2\hat{j}+4\hat{k} $ এবং $ \hat{i}-3\hat{j}+a\hat{k} $ ভেক্টরত্রয় সমতলীয়?


\begin{oneparchoices}
\choice 5
\choice 4
\choice 3
\choice 2

\end{oneparchoices}

\question  $ x $ অক্ষকে $ (4,0) $ বিন্দুতে স্পর্শ করে এবং কেন্দ্র $ 5x-7y+1=0 $ সরলখোর উপর অবস্থিত এমন বৃত্তের সমীকরণ হবে-

\begin{oneparchoices}
\choice $ x^{2}+y^{2}-8x-6y+9=0 $
\choice $ x^{2}+y^{2}-8x+6y+16=0 $
\choice $ x^{2}+y^{2}-8x+6y+9=0 $
\choice $ x^{2}+y^{2}-8x-6y+16=0 $

\end{oneparchoices}

\question একজন লোকের 3 জোড়া কালো মোজা এবং 2 জোড়া বাদামী মোজা আছে। একদিন অন্ধকারে তাড়াহুড়া করে লোকটি মোজা পড়ল। সে প্রথমে একটি বাদামী মোজা পরার পর পরবর্তী মোজাটিও বাদামী হওয়ার সম্ভাবনা-

\begin{oneparchoices}
\choice $ \dfrac{1}{3} $
\choice $ \dfrac{2}{15} $
\choice $ \dfrac{1}{10} $
\choice $ \dfrac{3}{10} $
\end{oneparchoices}

\question $ 3x^{2}-5x+1=0 $ সমীকরণের মূলদ্বয় $ \alpha, \beta $ হলে $ \dfrac{1}{\alpha}, \dfrac{1}{\beta} $ মূলবিশিষ্ট সমীকরণ হবে-

\begin{oneparchoices}
\choice $ 3x^{2}-5x+1=0 $
\choice $ x^{2}-5x+3=0 $
\choice $ 5x^{2}-3x-1=0 $
\choice $ 3x^{2}+5x-1=0 $
\end{oneparchoices}


\question $ \mathlarger{\int\sqrt{\dfrac{1+x}{1-x}}dx = f(x)+c} $ হলে $ f(x) $ এর মান-

\begin{oneparchoices}
\choice $ \sin^{-1}x + \sqrt{1-x^{2}} $
\choice $ \sin^{-1}x - \sqrt{1-x^{2}} $
\choice $ \cos^{-1}x + \sqrt{1-x^{2}} $
\choice $ \sin^{-1}x + \sqrt{1+x^{2}} $
\end{oneparchoices}

\question যদি $ y = \sqrt{\cos 2x} $ হয়, তবে $ \dfrac{dy}{dx} = $ 

\begin{oneparchoices}
\choice $ -\dfrac{\sin 2x}{\sqrt{\cos 2x}}  $
\choice $ -\dfrac{\cos 2x}{\sqrt{\sin 2x}}  $
\choice $ -\dfrac{\sin x}{\sqrt{\tan x}}  $
\choice $ \dfrac{\tan 2x}{\sqrt{\sin 2x}}  $
\end{oneparchoices}

\question $ \tan \Big(\tan^{-1}\dfrac{1}{2} + \tan^{-1}\dfrac{1}{3} \Big) $ এর মান হবে-

\begin{oneparchoices}
\choice $ \dfrac{5}{6} $
\choice $ 1 $
\choice $ \dfrac{\pi}{4} $
\choice $ - \dfrac{5}{6}$

\end{oneparchoices}

\question  $ \sin (ax +b) $ এর $ n $ তম অন্তরক হবে-

\begin{oneparchoices}
\choice  $a^{n} \sin\Big(\dfrac{n\pi}{2} + ax+b\Big)$
\choice  $a^{n} \cos\Big(\dfrac{n\pi}{2} + ax+b\Big)$\\
\choice  $(-1)^{n}a^{n} \sin (ax +b) $
\choice  $(-1)^{n}a^{n} \cos (ax +b) $
\end{oneparchoices}

\question $ \dfrac{(i+1)^{2}}{(i-1)^{4}} $  জটিল সংখ্যাটির আর্গুমেন্ট হবে-

\begin{oneparchoices}
\choice $ \pi $
\choice $ -\pi $
\choice $  \dfrac{\pi}{2} $
\choice  $  -\dfrac{\pi}{2} $

\end{oneparchoices}

\question $ 8+4\sqrt{5}i $ এর বর্গমূল হবে-

\begin{oneparchoices}
\choice $ \pm(3-2i) $
\choice $ \pm(\sqrt{10} + \sqrt{2}i) $
\choice $ \pm(\sqrt{10} - \sqrt{2}i) $
\choice $ \pm(3+2i) $

\end{oneparchoices}

\question  $ y=mx,\, y=m_{1}x $ এবং $ y=b  $ সরলরেখাদ্বয় দ্বারা গঠিত ত্রিভুজের বর্গএককে ক্ষেত্রফল হবে-

\begin{oneparchoices}
\choice $ \dfrac{b^{2}(m_{1}-m)}{2mm_{1}} $
\choice  $ \dfrac{b^{2}(m-m_{1})}{2mm_{1}} $
\choice $ \dfrac{b^{2}|m-m_{1}|}{mm_{1}} $
\choice $ \dfrac{b^{2}|m-m_{1}|}{2mm_{1}} $

\end{oneparchoices}


\question $ \dfrac{1}{2} + \dfrac{1}{3^{2}} + \dfrac{1}{2^{3}} + \dfrac{1}{3^{4}} + \dfrac{1}{2^{5}}+ \dfrac{1}{3^{6}}+\cdots $  ধারাটির সমষ্টি হবে – 

\begin{oneparchoices}
\choice  $ \dfrac{24}{19} $
\choice  $ \dfrac{19}{24} $
\choice  $ \dfrac{5}{24} $
\choice $ \dfrac{5}{19} $
\end{oneparchoices}

\question  একজন কৃষক আয়তার বাগানের তিনদিক বেড়াদিয়ে চর্তুদিক দেয়াল দিয়ে ঘেরাও দিলো। যদি তার কাছে $100m$ বেড়া থাকে তবে ঘেরাও দেয়া স্থানের সর্বোচ্চ আয়তন হবে-
 
\begin{oneparchoices}
\choice $ 2500m^{2} $
\choice $ 1250m^{2} $
\choice $ 750m^{2} $
\choice $ 2000m^{2} $
\end{oneparchoices}

\question $\begin{vmatrix}
a & 1 & b+c \\
b & 1 & c+a\\
c & 1 & a+b
\end{vmatrix} $  এর মান হবে-

\begin{oneparchoices}
\choice 0
\choice $ abc(a+b)(b+c)(c+a) $
\choice $ abs $
\choice  $ (a+b)(b+c)(c+a) $
\end{oneparchoices}

\question  $ x^{2}+3xy+5y^{2}=1 $ যদি হয় তাহলে $ \dfrac{dy}{dx} $ সমান হবে-

\begin{oneparchoices}
\choice $ -\dfrac{3x+2y}{3x+10y} $
\choice $ \dfrac{2x+3y}{3x+10y} $
\choice $ \dfrac{2x-3y}{3x+10y} $
\choice $ \dfrac{2x+3y}{3x-10y} $
\end{oneparchoices}

\question   দশমিক সংখ্যা 2013 এর দ্বিমিক প্রকাশ হবে-

\begin{oneparchoices}
\choice 11111011101
\choice 10111011111
\choice 10101110111
\choice 10101110101

\end{oneparchoices}

\question  $ u $ বেগে আনুভুমিকের সাথে $ \alpha $ কোণে প্রক্ষিপ্ত বস্তুর সর্বোচ্চ উচ্চতা হবে-

\begin{oneparchoices}
\choice $ \dfrac{u^{2}\sin 2\alpha}{2g} $
\choice $ \dfrac{u^{2}\sin^{2}\alpha}{2g} $
\choice $ \dfrac{u^{2}\sin 2\alpha}{g} $
\choice $ \dfrac{u^{2}\sin^{2}\alpha}{g} $

\end{oneparchoices}

\question  3P এবং 2P বলদ্বয়ের লদ্ধি R। প্রথম বলটিদ্বিগুণ করা হলে লদ্ধিও দ্বিগুণ হয়। বলদ্বয়ের অন্তর্গত কোণ হবে-

\begin{oneparchoices}
\choice $ 110^{\circ} $
\choice $ 150^{\circ} $
\choice $ 120^{\circ} $
\choice $ 135^{\circ} $

\end{oneparchoices}

\question $ 3x^{2}+5y^{2} =15 $ উপবৃত্তের উৎকেন্দ্রিকতা হবে -

\begin{oneparchoices}
\choice $ \sqrt{\dfrac{3}{5}} $
\choice $ \sqrt{\dfrac{5}{3}} $
\choice $ \sqrt{\dfrac{2}{5}} $
\choice $ \sqrt{\dfrac{5}{2}} $

\end{oneparchoices}

\question  $ x=y^{2} $  এবং $ y=x-2 $ দ্বারা আবদ্ধ ক্ষেত্রের ক্ষেত্রফল হবে -

\begin{oneparchoices}
\choice $ 1\dfrac{1}{3} $
\choice $ 3\dfrac{1}{6} $
\choice $ 4\dfrac{1}{2} $
\choice $ 4\dfrac{3}{4} $
\end{oneparchoices}

\question  একটি বস্তুকণা খাড়া উপরের দিকে প্রক্ষেপ করলে নির্দিষ্ট বিন্দু P তে পৌছাতে $ t_{1} $ সময় লাগে। যদি আরত্ত $ t_{2} $  সময় পর বস্তুটি ভুমিতে পতিত হয় তবে কণাটির সর্বোচ্চ উচ্চতা হবে - 

\begin{oneparchoices}
\choice $ \dfrac{1}{2}g(t_{1}+t_{2})^{2} $
\choice $ \dfrac{1}{8}g(t_{1}+t_{2})^{2} $
\choice $ \dfrac{1}{2}g(t_{1}^{2}+t_{2}^{2}) $
\choice $ \dfrac{1}{8}g(t_{1}^{2}+t_{2}^{2}) $
\end{oneparchoices}

\end{questions}
\begin{LARGE}
\begin{center}
গণিত (Mathematics - 2011)
\end{center}
\end{LARGE}
\begin{questions}

 \question  $ 3x+7y-2=0 $ সরলরেখার উপর লম্ব এবং $ (2,1) $ বিন্দুগামী সরলরেখার সমীকরণ -

\begin{oneparchoices}
\choice $ 3x+7y-13=0 $
\choice $ 7x-3y-11=0 $
\choice $ 7x+3y-17=0 $
\choice $ 7x-3y-2 =0 $
\end{oneparchoices}

\question  কোন স্তম্ভের শীর্ষ হতে $ 19.5\,m/sec $ বেগে খাড়া উপরের দিকে নিক্ষিপ্ত কোন কণা 5 সেকেন্ড পরে স্তম্ভের পাদদেশে পতিত হলে স্তম্ভের উচ্চতা – 

\begin{oneparchoices}
\choice $ 20\,m$
\choice $ 25\,m$
\choice $ 30\,m $
\choice $ 50\,m $
\end{oneparchoices}

\question 6 জন ছাত্র 5 জন ছাত্রী থেকে একটি কমিটি গঠন করতে হবে যাতে অন্তত: একজন ছাত্র এবং একজন ছাত্রী অর্ন্তভূক্ত থাকে। কত প্রকারে এ কমিটি গঠন করা যেতে পারে?

\begin{oneparchoices}
\choice 160
\choice 360
\choice 410
\choice 455

\end{oneparchoices}

\question  $\begin{pmatrix}
m-2 & 6 \\
2 & m-3
\end{pmatrix} $ ম্যাট্রিক্সটি ব্যতিক্রমী হলে $ m $ এর মান

\begin{oneparchoices}
\choice 6, -1
\choice -4,6
\choice -6, 4
\choice 1, -6
\end{oneparchoices}

\question  $ \mathlarger{\lim_{x\to 0} \dfrac{\sin^{-1}2x}{x}} $ এর মান

\begin{oneparchoices}
\choice 1
\choice 0
\choice 2
\choice $ \dfrac{1}{2} $
\end{oneparchoices}

\question  $ \lambda $ এর যে মানের জন্য $ y=\lambda x(1-x) $ বক্ররেখার স্পর্শকটি মূলবিন্দুতে $ x- $অক্ষের সাথে $ 30^{\circ} $ কোণ উৎপন্ন করে -

\begin{oneparchoices}
\choice $ \sqrt{3} $
\choice $ \dfrac{1}{\sqrt{3}} $
\choice  $ \dfrac{\sqrt{3}}{2} $
\choice 1

\end{oneparchoices}

\question  $ (2,4) $ কেন্দ্রবিশিষ্ট ও $ x- $অক্ষকে স্পর্শ করে এমন বৃত্তের সমীকরণ -

\begin{oneparchoices}
\choice  $ x^{2}+y^{2}-4x-8y+16=0 $
\choice  $ x^{2}+y^{2}-4x-8y+4=0 $
\choice  $ x^{2}+y^{2}-8x+4y+16=0 $
\choice  $ x^{2}+y^{2}-8x-4y+4=0 $
\end{oneparchoices}

\question $ i^{2}=-1 $ হলে $ \dfrac{i-i^{-1}}{i+2i^{-1}} $ এর মান

\begin{oneparchoices}
\choice $ 0 $
\choice $ -2i $
\choice $ 2i $
\choice  $ -2 $

\end{oneparchoices}

\question  $ \mathlarger{\int_{0}^{1}\dfrac{\sin^{-1}x}{\sqrt{1-x^{2}}}dx} $এর মান

\begin{oneparchoices}
\choice $ \dfrac{\pi}{2} $
\choice $ \dfrac{\pi^{2}}{8} $
\choice $ \dfrac{\pi^{2}}{4} $
\choice $ \dfrac{\pi^{2}}{16} $

\end{oneparchoices}

\question  $ \dfrac{(x+4)^{2}}{100} +\dfrac{(y-2)^{2}}{64} =1 $ হলে $ e=? $

\begin{oneparchoices}
\choice 1
\choice $ \dfrac{3}{5} $
\choice $ \dfrac{5}{3} $
\choice $ \dfrac{4}{5} $

\end{oneparchoices}

\question দশমিক সংখ্যা 181 কে দ্বিমিক পদ্ধতিতে প্রকাশ করলে হয় - 

\begin{oneparchoices}
\choice 10110101
\choice 10011011
\choice 11001010
\choice 10111011
\end{oneparchoices}

\question $ \cos\theta +\sqrt{3}\sin\theta =2 $ সমীকরণের সাধারণ সমাধান -

\begin{oneparchoices}
\choice $ \theta = 2n\pi -\dfrac{\pi}{3} $
\choice $ \theta = 2n\pi +\dfrac{\pi}{3} $
\choice $ \theta = 2n\pi +\dfrac{\pi}{6} $
\choice $ \theta = 2n\pi -\dfrac{\pi}{6} $
\end{oneparchoices}

\question যে সমীকরণের মূলগুলো $ x^{2}-5x-1=0 $ সমীকরণের মূল গুলো হতে 2 ছোট তা – 

\begin{oneparchoices}
\choice $ x^{2}+x+7=0 $
\choice $ x^{2}-x+7=0 $
\choice $ x^{2}-x-7=0 $
\choice $ x^{2}+x-7=0 $
\end{oneparchoices}


\question বাস্তব সংখ্যায় $ |3-2x|\le 1 $ অসমতাটির সমাধান-

\begin{oneparchoices}
\choice $ 1<x<2 $
\choice $ 1\le x \le 2 $
\choice $ x\le 1$ or $ x\ge 2 $
\choice $ 1<x\le 2 $
\end{oneparchoices}

\question $ \dfrac{\sin 75^{\circ}+\sin 15^{\circ}}{\sin 75^{\circ}-\sin 15^{\circ}} $ এর মান

\begin{oneparchoices}
\choice $ \sqrt{5} $
\choice $ \sqrt{3} $
\choice $ -\sqrt{5} $
\choice $ -\sqrt{3} $
\end{oneparchoices}

\question $ \mathlarger{\int \dfrac{e^{x}(1+x)}{\cos^{2}(xe^{x})}} $ সমান

\begin{oneparchoices}
\choice $ \sin (xe^{x}) +c $
\choice $ \cot (xe^{x}) +c $
\choice $ \tan (xe^{x}) +c $
\choice $ \cos (xe^{x}) +c$
\end{oneparchoices}

\question  $ x^{2}-2x+5=0 $ এর নূন্যতম মান – 

\begin{oneparchoices}
\choice  1
\choice  2
\choice  3
\choice  4
\end{oneparchoices}

\question  $ 1-\dfrac{1}{2}+\dfrac{1}{2^{2}}-\dfrac{1}{2^{3}}+\dfrac{1}{2^{4}}-\dfrac{1}{2^{5}}+\cdots \infty = $ 

\begin{oneparchoices}
\choice $ \dfrac{2}{3} $
\choice $ \dfrac{4}{3} $
\choice $ 2 $
\choice $ \dfrac{1}{3} $
\end{oneparchoices}

\question $ x^{2}-x+4y-4=0 $ পরাবৃত্তের শীর্ষবিন্দুর স্থানাংক -

\begin{oneparchoices}
\choice $ (-4,2) $
\choice $ (4,-2) $
\choice $ (4,5) $
\choice  $ (5,4) $
\end{oneparchoices}

\question স্রোত না থাকলে একটি ছেলে 5 মিনিটে সাতার কেটে সোজাসুজিভাবে 80 মিটার প্রশস্ত একটি খাল পার হতে পারে এবং স্রোত থাকলে দ্বিগুন সময় লাগে। স্রোতের বেগ -

\begin{oneparchoices}
\choice $ 15\,m/min $
\choice $ 16.5\,m/min $
\choice $ 12\,m/min $
\choice $ 13.86\,m/min $

\end{oneparchoices}

\question  $ \Big(2x^{2}+\dfrac{k}{x^{3}}\Big)^{10} $ এর বিস্তৃতিতে $ x^{5} $ এবং $ x^{15} $ এর সহগদ্বয় সমান হলে $ k $ এর ধনাত্মক মান -

\begin{oneparchoices}
\choice $ \dfrac{1}{\sqrt{2}} $
\choice  $ \dfrac{1}{\sqrt{3}} $
\choice $ \dfrac{1}{2} $
\choice $ \dfrac{1}{\sqrt{5}} $

\end{oneparchoices}


\question প্রতিবার প্রথমে ও শেষে U রেখে CALCULUS শব্দটির অক্ষরগুলোকে কতভাবে সাজানো যাবে?

\begin{oneparchoices}
\choice  90
\choice  180
\choice  280
\choice  360
\end{oneparchoices}

\question  $ y^{2}=16x $ ও $ y=4x $ দ্বারা আবদ্ধ ক্ষেত্রের ক্ষেত্রের ক্ষেত্রফল – 

\begin{oneparchoices}
\choice $ \dfrac{3}{2}\, sq.\, units $
\choice $ \dfrac{3}{4}\, sq.\, units $
\choice $ \dfrac{4}{3}\, sq.\, units $
\choice $ \dfrac{2}{3}\, sq.\, units $
\end{oneparchoices}

\question যদি $ \vec{AB} = 2\hat{i}+\hat{j} $ এবং $ \vec{AC} =3\hat{i}-\hat{j}+5\hat{k} $ হয় তবে $ \vec{AB} $ ও $ \vec{AC} $ কে সন্নিহিত বাহু ধরে অংকিত সামান্তরিকের ক্ষেত্রফল – 

\begin{oneparchoices}
\choice $ 2\sqrt{6} $
\choice $ 3\sqrt{6} $
\choice $ 4\sqrt{6} $
\choice  $ 5\sqrt{6} $
\end{oneparchoices}

\question $ \sqrt{3} $ এককের দুটি সমান বল $ 120^{\circ} $ কোণে একবিন্দুতে কাজ করে। তাদের লদ্ধির মান – 

\begin{oneparchoices}
\choice $ \sqrt{3}\, units $
\choice $ 4\sqrt{3}\, units $
\choice $ 3\, units $
\choice $2 \sqrt{3}\, units $
\end{oneparchoices}

\question   যদি $ y =\dfrac{\tan x-\cot x}{\tan x +\cot x} $ হয় তবে $ \dfrac{dy}{dx} $ সমান-

\begin{oneparchoices}
\choice $ 2\sin 2x $
\choice $ 2\cos 2x $
\choice $ 2\tan 2x $
\choice $ 2\cot 2x $

\end{oneparchoices}

\question একটি নিটল মুদ্রা ও একটি নিটল ছক্কা একত্রে  নিক্ষেপ করা হলো । একই সাথে হেড ও জোড় পাবার সম্ভাবনা কত?

\begin{oneparchoices}
\choice $ \dfrac{1}{2} $
\choice $ \dfrac{1}{3} $
\choice $ \dfrac{1}{4} $
\choice $ \dfrac{1}{5} $
\end{oneparchoices}

\question $ (2,-1),\,(a+1,a+3),\,(a+2, a) $  বিন্দুতিনটি সমরেখ হলে $ a $ এর মান -

\begin{oneparchoices}
\choice $ 4 $
\choice $ 2 $
\choice $ \dfrac{1}{4} $
\choice $ \dfrac{1}{2}$

\end{oneparchoices}

\question $ f(x) = x^{3}+3 $ এবং $ g(x) = \sqrt[3]{\dfrac{x-2}{3}} $ হলে $ (fog)(3) $এর মান


\begin{oneparchoices}
\choice 1
\choice 2
\choice 3
\choice 4

\end{oneparchoices}

\question  $ \cos\tan^{-1}\cot\sin^{-1}x $ সমান

\begin{oneparchoices}
\choice $ x $
\choice $ \dfrac{\pi}{2}-x $
\choice $ -x $
\choice $ x-\dfrac{\pi}{2} $
\end{oneparchoices}

\end{questions}
\begin{LARGE}
\begin{center}
গণিত (Mathematics - 2009)
\end{center}
\end{LARGE}
\begin{questions}

\question $ x^{2}-7x+12=0 $  সমীকরনের মূলদ্বয় $       \alpha $ এবং $ \beta $ হলে $ \alpha
+\beta $ এবং $ \alpha\beta $ মূলবিশষ্ট সমীকরণ –

\begin{oneparchoices}
\choice $ x^{2}-19x+84=0 $
\choice $ x^{2}+14x-144=0 $
\choice $ x^{2}+19x-84=0 $
\choice $ x^{2}-14x+144=0 $
\end{oneparchoices}

\question $ \omega $ যদি 1 এর একটি জটিল ঘনমূল হয়, তবে প্রদত্ত নির্ণায়কটির মান – $ \begin{vmatrix}
1 & \omega & \omega^{2} \\
\omega^{2} & \omega & 1\\
\omega^{2} & 1 & \omega
\end{vmatrix} $  

\begin{oneparchoices}
\choice $ 0 $
\choice $ 1 $
\choice $ \omega $
\choice $ \omega^{2} $
\end{oneparchoices}

\question  $\begin{pmatrix}
p+4 & 8 \\
2 & p-2
\end{pmatrix} $ ম্যাট্রিক্সটি ব্যতিক্রমী হলে $ p $ এর মান

\begin{oneparchoices}
\choice  -6, 4
\choice  -4,6
\choice -4, 2
\choice  -2, 4
\end{oneparchoices}

\question 6 জন ছাত্র 5 জন ছাত্রী থেকে একটি কমিটি গঠন করতে হবে যাতে অন্তত: একজন ছাত্র এবং একজন ছাত্রী অর্ন্তভূক্ত থাকে। কত প্রকারে এ কমিটি গঠন করা যেতে পারে?

\begin{oneparchoices}
\choice 160
\choice 360
\choice 410
\choice 455
\end{oneparchoices}

\question $ \Big(\dfrac{2}{3} x^{2}-\dfrac{1}{3}x\Big)^{9} $ এর বিস্তৃতিতে বর্জিত পদটি কত?

\begin{oneparchoices}
\choice $ \dfrac{224}{3^{8}} $
\choice $ -\dfrac{224}{3^{8}} $
\choice $ \dfrac{242}{3^{8}} $
\choice $ -\dfrac{242}{3^{8}} $

\end{oneparchoices}

\question  $ n- $তম পদ পর্যন্ত $ 1.2.3+2.3.4+3.4.5+\cdots $ ধারাটির যোগফল-

\begin{oneparchoices}
\choice $ n(n+1)(n+2)(n+3) $
\choice $ (n+1)(n+2)(n+3)(n+4) $
\choice $ \dfrac{1}{2} n(n+1)(n+2)(n+3) $
\choice $ \dfrac{1}{4} n(n+1)(n+2)(n+3) $
\end{oneparchoices}

\question  A, B, C বিন্দুগুলির স্থানাংক $ (a, bc),\,(b,ca),\,(c,ab) $ হলে $ \triangle ABC $ এর ক্ষেফল কত?

\begin{oneparchoices}
\choice $ \dfrac{1}{2}abc $
\choice $ \dfrac{1}{2}(a-b)(b-c)(c-a) $
\choice $ \dfrac{1}{2}(b-a)(b-c)(c-a) $
\choice $ 3abc $
\end{oneparchoices}

 \question  $ 2x-3y+=0 $ সরলরেখার উপর লম্ব এবং $ (1,-1) $ বিন্দুগামী সরলরেখার সমীকরণ -

\begin{oneparchoices}
\choice $ 3x+2y=1 $
\choice $ 3x-2y=5 $
\choice $ 3x+2y=5 $
\choice $ 2x+3y = 1 $
\end{oneparchoices}

\question  $ (2,3) $ কেন্দ্রবিশিষ্ট ও $ x+y-2=0 $ রেখাকে স্পর্শ করে এমন বৃত্তের সমীকরণ -

\begin{oneparchoices}
\choice  $ 2(x^{2}+y^{2})-8x-12y+17=0 $
\choice  $ 2(x^{2}+y^{2})-6x-10y+15=0 $\\
\choice  $ 2(x^{2}+y^{2})-4x-8y+11=0 $
\choice  $ 2(x^{2}+y^{2})-2x-6y+7=0 $
\end{oneparchoices}

\question $ y^{2}=4x+8y $ পরাবৃত্তের শীর্ষবিন্দুর স্থানাংক -

\begin{oneparchoices}
\choice $ (4,4) $
\choice $ (-4,-4) $
\choice $ (4,-4) $
\choice  $ (-4,4) $
\end{oneparchoices}

\question $ \vec{B} = 6\hat{i}-3\hat{j}+2\hat{k} $ ভেক্টরের  $ \vec{A} =2\hat{i}+2\hat{j}+\hat{k} $ উপর ভেক্টরের অভিক্ষেপ –

\begin{oneparchoices}
\choice $ \dfrac{8}{7} $
\choice $ \dfrac{7}{8} $
\choice $ \dfrac{8}{5} $
\choice  $\dfrac{5}{8} $
\end{oneparchoices}

\question $ \cos 198^{\circ} +\sin 432^{\circ} + \tan 168^{\circ} + \tan 12^{\circ}$ এর মান

\begin{oneparchoices}
\choice $ 0 $
\choice $ -1 $
\choice $ 1 $
\choice $ \dfrac{1}{2} $
\end{oneparchoices}

\question $ 4(\cos\theta +\sin^{2}\theta) =5 $ সমীকরণের সাধারণ সমাধান -

\begin{oneparchoices}
\choice $ \theta = 2n\pi \pm \dfrac{\pi}{2} $
\choice $ \theta = 2n\pi \pm \dfrac{\pi}{3} $
\choice $ \theta = 2n\pi \pm \dfrac{\pi}{4} $
\choice $ \theta = 2n\pi \pm \dfrac{\pi}{5} $
\end{oneparchoices}

\question $ i^{2}=-1 $ হলে $ \dfrac{i^{-1}-i}{i+2i^{-1}} $ এর মান

\begin{oneparchoices}
\choice $ 2 $
\choice $ -2i $
\choice $ 2i $
\choice  $ -2 $
\end{oneparchoices}

\question $ \cos\theta = \dfrac{12}{13} $ হলে $ \theta $ এর মান

\begin{oneparchoices}
\choice $ \pm \dfrac{5}{12} $
\choice $  \dfrac{25}{144} $
\choice $ \dfrac{13}{12} $
\choice  $ \pm \dfrac{13}{12} $
\end{oneparchoices}

\question   বাস্তব সংখ্যায় $ \dfrac{1}{|2x-3|}> 5 $ অসমতাটির সমাধান হলো-

\begin{oneparchoices}
\choice $ \Big(\dfrac{7}{5}, \dfrac{8}{5}\Big)$
\choice $ \Big[\dfrac{7}{5}, \dfrac{8}{5}\Big]$
\choice $ \Big(\dfrac{7}{5}, \dfrac{3}{5}\Big)\cup \Big(\dfrac{3}{2}, \dfrac{8}{5}\Big)  $
\choice $ \Big[\dfrac{7}{5}, \dfrac{3}{2}\Big]\cup \Big[\dfrac{3}{2}, \dfrac{8}{5}\Big]  $
\end{oneparchoices}

\question $ f(x) = \sin x,\, g(x)=x^{2} $ হলে $ f\Big(g\Big(\dfrac{\sqrt{\pi}}{2}\Big) \Big) $ এর মান - 

\begin{oneparchoices}
\choice $ \dfrac{\sqrt{2}}{2} $
\choice $  \dfrac{\sqrt{3}}{2} $
\choice $ \dfrac{1}{2} $
\choice  $ 1 $
\end{oneparchoices}

\question  $ \mathlarger{\lim_{x\to 0} \dfrac{\sin x^{2}}{x}}=? $ 

\begin{oneparchoices}
\choice 1
\choice $ -1 $
\choice 0
\choice $ 2 $
\end{oneparchoices}

\question $ x^{2}+xy+ y^{2}=0  $ হলে $ (3,-4) $ বিন্দুতে $ \dfrac{dy}{dx} $ এর মান - 

\begin{oneparchoices}
\choice $ \dfrac{2}{5} $
\choice $ \dfrac{5}{2} $
\choice $ \dfrac{3}{8} $
\choice $ \dfrac{8}{3} $
\end{oneparchoices}

\question যদি $ y=\ln (x+\sqrt{x^{2}+4})  $ হয় তবে   $\dfrac{dy}{dx} $ সমান - 

\begin{oneparchoices}
\choice $ \sqrt{x^{2}+4} $
\choice $ \dfrac{1}{1+\sqrt{x^{2}+4}} $
\choice $ 1+\sqrt{x^{2}+4} $
\choice $ \dfrac{1}{\sqrt{x^{2}+4}} $
\end{oneparchoices}

\question $ \mathlarger{\int \dfrac{dx}{e^{x}+e^{-x}}=?} $

\begin{oneparchoices}
\choice $ \tan^{-1} (e^{x}) +c $
\choice $ \tan (e^{x}+e^{-x}) +c $
\choice $ \tan (e^{x}) +c $
\choice $ \tan (e^{-x}) +c$
\end{oneparchoices}

\question  $ \mathlarger{\int_{1}^{e}\ln x\, dx} $ এর  মান – 

\begin{oneparchoices}
\choice  $ e $
\choice  $ e-1 $
\choice  1
\choice  $ 1-e $
\end{oneparchoices}

\question  $ \mathlarger{\int\dfrac{1}{\cos^{2}x\sqrt{\tan x}} dx} =?$  

\begin{oneparchoices}
\choice  $ \sqrt{\tan x}\ln (\cos^{2}x)+c $
\choice  $ 2\sqrt{\tan x} +c $
\choice  $ 2\sqrt{\tan x +c}  $
\choice  $ \dfrac{2}{3}(\tan x)^{\frac{3}{2}} +c $
\end{oneparchoices}

\question  $ \mathlarger{\int_{0}^{1}\dfrac{\cos^{-1}x}{\sqrt{1-x^{2}}}dx} $এর মান

\begin{oneparchoices}
\choice $ \dfrac{\pi^{2}}{2} $
\choice $ \dfrac{\pi^{2}}{8} $
\choice $ \dfrac{\pi^{2}}{4} $
\choice $ \dfrac{\pi^{2}}{16} $

\end{oneparchoices}

\question  $ u $ বেগে আনুভুমিকের সাথে $ \alpha $ কোণে প্রক্ষিপ্ত বস্তুর সর্বোচ্চ উচ্চতা হবে-

\begin{oneparchoices}
\choice $ \dfrac{u^{2}\sin 2\alpha}{2g} $
\choice $ \dfrac{u^{2}\sin^{2}\alpha}{2g} $
\choice $ \dfrac{u^{2}\sin 2\alpha}{g} $
\choice $ \dfrac{u^{2}\sin^{2}\alpha}{g} $

\end{oneparchoices}

\question  একটি বুলেট কোন দেয়ালের মধ্যে 2 ইঞ্চি ঢুকার পর এর বেগ অর্ধেক হারায়।বুলেটি দেয়ালে আর কতদুর ঢুকবে?

\begin{oneparchoices}
\choice $ 2^{"} $
\choice $ (\dfrac{2}{3})^{"} $
\choice $ 1^{"} $
\choice $ (\dfrac{1}{2})^{"} $

\end{oneparchoices}

\question  3P এবং 2P বলদ্বয়ের লদ্ধি R। প্রথম বলটিদ্বিগুণ করা হলে লদ্ধিও দ্বিগুণ হয়। বলদ্বয়ের অন্তর্গত কোণ হবে-

\begin{oneparchoices}
\choice $ 110^{\circ} $
\choice $ 150^{\circ} $
\choice $ 120^{\circ} $
\choice $ 135^{\circ} $

\end{oneparchoices}

\question দশমিক সংখ্যা 214 কে দ্বিমিক পদ্ধতিতে প্রকাশ করলে হয় - 

\begin{oneparchoices}
\choice 11010110
\choice 10100110
\choice 11001010
\choice 10111011
\end{oneparchoices}

\question নিন্মের লিনিয়ার প্রোগ্রামিং সমস্যার সমাধান কর:
গরিষ্ঠকরণ কর: $ z=3x+2y $\\
শর্ত হচ্ছে-  $ x+y\le 7,\, 2x+5y\le 20,\, x\ge 0,\,y\ge 0 $


\begin{oneparchoices}
\choice $ (5,2) $
\choice $ (7,0) $
\choice $ (10,0) $
\choice $ (0,7) $
\end{oneparchoices}

\question  40 থেকে 50 সংখ্যাগুলির মধ্যে দৈবচয়ন পদ্ধতিতে একটি সংখ্যা নেয়া হল। সংখ্যাটি মৌলিক না হওয়ার সম্ভাবনা-

\begin{oneparchoices}
\choice $ \dfrac{8}{11} $
\choice  $ \dfrac{5}{11} $
\choice $ \dfrac{3}{11} $
\choice $ \dfrac{1}{11} $

\end{oneparchoices}

\end{questions}

\begin{LARGE}
\begin{center}
গণিত (Mathematics - 2008)
\end{center}
\end{LARGE}
\begin{questions}

\question  $ 5x_{1}+10x_{2}\le 50,\, x_{1}+x_{2}\ge 1,\, x_{1}\ge 0,\,x_{2}\le 4,\, x_{2}\ge 0 $ শর্তাবলী সাপেক্ষে $ 2x_{1}+7x_{2} $ এর লঘিষ্ঠ মান – 

\begin{oneparchoices}
\choice $ 2 $
\choice $ 7 $
\choice $ 20 $
\choice $ 1 $
\end{oneparchoices}

\question $ x^{2}-5x+c=0 $ সমীকরনের একটি মূল 4 হলে অপর মূলটি -

\begin{oneparchoices}
\choice $ -5 $
\choice $ -4 $
\choice $ 4 $
\choice $ 1 $
\end{oneparchoices}

\question $ \log_{r}p =q $ এবং $ \log_{q}r = p$ হলে $ \log_{q}p $ এর মান

\begin{oneparchoices}
\choice $ \dfrac{p}{q} $
\choice $ \dfrac{q}{p} $
\choice $ pq $
\choice  $ p^{q} $
\end{oneparchoices}

\question $ \tan\theta = \dfrac{5}{12} $ হলে এবং $ \theta $ সূক্ষকোণ হলে $ \sin\theta + \sec (-\theta) $ এর মান

\begin{oneparchoices}
\choice $  \dfrac{21}{156} $
\choice $  \dfrac{229}{156} $
\choice $ \dfrac{219}{156} $
\choice  $ \dfrac{17}{13} $
\end{oneparchoices}

\question $ \cos^{2} 0^{\circ} +\cos^{2} 10^{\circ} + \cos^{2} 20^{\circ} +\cdots + \cos 90^{\circ}$ এর মান

\begin{oneparchoices}
\choice $ 6 $
\choice $ 3 $
\choice $ 5 $
\choice $ 4 $
\end{oneparchoices}

\question $ y=\dfrac{x+1}{x}  $ হলে - 

\begin{oneparchoices}
\choice $ x^{2}\dfrac{d^{2}y}{dx^{2}} =1 $
\choice $ x^{4}\dfrac{d^{2}y}{dx^{2}} =3x+2 $
\choice $ x^{3}\dfrac{d^{2}y}{dx^{2}} =2 $
\choice $ \dfrac{dy}{dx} = \dfrac{2x+1}{x^{2}}$
\end{oneparchoices}

\question $ y=x^{3}-12x+16 $ বক্ররেখার যে সমস্ত বিন্দুতে স্পর্শক $ x- $অক্ষের সমান্তরাল তাদের স্থানাংক -

\begin{oneparchoices}
\choice $ (2,0),\,(-2,24) $
\choice $ (2,0),\,(-2,0)$
\choice $ (4,12),\,(-4,12) $
\choice $ (2,0),\,(-2,32) $
\end{oneparchoices}

\question নিন্মের কোন সমীকরণ দ্বারা নির্দেশিত বৃত্তের স্পর্শক $ x-$ অক্ষ ?

\begin{oneparchoices}
\choice $ x^{2}+y^{2}-10x-6y+9=0 $
\choice $ x^{2}+y^{2}+10x+6y+25=0 $
\choice $ x^{2}+y^{2}+6x+10y+25=0 $
\choice $ x^{2}+y^{2}+6x+8y+25=0 $
\end{oneparchoices}

\question $ 3\hat{i}+2\hat{j}+\lambda\hat{k} $ এবং $ 4\hat{i}-3\hat{j}+\hat{k} $ ভেক্টরদ্বয় পরস্পর লম্ব হলে এর মান-

\begin{oneparchoices}
\choice 6
\choice $ -6 $
\choice $ 12 $
\choice $ 1 $
\end{oneparchoices}

\question কোন বিন্দুতে $ 60^{\circ} $ কোণে ক্রিয়ারত দুটি সমান বলকে একই বিন্দুতে ক্রিয়ারত 9N বলের সাহায্যে ভারসাম্য রাখলে সমান বলদ্বয়ের প্রতিটির মান – 

\begin{oneparchoices}
\choice $ 3\sqrt{3}\,N $
\choice $ \sqrt{3}\,N $
\choice $ 3\,N $
\choice $ 9\,N $
\end{oneparchoices}

\question দ্বিমিক সংখ্যা 100110100111 এর দশমিকে প্রকাশ -

\begin{oneparchoices}
\choice 2471
\choice 3673
\choice 2472
\choice 3674
\end{oneparchoices}

\question $ 5x-x^{2}-6>0 $ হলে 

\begin{oneparchoices}
\choice $ x<2 $
\choice $ 2>x>3 $
\choice $ 2<x<3 $
\choice $ x>3,\,x<2 $
\end{oneparchoices}

\question   $ \begin{vmatrix}
x+y & x & y \\
x & x+z & z\\
y & z & y+z
\end{vmatrix} $  নির্ণায়কটির মান –

\begin{oneparchoices}
\choice $ 4xyz $
\choice $ 3xyz $
\choice $ 2xyz $
\choice $ xyz $
\end{oneparchoices}

\question $ 2\cos\theta =1 $ সমীকরণের সাধারণ সমাধান -

\begin{oneparchoices}
\choice $ \theta = n\pi + \dfrac{\pi}{3} $
\choice $ \theta = 2n\pi \pm \dfrac{\pi}{6} $
\choice $ \theta = 2n\pi + \dfrac{\pi}{3} $
\choice $ \theta = 2n\pi \pm \dfrac{\pi}{3} $
\end{oneparchoices}

\question $ \cot\Big(\sin^{-1}\dfrac{1}{2}\Big) $ এর মান

\begin{oneparchoices}
\choice $ \dfrac{2}{\sqrt{3}} $
\choice $ \dfrac{1}{\sqrt{3}} $
\choice $ \sqrt{3} $
\choice $ \dfrac{\sqrt{3}}{2} $
\end{oneparchoices}

\question $ \dfrac{d}{dx}\Big( \log_{x}e \Big) = $ 

\begin{oneparchoices}
\choice $ \dfrac{\log_{x}e}{x} $
\choice $ \dfrac{1}{x\ln x} $
\choice $ -\dfrac{\ln x}{x} $
\choice $ -\dfrac{1}{x(\ln x)^{2}} $
\end{oneparchoices}

\question নিচের কোনটি $ \sin x\cos x $  এর অনির্দিষ্ট যোগজ নয় ?

\begin{oneparchoices}
\choice $ \dfrac{1}{4}\cos 2x $
\choice  $- \dfrac{1}{4}\cos 2x $
\choice  $ \dfrac{1}{2}\sin^{2} x $
\choice $ -\dfrac{1}{2}\cos^{2} x $
\end{oneparchoices}

\question  $ k $ এর কোন মানের জন্য $ (x-y+3)^{2} + (kx+2)(y-1) = 0 $ সমীকরণটি একটি বৃত্ত নির্দেশ করে?

\begin{oneparchoices}
\choice $ 1 $
\choice $ -1 $
\choice $ 2 $
\choice $ -2 $
\end{oneparchoices}

\question  যে পরাবৃত্তের উপকেন্দ্রের স্থানাংক $ (4,0) $ এবং নিয়ামক $ x+2=0 $, তার সমীকরণ 

\begin{oneparchoices}
\choice $ y^{2}=4(x-1) $
\choice $ y^{2}=6(x-2) $
\choice $ y^{2}=10(x-3) $
\choice $ y^{2}=12(x-1) $
\end{oneparchoices}

\question  একটি বিন্দুতে ক্রিয়াশীল P N এবং 12N মানের দুটি বলের লদ্ধি $ 3\sqrt{6}N $, যার ক্রিয়ারেখা P এর দিকে $ 90^{\circ} $ কোণ উৎপন্ন করে। P এর মান 

\begin{oneparchoices}
\choice 11 N
\choice 9 N
\choice 13 N
\choice $ 2\sqrt{7} $ N
\end{oneparchoices}

\question  2 থেকে 40 সংখ্যাগুলির মধ্যে দৈবচয়ন পদ্ধতিতে একটি সংখ্যা নেয়া হল। সংখ্যাটি মৌলিক হওয়ার সম্ভাবনা-

\begin{oneparchoices}
\choice $ \dfrac{11}{39} $
\choice  $ \dfrac{4}{13} $
\choice $ \dfrac{1}{3} $
\choice $ \dfrac{11}{38} $
\end{oneparchoices}

\question  $ \dfrac{x+17}{(x-3)(x+2)}=\dfrac{a}{x-3}+\dfrac{b}{x+2} $ হলে

\begin{oneparchoices}
\choice $ a=2,\,b=-5 $
\choice $ a=,\,b=-3 $
\choice $ a=-3,\,b=4 $
\choice $ a=4,\,b=-2 $
\end{oneparchoices}

\question  $A = \begin{bmatrix}
7 & 6 \\
8 & 7
\end{bmatrix} $ হলে $ A^{-1} $ 

\begin{oneparchoices}
\choice  $\begin{bmatrix}
-7 & 6 \\
8 & -7
\end{bmatrix} $
\choice  $\begin{bmatrix}
7 & -8 \\
-6 & 7
\end{bmatrix} $
\choice $\begin{bmatrix}
7 & -6 \\
-8 & 7
\end{bmatrix} $
\choice  $\begin{bmatrix}
-7 & 8 \\
6 & -7
\end{bmatrix} $
\end{oneparchoices}

 \question  $ 5+3x-x^{2} $ এর সর্বোচ্চ মান -

\begin{oneparchoices}
\choice $ 3 $
\choice $ \dfrac{11}{4} $
\choice $ \dfrac{29}{4} $
\choice $ \dfrac{27}{4} $
\end{oneparchoices}

\question  $ \cot A -\tan A $ সমান 

\begin{oneparchoices}
\choice  $ 2\tan 2A $
\choice  $ 2\cot 2A $
\choice  $ 2\cos^{2} A $
\choice  $ 2\sin^{2}A$
\end{oneparchoices}

\question  $ \mathlarger{\lim_{x\to 0} \dfrac{\sin (2x)^{2}}{x}}=? $ 

\begin{oneparchoices}
\choice 1
\choice $ \dfrac{1}{2} $
\choice 0
\choice $ 2 $
\end{oneparchoices}

\question  $ \mathlarger{\int_{0}^{1}\dfrac{dx}{\sqrt{2x-x^{2}}}} = $ 

\begin{oneparchoices}
\choice $ \dfrac{\pi}{2} $
\choice $ \dfrac{\pi}{4} $
\choice $ -\dfrac{\pi}{4} $
\choice $ -\dfrac{\pi}{2} $
\end{oneparchoices}

\question $ y^{2}=4x $ পরাবৃত্ত এবং $ y = x $ সরলরেখাদ্বারা বেষ্টিত এলাকার ক্ষেত্রফল বর্গ এককে-

\begin{oneparchoices}
\choice $ \dfrac{8}{3} $
\choice $ \dfrac{5}{3} $
\choice $ \dfrac{4}{3} $
\choice  $ \dfrac{2}{3} $
\end{oneparchoices}

\question $ y=3x+7 $ এবং $ 3y-x=8 $ সরলরেখাদ্বয়ের অর্ন্তভূক্ত সূক্ষকোণ-

\begin{oneparchoices}
\choice $ \tan^{-1}(1) $
\choice $ \tan^{-1}\Big(\dfrac{7}{8}\Big) $
\choice $ \tan^{-1}\Big(\dfrac{4}{3}\Big) $
\choice  $ \tan^{-1}\Big(\dfrac{3}{4}\Big) $
\end{oneparchoices}

\question $ f(x) = \dfrac{x-3}{2x+1} $ এবং $ x\neq \dfrac{1}{2}$ হলে $ f^{-1}(-2) $ এর মান 

\begin{oneparchoices}
\choice $ \dfrac{5}{3} $
\choice $ -\dfrac{5}{3} $
\choice $ \dfrac{1}{5} $
\choice  $ \dfrac{2}{5} $
\end{oneparchoices}

\end{questions}

\begin{LARGE}
\begin{center}
গণিত (Mathematics - 2007)
\end{center}
\end{LARGE}
\begin{questions}

\question  $ x^{2}-5x-1=0 $ সমীকরনের মূলদ্বয় হতে 2 কম মূলবিশষ্ট সমীকরণ –

\begin{oneparchoices}
\choice $ x^{2}+x+7=0 $
\choice $ x^{2}-x+7=0 $
\choice $ x^{2}+x-7=0 $
\choice $ x^{2}-x-7=0 $
\end{oneparchoices}

\question $ \begin{pmatrix}
p-4 & 8\\
2 & p+2
\end{pmatrix} $ ম্যাট্রিক্সটি ব্যতীক্রমী হবে যদি p এর মান 

\begin{oneparchoices}
\choice $ -4, 6 $
\choice $-6, 4$
\choice $ 4, 6$
\choice $-6, -4$
\end{oneparchoices}

\question $ \Bigg(2x + \dfrac{1}{6x} \Bigg)^{10} $ এর সম্প্রসারণে বর্জিত পদ হল – 

\begin{oneparchoices}
\choice $ \dfrac{28}{27} $
\choice $ \dfrac{27}{28} $
\choice $ 1 $
\choice  $ 3 $
\end{oneparchoices}

\question $ i^{2}=-1 $ হলে $ \dfrac{i-i^{-1}}{i+2i} $ এর মান –


\begin{oneparchoices}
\choice $ 0 $
\choice $ -2i $
\choice $ 2i $
\choice  $ -2 $
\end{oneparchoices}

\question $ \begin{vmatrix}
\alpha -3 & -1 \\
-8 & \alpha +4
\end{vmatrix} $ নির্ণায়কটির মান শূণ্য হলে এর মান – 

\begin{oneparchoices}
\choice $ 4$ or $-5 $
\choice $ 5$ or $-4 $
\choice $ 3 $
\choice $ 10 $
\end{oneparchoices}

\question SCHOOL শব্দটি হতে তিনটি অক্ষরদিয়ে পৃথকভাবে সাজানোর সংখ্যা – 

\begin{oneparchoices}
\choice 10
\choice 14
\choice 4
\choice 15
\end{oneparchoices}

\question $ y = 1 +\dfrac{1}{2+x} $ বক্ররেখা $ x-$অক্ষকে A বিন্দুতে $ y-$অক্ষকে B বিন্দুতে ছেদ করলে AB সরলরেখার সমীকরণ  হবে–

\begin{oneparchoices}
\choice $ x-2y+3=0 $
\choice $ x+2y+3=0 $
\choice $ 2x-y+3=0 $
\choice $ x-6y-3=0 $
\end{oneparchoices}

\question $ |2x-3|\le 1 $ বাস্তবসংখ্যায় অসমতাটির সমাধান – 

\begin{oneparchoices}
\choice $ 1<x<2 $
\choice $ 1\le x\le 2 $
\choice $ x\le 1 $ or $ x\ge 2 $
\choice $ x\le 2 $ or $x\ge 1$
\end{oneparchoices}

\question মূলবিন্দু হতে $ 3x+4y=10 $ সরলরেখাটির লম্বদুরত্ব - 

\begin{oneparchoices}
\choice 2
\choice 3
\choice 4
\choice  5
\end{oneparchoices}

\question $ 3x+7y-2=0 $ সরলরেখার উপর লম্ব এবং $ (2,-1) $ বিন্দুগামী সরলরেখার সমীকরণ – 
 
\begin{oneparchoices}
\choice $ 3x+7y-13=0 $
\choice $ 7x-3y-11=0 $
\choice $ 7x+3y-17=0 $
\choice $ 7x-3y-2 = 0 $
\end{oneparchoices}

\question সরলরেখা $ y= kx-1 $ বক্ররেখা $ y=x^{2}+3 $ এর স্পর্শক হবে যদি $ k $ এর একটি মান – 

\begin{oneparchoices}
\choice $ 1 $
\choice $ 2\sqrt{2}$
\choice $ 3 $
\choice $ 4 $
\end{oneparchoices}

\question $ (9,-9) $ও $ (-5,5) $ বিন্দুদ্বয়ের সংযোজক সরলরেখাকে ব্যাস ধরে অঙ্কিত বৃত্তের সমীকরণ –

\begin{oneparchoices}
\choice $ x^{2}+y^{2}-4x+4y+90=0 $
\choice $ x^{2}+y^{2}-4x+4y-90=0 $
\choice $ x^{2}+y^{2}+4x-4y-90=0 $
\choice $ x^{2}+y^{2}-4x-4y+90=0 $
\end{oneparchoices}

\question $ \cos 75^{\circ} $ এর সঠিক মান
এর মান – 

\begin{oneparchoices}
\choice $ \dfrac{\sqrt{3}+1}{2\sqrt{2}} $
\choice $ \dfrac{\sqrt{3}}{2\sqrt{2}} $
\choice $ -\dfrac{\sqrt{3}}{2\sqrt{2}} $
\choice $ \dfrac{\sqrt{3}-1}{2\sqrt{2}} $
\end{oneparchoices}

\question $ \tan^{-1}6+ \tan^{-1}\dfrac{7}{5} $  হলে এর মান – 

\begin{oneparchoices}
\choice $ \dfrac{\pi}{2} $
\choice $ \dfrac{3\pi}{2} $
\choice $ \dfrac{3\pi}{4} $
\choice $ \dfrac{\pi}{3} $
\end{oneparchoices}

\question $ 2\cos^{2}\theta + 2\sqrt{2}\sin\theta = 3 $ হলে $ \theta $ এর মান -  

\begin{oneparchoices}
\choice $ 30^{\circ} $
\choice $ 45^{\circ} $
\choice $ 60^{\circ} $
\choice  $ 135^{\circ} $
\end{oneparchoices}

\question $ \dfrac{(x+1)^{2}}{100} + \dfrac{(y-2)^{2}}{64} = 1 $ উপবৃত্তের উৎকেনন্দ্রিকতা – 

\begin{oneparchoices}
\choice $ 1 $
\choice $ \dfrac{3}{5} $
\choice $ \dfrac{5}{3} $
\choice $ \dfrac{4}{5} $
\end{oneparchoices}

\question $ x\to 0 $ হলে $ \dfrac{\tan^{-1}2x}{x} $ এর লিমিট কত?

\begin{oneparchoices}
\choice $ 1 $
\choice $ 0 $
\choice $ 2 $
\choice $ \dfrac{1}{2} $
\end{oneparchoices}

\question $ f(x) = x^{2}+4 $ এবং $ g(x) = 2x-1 $ হলে $ g(f(x)) $ এর মান -  

\begin{oneparchoices}
\choice $ 2x^{2}+7 $
\choice $ 7x^{2}+2 $
\choice $ x^{2}+2x-1 $
\choice $ x^{2}-2x+3 $
\end{oneparchoices}

\question যদি $ y= \ln (x+\sqrt{x^{2}+a^{2}}) $ হয় তবে $ \dfrac{dy}{dx} $ সমান

\begin{oneparchoices}
\choice $ \sqrt{x^{2}+a^{2}} $
\choice  $ \dfrac{1}{1+\sqrt{x^{2}+a^{2}}} $
\choice  $ 1+\sqrt{x^{2}+a^{2}} $
\choice $ \dfrac{1}{\sqrt{x^{2}+a^{2}}} $
\end{oneparchoices}

\question যদি $ x^{2}+3xy+5y^{2} = 1 $ হয় $ \dfrac{dy}{dx} $ সমান

\begin{oneparchoices}
\choice $ \dfrac{2x+3y}{3x+10y} $
\choice $ \dfrac{2x-3y}{3x+10y} $
\choice $ \dfrac{2x+3y}{3x-10y} $
\choice $ -\dfrac{2x+3y}{3x+10y} $
\end{oneparchoices}

\question $ \mathlarger{\int \dfrac{e^{x}(1+x)}{\cos^{2}(xe^{x})}} $ সমান

\begin{oneparchoices}
\choice $ \sin (xe^{x}) +c $
\choice $ \cot (xe^{x}) +c $
\choice $ \tan (xe^{x}) +c $
\choice $ \cos (xe^{x}) +c$
\end{oneparchoices}

\question  $ \mathlarger{\int}\dfrac{1}{\cos^{2}x\sqrt{\tan x}}dx $ সমান

\begin{oneparchoices}
\choice $ \sqrt{\tan x}\ln (\cos^{2} x) + c $
\choice $ \sin x\sqrt{\tan x} + c $
\choice $ 2\sqrt{\tan x} + c $
\choice $ \dfrac{2}{3}(\tan x)^{\frac{3}{2}} + c $
\end{oneparchoices}

\question $ \mathlarger{\int\limits_{0}^{1}}\dfrac{x}{\sqrt{1-x^{2}}}dx $ এর সমান

\begin{oneparchoices}
\choice $ \dfrac{1}{2} $
\choice $ \dfrac{\pi}{\sqrt{2}} $
\choice $ 1 $
\choice $ \dfrac{\pi}{2} $ 
\end{oneparchoices}

\question যদি $ \mathlarger{\int\limits_{0}^{4}}f(x)dx = 5$  হয় তবে $ \mathlarger{\int\limits_{1}^{5}}f(x-1)dx = 5$  এর মান

\begin{oneparchoices}
\choice $ 4 $
\choice $ 6 $
\choice $ 0 $
\choice $ 5 $
\end{oneparchoices}

\question  $ y^{2}=16x $ ও $ y =4x $ দ্বারা আবদ্ধ ক্ষেত্রের ক্ষেত্রফল – 

\begin{oneparchoices}
\choice $ \dfrac{3}{2} $ sq.units
\choice $ -\dfrac{3}{2} $ sq.units
\choice $ -\dfrac{2}{3} $ sq.units
\choice $ \dfrac{2}{3} $ sq.units
\end{oneparchoices}

\question  কোনবিন্দুতে ক্রিয়ারত দুটি বলের একটির মান অপরটির দ্বিগুন হলে এবং তাদের লদ্ধি ক্ষুদ্রতরটির উলম্ব হলে অন্তভুর্ক্ত কোণ হবে –


\begin{oneparchoices}
\choice $ 60^{\circ} $
\choice $ 120^{\circ} $
\choice $ 90^{\circ} $
\choice  $ 210^{\circ} $
\end{oneparchoices}

 \question  কোন স্তম্ভের শীর্ষ হতে $ 19.5\,ms^{-1} $ বেগে খাড়া উপরের দিকে কোন কণা 5 সেকেন্ড পরে স্তম্ভের পাদদেশে পতিত হলে স্তম্ভের উচ্চতা হবে –

\begin{oneparchoices}
\choice $ 20 $ m
\choice $ 25 $ m
\choice $ 30 $ m
\choice $ 50 $ m
\end{oneparchoices}

\question  দশমিক সংখ্যা 181 কে দ্বিমিকে প্রকাশ করলে হয় – 

\begin{oneparchoices}
\choice  $ 10220202 $
\choice  $ 10010111 $
\choice  $ 10101101 $
\choice  $ 11010011 $
\end{oneparchoices}

\question  30 থেকে 40 পর্যন্ত সংখ্যা হতে কোন একটিকে ইচ্ছামত নিলে সেই সংখ্যাটি মৌলিক হওয়ার সম্ভাবনা – 

\begin{oneparchoices}
\choice $ \dfrac{1}{2} $
\choice $ \dfrac{5}{11} $
\choice $ \dfrac{6}{11} $
\choice $ \dfrac{3}{5} $
\end{oneparchoices}

\question  একজন লোক তাঁর কাধে অনুভুমিকভাবে স্থাপিত 6 ফুট দীর্ঘ একটি লাঠির প্রান্তে হাত রেখে অপর প্রান্তে W ওজনের একটি বস্তু বহন করছে। কাধেঁর উপর চাপের পড়িমান বস্তুর ওজনের তিনগুন হলে কাঁধ হতে হাতের দুরত্ব -

\begin{oneparchoices}
\choice $ 3 $ feet
\choice $ 4 $ feet
\choice $ 2 $ feet
\choice $ 1 $ feet
\end{oneparchoices}

\end{questions}



\end{document}