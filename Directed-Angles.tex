\documentclass[11pt]{scrartcl}
\usepackage[mdthm,secthm]{evan}
%\renewcommand{\ol}[1]{\overleftrightarrow{#1}}

\begin{document}
\title{How to Use Directed Angles}
\date{May 31, 2015}
\maketitle



\begin{abstract}
	\textit{``WLOG, diagram as shown.''} -- Everyone
\end{abstract}

This is a very brief note on what a directed angle is and how to use it
to write olympiad solutions which are impervious to configuration issues.
It is short because I don't actually need to teach you how to
solve problems, but only to rewrite proofs of problems you've
already essentially solved.

\section{Introduction}
Everyone hates configuration issues (both contestants and graders).
Configuration issues come up because the most fundamental theorem in olympiad geometry actually has two cases.

\begin{proposition}
	[Cyclic Quadrilaterals]
	Let $A$, $B$, $X$, $Y$ be any four points, no three collinear.
	\begin{enumerate}[(i)]
		\ii If $A$ and $B$ lie on the same side of $\ol{XY}$,
		then the four points are concyclic
		if and only if $\angle XAY = \angle XBY$.
		\ii If $A$ and $B$ lie on different sides of $\ol{XY}$,
		then the four points are concyclic
		if and only if $\angle XAY + \angle XBY = 180\dg$.
	\end{enumerate}
	\label{prop:bad_cyclic_quad}
\end{proposition}

\begin{figure}
	\centering
	\begin{asy}
		size(4cm);
		draw(unitcircle);
		pair A = dir(110);
		pair X = dir(210);
		pair Y = dir(330);
		pair B = dir( 50);
		Drawing("A", A, A);
		Drawing("B", B, B);
		Drawing("X", X, X);
		Drawing("Y", Y, Y);
		draw(X--A--Y);
		draw(X--B--Y);
		draw(anglemark(X,A,Y));
		draw(anglemark(X,B,Y));
	\end{asy}
	\qquad
	\begin{asy}
		size(4cm);
		draw(unitcircle);
		pair A = dir(110);
		pair X = dir(210);
		pair Y = dir(330);
		pair B = dir(280);
		Drawing("A", A, A);
		Drawing("B", B, B);
		Drawing("X", X, X);
		Drawing("Y", Y, Y);
		draw(X--A--Y);
		draw(X--B--Y);
		draw(anglemark(X,A,Y));
		pair Z = 1.5*B-0.5*X;
		draw(anglemark(Z,B,Y));
		draw(B--Z);
	\end{asy}
	\caption{The two cases for cyclic quadrilaterals.}
	\label{fig:cyclic_quad}
\end{figure}

Doesn't that just look annoying?
Any time you want to invoke a cyclic quadrilateral, you have to actually check the points lie on the correct side of some line.

In fact, you even have to worry about configuration issues for something as simple as adding two angles.
\begin{ques}
	When does the assertion
	\[ \angle AOP + \angle POB = \angle AOB \]
	fail?
\end{ques}

Thus any time you want to add two angles, you technically need to also check that point $P$ lies ``inside'' $\angle AOB$.

Given all this disaster, you might wonder how we ever got any angle chasing done at all.
The secret is that \textbf{configuration issues are the object of widespread scorn}: they are glossed over, swept under a carpet, or ``left as an exercise''.
I've almost never seen them addressed seriously, except in the very rare circumstances in which they actually matter.

Let's fix this.

\section{Directed Angles}
In what follows I'm going to write $\measuredangle AOB$ for a directed angle
to distinguish it from a ``regular'' angle $\angle AOB$.
But I should warn that \textbf{this notation is absolutely not standard}.
Thus if you wish to use directed angles on an olympiad, you should explicitly say so in your solution.

Here's the very general definition.
\begin{definition}
	Given any two non-parallel lines $\ell$ and $m$,
	we define the directed angle
	\[ \dang(\ell, m) \]
	to be the measure of the angle
	\emph{starting} from $\ell$ and \emph{ending} at $m$,
	measured \emph{counterclockwise}.
\end{definition}

\begin{figure}[ht]
	\centering
	\begin{asy}
		size(5cm);
		pair L = dir(80);
		pair LL = -L;
		pair M = dir(130);
		pair MM = -M;
		draw(L--LL);
		draw(M--MM);
		MP("\ell", L, L);
		MP("m", M, M);
		draw(arc(origin,0.4*L,0.4*M), EndArrow);
		MP("50^{\circ}", 0.6*dir(105), origin);
	\end{asy}
	\caption{The directed angle $\dang(\ell, m) = 50\dg$.}
	\label{fig:dir_angle_first}
\end{figure}

Notice that 
\begin{equation}
	\dang(\ell, m) + \dang(m, \ell) = 180\dg
	\label{eq:directed_180_old}
\end{equation}
holds universally.
This is kind of nice, but it's a bit annoying to have that $180\dg$
lying around there, and so we will also \textbf{take all angle measures modulo $180\dg$}.
That means that $-70\dg = 110\dg = 290\dg = \dots$.
Once we take mod $180\dg$, \eqref{eq:directed_180_old} becomes
the following very important result.
\begin{proposition}
	For any lines $\ell$ and $m$,
	$\dang(\ell, m) = - \dang(m, \ell)$.
	(In other words, measuring the angle clockwise instead
	of counterclockwise corresponds to negation.)
\end{proposition}
Observe why this is intuitively true in \Cref{fig:dir_angle_first}.

You can verify now that with this identification, we have 
\[ \dang (\ell, m) + \dang(m, n) = \dang (\ell, n) \]
for all concurrent lines $\ell$, $m$, and $n$,
regardless of configuration.

With this definition in place, we can define $\dang AOB$
as the angle between the two lines $AO$ and $BO$.
\begin{definition}
	Given three points $A$, $O$, $B$ we define
	\[ \dang AOB \defeq \dang
		\left( \ol{AO}, \ol{BO} \right). \]
	Equivalently, if $\ell$ and $m$ are two lines which intersect at $O$,
	then $\dang(\ell, m) = \dang AOB$
	for any point $A$ on $\ell$ and $B$ on $m$.
\end{definition}
(Note that by $\ol{XY}$ I mean ``line $XY$'', not ``segment $XY$''.)


\begin{figure}[ht]
	\centering
	\begin{asy}
		size(5cm);
		pair L = dir(80);
		pair LL = -L;
		pair M = dir(130);
		pair MM = -M;
		draw(L--LL);
		draw(M--MM);
		MP("\ell", L, L);
		MP("m", M, M);
		draw(arc(origin,0.4*L,0.4*M), EndArrow);
		draw(arc(origin,-0.3*M,0.3*L), dotted, BeginArrow);
		Drawing("A", 0.8*L, dir(-10));
		Drawing("B", -0.6*M, dir(20));
		Drawing("O", origin, dir(190));
		MP("50^{\circ}", 0.6*dir(105), origin);
		MP("-130^{\circ}", 0.5*dir(15), origin);
	\end{asy}
	\caption{The directed angle $\dang AOB = -130\dg = 50\dg$.}
\end{figure}


Most of the time we will be using the $\dang AOB$ notation.
But it is sometimes useful to use the $\dang(\ell, m)$ notation
in problems where the intersection point $O$ of $\ell$ and $m$
is not yet named.

\section{Properties of Directed Angles}
You might ask whether this strange convention is actually useful.
Let me convince you it is with the following theorem.

\begin{theorem}[Directed Cyclic Quadrilaterals]
	\label{thm:dir_cyclic_quad}
	Let $A$, $B$, $X$, $Y$ be four points, no three collinear.
	Then they are concyclic if and only if
	\[ \dang XAY = \dang XBY. \]
\end{theorem}
\begin{exercise}
	Go back to \Cref{fig:cyclic_quad} and verify
	that this theorem actually works for both configurations.
\end{exercise}

Now you should be paying attention, because I just eliminated one of the biggest pains in olympiad geometry for you.
Unlike the \Cref{prop:bad_cyclic_quad} you grew up with,
\Cref{thm:dir_cyclic_quad} has no case distinctions.

Many other things become vastly simpler.
\begin{theorem}
	[Angle Addition Postulate]
	We have $\dang AOP + \dang POB = \dang AOB$.
\end{theorem}
Actually, this is a ``special case'' of the following result
for three concurrent lines $\ell$, $m$, $n$.
\begin{theorem}
	[Triangles Sum to $180\dg$]
	For any lines $\ell$, $m$, $n$ we have
	\[ \dang(\ell, m) + \dang(m,n) + \dang(n,\ell) = 0. \]
	In particular, for any points $A$, $B$, $C$ we have
	\[ \dang ABC + \dang BCA + \dang CAB = 0. \]
\end{theorem}
\begin{proof}
	Check it yourself.
\end{proof}
\begin{figure}[ht]
	\centering
	\begin{asy}
		size(5cm);
		pair A = Drawing("A", dir(70), dir(70));
		pair B = Drawing("B", dir(330), dir(310));
		pair C = Drawing("C", dir(210), dir(210));
		draw(A--B--C--cycle);

		real r = 0.25;
		draw(arc(A, A+r*dir(C-A), A+r*dir(B-A)), EndArrow);
		draw(arc(B, B+r*dir(A-B), B+r*dir(C-B)), EndArrow);
		draw(arc(C, C+r*dir(B-C), C+r*dir(A-C)), EndArrow);
		real k = 0.6;
		MP("\measuredangle CAB", k*dir(A), origin);
		MP("\measuredangle ABC", k*dir(B), origin);
		MP("\measuredangle BCA", 0.9*k*dir(C), origin);
	\end{asy}
	\caption{Triangle Sum}
\end{figure}


\begin{theorem}
	[Collinearity Criteria]
	Let $X$ be any point.
	Points $A$, $B$, $C$ are collinear if and only if
	\[ \dang XBC = \dang XBA. \]
\end{theorem}
\begin{proof}
	Prove this yourself.
	(Show that the assertion is equivalent to $\dang ABC=0$.)
\end{proof}
\begin{figure}[ht]
	\centering
	\begin{asy}
		pair X = 0.7*dir(110);
		Drawing("X", X, dir(X));
		pair A = Drawing("A", 0.8*dir(20), dir(-X));
		pair B = Drawing("B", -0.2*dir(20), dir(-X));
		pair C = Drawing("C", 0.3*dir(20), dir(-X));
		draw(dir(A)--dir(-A), Arrows);
		draw(X--B);
	\end{asy}
	\qquad
	\begin{asy}
		pair X = 0.7*dir(110);
		Drawing("X", X, dir(X));
		pair A = Drawing("A", -0.6*dir(20), dir(-X));
		pair B = Drawing("B", -0.2*dir(20), dir(-X));
		pair C = Drawing("C", 0.3*dir(20), dir(-X));
		draw(dir(A)--dir(-A), Arrows);
		draw(X--B);
	\end{asy}
	\caption{Collinearity with directed angles.}
\end{figure}

Also, note that right angles have the very nice property that
if $\ell \perp m$ we have
\[ \dang (\ell, m) = \dang (m, \ell) = 90\dg. \]
Hence any time you have perpendiculars, you can simply set
the measure of the directed angle as $90\dg$
without thinking or worrying about counterclockwise versus clockwise.

\section{Examples}
Writing a solution using directed angles should, in theory, not take any more additional effort.
One simply solves the problem by looking at \emph{one particular diagram},
but writes the solution up using the above rules so
that they hold for \emph{all diagrams}.

Let me actually give an example now.

\begin{example}
	[NIMO Winter 2013, Adapted] Let $ABC$ be a triangle with orthocenter $H$ and let $M$ be the midpoint of $\overline{BC}$.  Denote by $\omega_B$ the circle passing through $B$, $H$, and $M$, and denote by $\omega_C$ the circle passing through $C$, $H$, and $M$. Lines $AB$ and $AC$ meet $\omega_B$ and $\omega_C$ again at $P$ and $Q$, respectively. Rays $PH$ and $QH$ meet $\omega_C$ and $\omega_B$ again at $R$ and $S$, respectively. Prove that $M$, $R$, $S$ are collinear.
\end{example}
This is just angle chasing.
Unfortunately, the configurations when $\triangle ABC$ is acute and
$\triangle ABC$ is obtuse look quite different from each other.

\begin{center}
	\begin{asy}
		size(5cm);
		pair A = Drawing("A", dir(110), dir(110));
		pair B = Drawing("B", dir(210), dir(210));
		pair C = Drawing("C", dir(330), dir(330));
		draw(A--B--C--cycle, red);
		pair M = Drawing("M", midpoint(B--C), 2*dir(-70));
		pair H = Drawing("H", orthocenter(A,B,C), dir(orthocenter(A,B,C)-M)*2);

		Drawing(circumcircle(B,H,M), blue);
		Drawing(circumcircle(C,H,M), blue);
		pair OB = circumcenter(B,H,M);
		pair OC = circumcenter(C,H,M);
		pair P = Drawing("P", 2*foot(OB,B,A)-B, dir(160));
		pair Q = Drawing("Q", 2*foot(OC,C,A)-C, dir(90));
		pair R = Drawing("R", 2*foot(OC,P,H)-H, dir(45));
		pair S = Drawing("S", 2*foot(OB,Q,H)-H, dir(260));
		draw(P--R); draw(Q--S);
		draw(R--S, dotted);

		draw(circumcircle(P,Q,A), dashed);
	\end{asy}
	\qquad
	\begin{asy}
		size(7cm);
		pair A = Drawing("A", dir(50), dir(10));
		pair B = Drawing("B", dir(170), dir(210));
		pair C = Drawing("C", dir(10), dir(10));
		draw(A--B--C--cycle, red);
		pair M = Drawing("M", midpoint(B--C), 2*dir(-100));
		pair H = Drawing("H", orthocenter(A,B,C), dir(orthocenter(A,B,C)-M)*2);

		Drawing(circumcircle(B,H,M), blue);
		Drawing(circumcircle(C,H,M), blue);
		pair OB = circumcenter(B,H,M);
		pair OC = circumcenter(C,H,M);
		pair P = Drawing("P", 2*foot(OB,B,A)-B, dir(-50));
		pair Q = Drawing("Q", 2*foot(OC,C,A)-C, dir(100));
		pair R = Drawing("R", 2*foot(OC,P,H)-H, dir(-80));
		pair S = Drawing("S", 2*foot(OB,Q,H)-H, dir(130));
		draw(P--R); draw(Q--S);
		draw(R--S, dotted);

		draw(circumcircle(P,Q,A), dashed);
		draw(A--Q, red+dotted);
	\end{asy}
\end{center}

Follow along the proof in both diagrams, asking yourself why each equality is true.
The beauty of directed angles is that to \emph{write} this proof, I only
had to look at the first diagram;
it then works for the other diagram automatically.

\begin{proof}
	[Verbose Solution]
	Applying Miquel's Theorem (in the exercises later),
	we find that quadrilateral $APHQ$ is cyclic.
	Also from Triangle Sum we have
	\[ \dang \left( \ol{HB}, \ol{AB} \right)
		+ \dang \left( \ol{AB}, \ol{AC} \right)
		+ \dang \left( \ol{AC}, \ol{HB} \right)
		= 0 \]
	 we derive that $\dang HBA + \dang BAC = 90\dg$.

	Taking these for granted,
	we can compute
	\begin{align*}
		\dang BPS &= \dang BHS & \text{Cyclic quads} \\
		&= \dang PHS + \dang BHP & \text{Angle addition} \\
		&= \dang PHQ + \dang BHP & \text{$S$, $H$, $Q$ collinear} \\
		&= \dang PAQ + \dang BHP & \text{Cyclic quads} \\
		&= \dang PAQ - (\dang HPB + \dang PBH) & \text{Triangle Sum} \\
		&= \dang BAC - (\dang HPB + \dang ABH) & \text{Collinearity} \\
		&= \dang BAC + \dang BPH + \dang HBA & \text{Negation} \\
		&= 90\dg + \dang BPH. &
	\end{align*}
	But \[ \dang BPS - \dang BPH
		= \dang BPS + \dang HPB = \dang HPS. \]
	so we deduce that $\dang HPS = -90\dg = 90\dg$,
	hence $\dang HMS = 90\dg$ as well.
	In the same way, $\dang HMR = 90\dg$ as well.
	Hence $\dang HMS = \dang HMR$, so points $M$, $R$, $S$ are collinear.
\end{proof}
In an actual olympiad one need not be so explicit.
In particular, you can omit the part where I wrote out the explicit reasons
for each step; I only provided this for reference.
Here is a second example (much harder) in which I will be much more succinct.
See if you can follow along the logic.

\begin{example}
	[European Girl's MO 2012, Problem 7]
	Let $ABC$ be an acute-angled triangle with circumcircle $\Gamma$ and orthocenter $H$. Let $K$ be a point of $\Gamma$ on the other side of $BC$ from $A$. Let $L$ be the reflection of $K$ in the line $AB$, and let $M$ be the reflection of $K$ in the line $BC$. Let $E$ be the second point of intersection of $\Gamma $ with the circumcircle of triangle $BLM$.
Show that the lines $KH$, $EM$ and $BC$ are concurrent.
\end{example}
\begin{center}
	\begin{asy}
		unitsize(2.5cm);
		pair A = Drawing("A", dir(50), dir(50));
		pair B = Drawing("B", dir(209), dir(180));
		pair C = Drawing("C", dir(331), dir(0));
		Drawing(A--B--C--cycle);
		pair H = orthocenter(A,B,C);
		pair Ka = foot(A,B,C);
		pair Kc = foot(C,A,B);
		Drawing("H", H, dir(45));
		pair Ha = Drawing("H_A", 2*Ka-H, dir(-70));
		pair Hc = Drawing("H_C", 2*Kc-H, dir(135));
		pair K = Drawing("K", dir(-64), dir(-90));
		pair M = Drawing("M", reflect(B,C) * K, dir(180));
		pair L = Drawing("L", reflect(B,A) * K, dir(90));
		pair E = Drawing("E'", 2*foot(origin,M,Ha)-Ha, dir(70));
		Drawing(A--Ha--B--Hc--C);
		Drawing(L--Hc); Drawing(Hc--E, dotted);
		draw(unitcircle);
		draw(Ha--E);
		draw(B--H--K);
		draw(L--K--M,dashed);
		draw(circumcircle(B,L,E), dotted);
	\end{asy}
	\begin{asy}
		unitsize(2.5cm);
		pair A = Drawing("A", dir(100), dir(100));
		pair B = Drawing("B", dir(209), dir(250));
		pair C = Drawing("C", dir(331), dir(0));
		Drawing(A--B--C--cycle);
		pair H = orthocenter(A,B,C);
		pair Ka = foot(A,B,C);
		pair Kc = foot(C,A,B);
		Drawing("H", H, dir(45));
		pair Ha = Drawing("H_A", 2*Ka-H, dir(-70));
		pair Hc = Drawing("H_C", 2*Kc-H, dir(135));
		pair K = Drawing("K", dir(-84), dir(-70));
		pair M = Drawing("M", reflect(B,C) * K, dir(-10));
		pair L = Drawing("L", reflect(B,A) * K, dir(240));
		pair E = Drawing("E'", 2*foot(origin,M,Ha)-Ha, dir(70));
		Drawing(A--Ha--B--Hc--C);
		Drawing(L--Hc); Drawing(Hc--E, dotted);
		draw(unitcircle);
		draw(Ha--E);
		draw(B--H--K);
		draw(L--K--M,dashed);
		pair O = circumcenter(B,L,E);
		draw(arc(O, abs(O-L), 170, 400), dotted);
	\end{asy}
\end{center}

\begin{proof}
	[Solution] Let $H_A$ and $H_C$ be the reflections of $H$ across $\ol{BC}$ and $\ol{BA}$; it is well-known that these lie on $\Gamma$.  Let $E'$ be the second intersection of line $H_AM$ with $\Gamma$. By construction, lines $E'M$ and $HK$ concur on $\ol{BC}$, and our goal is to show that $B$, $L$, $E'$, $M$ are concyclic.

	First, we claim that $L$, $H_C$, and $E'$ are collinear. 
	Due to the reflections,
	\[ \measuredangle LH_CB = -\measuredangle KHB = \measuredangle MH_AB 
	= \measuredangle E'H_AB = \measuredangle E'H_CB \]
	which proves the claim.
	Then
	\[ \measuredangle LE'M = \measuredangle H_CE'H_A = \measuredangle H_CBH_A = 2\measuredangle ABC \]
	(the last equality following from reflections; verify it yourself) and
	\[ \measuredangle LBM = \measuredangle LBK + \measuredangle KBM = 2\measuredangle ABK + 2\measuredangle KBC = 2\measuredangle ABC \]
	so $B$, $L$, $E'$, $M$ are concyclic. Hence $E=E'$ and we are done.
\end{proof}



\section{A Word of Warning}
Never take half of a directed angle -- since we are working modulo $180\dg$,
taking half of an angle doesn't make sense.


\section{Practice Problems}
Here are some famous lemmas from olympiad geometry that you should know.
They are not hard to prove (in fact, they are all direct angle chasing) but they do have configuration issues.
See if you can solve them and then write a single proof which does not need to consider different cases.

\begin{center}
	\begin{asy}
		size(4cm);
		pair A = Drawing("A", dir(110), dir(110));
		pair B = Drawing("B", dir(210), dir(210));
		pair C = Drawing("C", dir(330), dir(330));
		pair D = Drawing("D", foot(A,B,C), dir(-90));
		pair E = Drawing("E", (A+C)/2, dir(40));
		pair F = Drawing("F", 0.7*A+0.3*B, dir(160));
		draw(circumcircle(A,E,F), dotted);
		draw(circumcircle(B,F,D), dotted);
		draw(circumcircle(C,D,E), dotted);
		draw(A--B--C--cycle);
	\end{asy}
	\begin{asy}
		size(3.5cm);
		pair Y = Drawing("Y", origin, dir(225));
		pair A = Drawing("A", dir(130), dir(130));
		pair B = Drawing("B", 1.3*dir(70), dir(110));
		pair r = 1.7*dir(-40);
		pair C = Drawing("C", A*r, dir(A*r));
		pair D = Drawing("D", B*r, dir(B*r));
		pair X = Drawing("X", extension(A,B,C,D), dir(50));
		draw(A--X);
		draw(C--D);
		draw(circumcircle(A,X,Y));
		draw(circumcircle(B,X,Y));
		draw(A--Y--B, dashed);
		draw(C--Y--D, dashed);
	\end{asy}
	\begin{asy}
		size(5cm);
		pair A = Drawing("A", dir(130), dir(130));
		pair B = Drawing("B", dir(210), dir(210));
		pair C = Drawing("C", dir(330), dir(330));
		pair I = Drawing(incenter(A,B,C));
		pair D = Drawing("D", foot(I,B,C), dir(-90));
		pair E = Drawing("E", foot(I,C,A), dir(70));
		pair F = Drawing("F", foot(I,A,B), dir(160));
		pair X = Drawing("X", extension(B,I,E,F), dir(70));
		pair Y = Drawing("Y", extension(C,I,E,F), dir(70));
		draw(arc(midpoint(B--C), C, B), dotted);
		draw(A--B--C--cycle);
		draw(CP(I,D));
		draw(B--X);
		draw(C--Y);
		draw(F--X);
	\end{asy}
\end{center}


\begin{problem}
	[Miquel's Theorem]
	Let $ABC$ be a triangle.
	Consider any points $D$, $E$, $F$ on lines $BC$, $CA$, $AB$
	(distinct from the vertices of $ABC$, but not necessarily
	in the interiors of the sides).
	Prove that the circumcircles of triangles $AEF$, $BFD$, and $CDE$
	intersect at a single point.
\end{problem}

\begin{problem}
	[Spiral Similarity Lemma]
	Two circles $\omega_1$ and $\omega_2$ meet at points $X$ and $Y$.
	A line through $X$ intersects $\omega_1$ and $\omega_2$ again at $A$ and $B$.
	A second line through $X$ intersects $\omega_1$ and $\omega_2$ again at $C$ and $D$.
	Show that $\triangle AYC \sim \triangle BYD$.
\end{problem}

\begin{problem}
	[Right Angles on Intouch Chord]
	Let $ABC$ be a triangle whose incircle touches the opposite sides
	at $D$, $E$, $F$.
	The angle bisectors of $\angle B$ and $\angle C$ meet
	line $EF$ at points $X$ and $Y$.
	Prove that $X$ and $Y$ lie on the circle with diameter $\ol{BC}$.
\end{problem}

\section{Contest Practice}
Some of the following problems are much more nontrivial;
the last one is infamously difficult.
Directed angles will help clean up a solution
but as stressed before will not actually help you find 
the solution in the first place.

\begin{problem}
	[Folklore]
	Let $\mathcal C_1$, $\mathcal C_2$, $\mathcal C_3$, $\mathcal C_4$
	be four distinct circles.
	For $i=1,2,3,4$, suppose that $\mathcal C_i$ and $\mathcal C_{i+1}$
	intersect at two distinct points $A_i$ and $B_i$
	(here $\mathcal C_5 = \mathcal C_1$).
	Prove that if $A_1A_2A_3A_4$ is cyclic then so is $B_1B_2B_3B_4$.
\end{problem}
\begin{problem}
	[Shortlist 2010/G1]
	Let $ABC$ be an acute triangle with $D, E, F$ the feet of the altitudes lying on $\ol{BC}, \ol{CA}, \ol{AB}$ respectively.
	One of the intersection points of the line $EF$ and the circumcircle is $P$.
	The lines $BP$ and $DF$ meet at point $Q$. Prove that $AP = AQ$.
\end{problem}
%\begin{problem}
%	[Zhautkyov Olympiad 2013]
%	Let $ABCD$ be a trapezoid with $\ol{AD} \parallel \ol{BC}$ and $\angle ABC > 90\dg$.
%	Point $M$ is chosen on side $AB$.
%	Let $O_1$ and $O_2$ be the circumcenters of $\triangle MAD$ and $\triangle MBC$, respectively.
%	The circumcircles of $\triangle MO_1D$ and $\triangle MO_2C$ meet again at point $N$.
%	Prove that $O_1$, $O_2$, $N$ are collinear.
%\end{problem}
\begin{problem}
	[USAMO 2013/1] In triangle $ABC$, points $P$, $Q$, $R$ lie on sides $BC$, $CA$, $AB$, respectively.
	Let $\omega_A$, $\omega_B$, $\omega_C$ denote the circumcircles of triangles $AQR$, $BRP$, $CPQ$, respectively. 
	Given the fact that segment $AP$ intersects $\omega_A$, $\omega_B$, $\omega_C$ again at $X$, $Y$, $Z$ respectively,
	prove that $YX/XZ=BP/PC$.
\end{problem}
\begin{problem}
	[Balkan 2009] Let $\ol{MN}$ be a line parallel to the side $\ol{BC}$ of a triangle $ABC$,
	with $M$ on side $AB$ and $N$ on side $AC$.
	The lines $BN$ and $CM$ intersect at point $P$.
	The circumcircles of $\triangle BMP$ and $\triangle CNP$ meet again at $Q$.
	Prove that $\angle BAQ = \angle CAP$.
\end{problem}
\begin{problem}
	[USA TST 2007/1] Circles $\omega_1$ and $\omega_2$ meet at $P$ and $Q$.
	Segments $AC$ and $BD$ are chords of $\omega_1$ and $\omega_2$ respectively, such that lines $AB$ and $CD$ meet at $P$.
	Lines $BD$ and $AC$ meet at $X$.
	Point $Y$ lies on $\omega_1$ such that $\ol{PY} \parallel \ol{BD}$.
	Point $Z$ lies on $\omega_2$ such that $\ol{PZ} \parallel \ol{AC}$.
	Prove that points $Q$, $X$, $Y$, $Z$ are collinear.
\end{problem}
\begin{problem}
	Let $ABC$ be a triangle with circumcircle $\Gamma$.
	Let $\ell$ be a line in the plane, and let $\ell_a$, $\ell_b$, $\ell_c$ be the lines obtained
	by reflecting $\ell$ in the lines $BC$, $CA$, and $AB$, respectively.
	Let $\triangle A'B'C'$ denote the triangle determined by the lines $\ell_a$, $\ell_b$, $\ell_c$.
	\begin{enumerate}[(a)]
		\ii (Iran 1995) Prove that the incenter of the $\triangle A'B'C'$ lies on $\Gamma$.
		\ii (IMO 2011/6) Assume $\ell$ is tangent to $\Gamma$.
		Show that the circumcircle $\triangle A'B'C'$ is tangent to the circle $\Gamma$.
	\end{enumerate}
\end{problem}



\end{document}
