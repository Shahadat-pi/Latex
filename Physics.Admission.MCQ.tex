\documentclass{article}
\usepackage[utf8]{inputenc}
\usepackage{amsmath}
\usepackage{mathtools}
\DeclarePairedDelimiter{\ceil}{\lceil}{\rceil}
\usepackage{geometry}
\usepackage[banglamainfont=Kalpurush, 
            banglattfont=Siyam Rupali
           ]{latexbangla}
        
\begin{document}
\begin{center}
\begin{large}
\textbf{ঢাকা বিশ্ববিদ্যালয়ের বিগত বছরের ভর্তি পরীক্ষার প্রশ্নপত্র}\\
পদার্থবিজ্ঞান 2017
\end{large}
\end{center}
\begin{enumerate}
\item  যদি $A = B^{n}C^{m}$ এবং A, B ও C এর মাত্রা যথাক্রমে LT, $L^{2}T^{-1}$ এবং $LT^{2}$ হয় তবে $m$ ও $n$ এর মান হবে - 
\end{enumerate}
\end{document}