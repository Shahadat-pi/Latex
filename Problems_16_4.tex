\documentclass[12pt]{article}
\usepackage{USAMTS_problems}

\chead{\parbox{1.6in}{\hskip-0.2in\scalebox{0.50}[0.50]{\includegraphics*[viewport=120 340 700 500]{USAMTSBW.pdf}}}\parbox{5in}{\begin{center}\large {\bf USA Mathematical Talent Search}
\\ Round 4 Problems \\
  Year 16 --- Academic Year 2004--2005\\{\normalsize www.usamts.org}\end{center}}}
\begin{document}

Please follow the rules below to ensure that your paper is graded properly.  

\begin{enumerate}

\item \label{imp}Put your name, username, and USAMTS ID$\#$ on {\bf every page you submit}.

\item Once you send in your solutions, that submission is final.  You cannot resubmit solutions.

\item If you have already sent in an Entry Form and a Permission Form, you do not need to resend them.

\item Confirm that your email address in your USAMTS Profile is 
correct.  You can do so by logging into the site, then clicking
on My USAMTS on the sidebar, then click Profile.  If you are registered for the USAMTS and haven't received 
any email from us about the USAMTS, your email address is probably wrong in your Profile.

\item Do not fax solutions written in pencil.

\item No single page should contain solutions to more than one problem.

\item Submit your solutions by March 14, 2005 (postmark deadline), via one of the methods below.
\begin{enumerate}
\item Email: solutions@usamts.org.  Please see usamts.org for a list of acceptable file types.
Do not send .doc Microsoft Word files.
\item Fax: (619) 445-2379
\item Snail mail: USAMTS, P.O. Box 2090, Alpine, CA 91903--2090.
\end{enumerate}

\item Re--read item~\ref{imp}.

\end{enumerate}

\pagebreak

\USprob{1/4/16.}{
\rightskip 0in
Determine with proof the number of positive integers $n$ such that a convex regular polygon with $n$ sides has 
interior angles whose measures, in degrees, are integers.
}

\vskip0.25in
\USprob{2/4/16.}{
\rightskip 0in
Find positive integers $a,b$, and $c$ such that
$$\sqrt a + \sqrt b + \sqrt c = \sqrt {219 + \sqrt{10080} + \sqrt {12600} + \sqrt {35280}}.$$
Prove that your solution is correct.  (Warning: numerical approximations of the values do not constitute a proof.)
}

\vskip0.25in
\USprob{3/4/16.}{
\rightskip 0in
Find, with proof, a polynomial $f(x,y,z)$ in three variables, with integer coefficients, such that for all integers $a,b,c$, the sign of $f(a,b,c)$ (that is, positive, negative, or zero) is the same as the sign of $a+b\sqrt[3]{2}+c\sqrt[3]{4}$.
}

\vskip0.25in
\USprob{4/4/16.}{
\rightskip 0in
Find, with proof, all integers $n$ such that 
there is a solution in nonnegative real numbers $(x,y,z)$ to 
the system of equations $$2x^2+3y^2+6z^2=n \quad \text{and} 
\quad 3x+4y+5z=23.$$
}

\USprob{5/4/16.}{
\rightskip 2.2in
Medians $AD$, $BE$, and $CF$ of triangle $ABC$ meet at $G$ as shown.  Six small triangles, each with a vertex at $G$, are formed. We draw the circles inscribed in triangles $AFG$, $BDG$, and $CDG$ as shown.  Prove that if these three circles are all congruent, then $ABC$ is equilateral.

\hskip4.5in\vskip-1.5in
\includegraphics[viewport=-2.5in 4in 6.5in 7.5in,scale=0.8,clip]{Incircles2.pdf}

}


\end{document}
