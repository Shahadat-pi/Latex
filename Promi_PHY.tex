\documentclass[addpoints]{exam}
\usepackage[utf8]{inputenc}
\usepackage{amsmath}
\usepackage{mathtools}
\usepackage{tikz}
\usepackage{pgfplots}
\usetikzlibrary{datavisualization}
\usetikzlibrary{datavisualization.formats.functions}
\usepackage{relsize}
\usepackage{dirtytalk}
\usepackage{graphicx}
\graphicspath{ {./drift/} }
\DeclarePairedDelimiter{\ceil}{\lceil}{\rceil}
\usepackage{geometry}
\usepackage{draftwatermark}
\SetWatermarkFontSize{2cm}
\SetWatermarkText{Physics}
\usepackage[banglamainfont=Kalpurush, 
            banglattfont=Siyam Rupali
           ]{latexbangla}
        
\begin{document}
\begin{LARGE}
\begin{center}
Physics
\end{center}
\end{LARGE}
\begin{questions}

\question  একটি বস্তু পুরো পথের অর্ধেক শেষ সেকেন্ডে অতিক্রম করে। একে স্থির অবস্থা থেকে ছেড়ে দিলে বস্তুর পতনকাল ও উচ্চতা কত?

\question 49 মিটার উচ্চতা থেকে একটি পাথর ফেলে দেয়ার 1 সেকেন্ড পর একই উচ্চতা থেকে আরেকটি পাথর নিক্ষেপ করা হল। পাথর দুটি একই সময়ে মাটিতে আঘাত করলে 2য়টির আদিবেগ কত? 

\question একটি নির্দিষ্ট উচ্চতা থেকে একটি তরলের প্রথম ফোটা মাটিতে স্পর্শ করলে চতুর্থ ফোটা পড়তে শুরু করে। তৃতীয় ফোটাটি ভুমি থেকে কত উচ্চতায়?

\question একটি গাড়ি সম্পূর্ণ পথের অর্ধেক 40 কি.মি/ঘন্টা এবং বাকি অর্ধেক 60 কি.মি/ঘন্টা বেগে গেলে গড়দ্রুতি কত?

\question একটি লিফট 4.8 ms$ ^{-2} $ ত্বরণে  নিচে নামছে। এর মেঝে থেকে 2 ফুট উচু থেকে একটি বল ছেড়ে দিলে মেঝেতে আঘাত করতে কত সময় লাগবে?

\question 5 ms$ ^{-1} $ একটি লিফট উপরের দিকে সমবেগে উঠছে। হটাৎ একটি স্ক্রু খুলে পড়লে 1 সেকেন্ড পর এর বেগ কত? ঐ মুহুর্তে লিফট ও স্ক্রুর মধ্যবর্তী দুরত্ব কত?

\question একটি বালক একটি ব্রীজের উপর থেকে দাড়িয়ে 20 ms$ ^{-1} $ বেগে একটি বল খাড়া উর্ধ্বে নিক্ষেপ করলো। পানির পৃষ্ঠ 10 মিটার নিচে।\\
(1)	সবোর্চ্চ উচ্চতা কত?\\
(2)	নিক্ষেপনের কতক্ষন পর পানি স্পর্শ করে?\\
(3)	যদি 20 ms$ ^{-1} $ বেগে আনুভুমিক ভাবে নিক্ষেপ করা হয় তবে কতক্ষণ লাগে?\\
(4)	শুধু ছেড়ে দিলে কতক্ষণ লাগে?


\question দেখাও যে নির্দিষ্ট আদিবেগের জন্য দুটি কোণ পাওয়া যাবে যাদের জন্য পাল্লা নির্দিষ্ট।

\question  দুটি প্রাসের একটি $ \alpha $ কোণে অপরটিকে $ 90-\alpha $ কোণে বেগে ছোড়া হলে 2য়টি 1মটির চেয়ে 19.6 মিটার উচ্চতায় উঠে। প্রত্যেকের সবোর্চ্চ উচ্চতা কত? 

\question একটি বুলেট $ 30^{\circ} $ কোণে 3কি.মি. দুরে আঘাত করে। কোণ পরিবর্তন করে 5 কিমি দুরে আত করা যাবে কি? 

\question একজন ক্রিকেটার সর্বোচ্চ 100 মিটার দুরে একটি বল পাঠাতে পার। একই আদিবেগে সর্বাধিক কত উচ্চতায় পাঠাতে পারবে? 




\end{questions}

\end{document}