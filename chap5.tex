\documentstyle[12pt,psfig]{article}
\message{
 	Copyright, 1993, all rights reserved, Charlie Ebner (Dept.of Physics, 
 The Ohio State University, Columbus OH 43210) and Mark Jarrell (Dept.of 
 Physics, The University of Cincinnati, Cincinnati, OH 45221-0011).  This 
 material may not be reproduced for profit, modified or published in any 
 form (this includes electronic redistribution) without the prior written 
 permission of the authors listed above.	
}


\begin{document}


\section{Introduction and Definitions}
As far as anyone knows, there is no such thing as a free magnetic charge or
{\em magnetic monopole}, although there are people who look for them (and
occasionally claim to have found one); certainly they may exist. Because
no known phenomena require their existence, we shall
develop magnetostatics and eventually electrodynamics assuming that they do
not exist. In this case there is a fundamental difference between
electrostatics and magnetostatics, explaining in part why the two subjects
developed independently and were regarded as distinct rather than
different limits or aspects of one type of phenomenon (electromagnetic
phenomena). 

\subsection{Magnetic Induction}
In the absence of monopole moments, the most fundamental source
of magnetic effects is the {\em magnetic dipole}. In the presence of other
magnetic materials, a point dipole will experience some force. One
defines the {\em magnetic flux density} or {\em magnetic induction} $\B$ in
terms of the torque $\N$ exerted on the dipole. Given that the dipole
moment is $\mm$, the defining relation is
\beq
\N\equiv\mm\times\B.
\eeq
Thus, as for electrostatics, the basic field of magnetostatics is defined
by the effect produced on an elementary source.

\subsection{Current Density and Conservation}
Among the first known manifestations of magnetic phenomena were the forces
observed to act on some materials (magnets) as a consequence of the earth's
magnetic field. In 1819, Hans Christian Oersted (1777-1851) found that very
similar effects could be produced by placing a magnet close to a
current-carrying wire, indicating a connection between electrical current
and magnetism. Much of what we have to say about magnetostatics will
involve the use of currents as sources of $\B$, so let us say a few words
about the properties of stationary, \ie time-independent, currents.
We shall write the {\em current density} as $\Jx$; it has dimensions of
charge/area-time and is by definition such that a component $J_i$ is the
amount of charge that crosses unit area in unit time given that the normal
to the surface is in the $i$-direction. Given a charge density $\rhx$ moving
at a velocity $\v$, there is a current density
\beq
\Jx=\rhx\v.
\eeq

It is an experimental fact that charge is conserved. We may determine a
continuity or conservation equation which expresses this fact. Consider
\beq
\inv\,\div\Jx=\ina\,\J\cdot\nn.
\eeq
The surface integral gives the rate at which charge flows out of the domain
V through the surface S. Because charge is conserved, this must be equal to
the negative of the rate at which the total charge inside of V changes:
\beq
\ina\,\Jx\cdot\nn=-\pde{}{t}\lep\inv\,\rhx\rip.
\eeq
Assuming that V is independent of time, we may move the derivative with
respect to time inside of the integral and so have
\beq
\inv\,\div\Jx=\ina\,\Jx\cdot\nn=-\inv\,\pde{\rhx}{t}.
\eeq
Now argue in the usual fashion: Because V is an arbitrary domain, this
equation can only be true if the integrands on the two sides are the same
everywhere. Hence we have
\beq
\div\Jx+\pde{\rhx}{t}=0.
\eeq
This equation is true so long as charge is {\em locally} conserved, meaning
that the only way for charge to appear in V (or to disappear from V) is by
flowing across the boundary. The equation has the typical form of a
continuity equation which is that the divergence of the current density
of some quantity plus the time derivative of the density of that quantity
equals zero. If there are sources (or sinks) of the quantity in question,
there is an additional term in the equation expressing the contribution of
these sources.

\section{Amp\`ere's Law}
Even as currents in wires produce forces on magnetic materials, so do they
produce forces on other current loops.
F\'elix Savart (1791-1841) and Jean-Baptiste Biot (1774-1862)
began experiments on these forces soon after Oersted's discovery, as did
Andr\'e-Marie Amp\`ere. Amp\`ere continued his experiments for some years
and published his collected results in 1825. The basic law emerging from
Amp\`ere's work deals with the forces acting between closed current loops.
Suppose that we have a current $I$ in one loop and a current $I'$ in a
second. Let $\F$ be the force acting on the loop carrying current $I$.

\centerline{\psfig{figure=fig1.ps,height=1.75in,width=4.25in}}

\noindent Then {\em Amp\`ere's Law} may be expressed as follows:
\beq
\F=kII'\int\int\frac{d\l\times[d\l'\times(\x-\xp)]}{|\x-\xp|^3}
\eeq
where the integrals over $\l$ and $\l'$ are, respectively, around the loops
carrying currents $I$ and $I'$; $\x$ and $\xp$ are the
position vectors of the integration points. The constant $k$ depends on the
units employed. For our units, with current expressed as $statcoul/sec$, or
$statamp$, $k$ has dimensions of (inverse speed)$^2$ (remember that charge
has has dimensions of $M^{1/2}L^{3/2}/T$ in our units). Hence one writes
$k\equiv1/c^2$ where $c$ is a speed. From appropriate experiments one may
find that its numerical value in cgs units is $c=2.998\times10^10\,cm/sec$.
We recognize this as the speed of light, but that is, for the moment, not
important.

At this juncture we may introduce the magnetic induction by writing the
force as
\beq
\F\equiv\frac{I}{c}\int d\l\times\Bx
\eeq
where $\Bx$ is the magnetic induction produced by the other loop's current.
It is not yet clear apparent this $\B$ is the same as the one introduced in
\eq{1}; nevertheless, it is, as we shall see presently.

Comparison of \eq{7} with \eq{8} shows that the magnetic induction
produced by the loop carrying current $I'$ may be written as an integral over
that loop,
\beq
\Bx=\frac{I'}{c}\int\frac{\dvl'\times(\x-\xp)}{|\x-\xp|^3};
\eeq
this equation is often called the {\em Biot-Savart Law}.

One also writes these equations in differential form, although that may
introduce some inaccuracies and even misunderstanding. First, the force
acting on just an infinitesimal piece of the loop carrying current $I$ is
\beq
d\F(\x)=\frac Ic\dvl\times\Bx.
\eeq

\centerline{\psfig{figure=fig2.ps,height=1.0in,width=6.375in}}

\noindent The correct interpretation of this equation is that it 
expresses the force
on the element $\dvl$ of the loop carrying current $I$ which is produced by
the current $I'$ in the other loop; $\Bx$ is the magnetic induction
produced by this other loop. There will be additional forces on the element
$dl$ produced by the current in other parts of its own loop.

Another equation one frequently sees is an expression for the infinitesimal
magnetic induction produced at a point $\x$ by an infinitesimal element of
a source loop. Given that the source loop is, as above, the one carrying
current $I'$, this expression is
\beq
d\Bx=\frac{I'}{c}\frac{\dvl'\times(\x-\xp)}{|\x-\xp|^3}.
\eeq
This is, however, not a correct statement in that the element $\dvl'$ of this
circuit acting alone does not produce such a magnetic induction. First, it is
impossible to have such a source acting alone if there is no time
dependence in the sources and fields; the flowing charge which gives the
current $I'$ in $\dvl'$ has to come from somewhere and go somewhere and so
if this element is the entire source current distribution, then there must be
some time dependence associated with the accumulation and depletion of
charge at the two ends of the element. When this time dependence is 
included, \eq{11} will not give the magnetic induction correctly.

\subsection{Induction of an Arbitrary Current Density}
The preceding results are capable of generalization to arbitrary current
distributions (not just filaments). One has to replace $I\dvl$ by $\Jx dadl$
where $\Jx$ is the current density, $da$ is the cross-sectional area of the
filament, and $dl$ is the magnitude of $\dvl$. Then note that $dadl=d^3x$, a
volume element. 

\centerline{\psfig{figure=fig3.ps,height=1.0in,width=6.375in}}

\noindent Hence one finds that the flux produced by an extended
current distribution is
\beq
\Bx=\frac1c\inivp\,\frac{\Jxp\times(\x-\xp)}{|\x-\xp|^3}
\eeq
while the force on an extended current distribution $\Jx$ produced by some
externally applied $\Bx$ is
\beq
\F=\frac1c\iniv\,\Jx\times\Bx.
\eeq
The corresponding force density at point $\x$ is
\beq
d\F=\frac1cd^3x\,\Jx\times\Bx,
\eeq
and the expression for $d\B$ (not to be trusted) becomes
\beq
d\Bx=\frac1cd^3x\,\frac{\Jxp\times(\x-\xp)}{|\x-\xp|^3}.
\eeq

Finally, let's add to our arsenal of equations one for the torque $\N$
felt by a current distribution $\Jx$ acted upon by an externally applied
magnetic induction $\Bx$. Given an object experiencing a force
$\F$, the torque relative to a point $O$ is just $\x\times\F$ where $\x$ is
the location of the object relative to $O$. Hence the torque on the current
distribution is
\beq
\N=\frac1c\iniv[\x\times(\Jx\times\Bx)].
\eeq

\subsection{An Alternate Form of Amp\`ere's law}

Before going on to additional formalism, let us express Amp\`ere's law,
\eq{7}, in a more symmetric form by applying some vector manipulations.
First, we will make use of the general identity $\A\times(\B\times\C)=
(\A\cdot\C)\B-(\A\cdot\B)\C$ to have
\beq
\F=\frac{II'}{c^2}\int\int\frac{[\dvl\cdot(\x-\xp)]d\l'-[\dvl\cdot \dvl']
(\x-\xp)}{|\x-\xp|^3}.
\eeq
According to Newton, the force of the loop carrying current $I$ should be
equal and opposite to the force on the other loop, implying certain
symmetries in the integrand above. The second term in the numerator changes
sign under interchange of $\x$ and $\xp$, but the first does not. Let us
study the latter more closely. Consider
\beq
\int \dvl'\int\dvl\cdot\leb-\grad\lep\frac1\xxpa\rip\rib=-\int\dvl'\int
d\lep\frac1\xxpa\rip\equiv0
\eeq
where the last step follows from the fact we integrate over a 
closed path\footnote{For example, $\oint d\phi = \phi(x)-\phi(x)$, where $x$
is an arbitrary point along the contour}.
Thus we find that the force $\F$ may be written simply as
\beq
\F=-\frac{II'}{c^2}\int\int\frac{\dvl\cdot \dvl'}{|\x-\xp|^3}(\x-\xp).
\eeq

{\subsection{Example: Force Between Parallel Wires}
Let us look at an example. Suppose we have two parallel wires a distance
$d$ apart carrying currents $I$ and $I'$ and wish to find the force per
unit length acting on one of them. At a point $\x$ on the one carrying
current $I$, there is an induction $B$ from the other one which is directed
perpendicular to the plane containing the wires. This field is given by the
integral over the sources in the other wire,

\centerline{\psfig{figure=fig4.ps,height=1.0in,width=4.25in}}

\beq
|\B|=B=\frac{I'}{c}\int\frac{\dvl'\times(\x-\xp)}{|\x-\xp|^3}=
\frac{I'd}{c}\int_{-\infty}^\infty \frac{dz'}{(d^2+z'^2)^{3/2}}
=\frac{I'}{cd}\int_{-\infty}^\infty \frac{du}{(1+u^2)^{3/2}}=\frac{2I'}
{cd}.
\eeq
The consequent force on the current $I$ in length $dz$ is
\beq
|d\F|=\frac Ic\,dz\,B=\frac{2II'}{c^2d}.
\eeq
The direction of $\F$ is such that the wires are attracted toward each
other if the currents are in the same direction and they are repelled if
the currents are in opposite directions. In doing this calculations we
have conveniently ignored the fact that we didn't deal with closed current
loops; in principal the wires have to be bent into closed loops somewhere
far away from where we have calculated the force. With a little work one
can convince himself that for sufficiently large loops, the contributions
from the part that we ignored will be as small as desired.

\section{Differential Equations of Magnetostatics}
Even as we found equations for the divergence and curl of the electric
field, so can we find such equations for the magnetic induction. Our
starting point is the integral expression for the induction produced by a
source distribution $\Jx$,
\beq
\Bx=\frac1c\inivp\Jxp\times\frac{(\x-\xp)}{|\x-\xp|^3},
\eeq
or, proceeding in the same manner as we have done before,
\beq
\Bx=-\frac1c\inivp\Jxp\times\grad\lep\xxpi\rip.
\eeq
Now apply the identity $\curl(f\A)=(\curl\A)f+\grad f\times\A$:
\beq
\curl\lep\frac{\Jxp}{\xxpa}\rip=(\curl\Jxp)\xxpi+\grad\lep\xxpi\rip\times
\Jxp.
\eeq
The curl (taken with respect to components of $\x$) of $\Jxp$ is zero, so,
upon substituting into \eq{24}, we find
\beq
\Bx=\frac1c\inivp\curl\lep\frac{\Jxp}{\xxpa}\rip=\frac1c\curl\lep
\inivp\frac{\Jxp}{\xxpa}\rip.
\eeq
Because $\Bx$ is the curl of a vector field, its divergence must be zero,
\beq
\div\Bx=0.
\eeq
This is our equation for the divergence of the magnetic induction. It tells
us that there are no magnetic charges.

To find a curl equation for $\Bx$, we take the curl of \eq{25},
\beq
\curl\Bx=\frac1c\curl\leb\curl\lep\inivp\frac{\Jxp}{\xxpa}\rip\rib.
\eeq
Now employ the identity
\beq
\curl(\curl\A)=\grad(\div\A)-\lap\A,
\eeq
valid for any vector field $\Ax$; the last term on the right-hand side of this
identity is to be interpreted as
\beq
\lap\A\equiv(\lap A_x)\xh+(\lap A_y)\yh+(\lap A_z)\zh;
\eeq
it is important that this relation is written in Cartesian coordinates.
Using the identity, we find
\beq
\curl\leb\curl\lep\frac{\Jxp}{\xxpa}\rip\rib=\grad\leb\Jxp\cdot\grad
\lep\xxpi\rip\rib-\Jxp\lap\lep\xxpi\rip.
\eeq
The second term is just $4\pi\Jxp\de(\x-\xp)$, so we have
\beq
\curl\Bx=\frac{4\pi}c\Jx+\frac1c\grad\leb\inivp\Jxp\cdot\grad\lep\xxpi\rip\rib.
\eeq
The remaining integral may be completed as follows:
\beqa
\grad\inivp\Jxp\cdot\grad\lep\xxpi\rip=-\grad\inivp\Jxp\cdot\grad'\lep\xxpi\rip
\nonumber\\=-\grad\inivp\divp\lep\frac{\Jxp}{\xxpa}\rip+\grad\inivp\frac
{\divp\Jxp}{\xxpa}.
\eeqa
The first term in the final expression can be converted to a surface
integral which then vanishes for a localized current distribution which
lies totally within the domain enclosed by the surface. The integrand of
the second term involves $\divp\Jxp=-\partial\rh(\xp)/\partial t\equiv0$
for a steady-state current distribution. Consequently we have the curl
equation
\beq
\curl\Bx=\frac{4\pi}{c}\Jx \mbox{ Sometimes called Ampere's Law}
\eeq

The curl and divergence equations, $\div\Bx=0$ and $\curl\Bx=(4\pi/c)\Jx$
plus an appropriate statement about the behavior of $\Bx$ on a boundary
tell us all we need to know to find the magnetic induction for a given set
of sources $\Jx$. There are, of course, also integral versions of these
differential equations. One finds the most common forms of them from the
Stokes theorem and the divergence theorem.
\beq
\inv\div\Bx=\ina\Bx\cdot\nn=0
\eeq
and
\beq
\ina(\curl\Bx)\cdot\nn=\invl\cdot\Bx=\frac{4\pi}{c}\ina\Jx\cdot\nn.
\eeq
The last of these is also commonly written as
\beq
\invl\cdot\Bx=\frac{4\pi}{c}I_s
\eeq
where $I_s$ is the total current passing through the surface S in the
direction of a right-hand normal relative to the direction in which the
line integral around C is done.

\centerline{\psfig{figure=fig5.ps,height=1.25in,width=4.25in}}

\noindent This relation is frequently called
Amp\`ere's law. One interesting feature of this equation is that the result
is independent of the actual surface S employed so long as it is an open
surface that ends on the path C. The amount of charge passing through all
such surfaces per unit time \ie $I_s$, is independent of S because $\div\Jx
=0$.

\section{Vector and Scalar Potentials}
In all electromagnetic systems it is possible to devise a potential
which is a vector field for $\Bx$; it is called a {\em vector potential}.
Sometimes, it is also possible to devise a scalar potential for $\Bx$. We
shall consider the latter first.

\subsection{Scalar Potential}
As we have seen in the study of electrostatics, one can write a vector
field as the gradient of a scalar if the vector field has zero divergence.
This is the case for $\Bx$ in those regions of space where $\Jx=0$. Thus,
in a source-free domain, we can write
\beq
\Bx=-\grad\Ph_M(\x).
\eeq
The {\em magnetic scalar potential} satisfies a differential equation which
follows from the requirement that $\div\Bx=0$; it is
\beq
\lap\Ph_M(\x)=0.
\eeq
Thus, wherever there is a magnetic scalar potential, it satisfies the
Laplace equation. In order to solve for it, we may apply any of the
techniques we learned for finding the electrostatic potential in
charge-free regions of space. Hence no more will be said about the
magnetic scalar potential in general in this chapter.

\subsection{Vector Potential and Gauge Invariance}
Consider now the vector potential for a magnetostatic field. Because
$\div\Bx=0$, it is possible to write the magnetic induction as the curl of
another vector field (since $\div\lep\curl\A\rip$).  We have in fact already 
constructed such a vector field because in \eq{25} we have written
\beq
\Bx=\curl\lep\frac1c\inivp\frac{\J(\xp)}{\xxpa}\rip.
\eeq
The field within the parentheses is a vector potential for $\Bx$. We shall
write it as $\Ax$. It is, in contrast to the electrostatic scalar
potential, not unique because one can always add to it the gradient of any
scalar field $\chx$ and have a vector field whose curl is still $\Bx$.
That is, given
\beq
\Ax=\frac1c\inivp\frac\Jxp\xxpa,
\eeq
which is such that $\Bx=\curl\Ax$, we can write
\beq
\A'(\x)=\Ax+\grad\chx,
\eeq
where $\chx$ is arbitrary. Then it is true that $\Bx=\curl\A'(\x)$ because
the curl of the gradient of a scalar field is zero.

By writing $\Bx=\curl\Ax$, we have automatically satisfied the requirement
that $\div\Bx=0$; Hence we may find a single (vector) equation for the vector
potential by substituting $\B=\curl\A$ into Amp\`ere's law:
\beq
\curl\Bx=\curl(\curl\Ax)=\frac{4\pi}c\Jx,
\eeq
or, using $\curl(\curl\A)=\grad(\div\A)-\lap\A$,
\beq
\lap\Ax-\grad(\div\Ax)=-\frac{4\pi}c\Jx.
\eeq
This equation would be considerably simpler if we could make the divergence
of $\Ax$ vanish. It is possible to do this by using a vector potential
constructed with an appropriate choice of $\chx$. The underlying
mathematical point is the following: so far, the only condition we have
placed on the vector potential is that its curl should be the magnetic
induction. We are free to choose it so that its divergence conforms to our
wishes because a vector field is sufficiently flexible that its curl and
divergence can both be specified arbitrarily. Consider then the divergence of
the most general vector potential. We write this potential as
\beq
\Ax=\frac1c\inivp\frac{\J(\xp)}\xxpa+\grad\chx.
\eeq
Then
\beq
\div\Ax=\frac1c\inivp\J(\xp)\cdot\grad\lep\xxpi\rip+\lap\chx.
\eeq
Now change the $\grad$ operator to $-\grad'$ and then do an integration by
parts. The surface integral may be discarded if the volume integral is over
all space and $\Jxp$ is localized (so that it vanishes on the surface of
the volume of integration). What one then has left is
\beq
\div\Ax=\frac1c\inivp\frac{\divp\J(\xp)}\xxpa+\lap\chx=\lap\chx,
\eeq
where the last step follows from the fact that a time-independent set of
sources give a current distribution having zero divergence.

From this result we can see two things: first, the
vector potential has zero divergence if we forget about $\chx$, \ie if we
set it equal to zero. Second, we can make the divergence be any
scalar function $f(\x)$ we want by setting $\lap\chx=f(\x)$. We would then
have to solve for $\chx$ which is in principle straightforward because the
differential equation for $\ch$ is just the Poisson equation and we know
how to solve that. The solution is
\beq
\chx=-\frac1{4\pi}\inivp\frac{f(\xp)}\xxpa.
\eeq

Specifying the divergence of $\Ax$ is called choosing the {\em gauge} of
the vector potential, and changing the function $\chx$ is called making a
{\em gauge transformation}. Notice that one can make a transformation
without changing the gauge (\ie without changing $\div\A$); to do this one
must change $\ch$ by a function which satisfies the Laplace equation. The
particular gauge specified by $\div\Ax=0$ is called the
{\em Coulomb gauge}.

Returning to our original point, we have found that we can pick a vector
potential with zero divergence. In particular,
\beq
\Ax=\frac1c\inivp\frac{\J(\xp)}\xxpa
\eeq
has this property. In this, the Coulomb gauge, the vector potential
satisfies the equations
\beq
\lap\Ax=-\frac{4\pi}c\Jx,
\eeq
which {\bf{must}} be interpreted as
\beq
\lap A_i(\x)=-\frac{4\pi}cJ_i(\x)
\eeq
where the subscript $i$ denotes any Cartesian component of $\A$.

\subsection{Example: A Circular Current Loop}

Consider a circular loop of radius $a$ carrying a current $I$. Let the loop
lie in the $z=0$ plane and be centered at the origin of coordinates. 

\centerline{\psfig{figure=fig6.ps,height=2.0in,width=6.375in}}

\noindent Then
we may write $\Jx=J_\ph(\x)\phh$ with
\beq
\J_\ph(\x)=\frac Ia\de(\cos\th)\de(r-a).
\eeq
It is a natural temptation to do the following INCORRECT thing:
write $\Ax=A_\ph(\x)\phh$ (correct so far) with
\beq
A_\ph(\x)=\frac1c\inivp\frac{J_\ph(\xp)}\xxpa.
\eeq
This integral is not correct because the unit vector $\phh$ at the field
point $\x$ is not the same as the unit vector $\phh'$ at the field point
$\xp$.

What we can do instead is to make use of the azimuthal symmetry of the
system and trivially generalize to all values of $\ph$ after having first
evaluated the vector potential at some particular value of $\ph$.
Let us look at the potential at $\ph=0$, \ie in the $x$-$z$ plane. Here,
the vector potential will be in the $y$ direction, so we can say that

\beq
A_\ph(r,\th)=A_y(r,\th,\ph=0)=\frac1c\inivp\frac{J_y(\xp)}\xxpa
\eeq
at $\ph=0$. Now, $J_y(\xp)=J_{\ph'}\cos\ph'$, so
\beq
A_\ph(r,\ph)=\frac{I}{ac}\inivp\frac{\de(r'-a)\de(\cos\th')\cos\ph'}{(r^2+
a^2-2ar\cos\ga)^{1/2}}
\eeq
where $\ga$ is the angle between the directions of $\x$ and $\xp$,
$\cos\ga=\sin\th\cos\ph'$. 

The integral is an elliptic integral; rather
than deal with its arcane properties, we will expand it in the usual way:
\beqa
A_\ph(r,\th)=\frac{I}{ca}\inivp\frac{\de(r'-a)\de(\cos\th')\cos\ph'}
{\xxpa}\nonumber\\=Re\lec\frac{I}{ca}\inivp\frac{\de(r'-a)\de(\cos\th')}
{\xxpa} e^{i\ph'}\ric\nonumber\\=Re\lec\frac{I}{ca}\inivp\de(r'-a)\de(\cos
\th')e^{i\ph'}\sum_{l,m}\frac{4\pi}{2l+1}\frac{r_<^l}{r_>^{l+1}}\ylmsp
\ylm|_{\ph=0}\ric\nonumber\\=Re\lec\frac{Ia}{c}\sum_{l,m}\frac{(l-m)!}{(l+m
)!}P_l^m(\cos\th)P_l^m(0)\frac{r_<^l}{r_>^{l+1}}\int_0^{2\pi}d\ph'e^{i\ph'}
e^{-im\ph'}\ric\nonumber\\=\frac{2\pi Ia}{c}\sum_{l=1}^\infty\frac{P_l^1(0)
P_l^1(\cos\th)}{l(l+1)}\frac{r_<^l}{r_>^{l+1}}
\eeqa
where $r_<$ and $r_>$ refer to the smaller and larger of $r$ and $a$.
Next one notes that $P_l^1(0)=0$ for $l$ even and
\beq
P_l^1(0)=(-1)^{(l+1)/2}l!/[(l-1)!!]^2
\eeq
for $l$ odd; set $l=2n+1$ and have
\beqa
A_\ph(r,\th)=\frac{2\pi Ia}{c}\sum_{n=0}^\infty\frac{(-1)^{n+1}(2n+1)!}{
(2n+1)(2n+2)[(2n)!!]^2}\frac{r_<^{2n+1}}{r_>^{2n+2}}P_{2n+1}^1(\cos\th)
\nonumber\\=\frac{\pi  Ia}{c}\sum_{n=0}^\infty\frac{(-1)^{n+1}(2n-1)!!}{2^n
(n+1)!}\frac{r_<^{2n+1}}{r_>^{2n+2}}P_{2n+1}^1(\cos\th).
\eeqa
At large $r$, \ie $r>>a$, we can keep just the leading term (n=0) and find
\beq
A_\ph(r,\th)=\frac{\pi Ia}{c}\frac{a}{r^2}\sin\th.
\eeq
The corresponding components of the magnetic induction are
\beq
B_r(r,\th)=\frac1{r\sin\th}\pde{}{\th}(\sin\th A_\ph)=\frac{2\pi Ia^2}{cr^3}
\cos\th,
\eeq
and
\beq
B_\th(r,\th)=-\frac1r\pde{}{r}(rA_\ph)=\frac{\pi Ia^2}{cr^3}\sin\th.
\eeq

\centerline{\psfig{figure=fig7.ps,height=2.0in,width=6.375in}}

We recognize these as having the same form as the 
field of an electric
powers of $a/r$, the current loop is treated as a {\em magnetic dipole}
with {\em magnetic dipole moment} $\mdm\equiv (\pi I a^2/c)\zh$; the
corresponding vector potential, \eq{58},  is
\beq
\Ax=A_\ph(r,\th)\phh=\frac{|\mdm|\sth}{r^2}\phh=\frac{\mdm\times \x}
{|\x|^3}.
\eeq
We may compare this potential to that of an electric dipole which is
\beq
\Phx=\frac{\edm\cdot\x}{|\x|^3}.
\eeq

\section{The Field of a Localized Current Distribution}
The example of the preceding section is a special case of the field of a
localized current distribution. Let us now suppose that we have some such
distribution $\Jxp$ around the origin of coordinates. We shall find the
potential of this distribution in the dipole approximation. To this end we
must expand the function
\beq
\xxpi\approx\frac1{|\x|}\leb1+\frac{\x\cdot\xp}{|\x|^2}+...\rib
\eeq
in powers of $r'/r$.

\centerline{\psfig{figure=fig8.ps,height=1.5in,width=6.375in}}

Of particular interest to us is the first-order term in the brackets.
Using this expansion, we find that the vector potential is
\beq
\Ax=\frac1c\inivp\Jxp\leb\frac1{|\x|}+\frac{\x\cdot\xp}{|\x|^3}+...\rib.
\eeq
The first term $\sim\inivp\Jxp$ is zero for a localized steady current
distribution (Why?)\footnote{The short answer is just that in lieu of sources
or sinks of charge the total current density in a  current distribution must 
be zero.  More formally, consider the 3-vector with components $U_i$
\[
U_i=0=\inv x_i\div\J=\inv\leb \div\lep x_i \J\rip-\J\cdot\grad x_i\rib
\]
Using the divergence theorem, this becomes
\[
\ina x_i \J\cdot\nn -\inv \J\cdot{\hat{\x}_i}
\]
which follows since $\grad x_i={\hat{\x}_i}$.  Then if we take the surface to
infinity, where there is no current density, we obtain the desired result
\[
0=-\inv J_i
\]
}
; the second one is
\beq
\A_d(\x)=\frac1{c|\x|^3}\inivp(\x\cdot\xp)\Jxp.
\eeq
However, $(\x\cdot\xp)\J=-\x\times(\xp\times\J)+(\x\cdot\J)\xp$, so
\beq
\A_d(\x)=\frac1{c|\x|^3}\lec\inivp(\x\cdot\Jxp)\xp-\x\times\inivp(\xp\times
\Jxp)\ric.
\eeq
Consider the $j^{th}$ component of the first integral:
\beq
\inivp x_j'(\x\cdot\Jxp)=\sum_{i=1}^3\inivp x_j'x_iJ_i(\xp).
\eeq
Now, it is the case that
\beq
\divp(x_i'\Jxp)=(\grad'x_i')\cdot\Jxp+x_i'(\divp\Jxp)=J_i(\xp),
\eeq
since the divergence of $\J$ is zero. This allows us to write
\beqa
\inivp(\x\cdot\Jxp)x_j'=\sum_{i=1}^3x_i\inivp[\divp(x_i'\Jxp)]x_j'
\nonumber\\ \hbox{Parts integration:}\nonumber \\
=-\sum_{i=1}^3x_i\inivp(x_i'\Jxp)\cdot(\grad'x_j')=-\inivp(\x\cdot\xp)J_j(
\xp).
\eeqa
Generalizing now to all three components of this integral, we find
\beq
\inivp(\x\cdot\Jxp)\xp=-\inivp(\x\cdot\xp)\Jxp.
\eeq
Comparison with the expressions for $\A_d$ shows that
\beq
\A_d(\x)=-\frac1{2c}\frac1{|\x|^3}\x\times\leb\inivp(\xp\times\Jxp)\rib
\equiv\frac{\mdm\times\x}{|\x|^3}=\curl\lep\frac{\mdm}{|\x|}\rip
\eeq
where the {\em magnetic dipole moment} of the current distribution is defined as
\beq
\mdm\equiv\frac1{2c}\inivp(\xp\times\Jxp).
\eeq
The {\em magnetic moment density} of the distribution is defined by
\beq
{\cal M}(\x)\equiv\frac1{2c}\x\times\Jx
\eeq
so that the magnetic moment is the integral of this density over all space.

The field produced by the source in the magnetic dipole approximation is
\beq
\B_d(\x)=\curl\A_d(\x)=\curl\lep\frac{\mdm\times\x}{|\x|^3}\rip.
\eeq
Using the identity
\beq
\curl(\A\times\B)=\A(\div\B)-\B(\div\A)+(\B\cdot\grad)\A-(\A\cdot\grad)\B,
\eeq
we can write this equation as
\beq
\B_d(\x)=\frac{3\nn(\nn\cdot\mdm)-\mdm}{|\x|^3}
\eeq
where $\nn=\x/|\x|$. 

	This is, as was the case for the electric dipole, only
the field outside of the current distribution. For a point magnetic dipole,
it would be the field away from the location of the dipole (the origin),
and at the origin there is a delta-function field as for the point electric
dipole. We may find the magnitude and direction of this singular field by
a more careful analysis of what happens as $r\rightarrow0$. To this end it
is useful to write the vector potential of the dipole as
\beq
\A_d(\x)=\curl\lep\frac{\mdm}{|\x|}\rip.
\eeq
Then we can write

\beq
\Bx=\curl\Ax=\curl\leb\curl\lep\frac\mdm r\rip\rib=\grad\leb\div\lep\frac
\mdm r\rip\rib-\lap\lep\frac\mdm r\rip.
\eeq
The last term on the right-hand side is just $4\pi\mdm\de(\x)$. The first
one we have seen before, as it is the same as the electric field of an
electric dipole; we already know what singularity is contained therein but
will figure it out again as an exercise. Start by integrating this term
over a small sphere of radius $\ep$ centered at the origin and then take
the limit as $\ep\rightarrow0$:
\beq
\int_{r<\ep}d^3x\grad\leb\div\lep\frac\mdm r\rip\rib=\int_{r=\ep}d^2x
\nn\leb\div\lep\frac\mdm r\rip\rib,
\eeq
where we have used the identity, valid for any scalar function $f(\x)$:
\beq
\iniv\grad f(\x)=\inia\nn f(\x);
\eeq
S is the surface enclosing the domain V and $\nn$ is the usual outward unit
normal. Continuing, we have
\beq
\int_{r=\ep}d^2x\nn\leb\div\lep\frac\mdm r\rip\rib=-\int_{r=\ep}d^2x\,\rhh\lep
\frac{\mdm\cdot\x}{r^3}\rip=-\frac{4\pi}3\mdm.
\eeq
Hence we find that $\grad[\grad\cdot(\mdm/r)]$ contains the singular piece
$(4\pi/3)\mdm\de(\x)$. Putting it into \eq{78}, we conclude that the
delta-function piece of the magnetic field is $(8\pi/3)
\mdm\de(\x)$, and hence the total field of the magnetic dipole is
\beq
\Bx=\frac{3(\nn\cdot\mdm)\nn-\mdm}{r^3}+\frac{8\pi}3\mdm\de(\x).
\eeq

The consequences of the presence of the delta-function piece are observed
in atomic hydrogen where the magnetic moment of the electron interacts with
that of the nucleus, or proton. Without this interaction, all total-spin
states of the atom would be degenerate. As a consequence of the
interaction, the ``triplet'' or ``spin-one'' states are raised slightly in
energy relative to the ``singlet'' or ``spin-zero'' state. The splitting is
small even on the scale of atomic energies, being about $10^{-17} erg$ or
$10^{-5}\;ev$. A photon which has the same energy as this
splitting, and which is thus produced or absorbed by a transition between
the two atomic energy levels, has a wavelength of 21 $cm$. The delta-function
part of the field also plays an important role in the scattering of neutrons
from magnetic materials.

\section{Forces on a Localized Current Distribution}
We shall look next at the interaction of a localized current distribution
with an externally applied field, again using procedures reminiscent of the
multipole expansion for a localized charge distribution. First, we expand
the applied field around some suitably chosen origin (at the center of
the current distribution),

\centerline{\psfig{figure=fig9.ps,height=1.5in,width=6.375in}}

\beq
\Bx=\B(0)+(\x\cdot\grad')\B(\xp)|_{\xp=0}+...;
\eeq
The force that the field exerts on a localized current distribution
located around the origin is then expanded as follows:
\beqa
\F=\frac1c\iniv(\Jx\times\Bx)\nonumber\\
=-\frac1c\B(0)\times\iniv\Jx+\frac1c\iniv\Jx\times[(\x\cdot\grad')\B(\xp)
|_{\xp=0}]+...  .
\eeqa
Now, the first integral in the last line vanishes for a localized
steady-state current distribution (there can't be any net flow of charge
in any direction), and we can manipulate the integrand in the final
integral as follows:
\beq
(\x\cdot\grad')\B(\xp)=\grad'(\x\cdot\B(\xp))-\x\times(\grad'\times\B(\xp))
,\eeq
and we may suppose that $\B$ is due entirely to external sources so that
$\grad'\times\B(\xp)=0$ for $\xp$ around the origin. Thus we find that the
force is
\beqa
\F=\frac1c\iniv\Jx\times[\grad'(\x\cdot\B(\xp))|_{\xp=0}]\nonumber\\
=-\frac1c\iniv\grad'\times[(\x\cdot\B(\xp))|_{\xp=0}\Jx]\nonumber\\
=-\frac1c\grad'\times\iniv[\x\cdot\B(\xp))|_{\xp=0}\Jx].
\eeqa
Now, we can write the last integral as
\beq
\iniv(\x\cdot\B(\xp))\Jx=-\iniv[\B(\xp)\times(\x\times\Jx)-\x(\B(\xp)\cdot
\Jx)].
\eeq
Further, it is true that
\beq
\iniv\x(\B(\xp)\cdot\Jx)=-\iniv(\x\cdot\B(\xp))\Jx;
\eeq
we can demonstrate this fact by considering the $i^{th}$ component of the
integral.
\beqa
-\iniv(\x\cdot\B(\xp))J_i(\x)=-\iniv(\x\cdot\B(\xp))\div(x_i\Jx)\nonumber\\
=\iniv[\grad(\x\cdot\B(\xp))]\cdot(x_i\Jx)=\iniv\B(\xp)\cdot(x_i\Jx)\nonumber
\\=\iniv x_i(\B(\xp)\cdot\Jx).
\eeqa
Hence, from \eqss{84}{85}{86} we find that
\beqa
\F=\frac1{2c}\grad'\times\lec\iniv[\B(\xp)\times(\x\times\Jx)]\ric|_{\xp=0}
\nonumber\\=\frac1{2c}\grad'\times\lec\B(\xp)\times\iniv(\x\times\Jx)
\ric|_{\xp=0}\nonumber\\
=\grad'\times(\B(\xp)\times\mdm)|_{\xp=0}=(\mdm\cdot\grad')\B(\xp)|_{\xp=0}-
\mdm(\grad'\cdot\B(\xp))|_{\xp=0}\nonumber\\
=\grad'(\mdm\cdot\B(\xp))|_{\xp=0}-\mdm\times(\grad'\times\B(\xp))|_{\xp=
0}.
\eeqa

Along the way in this derivation we have made use of the facts that the
divergence and curl of $\B$ are zero in the region near the origin. The
final result has the form of the gradient of a scalar function,
\beq
\F=-\grad(-\mdm\cdot\Bx)
\eeq
where the gradient is to be evaluated at the center of the current
distribution. Notice in particular that there is no force if the applied
magnetic induction is uniform. More generally, the force is in the
direction of the gradient of the component of $\B$ in the direction of
$\mdm$.

The torque $\N$ on the current distribution may be found using similar
manipulations.
\beqa
\N=\frac1c\iniv\,\x\times(\Jx\times\Bx)\approx\frac1c\iniv\,\x\times(\Jx\times
\B(0))\nonumber\\
=\frac1c\iniv\,[(\x\cdot\B)\Jx-(\x\cdot\Jx)\B]=
-\frac1{2c}\iniv\B\times(\x\times\Jx
)\nonumber\\-\frac1c\B\iniv(\x\cdot\Jx)=\mdm\times\B-\frac1c\B\iniv(\x
\cdot\Jx)
\eeqa
However,
\beq
\div(r^2\Jx)=2\x\cdot\Jx+r^2(\div\Jx);
\eeq
the final term here is zero, so it is the case that
\beq
\iniv(\x\cdot\Jx)=\frac12\iniv\div(r^2\Jx)=0,
\eeq
the final step following from the fact that the current distribution is
localized. Thus the torque on the localized source reduces to
\beq
\N=\mdm\times\B,
\eeq
in the dipole approximation. Compare this result with \eq{1}.

\section{Macroscopic Magnetostatics}
If one has a macroscopic object with lots of microscopic currents flowing
around in it (electrons on molecules give many small current loops), then
it is best to carry out suitable averages over volumes small compared to
anything macroscopic but large compared to molecules and to calculate only
the average quantities. The elementary procedure, in which many
complications are blissfully ignored, is to break the sources up into
free and bound:
\begin{enumerate}
\item the {\em macroscopic current density}, written as $\Jx$
\item the
{\em magnetization} $\Mx$ which is defined as the magnetic dipole moment per
unit volume. 
\end{enumerate} 
The latter comes from the molecules, atoms, ions, etc., in a
material medium. The macroscopic current density comes from moving charges
which are not bound on molecules, \ie not localized. Notice that within
this approach we can accommodate sources which cannot actually be described
as the motion of charges, or currents; that is, the spin magnetic moments of
``elementary'' particles such as the electron. These may be treated as
point dipoles and so simply contribute to the magnetization.

Given these sources, use of \eqs{48}{71} tell us that the (macroscopic)
vector potential is
\beq
\Ax=\frac1c\inivp\leb\frac\Jxp\xxpa+\frac{c\Mxp\times\xxp}{\xxpa^3}\rib.
\eeq
We can manipulate the term involving the magnetization in a manner which
should be familiar:
\beqa
\Ax=\frac1c\inivp\lec\frac\Jxp\xxpa+\frac{c\Mxp\times\xxp}{\xxpa^3}\ric
\nonumber\\=\frac1c\inivp\lec\frac\Jxp\xxpa+c\Mxp\times\grad'\lep\frac1\xxpa
\rip\ric\nonumber\\=\frac1c\inivp\lec\frac{\Jxp+c(\grad'\times\Mxp)}\xxpa\ric
-\inivp\grad'\times\lep\frac\Mxp\xxpa\rip
\eeqa
However, there is an identity which allows us to convert the integral of a
curl to a surface integral:
\beq
\inv\curl\V=\inac\nn\times\V.
\eeq
In the present instance, when the last integral in \eq{97} is converted to a
surface integral, it will vanish because if the surface encloses all of the
magnetic materials, $\M$ will be zero on the boundary. Hence we have the
result
\beq
\Ax=\frac1c\inivp\,\frac{\Jxp+c[\grad'\times\Mxp]}\xxpa
\eeq
for the macroscopic vector potential.

\subsection{Magnetization Current Density}

The form of this equation implies that $c\curl\Mx$ is a current density 
associated with spatial variations of the magnetization. 
It is possible to imagine how the curl of the magnetization yields a current
source, if we consider each magnetic dipole as originating from a vanishingly
small current loop.

\centerline{\psfig{figure=fig10.ps,height=1.9in,width=6.375in}}

\noindent This may or may not
really be the case. If the magnetization is the consequence of the motions of
bound charges on molecules, then it is indeed a current. But if the
magnetization is the consequence of the intrinsic dipole moments of elementary
particles such as the electron, then it is not reasonable to think of the curl
of the magnetization as a current density. Nevertheless, we shall define
the {\em magnetization current density} $\J_M(\x)$ as\footnote{Notice that
this object has zero divergence, as a steady-state current density should.}
\beq
\J_M(\x)=c\curl\Mx.
\eeq
In terms of the magnetization current density, we have
\beq
\Ax=\frac1c\inivp\frac{\Jxp+\J_M(\xp)}\xxpa.
\eeq

The differential field equations obeyed by the {\em macroscopic magnetic
induction}, which is the curl of the macroscopic vector potential,
\beq
\Bx=\curl\Ax,
\eeq
are easy to write down by referring to our earlier derivation; they are
\beq
\curl\Bx=\frac{4\pi}c(\Jx+\J_M(\x))
\eeq
and, of course,
\beq
\div\Bx=0.
\eeq

\subsection{Magnetic Field}

The difficulty one faces in solving these equations is much the same as
faced in macroscopic electrostatics. One generally does not know
magnetization until after solving for the induction. Further, the relation
between $\Bx$ and $\Mx$ depends on the material. It is customary to define
an additional macroscopic field $\Hx$, called the {\em magnetic field},
\beq
\Hx\equiv\Bx-4\pi\Mx.
\eeq
Then \eq{103} can be rewritten as
\beq
\curl\Bx=\frac{4\pi}c(\Jx+c\curl\Mx)
\eeq
or
\beq
\curl(\Bx-4\pi\Mx)=\frac{4\pi}c\Jx
\eeq
or
\beq
\curl\Hx=\frac{4\pi}c\Jx.
\eeq
We haven't resolved anything by making this definition, of course; we have
simply phrased the problem in a somewhat different form. The advantage in
introducing the magnetic field is that it obeys a relatively simple
differential equation and that, in many materials, there is a simple
approximate relation between $\B$ and $\H$. The simplest materials are,
like dielectrics, linear, isotropic, and uniform, in which case the relation
between $\B$ and $\H$ is
\beq
\Hx=\frac1\mu\Bx
\eeq
where $\mu$ is a constant known as the {\em magnetic permeability}. This
positive constant can be smaller or larger than unity, leading to two
classifications of magnetic materials: If $\mu<1$, the material is said to
be a {\em diamagnet}; if $\mu>1$, it is called a {\em paramagnet}. 

\centerline{\psfig{figure=diapara.ps,height=2.5in,width=4.25in}}

\noindent For
linear, isotropic materials, it is also common to introduce the
{\em magnetic susceptibility} $\ch_m$,
\beq
\ch_m\equiv\frac{\mu-1}{4\pi}
\eeq
so that $\M=\ch_m\H$.

Many materials have dramatically different relations among $\B$, $\M$, and
$\H$. {\em Ferromagnets} are a prime example; they can have a finite
magnetic induction with zero magnetic field as well as the converse. In
addition, as shown in the figure below
the relation between these two fields in ferromagnets is usually not
single-valued; the particular magnetic field one finds for a given value of
$\B$ depends on the ``history'' of the sample, meaning it depends on what
external fields it was subjected to prior to determining $\B$ and $\H$. We
will not concern ourselves with the origins of this behavior.

\centerline{\psfig{figure=fig12.ps,height=2.0in,width=4.25in}}
 
\subsection{Boundary Conditions}
We have the general differential equations of magnetostatics
\beq
\curl\Hx=\frac{4\pi}c\Jx
\eeq
and
\beq
\div\Bx=0.
\eeq

These are valid for any system, independent of the particular relation
between the magnetic induction and the magnetic field. From them we can
derive general boundary or continuity conditions. From the divergence
equation, we infer that the normal component of the magnetic induction is
continuous at an interface between two materials,

\centerline{\psfig{figure=bc1.ps,height=2.5in,width=5.0in}}

\beq
(\B_2-\B_1)\cdot\nn=0
\eeq
where the subscripts 1 and 2 refer to the induction in each of two
materials meeting at an interface, and $\nn$ is a unit vector normal to the
interface and pointing into medium 2.

From the curl equation, one finds that the tangential components of $\H$ can
be discontinuous. The discontinuity is related to the current flowing along
(parallel to) the interface. Consider Stokes' theorem as applied to the
curl equation. 

\centerline{\psfig{figure=bc2.ps,height=3.0in,width=5.0in}}

\noindent Thus,
\beq
\ina[\curl\Hx]\cdot\nn'=\invlc\cdot\Hx=\frac{4\pi}c\ina\Jx\cdot\nn'
\eeq
where the unit vector $\nn'$ is directed normal to the surface S over which
the integration is done. Using a rectangle of dimensions $a$ by $h$, where
$a<<h$ and $a$ is directed perpendicular to the interface while $h$ is
parallel to it, we find that the line integral comes down to
\beq
\invlc\cdot\Hx=h(\H_2-\H_1)\cdot(\nn'\times\nn)
\eeq
which is to say, we find the discontinuity in a particular tangential
component of $\H$ across the interface. The integral over the current, on
the other hand, gives, for a finite, slowly varying current,
\beq
\frac{4\pi}c\ina\Jx\cdot\nn'\sim ah\J\cdot\nn';
\eeq
this is proportional to $a$ and so is insignificant for $a<<h$ assuming a
finite well-behaved current density. In this case, there is no
discontinuity in the tangential components of $\H$,
\beq
(\H_2-\H_1)\times\nn=0.
\eeq

There is also the possibility of a singular term in $\Jx$ which would be of
the form
\beq
\J_s(\x)=\K(\x)\de(\xi)
\eeq
where $\xi$ is the distance from the interface and the vector $\K$ points
in a direction parallel to the interface. This vector is a {\em
surface-current density} and has dimensions of charge per unit length per
unit time. This expression is, of course, an idealization. When currents
run along the surface of a material, they typically are not localized
precisely at the surface (\ie, within an atomic size) but are spread over a
surface layer of thickness ranging from a few hundred Angstroms to some
microns. If the length $a$ is significantly larger than this layer's
thickness, then it is reasonable to talk about a surface-current density
and quite acceptable to regard it as being localized at the interface.
Proceeding on this basis, we find that the continuity condition becomes
\beq
h(\H_2-\H_1)\cdot(\nn'\times\nn)=h\frac{4\pi}{c}\K\cdot\nn'.
\eeq
This relation can be written in the more general form
\beq
\nn\times(\H_2-\H_1)=\frac{4\pi}c\K
\eeq
as may be shown by (1) realizing that $\K\cdot\nn=0$ and (2) taking the inner
product of \eq{120} with any vector (such as $\nn'$) lying in the
plane of the interface.

\section{Examples of Boundary-Value Problems in Magnetostatics}

\subsection{Uniformly Magnetized Sphere}

The principal example is a uniformly magnetized sphere meaning a material that
maintains a constant magnetization $\M_0$ in the absence of any applied
field. 

\centerline{\psfig{figure=mag_sphere.ps,height=1.5in,width=1.5in}}

\noindent We shall let the direction of $\M$ be the z-direction and so have
\beq
\Mx=\lec\barr{cc} M_0\zh  & r<a\\0 & r>a.\ear \right.
\eeq
In the region $r>a$, $\B=\H$. For $r<a$, $\B=\H+4\pi\M$. However, $\M$ is
constant here and so has no curl or divergence. This means that the curls
of both $\B$ and $\H$ are zero everywhere except at the boundary of the
sphere.

Consider the magnetization current density, $\J_M=c\curl\M$; it is non-zero
only at $r=a$ where it is singular. By applying Stokes' theorem to this
equation in the same manner as was done to find the continuity condition on
the tangential components of $\H$ at an interface, one finds that there is
a magnetization surface-current density $\K_M$ given by
\beq
\K_M=c\nn\times(\M_2-\M_1)
\eeq
where $\nn$ is the unit normal at the surface pointing into material 2.
In the present application, $\M_1=\M_0$ and $\M_2=0$. Since $\rhh\times
\zh=-\sth\phh$, we have $\K_M=K_M\phh$ where $K_M=cM_0\sth$.

\subsubsection{Scalar Potential for the Induction}
Now that we have identified the sources, let's look at some methods of
solution. First, consider a scalar potential approach. Because the curl of
$\B$ is zero for $r<a$ and $r>a$, we can devise scalar potentials for the
magnetic induction in these two regions. Because $\div\Bx=0$, the
potentials satisfy the Laplace equation. Further, the system is invariant
under rotation around the $z$-axis, implying that the potential is
independent of $\ph$. Hence it must be possible to write
\beq
\Ph_<(\x)=M_0a\sum_lA_l\lep\frac ra\rip^lP_l(\cth) \hbox{ for $r<a$}
\eeq
 and
\beq
\Ph_>(\x)=M_0a\sum_lC_l\lep\frac ar\rip^{l+1}P_l(\cth) \hbox{ for $r>a$}.
\eeq
Given that $\B=-\grad\Phx$, the condition that the normal component of $\B$
is continuous at $r=a$ becomes (making use of the orthogonality of the
Legendre polynomials in the usual way)
\beq
lA_l=-(l+1)C_l
\eeq
for all $l$. The other boundary condition is that the tangential component
of $\H$, or $H_\th$ is continuous. Since $\B=\H+4\pi\M$, and since the
$\th$-component of the magnetization is $-M_0\sth$ (and $\sth=-dP_1(\cth)/d
\th$, this second condition leads to
\beq
A_l=C_l \hbox{ for $l$ not equal to 1}
\eeq
and
\beq
C_1=A_1+4\pi.
\eeq
>From these equations it is easy to see that $A_l=C_l=0$, $l\ne1$, while
$C_1=4\pi/3$ and $A_1=-8\pi/3$.

Having found the potential, we may compute the fields. The magnetic
induction at $r>a$ has the familiar dipolar form; the field within the
sphere is a constant:
\beq
\B_<=\frac{8\pi}3\M_0 \hbox{ and } \H_<=-\frac{4\pi}3\M_0.
\eeq

\subsubsection{Scalar Potential for the Field}
A second approach to this problem is to devise a scalar potential for the
magnetic field. Since $\Jx=0$ everywhere, the curl of $\H$ is zero
everywhere and so there is a potential $\Ph_H$ for $\H$ {\bf everywhere}
such that $\Hx=-\grad\Ph_H(\x)$. The
divergence of $\H$ obeys
\beq
\div\Hx=-\lap\Ph_H(\x)=\div\Bx-4\pi\div\Mx=-4\pi\div\Mx.
\eeq
Looking upon this as a Poisson equation, we can immediately see that the
solution for the potential is
\beq
\Ph_H(\x)=-\inivp\frac{\divp\Mxp}\xxpa=-\inivp\leb\divp\lep\frac\Mxp\xxpa
\rip-\Mxp\cdot\grad'\lep\frac1\xxpa\rip\rib
\eeq
The first term in the final expression may be converted to a surface
integral which is seen to be zero from the fact that the magnetization is
non-zero only within the sphere. Hence
\beqa
\Ph_H(\x)=M_0\zh\cdot\int_{r'<a}d^3x'\grad'\lep\frac1\xxpa\rip
=-\div\leb M_0\zh\int_{r'<a}\frac{d^3x'}\xxpa\rib\nonumber\\
=-\div\leb M_0\zh\int_0^ar'^2dr'4\pi\frac1{r_>}\rib
\eeqa
where $r_>$ is the larger of $a$ and $r$. For $r>a$, we find
\beq
\Ph_H(\x)=-\div\lep M_0\zh 4\pi\frac{a^3}{3r}\rip=\frac{4\pi}3a^3M_0
\frac{\cth}{r^2},
\eeq
and for $r<a$,
\beq
\Ph_H(\x)=-\div\leb M_0\zh4\pi\lep\frac{a^2}2-\frac{r^2}6\rip\rib
=\frac{4\pi}3M_0z.
\eeq

Notice how the preceding calculation avoids having to think about what happens
at the surface of the sphere; the integration by parts leaves us with a simpler
integration over the magnetization. It is, of course, possible to evaluate the
divergence of the magnetization and complete the integral directly without
using the integration by parts. To this end, write the magnetization as
\beq
\Mx=M_0\zh\th(a-r)
\eeq
where the $\th$-function is a step function,
\beq
\th(x)=\lec\barr{cc} 1,  & x>0\\0,   &  x<0.\ear\right.
\eeq
Then
\beq
\div\Mx=M_0\pde{}{z}\th(a-r)=-M_0\de(a-r)\pde rz
\eeq
since
\beq
\der{\th(x)}x=\de(x).
\eeq
Hence
\beqa
\Ph_H(\x)=\inivp\frac{\divp\Mxp}\xxpa=M_0a^2\int d\Om'\frac{\cth'}\xxpa \\
\nonumber\\=M_0a^2\frac{4\pi}3\frac3{4\pi}\frac{4\pi}3\frac{r_<}{r_>^2}
\cth=\frac{4\pi}3 M_0a^2\frac{r_<}{r_>^2}\cth.
\eeqa
Here, $\xp$, after integrating over $r'$, has the magnitude $a$.

\subsubsection{Direct Calculation of $\B$}
Another approach to this problem is to calculate the vector potential for $
\Bx$ directly. Since $\Jx=0$ everywhere, only the curl of the magnetization
acts as a source of this potential. Thus we have
\beqa
\Ax=\frac1c\inivp\frac{c\grad'\times\Mxp}\xxpa=\inivp\leb\grad'\times\lep
\frac\Mxp\xxpa\rip-\grad'\lep\frac1\xxpa\rip\times\Mxp\rib\nonumber\\
=\inacp\nn'\times\lep\frac\Mxp\xxpa\rip+\curl\lep\inivp\frac\Mxp\xxpa\rip
\nonumber\\=\curl M_0\zh4\pi\lec\barr{cc} a^3/3r, & r>a\\
a^2/2-r^2/6, & r<a. \ear\right.
\eeqa
One may easily work out the curl to find
\beq
\Ax=\frac{4\pi}3M_0a^3\frac\sth{r^2}\phh \hbox{ for $r>a$}
\eeq
and
\beq
\Ax=\frac{4\pi}3M_0r\sth\phh \hbox{ for $r<a$.}
\eeq
These expressions may be summarized as
\beq
\Ax=\frac{4\pi M_0}3a^2\lep\frac{r_<}{r_>^2}\rip\sth\phh.
\eeq

The preceding example is a typical one involving a permanently magnetized
material. Then one knows $\Mx$ and so can proceed with a solution of the
equations for $\B$ or $\H$ via the route of one's choice. A quite different
sort of problem is one involving linear, isotropic magnetic materials with
$\Jx=0$. These are entirely equivalent to problems in macroscopic
electrostatics (Laplace equation problems) with which we have extensive
experience. The point is that one has
\beq
\curl\Hx=0
\eeq
 which means there is a potential $\Ph_H(\x)$ such that
\beq
\Hx=-\grad\Ph_H(\x).
\eeq
Further, since $\div\Bx=0$, and $\Bx=\mu\Hx$, we have
\beq
0=\div\Bx=\div(\mu\Hx)=(\grad\mu)\Hx+\mu\div\Hx=-(\grad\mu)\grad\Ph_H(\x)
-\mu\lap\Ph_H(\x),
\eeq
or
\beq
\lap\Ph_H(\x)=-\frac{\grad\mu}{\mu}\grad\Ph_H(\x).
\eeq
In any region of space where $\mu$ is a constant, this is just the Laplace
equation,
\beq
\lap\Ph_H(\x)=0.
\eeq
Hence we may write down an appropriate solution of the Laplace equation for
$\Ph_H(\x)$ in each region of space where the magnetic properties are
uniform and then make sure the continuity conditions on the tangential
components of $\Hx$ and on the normal components of $\Bx$ are satisfied on
the boundaries between such regions. The techniques used are the same as
for boundary value problems in electrostatics.

Notice that in problems of this kind, we can equally well make a scalar
potential for $\Bx$ in each of the regions where $\mu$ is constant.

\subsection{Shielding by a Paramagnetic Cylinder}
A standard example is magnetic shielding in which a shell of magnetic
material with a very large value of $\mu$ (a strongly paramagnetic
material) is placed around some region of space. Suppose, for example, that
a long cylindrical shell of inner radius $a$ and outer radius $b$ is placed
around the z-axis and that this system is subjected to a transverse applied
field $\H_0=H_0\xh$. Then we have three regions, $\rh<a$, $a<\rh<b$, and
$b<\rh$, in which we may make potentials $\Ph_H$ which satisfy the Laplace
equation. These potentials will have the forms\newline
(i) $\rh<a$:
\beq
\Ph_H(\x)=H_0A\rh\cph
\eeq
(ii) $a<\rh<b$:
\beq
\Ph_H(\\x)=H_0\leb C\rh\cph+Da^2\cph/\rh\rib
\eeq
(iii) $b<\rh$:
\beq
\Ph_H(\x)=H_0\leb -\rh\cph+Eb^2\cph/\rh\rib.
\eeq
Here, we have guessed (on the basis of long experience) that the only terms
in the potential will be the ones with the same dependence on $\ph$ as the
potential of the applied field. The latter is $-H_0x=-H_0\rh\cph$.

Continuing, one has the boundary conditions at $\rh=a$ and $\rh=b$ which
are that $H_\ph$ and $B_\rh$ must be continuous. These lead to the four
equations
\beqa
A=C+D\nonumber\\
C+D\frac{a^2}{b^2}=-1+E\nonumber\\
A=\mu C-\mu D\nonumber\\
\hbox{and}\nonumber\\
\mu C-\mu D\frac{a^2}{b^2}=-1-E.
\eeqa
The solution for $A$ in particular is
\beq
A=-\lep\frac{4\mu}{(\mu+1)^2-(a^2/b^2)(\mu-1)^2}\rip,
\eeq
from which one finds that the field inside of the shield is
\beq
\Hx=\Bx=\frac{4\mu}{(\mu+1)^2-(a^2/b^2)(\mu-1)^2}\H_0.
\eeq
The field inside is decreased relative to the applied field by a factor of
$A$. If one employs a ``high-$\mu$'' material with a permeability of order
$10^4$ or $10^5$, then $a\sim4/\{\mu[1-(a^2/b^2)]\}$ which will be much
smaller than unity if $a$ is significantly smaller than $b$. This is the
limit in which the sleeve acts as a good shield against the applied
magnetic field.
\edo

\end{document}