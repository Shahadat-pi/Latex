\documentclass[a4paper,10.5pt,fleqn]{article}
\usepackage[sexy,hints]{evan}
\usepackage{dirtytalk,array,tabu}
\usepackage{pgf,tikz,empheq}
\usetikzlibrary{datavisualization.formats.functions}
\newcommand\perm[2][^n]{\prescript{#1\mkern-2.5mu}{}P_{#2}}
\newcommand\comb[2][^n]{\prescript{#1\mkern-0.5mu}{}C_{#2}}
\usetikzlibrary{calc,trees,positioning,arrows,fit,shapes,calc}
\hypersetup{hidelinks=yes}
\begin{document}
\section{Definition}
\subsubsection{ The complex number system}
There is no real number $ x $ which satisfies the polynomial equation $x^2+ 1 = 0.$ To permit solutions of this and similar equations, the set of complex numbers is introduced. We can consider a complex number as having the form $a + ib$ where a and b are real number and $ i $,
which is called the imaginary unit, has the property that $ i^2= – 1.$
It is denoted by $ z $. Therefore,  $z = a + ib$ where \say{$ a $} is called as real part of $ z $ which is denoted by $ (Re\, z) $ and \say{$ b $} is called as imaginary part of $ z $ which is denoted by $(Im\,z).$\\

Any complex number is :\\

$ (i) $ Purely real, if $b = 0$  \quad $ (ii)$ Purely imaginary, if $a = 0$ \quad $(iii)$ Imaginary, if $ b \neq 0.$\\

NOTE :\\ 

$(a)$ The set $\mathbb{R}$ of real numbers is a proper subset of the Complex Numbers. Hence the complete
number system is $\mathbb{N} \subset \mathbb{W} \subset \mathbb{I} \subset \mathbb{Q} \subset \mathbb{R} \subset \mathbb{C}$\\

$(b)$ Zero is purely real as well as purely imaginary but not imaginary.\\

$(c)$ $ i = \sqrt{-1}$ is called the imaginary unit. Also $i^2 = -1,\; i^3= -i ,\; i^4= 1\; etc.$\\

$(d)\; \sqrt{a} \sqrt{b} = \sqrt{ab} $ only if atleast one of a or b is non - negative.\\

$(e)$ is $z = a + ib$, then $a - ib$ is called complex conjugate of $z$ and written as $ \bar{z} = a -ib$.\\

\textbf{Self Practice Problems}\\

\textbf{1. }Write the following as complex number
$(i)\, \sqrt{-16} \quad (ii)\, \sqrt{x} , (x > 0)\quad
(iii)\, –b + \sqrt{-4ac},\, (a, c> 0)$\\

\textbf{Ans.} $(i)\, 0 + i \sqrt{16}\; (ii)\, \sqrt{x}  + i0\; (iii)\, -b + i\sqrt{4ac}$\\

\textbf{2.} Write the following as complex number:\\

$(i)\,\sqrt{x}\,  (x < 0) \quad(ii)\, \text{ roots of}\, x^2- (2\cos\theta)x + 1 = 0$

\textbf{2. Algebraic Operations:}
Fundamental operations with complex numbers
In performing operations with complex numbers we can proceed as in the algebra of real numbers,
replacing $i^2$ by $-1$ when it occurs.\\

\textbf{1. } Addition $(a + bi) + (c + di) = a + bi + c + di = (a + c) + (b + d) i$\\
\textbf{2. }Subtraction $(a + bi) – c + di) = a + bi – c – di = (a – c) + (b – d) i$\\
\textbf{3.} Multiplication $(a + bi) (c + di) = ac + adi + bci + bdi= (ac – bd) + (ad+ bc)i$

4. Division

c di
a bi


=
c di
a bi


.
c di
c bi


= 2 2 2
2
c d i
ac adi bci bdi

  

= 2 2
c d
ac bd (bc ad)i

  

= 2 2
c d
ac bd


+ i
c d
bc ad
2 2



Inequalities in complex numbers are not defined. There is no validity if we say that complex number is
positive or negative.
e.g. z > 0, 4 + 2i < 2 + 4 i are meaningless.
In real numbers if a2
+ b2
= 0 then a = 0 = b however in complex numbers,

z1
2
+ z2
2
= 0 does not imply z1
= z2
= 0.
Example : Find multiplicative inverse of 3 + 2i.
Solution Let z be the multiplicative inverse of 3 + 2i. then

 z . (3 + 2i) = 1
 z = 3 2i
1

= 3 2i 3 2i
3 2i
 


 z = 13
3
– 13
2
i






 i
13
2
13
3

Ans.

Self Practice Problem
1. Simplify in+100 + in+50 + in+48 + in+46 , n  .
Ans. 0
3. Equality In Complex Number:
Two complex numbers z1
= a1 + ib1
 z2
= a2 + ib2
are equal if and only if their real and imaginary parts

are equal respectively
i.e. z1
= z2  Re(z1
) = Re(z2
) and m
(z1
) = m
(z2The complex number system
There is no real number x which satisfies the polynomial equation x2

+ 1 = 0. To permit solutions of this

and similar equations, the set of complex numbers is introduced.
We can consider a complex number as having the form a + bi where a and b are real number and i,
which is called the imaginary unit, has the property that i2
= – 1.

It is denoted by z i.e. z = a + ib. ‘a’ is called as real part of z which is denoted by (Re z) and ‘b’ is called
as imaginary part of z which is denoted by (Im z).
Any complex number is :
(i) Purely real, if b = 0 ; (ii) Purely imaginary, if a = 0
(iii) Imaginary, if b  0.
NOTE : (a) The set R of real numbers is a proper subset of the Complex Numbers. Hence the complete
number system is N
(b) Zero is purely real as well as purely imaginary but not imaginary.
(c) i = 1 is called the imaginary unit.
Also i2 =  1; i3
=  i ; i4
= 1 etc.

(d) a b = a b only if atleast one of a or b is non - negative.
(e) is z = a + ib, then a – ib is called complex conjugate of z and written as z = a – ib
Self Practice Problems
1. Write the following as complex number
(i) 16 (ii) x , (x > 0)
(iii) –b +  4ac , (a, c> 0)
Ans. (i) 0 + i 16 (ii) x + 0i (iii) –b + i 4ac
2. Write the following as complex number
(i) x (x < 0) (ii) roots of x2

– (2 cos)x + 1 = 0

2. Algebraic Operations:
Fundamental operations with complex numbers
In performing operations with complex numbers we can proceed as in the algebra of real numbers,
replacing i2 by – 1 when it occurs.
1. Addition (a + bi) + (c + di) = a + bi + c + di = (a + c) + (b + d) i
2. Subtraction (a + bi) – c + di) = a + bi – c – di = (a – c) + (b – d) i
3. Multiplication (a + bi) (c + di) = ac + adi + bci + bdi2

= (ac – bd) + (ad+ bc)i

4. Division

c di
a bi


=
c di
a bi


.
c di
c bi


= 2 2 2
2
c d i
ac adi bci bdi

  

= 2 2
c d
ac bd (bc ad)i

  

= 2 2
c d
ac bd


+ i
c d
bc ad
2 2



Inequalities in complex numbers are not defined. There is no validity if we say that complex number is
positive or negative.
e.g. z > 0, 4 + 2i < 2 + 4 i are meaningless.
In real numbers if a2
+ b2
= 0 then a = 0 = b however in complex numbers,

z1
2
+ z2
2
= 0 does not imply z1
= z2
= 0.
Example : Find multiplicative inverse of 3 + 2i.
Solution Let z be the multiplicative inverse of 3 + 2i. then

 z . (3 + 2i) = 1
 z = 3 2i
1

= 3 2i 3 2i
3 2i
 


 z = 13
3
– 13
2
i






 i
13
2
13
3

Ans.

Self Practice Problem
1. Simplify in+100 + in+50 + in+48 + in+46 , n  .
Ans. 0
3. Equality In Complex Number:
Two complex numbers z1
= a1 + ib1
 z2
= a2 + ib2
are equal if and only if their real and imaginary parts

are equal respectively
i.e. z1
= z2  Re(z1
) = Re(z2
) and m
(z1
) = m
(z2
\end{document}