\documentclass{article}
\usepackage[utf8]{inputenc}
\usepackage{amsmath}
\usepackage{mathtools}
\DeclarePairedDelimiter{\ceil}{\lceil}{\rceil}
\usepackage{geometry}
\usepackage[banglamainfont=Kalpurush, 
            banglattfont=Siyam Rupali
           ]{latexbangla}
\usepackage{tikz}
\usetikzlibrary{scopes}
\begin{document}

\def\iangle{35} % Angle of the inclined plane

\def\down{-90}
\def\arcr{0.5cm} % Radius of the arc used to indicate angles
\begin{center}

\begin{tikzpicture}[
    force/.style={>=latex,draw=blue,fill=blue},
    axis/.style={densely dashed,gray,font=\small},
    M/.style={rectangle,draw,fill=lightgray,minimum size=0.5cm,thin},
    m/.style={rectangle,draw=black,fill=lightgray,minimum size=0.3cm,thin},
    plane/.style={draw=black,fill=blue!10},
    string/.style={draw=red, thick},
    pulley/.style={thick},
    scale=1.5
]


    %% Sketch
    \draw[plane] (0,-1) coordinate (base)
                     -- coordinate[pos=0.5] (mid) ++(\iangle:3) coordinate (top)
                     |- (base) -- cycle;
    \path (mid) node[M,rotate=\iangle,yshift=0.25cm] (M) {};
    \draw[->] (base)++(\arcr,0) arc (0:\iangle:\arcr);
    \draw[->] ([shift=(\iangle:\arcr)]M.east) arc (\iangle:80:\arcr);
    \path (base)++(\iangle*0.5:\arcr+5pt) node {$\alpha$};
     \node at (1.7,.8) {$\theta$};
    {[force,->]
            % Assuming that Mg = 1. The normal force will therefore be cos(alpha)
            \draw (M.center)[rotate=\iangle] -- ++(0,{cos(\iangle)}) node[above] {$N$};
            \draw (M.west) -- ++(\iangle:-1.2) node[left] {$\mu N$};
            \draw (M.east) -- ++(\iangle:1.2) ;
            \draw (M.east) -- ++(80:1.5) node[right] {$P$};
            \draw[force,->] (M.center) -- ++(0,-1.5) node[below] {$W$};
        }
                {[axis,->]
             \draw (M.center)[rotate=305] -- ++(\iangle:1.2);
            % Indicate angle. The code is a bit awkward.

            \draw[solid,shorten >=0.5pt,rotate=\iangle] (0.58,-1.1)
                arc(\down-\iangle:\down:\arcr);
            \node at (1.3,-.8) {$\alpha$};
        }

    %%
\end{tikzpicture}
\end{center}
ধরা যাক $W$ ওজন বিশিষ্ট একটি বস্তু ভূমির সাথে $\alpha$ কোণে আনত একটি অমসৃণ একটি তলের উপর তলের সাথে $\theta$ কোণে আনত একটি বল $P$ প্রয়োগ করার ফলে সাম্যবস্থায় রয়েছে।\\
$(a)$ প্রথমে মনে করা যাক বস্তুটি উপরে উঠার জন্য গতির উপক্রম হওয়ার বিন্দুতে নিন্মোক্ত বলগুলোর জন্য সাম্যবস্থায় আছে।\\
এর ওজন $W$;\\
আনত তলের সাধারণ প্রতিক্রিয়া $N$;\\
তলের উপর ক্রিয়াশীল প্রান্তীয় ঘর্ষণ $\mu N$ যেহেতু বস্তুটি উপরে উঠার উপক্রম হচ্ছে;\\
এবং বল $P$ যা আনত তলের সাথে $\angle \theta$ কোণে ক্রিয়া রত রয়েছে।\\
আনত তলের সাপেক্ষে লম্বাংশ উপপাদ্যের সাহায্যে লেখা যায়, লদ্ধির আনুভূমিক উপাংশ
\begin{equation}
\begin{split}
P\cos \theta &=N\cos 90 +\mu N \cos 180 + W\cos (270-\alpha) \\&= \mu N  + W\sin \alpha 
\end{split}
\end{equation} 
 উলম্বিক উপাংশ
\begin{equation}
\begin{split}
P\sin \theta &=N\sin 90 +\mu N \sin 180 + W\sin (270-\alpha) \\&=  W\cos \alpha -N \\N &= W\cos \alpha - P\sin \theta  
\end{split}
\end{equation} 
$N$ এর মান এ বসালে\\
\begin{equation}
\begin{split}
P\cos \theta &= \mu (W\cos \alpha - P\sin \theta)  + W\sin \alpha\\P (\cos \theta + \mu\sin \theta) &= W(\sin\alpha + \mu\cos \alpha)\\P &= \dfrac{\sin \alpha + \mu\cos\alpha}{\cos\theta + \mu \sin\theta}W\\&=  \dfrac{\sin \alpha + \tan \phi \cos\alpha}{\cos\theta + \tan\phi \sin\theta}W\\&= \dfrac{\sin \alpha\cos\phi + \cos\alpha\sin\phi}{\cos\theta\cos\phi + \sin\phi \sin\theta}W\\&= \dfrac{\sin (\alpha +\phi)}{\cos(\theta -\phi)}W
\end{split}
\end{equation} 
$R = P\sin \alpha +W \cos \alpha$
\begin{tikzpicture}[
    force/.style={>=latex,draw=blue,fill=blue},
    axis/.style={densely dashed,gray,font=\small},
    M/.style={rectangle,draw,fill=lightgray,minimum size=0.5cm,thin},
    m/.style={rectangle,draw=black,fill=lightgray,minimum size=0.3cm,thin},
    plane/.style={draw=black,fill=blue!10},
    string/.style={draw=red, thick},
    pulley/.style={thick},
]


    %% Sketch
    \draw[plane] (0,-1) coordinate (base)
                     -- coordinate[pos=0.5] (mid) ++(\iangle:3) coordinate (top)
                     |- (base) -- cycle;
    \path (mid) node[M,rotate=\iangle,yshift=0.25cm] (M) {};
    \draw[->] (base)++(\arcr,0) arc (0:\iangle:\arcr);
     \draw[->] (M.east)++(\arcr,0) arc (0:\iangle:\arcr);
    \path (base)++(\iangle*0.5:\arcr+5pt) node {$\alpha$};
     \path (M.east)++(\iangle*0.5:\arcr+5pt) node {$\alpha$};
    {[force,->]
            % Assuming that Mg = 1. The normal force will therefore be cos(alpha)
            \draw (M.center)[rotate=\iangle] -- ++(0,{cos(\iangle)}) node[above] {$R$};
            \draw (M.west) -- ++(\iangle:-1.2) node[left] {$\mu R$};
            \draw (M.east) -- ++(\iangle:1.2) ;
            \draw (M.east) -- ++(1.5,0) node[right] {$P$};
            \draw[force,->] (M.center) -- ++(0,-1.5) node[below] {$W$};
        }
                {[axis,->]
             \draw (M.center)[rotate=305] -- ++(\iangle:1.2);
            % Indicate angle. The code is a bit awkward.

            \draw[solid,shorten >=0.5pt,rotate=\iangle] (0.58,-1.1)
                arc(\down-\iangle:\down:\arcr);
            \node at (1.3,-.8) {$\alpha$};
        }

    %%
\end{tikzpicture}

The spring considered here is ideal, that it has no mass. The flat attachment on the spring is also assumed to be mass less.

This is to support the assumption that the spring initially has no compression or elongation. It thus is in its natural length. No energy is stored in it.

The height $H$ is measured from the bottom of the block. As the block falls and touches the flat extension on the spring, the bottom of the block has moved the distance $H$. The center of the block also moves the same distance. The block thus loses gravitational potential energy  equal to $MgH$.

Now the spring gets compressed and the block also moves downward with it.

Assume that the maximum downward movement, the maximum compression of the spring is $x$.

Then the bottom of the block  has moved  a total distance $H+x$ downward from its initial position.

By the time the spring gets compressed to maximum, the  lost  gravitational potential energy  equal to $Mg(H+x)$.

Where does this energy go?

When the mass falls on the spring, the impact compresses the spring through a

distance $x$. Then the total vertical fall of the block will be $(H+x)$.

This fall reduces the gravitational potential energy by $Mg(H+x)$

Now the spring starts getting compressed.

When the spring is compressed it develops a restoring force that acts opposite to the velocity of the block. 

This force will retard the mass and it will come to a momentary rest.

At this instant, potential energy lost by the block will appear as the elastic potential energy of the spring.

Therefore, $\frac{1}{2}Kx^2\:=\:Mg(H+x)$

Or, $\frac{1}{2}Kx^2\:=\:Mg(H+x)$

or, $\frac{1}{2}Kx^2\:-\:Mg(H+x)\:=0$                                   

Which becomes, $Kx^2\:-\:2Mgx\:-\:2MgH\:=0$

The compression of the spring

$x\:=\:\frac{\frac{2Mg}{K}\pm \sqrt{(\frac{2Mg}{K})^2 \:+\:\frac{8Mgh}{K}}}{2}$
\end{document}