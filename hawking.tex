\documentclass[a4paper,11pt]{article}
\usepackage[T1]{fontenc}
\usepackage[utf8]{inputenc}
\usepackage{evan}
\usepackage{lmodern}
\usepackage{amsfonts}
\usepackage{amssymb}
\usepackage{amsmath}
\usepackage{relsize}
\usepackage{xfrac}
\usepackage{hyperref}
\usepackage{amsfonts}
\usepackage{asymptote}
\usepackage{hyperref}
\usepackage{graphicx}
\usepackage[english]{babel}
\usepackage{amsmath}
\usepackage{dirtytalk}
\usepackage{mathtools}
\usepackage{tikz}
\usetikzlibrary{arrows,positioning} 
\usepackage{array}
\usepackage{wrapfig}
\usepackage{multirow}
\usepackage{tabu}
\usetikzlibrary{calc}
\usepackage{changepage}
\usepackage{caption,setspace}
\usepackage{draftwatermark}
\newcommand\perm[2][^n]{\prescript{#1\mkern-2.5mu}{}P_{#2}}
\newcommand\comb[2][^n]{\prescript{#1\mkern-0.5mu}{}C_{#2}}
\SetWatermarkText{$Hawking$}
\begin{document}

\begin{center}
\begin{LARGE}
Farewell to a brilliant mind
\end{LARGE}
\end{center}

Stephen Hawking died on Pi Day—March 14, which is also Einstein's birthday. He was born on another special day—Galileo's 300th death anniversary. It goes without saying that Hawking had a massive impact on not only gravitational physics, but also the world at large.

While I was a student at Cambridge, Hawking's office was down the hall from mine. We were in the same department—the Department of Applied Mathematics and Theoretical Physics in the Mathematics Faculty. Although it was difficult to talk to him because he could only speak using a computer, he was available to students like myself.

At the time I was in Cambridge, string theory was the hottest topic, and Hawking was unambiguously cold about it. He is even rumoured to have used a four-letter word to describe string theory. Being a young student working on string theory and cosmology, I therefore didn't feel much need to talk to him. But like many others, I was eventually deeply influenced by him and his vision. 

Over the last decade, Hawking's work on “black hole information paradox” has become one of the central issues to be understood by string theorists and the high-energy physics community all over the world. Even my students in Bangladesh, from BRAC University and from BUET, were inspired and influenced by him and wrote their theses on the black hole information paradox. For example, the first calculation some of them had to perform was Hawking's famous calculation that black holes emit radiation. Mishkat Al Alvi, Avik Roy, and Moinul Hossain Rahat from BUET wrote a thesis titled, “A quantum information theoretic analysis of the black hole information paradox...” My BRAC University students, Reefat and Ashiq Rahman, wrote their theses on different aspects of the black hole information paradox. Even a group of my computer science students at BRAC University, Paresha Farastu, Rafiduzzaman Sonnet, Saddat Hasan, and Sandipon Paul, who worked on quantum algorithms, had to repeat Hawking's classic calculations to understand the basics of quantum computing.

When students from different departments—maths, physics and computer science—living in various developing countries have to study your work, then you know you have made an incredibly rich contribution to global scientific culture. Professor Hawking's contributions to black hole physics, quantum information and early universe cosmology are certainly of fundamental importance and are now studied everywhere. However, perhaps his more important contribution was the creation of a whole generation of scientists, particularly British scientists who have led the field of mathematical relativity, quantum gravity, black hole physics and quantum information over several decades.

Many of the top relativity experts in the world were Hawking's students, post-docs or close collaborators. For example, Gary Gibbons, Don Page, Chris Pope, Raphael Bousso, Kip Thorne, Malcolm Perry and many others were Hawking's protégés. Hawking can even be considered the academic grandfather of some of the best young Bangladeshi scientists. The mathematician Mohammad Akbar, now at the University of Texas, was Gary Gibbons' student, and the string theorist Tibra Ali, now at the Perimeter Institute, was Malcolm Perry's student.

In some sense, Hawking's impact as a nation builder using math and science may be even more important than his role as a creative genius. He is primarily responsible for creating the largest mathematics department in the world—the Centre for Mathematical Sciences (CMS) at the University of Cambridge, where I was a PhD student. This is one of the greatest sources of pride and assets of the United Kingdom and can be said to be the biggest showpiece institution at any British university. CMS has ensured that Great Britain and Cambridge will remain at the forefront of not just mathematics, but science over the next 50 years. Without Professor Hawking, Cambridge would not have been able to raise over 100 million pounds in donations for the Centre, let alone the 10 million pounds needed just for its library. Nobody before this was able to afford spending 100 million pounds on a mathematics department. But with Hawking's influence, this became possible. This is an everlasting legacy for mankind. This is a “sadqajariya” in Bangla.

This vision of nation building through mathematics is something that I, and others at the Bangladeshi Mathematical Olympiad, such as Munir Hasan, Professors Kaykobad, Jamilur Reza Choudhury and Zafar Iqbal, have been pursuing over the past decade. We have helped create a generation of immensely talented students who will shape future Bangladeshi science and culture. We have sent dozens of unbelievably brilliant students to places like MIT, Cambridge and Harvard. We had to send them abroad because the infrastructure doesn't exist in Bangladesh to develop them into world-class scientists and engineers. We also have a much larger group of exceptionally talented students who have stayed in Bangladesh. Some of them have already left or are in the process of leaving to study at outstanding institutions abroad. For example, University of Dhaka graduates Nafiz Ishtiaque and Wasif Ahmed will be at the Perimeter Institute. BUET graduates like Avik and Rahat have already left. Even some of my students at BRAC University will go off to world-class places like Utrecht and Waterloo this year. They will become stars in the future.

We are already seeing some of the fruits of this nation building. Some of these students have already started returning and are making contributions. For example, Tamanna Islam Urmi recently came back from MIT. She is working on green energy and at my insistence is now teaching part-time in the computer science department at BRAC University. We hope to be able to completely emulate Hawking's nation building achievements in the future by creating a world-class math and science facility like Hawking's Centre for Mathematical Sciences to enable these exceptional people to find an academic home and make contributions.

It is a testament of Stephen Hawking's uniqueness that we start by describing his contributions to black hole physics and end up talking about his impact on Bangladeshi students and education. Today, we mourn his death, but celebrate his influence and impact—even on Bangladesh.
\end{document}