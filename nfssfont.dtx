% \iffalse meta-comment
%
% Copyright 1993-2016
% The LaTeX3 Project and any individual authors listed elsewhere
% in this file.
%
% This file is part of the LaTeX base system.
% -------------------------------------------
%
% It may be distributed and/or modified under the
% conditions of the LaTeX Project Public License, either version 1.3c
% of this license or (at your option) any later version.
% The latest version of this license is in
%    http://www.latex-project.org/lppl.txt
% and version 1.3c or later is part of all distributions of LaTeX
% version 2005/12/01 or later.
%
% This file has the LPPL maintenance status "maintained".
%
% The list of all files belonging to the LaTeX base distribution is
% given in the file `manifest.txt'. See also `legal.txt' for additional
% information.
%
% The list of derived (unpacked) files belonging to the distribution
% and covered by LPPL is defined by the unpacking scripts (with
% extension .ins) which are part of the distribution.
%
% \fi
\def\nfssfontfileversion{v2.2d}
\def\nfssfontfiledate{2015/02/01}

%
% \iffalse
% Copyright (C) 1994 Donald E. Knuth
% Copyright (C) 1994-94, 2003, 2006 Frank Mittelbach,
% all rights reserved.
% \fi
%
%
% \title{The program \texttt{nfssfont}}
% \author{Frank Mittelbach\thanks{Adapted from code written
%         by Donald E.~Knuth}}
% \date{\nfssfontfiledate}
%
%
% \def\dst{{\normalfont\scshape docstrip}}
% \def\NFSS{\textsf{NFSS}}
% \setcounter{StandardModuleDepth}{1}
%
%
% \MaintainedByLaTeXTeam{latex}
% \maketitle
%
% \section{Introduction}
%
% Within the \NFSS{} distribution there is a \LaTeX{} file
% \texttt{nfssfont.tex} which can be used to test new fonts, produce
% font tables showing all characters, etc. This is an addaption of a
% program originally written by Don Knuth.

% When you run this file through \LaTeX{} you will be asked to enter
% the name of the font to test. Your answer should be the external
% font name without any extension, e.g.\ \texttt{cmr10} (Computer
% Modern Roman 10pt) or \texttt{yinit} (Yannis Haralambous' Initial
% font).
%
% Alternatively (since version 2.1), you can press \meta{return} at
% this point. In that case a dialog is started that asks you for the
% font characteristics in \NFSS{} notation, i.e., first for the
% encoding scheme (default \texttt{T1}), then for the family name
% (default \texttt{cmr}), then for the series (default \texttt{m}),
% then for the shape (default \texttt{n}), and finally for the size
% (default \texttt{10pt}).
%
% You are then requested to enter a command. Probably the most
% important one is |\table|, which will produce a font chart.  To
% switch to a new test font, type |\init|; to finish the test, type
% |\bye|; and to learn about all the other possible tests type
% |\help|.
%
% This is at
% the moment only a crude addaption of the test program for the
% Computer Modern fonts by Donald Knuth.
%
% \StopEventually{}
%
% \section{A driver for this document}
%
% The next bit of code contains the documentation driver file for
% \TeX{}, i.e., the file that will produce the documentation you are
% currently reading. It will be extracted from this file by the
% \dst{} program.
%    \begin{macrocode}
%<*driver>
\documentclass{ltxdoc}
 % \OnlyDescription   % comment out for implementation details
\begin{document}
   \DocInput{nfssfont.dtx}
\end{document}
%</driver>
%    \end{macrocode}
%
% \section{The code}
%
% A testbed for font evaluation (see The METAFONTbook, Appendix H)
% changed for \LaTeX{} with \NFSS.
%
% This program should get a complete rewrite some day or at least some
% code documentation. Any volunteers?
%    \begin{macrocode}
%<*code>
\documentclass{article}
\nofiles
\setlength\textwidth{470pt}
\setlength\oddsidemargin{0pt}
\addtolength\textheight{7\baselineskip}
\addtolength\topmargin{-3\baselineskip}

\def\sevenrm{\fontencoding{OT1}\fontsize{7}{9pt}\rmfamily}

\typeout{**********************************************}
\typeout{* NFSS font test program version <\nfssfontfileversion>}
\typeout{*}
\typeout{* Follow the instructions}
\typeout{**********************************************}
\typeout{}

\tracinglostchars=0
\tolerance=1000
\raggedbottom
\parindent=0pt
%\newlinechar=`@               %FMi
\hyphenpenalty=200
\doublehyphendemerits=30000
%    \end{macrocode}
%
%    \begin{macrocode}
\newcount\m \newcount\n \newcount\p \newdimen\dim
\chardef\other=12

\def\today{\ifcase\month\or
  January\or February\or March\or April\or May\or June\or
  July\or August\or September\or October\or November\or December\fi
  \space\number\day, \number\year}
\def\hours{\n=\time \divide\n 60
  \m=-\n \multiply\m 60 \advance\m \time
  \twodigits\n\twodigits\m}
\def\twodigits#1{\ifnum #1<10 0\fi \number#1}
%    \end{macrocode}
%
%    We first ask for a real external font name\ldots
%    \begin{macrocode}
\def\init{%
 \typein[\currfontname]%
   {Input external font name, e.g., cmr10^^J%
    (or <enter> for NFSS classification of font):}%
%    \end{macrocode}
%    If the answer is a real font name (like ``cmr10'') then we have no idea
%    what the encoding of this font is, so in this case we record the word
%    ``unknown'' and internally assume |T1| encoding. Commands other than
%    |\table| are then most likely going to fail, unless that assumption was
%    correct.
%    \begin{macrocode}
 \def\encoding{unknown}%
 \fontencoding{T1}\selectfont
%    \end{macrocode}
%    If the user answers with ``return'' we ask for the \NFSS{}
%    classification of the font in form of encoding, family, series,
%    shape, and size offering a default each time.
%    \begin{macrocode}
 \ifx\currfontname\empty
  \typein[\encoding]{^^J^^J*** NFSS classification
                     ***^^J^^JFont encoding [T1]:}%
  \ifx\encoding\empty
   \fontencoding{T1}%
   \def\encoding{T1}%
  \else
%    \end{macrocode}
%    Depending on the chosen encoding we have the problem, that the encoding
%    support information may not be available (only for |T1| and |OT1| this is
%    preloaded in the kernel, all other encodings normally need a declaration in
%    the preamble). So here we attempt to load this support file even though
%    it is actually already too late for this.
%    \begin{macrocode}
    \edef\next{%
      \lowercase{\def\noexpand\next{\encoding enc.def}}}%
    \next
    \makeatletter
    \InputIfFileExists\next
       {}%
%    \end{macrocode}
%    If the encoding support file is not found this may indicate an error, eg
%    something misspelled. However, it may equally mean that we dealing
%    with a math font encoding for which no encoding support is
%    available.
%    \begin{macrocode}
       {\PackageWarningNoLine{nfssfont}%
        {Encoding file `\next' not found.%
          \MessageBreak
           You might have misspelt the name of the encoding
          \MessageBreak
           or perhaps this encoding is not a text encoding,
         \MessageBreak
           in which case you will probably only be able to
         \MessageBreak
           run the \noexpand\table command successfully}%
%    \end{macrocode}
%    But actually the situation is even worse, as some math
%    encodings do have such support files. So for them we need to make the encoding
%    known to NFSS in some other way, otherwise we could not even find the
%    external font name, let alone typesetting a |\table|.
%    \begin{macrocode}
       \expandafter\let\csname T@\encoding\endcsname\empty
       }%
    \makeatother
    \let\next\relax
    \fontencoding\encoding
  \fi
  \typein[\family]{Font family [cmr]:}%
  \relax
  \ifx\family\empty
    \fontfamily{cmr}%
  \else
    \fontfamily\family
  \fi
  \typein[\series]{Font series [m]:}%
  \ifx\series\empty
    \fontseries{m}%
  \else
    \fontseries\series
  \fi
  \typein[\shape]{Font shape [n]:}%
  \ifx\shape\empty
    \fontshape{n}%
  \else
    \fontshape\shape
  \fi
  \typein[\size]{Font size [10pt]:}%
  \ifx\size\empty
    \fontsize{10}{10}%
  \else
    \fontsize\size\size
  \fi
%    \end{macrocode}
%    Finally select the font and assign it to |\currfontname|.
%    \begin{macrocode}
  \selectfont\edef\currfontname{\fontname\font}%
 \fi
 \startfont
 \message{Now type a test command (\string\help\space for help):}}
%    \end{macrocode}
%
%    \begin{macrocode}
\def\startfont{\font\testfont=\currfontname
  \leftline{\sevenrm Test of font \currfontname\unskip\ (encoding
            \encoding\unskip) on \today\ at \hours}
  \medskip
  \testfont \setbaselineskip
  \ifdim\fontdimen6\testfont<10pt \rightskip=0pt plus 20pt
  \else\rightskip=0pt plus 2em \fi
  \spaceskip=\fontdimen2\testfont % space between words (\raggedright)
  \xspaceskip=\fontdimen2\testfont \advance\xspaceskip
  by\fontdimen7\testfont}
%    \end{macrocode}
%
%    \begin{macrocode}
{\catcode`\|=0 \catcode`\\=\other
|gdef|help{|message{%
\init switches to another font;^^J%
\stop or \bye finishes the run;^^J%
\table prints the font layout in tabular format;^^J%
\text prints a sample text, assuming TeX text font conventions;^^J%
\sample combines \table and \text;^^J%
\mixture mixes a background character with a series of others;^^J%
\alternation interleaves a background character with a series;^^J%
\alphabet prints all lowercase letters within a given background;^^J%
\ALPHABET prints all uppercase letters within a given background;^^J%
\series prints a series of letters within a given background;^^J%
\lowers prints a comprehensive test of lowercase;^^J%
\uppers prints a comprehensive test of uppercase;^^J%
\digits prints a comprehensive test of numerals;^^J%
\math prints a comprehensive test of TeX math italic;^^J%
\names prints a text that mixes upper and lower case;^^J%
\punct prints a punctuation test;^^J%
\bigtest combines many of the above routines;^^J%
\help repeats this message;^^J%
and you can use ordinary TeX commands (e.g., to \input a file).}}}

\def\setbaselineskip{\setbox0=\hbox{\n=0
\loop\char\n \ifnum \n<255 \advance\n 1 \repeat}
\baselineskip=6pt \advance\baselineskip\ht0 \advance\baselineskip\dp0 }
%    \end{macrocode}
%
%    \begin{macrocode}
\def\setchar#1{{\escapechar-1\message{\string#1 character = }%
  \def\do##1{\catcode`##1=\other}\dospecials
  \read-1 to\next
  \expandafter\finsetchar\next\next#1}}
\def\finsetchar#1#2\next#3{\global\chardef#3=`#1
  \ifnum #3=`\# \global\chardef#3=#2 \fi}
\def\promptthree{\setchar\background
  \setchar\starting \setchar\ending}

\def\mixture{\promptthree \domix\mixpattern}
\def\alternation{\promptthree \domix\altpattern}
\def\mixpattern{\0\1\0\0\1\1\0\0\0\1\1\1\0\1}
\def\altpattern{\0\1\0\1\0\1\0\1\0\1\0\1\0\1\0\1\0}
\def\domix#1{\par\chardef\0=\background \n=\starting
  \loop \chardef\1=\n #1\endgraf
  \ifnum \n<\ending \advance\n 1 \repeat}

\def\!{\discretionary{\background}{\background}{\background}}
\def\series{\promptthree \!\doseries\starting\ending\par}
\def\doseries#1#2{\n=#1\loop\char\n\!\ifnum\n<#2\advance\n 1 \repeat}
\def\complower{\!\doseries{`a}{`z}\doseries{'31}{'34}\par}
\def\compupper{\!\doseries{`A}{`Z}\doseries{'35}{'37}\par}
\def\compdigs{\!\doseries{`0}{`9}\par}
\def\alphabet{\setchar\background\complower}
\def\ALPHABET{\setchar\background\compupper}

\def\lowers{\docomprehensive\complower{`a}{`z}{'31}{'34}}
\def\uppers{\docomprehensive\compupper{`A}{`Z}{'35}{'37}}
\def\digits{\docomprehensive\compdigs{`0}{`4}{`5}{`9}}
\def\docomprehensive#1#2#3#4#5{\par\chardef\background=#2
  \loop{#1} \ifnum\background<#3\m=\background\advance\m 1
  \chardef\background=\m \repeat \chardef\background=#4
  \loop{#1} \ifnum\background<#5\m=\background\advance\m 1
  \chardef\background=\m \repeat}
%    \end{macrocode}
%
%    \begin{macrocode}
\def\names{ {\AA}ngel\aa\ Beatrice Claire
  Diana \'Erica Fran\c{c}oise Ginette H\'el\`ene Iris
  Jackie K\=aren {\L}au\.ra Mar{\'\i}a N\H{a}ta{\l}{\u\i}e {\O}ctave
  Pauline Qu\^eneau Roxanne Sabine T\~a{\'\j}a Ur\v{s}ula
  Vivian Wendy Xanthippe Yv{\o}nne Z\"azilie\par}
\def\punct{\par\dopunct{min}\dopunct{pig}\dopunct{hid}
  \dopunct{HIE}\dopunct{TIP}\dopunct{fluff}
  \$1,234.56 + 7/8 = 9\% @ \#0\par}
\def\dopunct#1{#1,\ #1:\ #1;\ `#1'\ ?`#1?\ !`#1!\ (#1)\ [#1]\
               #1*\ #1.\par}

\def\bigtest{\sample
  hamburgefonstiv HAMBURGEFONSTIV\par
  \names \punct \lowers \uppers \digits}

\def\math{\textfont1=\testfont \skewchar\testfont=\skewtrial
 \mathchardef\Gamma="100 \mathchardef\Delta="101
 \mathchardef\Theta="102 \mathchardef\Lambda="103 \mathchardef\Xi="104
 \mathchardef\Pi="105 \mathchardef\Sigma="106 \mathchardef\Upsilon="107
 \mathchardef\Phi="108 \mathchardef\Psi="109 \mathchardef\Omega="10A
 \def\ii{i} \def\jj{j}
 \def\\##1{|##1|+}\mathtrial
 \def\\##1{##1_2+}\mathtrial
 \def\\##1{##1^2+}\mathtrial
 \def\\##1{##1/2+}\mathtrial
 \def\\##1{2/##1+}\mathtrial
 \def\\##1{##1,{}+}\mathtrial
 \def\\##1{d##1+}\mathtrial
 \let\ii=\imath \let\jj=\jmath \def\\##1{\hat##1+}\mathtrial}
\newcount\skewtrial \skewtrial='177
\def\mathtrial{$\\A \\B \\C \\D \\E \\F \\G \\H \\I \\J \\K \\L \\M \\N
 \\O \\P \\Q \\R \\S \\T \\U \\V \\W \\X \\Y \\Z \\a \\b \\c \\d \\e \\f
 \\g \\h \\\ii \\\jj \\k \\l \\m \\n \\o \\p \\q \\r \\s \\t \\u \\v \\w
 \\x \\y \\z \\\alpha \\\beta \\\gamma \\\delta \\\epsilon \\\zeta
 \\\eta \\\theta \\\iota \\\kappa \\\lambda \\\mu \\\nu \\\xi \\\pi
 \\\rho \\\sigma \\\tau \\\upsilon \\\phi \\\chi \\\psi \\\omega
 \\\vartheta \\\varpi \\\varphi \\\Gamma \\\Delta \\\Theta \\\Lambda
 \\\Xi \\\Pi \\\Sigma \\\Upsilon \\\Phi \\\Psi \\\Omega \\\partial
 \\\ell \\\wp$\par}
\def\mathsy{\begingroup\skewtrial='060 % for math symbol font tests
 \def\mathtrial{$\\A \\B \\C \\D \\E \\F \\G \\H \\I \\J \\K \\L
  \\M \\N \\O \\P \\Q \\R \\S \\T \\U \\V \\W \\X \\Y \\Z$\par}
 \math\endgroup}
%    \end{macrocode}
%    Here we have to ensure that we use a suitable encoding otherwise our
%    octal and hexadecimal constants will appear in whatever encoding the
%    current font is.
%    \begin{macrocode}
\def\oct#1{\hbox{\fontencoding{OT1}\rmfamily\'{}\kern-.2em\itshape
           #1\/\kern.05em}}% octal constant
\def\hex#1{\hbox{\fontencoding{OT1}\rmfamily
           \H{}\ttfamily#1}}% hexadecimal constant
\def\setdigs#1"#2{\gdef\h{#2}% \h=hex prefix; \0\1=corresponding octal
 \m=\n \divide\m by 64 \xdef\0{\the\m}%
 \multiply\m by-64 \advance\m by\n \divide\m by 8 \xdef\1{\the\m}}
\def\testrow{\setbox0=\hbox{\penalty 1\def\\{\char"\h}%
 \\0\\1\\2\\3\\4\\5\\6\\7\\8\\9\\A\\B\\C\\D\\E\\F%
 \global\p=\lastpenalty}} % \p=1 if none of the characters exist
\def\oddline{\cr
  \noalign{\nointerlineskip}
  \multispan{19}\hrulefill&
  \setbox0=\hbox{\lower 2.3pt\hbox{\hex{\h x}}}\smash{\box0}\cr
  \noalign{\nointerlineskip}}
\newif\ifskipping
\def\evenline{\loop\skippingfalse
 \ifnum\n<256 \m=\n \divide\m 16 \chardef\next=\m
 \expandafter\setdigs\meaning\next \testrow
 \ifnum\p=1 \skippingtrue \fi\fi
 \ifskipping \global\advance\n 16 \repeat
 \ifnum\n=256 \let\next=\endchart\else\let\next=\morechart\fi
 \next}
\def\morechart{\cr\noalign{\hrule\penalty5000}
 \chartline \oddline \m=\1 \advance\m 1 \xdef\1{\the\m}
 \chartline \evenline}
\def\chartline{&\oct{\0\1x}&&\:&&\:&&\:&&\:&&\:&&\:&&\:&&\:&&}
\def\chartstrut{\lower4.5pt\vbox to14pt{}}
\def\table{$$\global\n=0
  \halign to\hsize\bgroup
    \chartstrut##\tabskip0pt plus10pt&
    &\hfil##\hfil&\vrule##\cr
    \lower6.5pt\null
    &&&\oct0&&\oct1&&\oct2&&\oct3&&\oct4&&\oct5&&\oct6&&\oct7&\evenline}
\def\endchart{\cr\noalign{\hrule}
  \raise11.5pt\null&&&\hex 8&&\hex 9&&\hex A&&\hex B&
  &\hex C&&\hex D&&\hex E&&\hex F&\cr\egroup$$\par}
\def\:{\setbox0=\hbox{\noboundary\char\n\noboundary}%
  \ifdim\ht0>7.5pt\reposition
  \else\ifdim\dp0>2.5pt\reposition\fi\fi
  \box0\global\advance\n 1 }
\def\reposition{\setbox0=\vbox{\kern2pt\box0}\dim=\dp0
  \advance\dim 2pt \dp0=\dim}
\def\centerlargechars{
  \def\reposition{\setbox0=\hbox{$\vcenter{\kern2pt\box0\kern2pt}$}}}
%    \end{macrocode}
%
%    \begin{macrocode}
\def\text{{\advance\baselineskip-4pt
\setbox0=\hbox{abcdefghijklmnopqrstuvwxyz}
\ifdim\hsize>2\wd0
   \ifdim 15pc>2\wd0 \hsize=15pc \else \hsize=2\wd0 \fi
\fi
On November 14, 1885, Senator \& Mrs.~Leland Stanford called together
at their San Francisco mansion the 24~prominent men who had been
chosen as the first trustees of The Leland Stanford Junior University.
They handed to the board the Founding Grant of the University, which
they had executed three days before. This document---with various
amendments, legislative acts, and court decrees---remains as the
University's charter.  In bold, sweeping language it stipulates that
the objectives of the University are ``to qualify students for
personal success and direct usefulness in life; and to promote the
publick welfare by exercising an influence in behalf of humanity and
civilization, teaching the blessings of liberty regulated by law, and
inculcating love and reverence for the great principles of government
as derived from the inalienable rights of man to life, liberty, and
the pursuit of happiness.'' \moretext
(!`THE DAZED BROWN FOX QUICKLY GAVE 12345--67890 JUMPS!)\par}}
\def\moretext{?`But aren't Kafka's Schlo\ss{} and \AE sop's
\OE uvres often na\"\i ve vis-\`a-vis the d\ae monic ph\oe nix's
official r\^ole in fluffy souffl\'es? }
\def\omitaccents{\let\moretext=\relax}

\def\sample{\table\text}
%    \end{macrocode}
%
%    \begin{macrocode}
\def\bye{\end{document}}
%    \end{macrocode}
%
%    The redefinition of |\typeout| around the |\begin{document}|
%    prevents \LaTeX{} from reporting that there is no |.aux| file.
%    This is a bad hack, I agree :-)
%    \begin{macrocode}
\let\oldtypeout\typeout
\def\typeout#1{}
%    \end{macrocode}
%    And another bad hack: to be able to load an encoding definition file
%    after |\begin{document}| I disable the preamble only setting.
%    \begin{macrocode}
\makeatletter
\let\@preamblecmds\@empty
\makeatother

\begin{document}
\let\typeout\oldtypeout

\ifx\noinit!\else\init\fi
%</code>
%    \end{macrocode}
%
%
% \DeleteShortVerb{\|}
% \Finale
%
\endinput
