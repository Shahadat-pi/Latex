\documentclass[a4paper,11pt]{book}
\usepackage[T1]{fontenc}
\usepackage[utf8]{inputenc}
\usepackage{lmodern}
\usepackage[sexy,hints]{evan}
\usepackage{hyperref}
\usepackage{graphicx}
\usepackage[english]{babel}
\usetikzlibrary{decorations.markings}
\usepackage{amsmath}
\usepackage{dirtytalk}
\usepackage{mathtools}
\usepackage{tikz}
\usetikzlibrary{arrows,positioning} 
\usepackage{array}
\usepackage{wrapfig}
\usepackage{multirow}
\usepackage{tabu}
\usetikzlibrary{calc}
\usepackage{changepage}
\usepackage{caption,setspace}
\begin{document}
[From the Transactions of the Cambridge Philosophical Society, VoL x. Part i.]
VIII. On Faraday's Lines of Force.
[Read Dec. 10, 1855, and Feb. 11, 1856.]
The present state of electrical science seems peculiarl unfavourable to speculation.
The laws of the distribution of electricity on the surface of conductors
have been analytically deduced from experiment; some parts of the mathematical
theory of magnetism are established, while in other parts the experimental data
are wanting ; the theory of the conduction of galvanism and that of the mutual
attraction of conductors have been reduced to mathematical formulae, but have
not fallen into relation with the other parts of the science. No electrical theory
can now be put forth, unless it shews the connexion not only between electricity
at rest and current electricity, but between the attractions and inductive effects
of electricity in both states. Such a theory must accurately satisfy those laws,
the mathematical form of which is known, and must afford the means of calculating
the effects in the limiting cases where the known formulae are inapplicable.
In order therefore to appreciate the requirements of the science, the student
must make himself familiar with a considerable body of most intricate mathematics,
the merfi retention of which in the memory materially interferes with
further progress. The first process therefore in the effectual study of the science,
must be one of simplification and reduction of the results of previous investigation
to a form in which the mind can grasp them. The results of this simplification
may take the form of a purely mathematical formula or of a physical
hypothesis. In the first case we entirely lose sight of the phenomena to be
explained ; and though we may trace out the consequences of given laws, we
can never obtain more extended views of the connexions of the subject If,
on the other luiml, we adopt a physical hypothesis, we see the phenomena only
throucrh a medium, and are liable to that blindness to facts and rashness m
[From the Transactions of the Cambridge Philosophical Society, VoL x. Part i.]
VIII. On Faraday's Lines of Force.
[Read Dec. 10, 1855, and Feb. 11, 1856.]
The present state of electrical science seems peculiarlunfavourable to speculation.
The laws of the distribution of electricity on the surface of conductors
have been analytically deduced from experiment; some parts of the mathematical
theory of magnetism are established, while in other parts the experimental data
are wanting ; the theory of the conduction of galvanism and that of the mutual
attraction of conductors have been reduced to mathematical formulae, but have
not fallen into relation with the other parts of the science. No electrical theory
can now be put forth, unless it shews the connexion not only between electricity
at rest and current electricity, but between the attractions and inductive effects
of electricity in both states. Such a theory must accurately satisfy those laws,
the mathematical form of which is known, and must afford the means of calculating
the effects in the limiting cases where the known formulae are inapplicable.
In order therefore to appreciate the requirements of the science, the student
must make himself familiar with a considerable body of most intricate mathematics,
the merfi retention of which in the memory materially interferes with
further progress. The first process therefore in the effectual study of the science,
must be one of simplification and reduction of the results of previous investigation
to a form in which the mind can grasp them. The results of this simplification
may take the form of a purely mathematical formula or of a physical
hypothesis. In the first case we entirely lose sight of the phenomena to be
explained ; and though we may trace out the consequences of given laws, we
can never obtain more extended views of the connexions of the subject If,
on the other luiml, we adopt a physical hypothesis, we see the phenomena only
throucrh a medium, and are liable to that blindness to facts and rashness m
\end{document}