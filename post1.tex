
% article.tex, a sample LaTeX file.
% Run LaTeX on this file twice for proper section numbers.
% A '%' causes LaTeX to ignore remaining text on the line










\documentclass{seminar}

%\documentclass{slides}   % Specifies the document style.
%\usepackage{color}

%\setlength{\oddsidemargin}{0.25in}


%%%%%%%%%%%%%%%%%%%%%%%%%%%%%%%%%%%%%%%%%%%%%%%%%%%%%%%%%%%%%%

%\documentclass[portrait]{slides}
%\usepackage{epsfig}
%\usepackage{rotating}
\usepackage{fancybox}
\usepackage{color}
\usepackage{pstricks}
%\usepackage{include_format}


\usepackage{ifpdf}

\ifpdf
  \usepackage[pdftex]{graphicx}
\else
  \usepackage[dvips]{graphicx}
\fi

%\input{pst-3d}

%\graphicspath{{figures/}}

\pagestyle{empty}
\slidestyle{bottom}

%\slideframe{double}
%\slideframe{shadow}
%\slideframe{oval}
%\slideframe{none}

\def\slidelabel{{\tiny
Physics 170 Week 6, Lecture 1\hspace*{\fill} \
    \thepage}}

%\newenvironment{landslide}
%{\renewcommand{\slideleftmargin}{11.5cm}
%\renewcommand{\slidetopmargin}{-5.5cm}
%\begin{slide}[225mm,150mm] \special{landscape}}
%{\end{slide}}

%\newenvironment{portslide}
%{\renewcommand{\slideleftmargin}{1.5cm}
%\renewcommand{\slidetopmargin}{2.5cm}
%\begin{slide*}[220mm,155mm] \special{landscape}}
%{\end{slide*}}

\begin{document}

%\newenvironment{landslide}
%{\renewcommand{\slideleftmargin}{11.5cm}
%\renewcommand{\slidetopmargin}{-5.5cm}
%\begin{slide}[225mm,150mm] \special{landscape}}

%\special{landscape}\special{! /landscape90 true store}
\begin{slide}



 {\large\bf Physics 170 Week 6, Lecture 1}

\vskip .5cm

http://www.phas.ubc.ca/$\sim$gordonws/170 \\

\includegraphics[scale=0.25]{cover1.eps}\vfill
\end{slide}





%%%%%%%%%%%%%%%%%%%%%%%%%%%%%%%%%%%%%%%%%%%%%%%%%%%%%%%%%%%%%%
% Set width of the text - What is left will be the right margin.
% In this case, right margin is 8.5in - 1.25in - 6in = 1.25in.
\setlength{\textwidth}{6in}
% command sets a 0.75-inch top margin.
\setlength{\topmargin}{-0.25in}

% Set height of the text - What is left will be the bottom margin.
% In this case, bottom margin is 11in - 0.75in - 9.5in = 0.75in
\setlength{\textheight}{8in}
%\begin{document}




\begin{slide}
\centerline{\huge\bf\red\it Textbook Chapter 8: \\Section 8.1-8.2}
\end{slide}



\begin{slide}
{\bf\it \red Learning Goals:}

\begin{itemize}
\item{}Learn about the force of static friction.
\item{}Learn how to analyze systems in the state of ``impending motion''.
\item{}Learn about the ``coefficient of static friction'' and 
the ``angle of friction''.
\item{}Solve an example which illustrates how the coefficient of
friction can be used to gain information in a statics problem.
\item{}Learn how to incorporate the force of friction into a 
statics problem which has both forces and moments.
\item{}Solve an example using the force of friction and
the equations of equilibrium for a static rigid body.
\end{itemize}
\vfill
\end{slide}









\begin{slide}

{\bf \blue The Force of friction:}

The ``force of friction'' is a force which resists slippage
at an interface between two materials.

It depends on the details of 
\begin{itemize}
\item{}the shape and nature of the surfaces
\item{}the materials involved
\end{itemize}

We will {\bf model} friction and use it to get information
about a static system.


The threshold force needed to begin slippage at an interface where
the normal force between the two surfaces is $N$ is
$$
F_f= \mu_s N
$$
where $\mu_s$ is the {\bf\it ``coefficient of static friction''}


\vfill
\end{slide}





\begin{slide}
{\bf\blue The state of impending motion}\\
\includegraphics[scale=0.09]{impending.eps}
\includegraphics[scale=0.09]{fg08_03.eps}\\
Before the box begins to move, force of friction cancels pull: 
$\vec F_f=-\vec P$\\
At ``impending motion'', the force $\vec P$ is just enough so that the box 
begins to move: $F_s=|\vec F_f|_{\rm max}=|\vec P|=\mu_s N$\\
Once it starts moving, friction obeys a different formula,
$F_k=|\vec F_f|=\mu_k N$ with $\mu_k$ is {\bf\it ``coefficient of kinetic friction''}.\\
Usually $\mu_k<\mu_s$. 
\vfill\end{slide}



\begin{slide}
{\bf\blue Coefficients of static friction}

\vskip 1cm

~~$\mu_s$~~~~~~~{\bf Contact materials}\\
0.03-0.05~~~~metal on ice \\
0.30-0.70~~~~wood on wood\\
0.20-0.50~~~~leather on wood\\
0.30-0.60~~~~leather on metal\\
1.10-1.70~~~~aluminum on aluminum\\
\vfill
\end{slide}






\begin{slide}
\includegraphics[scale=0.18]{8-49.eps}\\
{\bf\blue Example:} The block of weight W is being pulled up the
inclined plane of slope $\alpha$ using a force $\vec P$.  If $\vec
P$ acts at an angle $\phi$ as shown, show that for slipping to occur
$P=W\sin(\alpha+\theta)/\cos(\phi-\theta)$ where $\theta$ is the
angle of friction $\theta=\arctan\mu_s$.


\vfill
\end{slide}










\begin{slide}
\includegraphics[scale=0.14]{8-49a.eps}\\
 In order to understand what
the ``angle of friction'' is, we will begin by putting $\vec P=0$ and
asking at what angle the block will begin to slide?\\ We will assume
the situation of ``impending motion'' that the slope has been
increased from zero to its current slope with angle $\alpha$ where it
is just about to start sliding.  \\
We will begin by analyzing the forces acting on the block.
\vfill
\end{slide}



\begin{slide}
\includegraphics[scale=0.14]{8-49a.eps}\\
{\bf\blue The Forces acting on the block are:}\\
\begin{itemize}
\item{}Gravity: $\vec F_G = -W\hat j$
\item{}$\vec P=0$.
\item{}Normal reaction force: $\vec N = N\left( -\sin\alpha \hat i +
\cos\alpha\hat j\right)$
\item{}Friction force: $\vec F_f = \mu_s N \left( \cos\alpha \hat i
+\sin\alpha\hat j\right)$
\end{itemize}
\vfill
\end{slide}



\begin{slide}
Equilibrium of forces, $\sum_i \vec F_i = 0$ implies

$$
\vec F_G + \vec N + \vec F_f=0
$$
$$
\left(-W\hat j\right) + N\left( -\sin\alpha \hat i +
\cos\alpha\hat j\right)+ \mu_s N \left( \cos\alpha \hat i
+ \sin\alpha\hat j\right) =0
$$
Or, in components:\\
$\sum_i F_{ix}=0$: $-N\sin\alpha +N\mu_s\cos\alpha=0$\\
This is all we need for now! 
$N$ cancels from this equation.\\
The remaining equation can be written as
$$
-\sin\alpha+\mu_s\cos\alpha=0~,~\mu_s=\frac{\sin\alpha}{\cos\alpha}
$$
or 
\fbox{$
\mu_s = \tan\alpha
$}~but remember $\mu_s=\tan\theta$ = tan ``angle of friction''\\
$\alpha$ is equal to the angle of friction $\alpha=\theta$!
The angle of friction is the angle to which you can increase
the slope of the plane before the block starts to slide.
This is independent of $W$ or $N$!
\vfill
\end{slide}










\begin{slide}

We can measure the coefficient of friction $\mu_s$ by \\
{\bf A.~}Lifting the ramp until the block slides, then measuring the
 angle $\alpha$\\

{\bf\magenta Then compute $\mu_s=\tan\alpha$.}

\vfill
\end{slide}











\begin{slide}
{\bf\blue Back to the example:}\\
\includegraphics[scale=0.18]{8-49.eps}\\ 
The block of weight W is being pulled up the inclined plane of slope
$\alpha$ using a force $\vec P$.  If $\vec P$ acts at an angle $\phi$
as shown, show that for slipping to occur
$P=W\frac{\sin(\alpha+\theta)}{\cos(\phi-\theta)}$ where $\theta$ is
the angle of friction $\theta=\arctan\mu_s$.  \vfill
\end{slide}






\begin{slide}
{\bf \blue Strategy for finding a solution:} ~~
\includegraphics[scale=0.1]{8-49.eps}
\begin{itemize}
\item{}This is a two-dimensional problem:
we shall take the x-coordinate as horizontal and
y-coordinate as vertical.  
\item{}The z-direction is toward the
viewer out of the page -- and will not be used here -- z-component
of all vectors will be zero.
\item{}We recognize that this is a situation of ``impending moton''. 
The force needed to
begin sliding should be that which just overcomes the force of
static friction plus gravity.  
\item{}The force of static friction is equal to the
coefficient of static friction, $\mu_s=\tan\theta$, times
the normal force.
\vfill
\end{itemize}
\end{slide}



\begin{slide}
{\bf\blue Free body diagram}\\

\includegraphics[scale=0.06]{8-49.eps}\\

\includegraphics[scale=0.18]{drawingblock.eps}


\vfill
\end{slide}




\begin{slide}
{\bf\blue Find the forces:}~~
\includegraphics[scale=0.1]{drawingblock.eps}\\
\begin{itemize}
\item{}Gravity: $\vec F_G = -W\hat j$
\item{}Pull:  $\vec P = P\left( \cos(\phi+\alpha)\hat i +
\sin(\phi+\alpha)\hat j\right)$
\item{}Normal reaction force: $\vec N = N\left( -\sin\alpha \hat i +
\cos\alpha\hat j\right)$
\item{}Friction force: $\vec F_f = -\mu_s N \left( \cos\alpha \hat i
+ \sin\alpha\hat j\right)$
\end{itemize}
\vfill
\end{slide}








\begin{slide}

{\bf\blue Equilibrium of forces: }
$\vec F_G = -W\hat j~~,~~\vec P = P\left( \cos(\phi+\alpha)\hat i +
\sin(\phi+\alpha)\hat j\right)$\\
$\vec N = N\left( -\sin\alpha \hat i +
\cos\alpha\hat j\right)~,~\vec F_f = -\mu_s N \left(\cos\alpha \hat i
+\sin\alpha\hat j\right)$
$$
 \vec F_G +\vec P +\vec N +\vec F_f = 0
$$
\begin{eqnarray}
-W\hat j+ P\left( \cos(\phi+\alpha)\hat i + \sin(\phi+\alpha)\hat
j\right)+ N\left( -\sin\alpha \hat i + \cos\alpha\hat j\right)\nonumber \\
-\mu_s N \left( \cos\alpha \hat i + \sin\alpha\hat
j\right)=0\nonumber
\end{eqnarray}
components:
$$
 P \cos(\phi+\alpha)-N \sin\alpha -\mu_s N  \cos\alpha  =0
$$
$$
-W +  P  \sin(\phi+\alpha)+ N  \cos\alpha -\mu_s N  \sin\alpha=0
$$
\vfill
\end{slide}






\begin{slide}
components:
$$
 P \cos(\phi+\alpha)-N \sin\alpha -\mu_s N  \cos\alpha  =0
$$
$$
-W +  P  \sin(\phi+\alpha)+ N  \cos\alpha -\mu_s N  \sin\alpha=0
$$
Solve the first equation to get
$$
N=P\frac{ \cos(\phi + \alpha)}{\sin\alpha + \mu_s\cos\alpha}
$$
Then, use the second equation to find
$$
P\left[ \sin(\phi+\alpha)+ (\cos\alpha -\mu_s   \sin\alpha)\frac{
\cos(\phi + \alpha)}{\sin\alpha + \mu_s\cos\alpha}\right]=W
$$
\vfill
\end{slide}






\begin{slide}
Bringing forward the last equation from the previous page:
$$
P\left[ \sin(\phi+\alpha)+ (\cos\alpha -\mu_s   \sin\alpha)\frac{
\cos(\phi + \alpha)}{\sin\alpha +\mu_s\cos\alpha}\right]=W
$$
Now we solve for $P$:
$$
P=  W\frac{  (\sin\alpha+ \mu_s\cos\alpha) } { (\sin\alpha +
\mu_s\cos\alpha) \sin(\phi+\alpha)+ (\cos\alpha -\mu_s   \sin\alpha)
\cos(\phi + \alpha)  }
$$
Now we substitute $\mu_s=\tan\theta=\frac{\sin\theta}{\cos\theta} $ 
{\small
$$
P=  \frac{  W(\cos\theta\sin\alpha + \sin\theta\cos\alpha) } {
(\cos\theta\sin\alpha + \sin\theta\cos\alpha) \sin(\phi+\alpha)+
(\cos\theta\cos\alpha -\sin\theta \sin\alpha) \cos(\phi + \alpha)  }
$$}
Remember the double angle formulas:
$$
\cos(a\pm b)=\cos a\cos b \mp \sin a \sin b
$$
$$
\sin(a\pm b)=\sin a \cos b \pm \cos a \sin b
$$
\vfill
\end{slide}



\begin{slide}
{\small $$
P=  \frac{  W(\cos\theta\sin\alpha + \sin\theta\cos\alpha) } {
(\cos\theta\sin\alpha + \sin\theta\cos\alpha) \sin(\phi+\alpha)+
(\cos\theta\cos\alpha -\sin\theta \sin\alpha) \cos(\phi + \alpha)  }
$$}
Using
$$
\cos(a\pm b)=\cos a\cos b \mp \sin a \sin b
$$
$$
\sin(a\pm b)=\sin a \cos b \pm \cos a \sin b
$$
we get
$$
P=  W\frac{  \sin(\alpha+\theta) } { \sin(\alpha+\theta)
\sin(\phi+\alpha)+ \cos(\alpha+\theta)\cos(\phi + \alpha)  }
$$
Finally\\
\fbox{$
P=W\frac{\sin(\alpha+\theta)}{\cos(\phi-\theta)}
$}
 \vfill
\end{slide}



















\begin{slide}
{\bf\blue Equivalent systems and resultant normal force:}\\
Normal force is usually distributed.\\
 A distributed normal reaction force on
flat interface $~=~$ reaction force at a single point.
\begin{itemize}
\item{}Consider a flat interface with a variety of forces $\vec F_i$
acting normal to the interface. (For simplicity, take the force
vectors to be parallel.)
\item{}The forces can all be moved to a single point by adding a couple moment.
\item{}The point can then be adjusted until the additional couple
moment vanishes. The point at which the force is then located is the
{\bf \red ``location of the resultant normal force''}.
\end{itemize}
\vfill
\end{slide}












\begin{slide}

{\bf\blue Example:} \\
\includegraphics[scale=0.15]{8-38.eps}\\
The crate has a weight of 200lb and a center of
gravity at G. Determine the horizontal force P required to tow it.
Also determine the location of the resultant normal force  measured
from $A$. Take $h=4ft$ and $\mu_s=0.4$.
\vfill
\end{slide}
































\begin{slide}
{\bf\blue Strategy for finding a solution:} 
\begin{itemize} 
\item{}This is a two-dimensional problem.  We take the x-coordinate horizontal and
y-coordinate vertical with origin at $G$. The z-direction is 
toward the viewer out of the page.
\item{}We recognize that the force needed to begin towing the crate should be that 
which puts it in a state of impending motion, where the force of friction is
$|\vec F_f|=\mu_s|\vec N|$.
\item{}We will assume that the reaction normal force acts at a point which is
a distance $x$ along the horizontal from the point $A$ at the bottom right-hand corner
of the crate.  As we have discussed, the normal force is distributed over the contact.  Placing
it all at a single point is the same as finding an equivalent system which has the same total
force and exerts the same total moments on the object. 
\end{itemize}
\vfill
\end{slide}























\begin{slide}
\begin{itemize}
\item{}We must also decide where to place the point of action of the force of friction.  We
remember that we can consider an equivalent system with this force moved to an point along
its line of action.  For a flat interface, we can thus think of the force as acting at any point
on the interface.  
\item{}Similarly for the towing force, it is exerted by a rope which surrounds the crate and is
likely distributed over the rope.  Since the rope is located along the line of action
of the force, we can consider it a single concentrated force at any point along its line
of action. 
\item{}We will find a mathematical expression for all of the forces
acting on the crate, including the reaction normal force and the force
of friction.
\item{}We will compute the moments of each of the forces.
\end{itemize}
\vfill
\end{slide}

















\begin{slide}
\begin{itemize} 
\item{}We will then impose the conditions of equilibrium,
$
\sum_i \vec F_i=0$, $\sum_i \vec M_{{\cal O}i}=0$
\item{}There are three equations which we can use to solve for the unknown quantities. 
\end{itemize}
\vfill
\end{slide}






\begin{slide}
{\bf\blue Free body diagram}\\
\includegraphics[scale=0.1]{8-38.eps}\\~~\\
\includegraphics[scale=0.14]{crate.eps}
\vfill
\end{slide}











\begin{slide}

{\bf \blue Find the forces and the position vectors of points where they act:}\\

\includegraphics[scale=0.1]{crate.eps}
\begin{itemize}
\item{}Gravity: $\vec W = -(200lb) \hat j$ acting at $\vec r_G=\vec 0$
\item{}Towing:  $\vec P = P\hat i$ acting at $\vec r_P=(h-(3ft))\hat j$
\item{}Reaction: $\vec N=N\hat j$ acting at $\vec r_N=((2ft)+x)\hat i - (3ft)\hat j$
\item{}Friction:   $\vec F_f = -\mu_s N \hat i$ acting at
$- (3ft)\hat j$
\end{itemize}
 \vfill
\end{slide}










\begin{slide}
{\bf \blue Equilibrium of forces:} \\
We can find $P$ and $N$ from the force equations:
$$\vec W +\vec P  +\vec N+\vec F_f=0$$
$$
  -(200lb) \hat j+ P\hat i  + N\hat j -\mu_s N\hat i=0$$
In components:
$$
P=\mu_sN~~,~~N=W=200lbs
$$
\fbox{$
N=(200lb)$}~~,~~\fbox{$P=(80lb)$}\\
To find the point of action of the resultant normal force,
we must analyze the moments.
\vfill
\end{slide}




\begin{slide}
{\bf\blue Equilibrium of Moments ``in two dimensions'':}\\
~~\\
\includegraphics[scale=0.1]{crate.eps}\\
Moment due to Friction = $-3\mu_sN$\\
Moment due to Weight =0\\
Moment due to Normal $= (2+x)N$\\
Moment due to Pull $= -(h-3)P$\\
Total moments $M=-3\mu_sN+(2+x)N-(h-3)P$\\
Also, we know that $N=W$. $P=\mu_sN$ so $M=(-h\mu_s+2+x)W$\\
$M=0~\to~\mu_s h=(x+2)/\mu_s$\\
Note that this equation is independent of $W$!
\vfill
\end{slide} 





\begin{slide}
{\bf \blue Computation of Moments using vectors:}\\
For completeness, let us also compute the moments using our
vector techniques. We compute moments about $G$:
\begin{itemize}
\item{}Gravity: $\vec M_G=\vec 0\times( -(200lb)) \hat j=\vec 0$
\item{}Towing:  $\vec M_P=((h-(3ft))\hat j)\times  P\hat i=-(h-(3ft))P\hat k$
\item{}Reaction: $\vec M_N=(((2ft)+x)\hat i - (3ft)\hat j)\times\left( N\hat j\right)
=((2ft)+x)N\hat k$
\item{}Friction: $\vec M_f=(-(3ft)\hat j)\times (-\mu_sN\hat i)=-\mu_s(3ft)N\hat k$
\end{itemize}
\vfill
\end{slide}







\begin{slide}
{\bf\blue Equilibrium of Moments using vectors:}
 $$\sum \vec M=0~~~~~~~~~~~~~~~~~~~~~~~~~$$
$$ \vec M_G + \vec M_P + \vec M_N + \vec M_f=0
$$
Explicitly substituting the moment vectors gives
$$
 \vec 0 -(h-(3ft))P\hat k+((2ft)+x)N \hat
 k   -\mu_s(3ft)N\hat k =0
$$
or, remembering that $N=W$ and $P=\mu_sN$, 
$$
[-(h-3)\mu_s +2+x-3\mu_s]N\hat k=0~\to~ h\mu_s=x+2
$$
with $x$ and $h$ in ft.  We are given that
$$
h=4~ft~,~\mu_s=0.4
$$
\fbox{$x=-.400~ft$}
 \vfill
\end{slide}











\begin{slide}
{\bf\blue Example:}\\
\includegraphics[scale=0.15]{8-38.eps}\\
The crate has a weight of 200lb and a center of
gravity at $G$. Determine the height $h$ of the tow rope so that the
crate slips and tips at the same time.  What horizontal force $P$ is
required to do this?
\vfill
\end{slide}










\begin{slide}
{\bf\blue Example:}\\
\includegraphics[scale=0.14]{8-38.eps}\\
The crate has a weight of 200lb and a center of
gravity at $G$. Determine the height $h$ of the tow rope so that the
crate slips and tips at the same time.  What horizontal force $P$ is
required to do this?
\vfill
\end{slide}






\begin{slide}
\includegraphics[scale=0.1]{crate.eps}\\
{\bf\blue Strategy for finding a solution:}
\begin{itemize}
\item{}The crate will slip and tip at the same time when the resultant
normal reaction force is at $A$ and when the force of friction is just
compensated by $P$ so that the crate is in a state of impending motion.
\item{}In the previous problem, we found that
$$
h\mu_s=x+2
$$
Now, we set $x=0$ and we get \fbox{$h=2/\mu_s=\frac{(2)}{(0.4)}~=~5~ft$}
\end{itemize}
\vfill
\end{slide}









\begin{slide}
\centerline{\huge\bf\red\it For the next lecture, please read}

\vskip .5cm

\centerline{\huge\bf\red\it Textbook Chapter 8:Section 8.2-8.3}
\end{slide}

\end{document}



