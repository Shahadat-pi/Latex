\documentclass{article}
\usepackage[utf8]{inputenc}
\usepackage{amsmath}
\usepackage{graphicx,relsize}
\graphicspath{ {images/} }
\usepackage{mathtools}
\DeclarePairedDelimiter{\ceil}{\lceil}{\rceil}
\usepackage{geometry}
\usepackage[banglamainfont=Kalpurush, 
            banglattfont=Siyam Rupali
           ]{latexbangla}
\usepackage{tikz}
\usetikzlibrary{scopes}
\begin{document}
আশিকুর যখন বুকে গুলি নিয়ে হাসপাতালের আইসিইউতে শুয়ে আছে, তখন রাজীব চড়ে বসেছে অ্যাম্বুলেন্সে। অনেক দিন পর বাড়ি ফিরছে ছেলেটা। এবার চিরবিশ্রাম। সেই ছোটবেলাকার মতো আবার মায়ের পাশে শুয়ে পড়বে ছেলেটা। তৃতীয় শ্রেণিতে থাকতে শেষ দেখেছিল মাকে-বাবাকে। এবার তাঁদের কবরের পাশে শুয়ে পড়বে। সন্তানকে মাঝে নিয়ে দুপাশে বাবা-মা। কী সুন্দর সুখী একটা ছবি! কিন্তু তা হওয়ার নয়। কারণ, সবাই তাঁরা অকালমৃত। দারিদ্র্যের হুতাশন আর দুর্ঘটনা নামের কাঠামোগত হত্যাকাণ্ডে যদি একটা পরিবার এভাবে শেষ হয়ে না যেত! ভাবা যাক, রাজীব ঢাকা শহরে মেসে থেকে বাসে ঝুলে টিউশনি করে একদিন বিসিএস দিয়ে সরকারি চাকরি পেল। ভাবা যাক, সেই সামান্য সফলতা নিয়ে সে গিয়ে দাঁড়াল তার বাবা-মায়ের কবরের সামনে। ছোট দুই ভাইকে পাশে নিয়ে সে তাঁদের কবর জিয়ারত করছে আর মনে মনে ভাবছে, যাক, ওপার থেকে বাবা-মা বোধ হয় তাকে আশীর্বাদ করছেন। ছেলের সাফল্যে তাঁদের বিদেহী আত্মা বোধ হয় এবার একটু জুড়াল।

কিন্তু সেই দৃশ্য আর জন্মাবে না; বরং কাটা পড়া হাতটা ঢাকা শহরে ফেলে রেখেই পটুয়াখালীর বাউফলের দাসপাড়া গ্রামে ফিরে যাচ্ছে রাজীব। এই শহর তার হাত নিল, প্রাণ নিল। দরিদ্র কোটায় আটকে পড়া ছেলেটার আশৈশবের সব সংগ্রাম ব্যর্থ করে দিল। নিজের পায়ে দাঁড়াতে ছুটে চলেছিল সে। তিতুমীর কলেজে স্নাতক দ্বিতীয় বর্ষে পড়ার পাশাপাশি কম্পিউটার কম্পোজ, গ্রাফিকস ডিজাইনের কাজ শিখছিল। ছাত্র পড়াত। দম ফেলার ফুরসত ছিল না। ভাই দুটির দায়িত্ব নেওয়ার ইচ্ছা ছিল। হয়তো নিজের করে একটা স্বপ্নও ছিল। সব শেষ।

হ‌ুমায়ূন আহমেদ এই নিম্নবিত্তের মৃদু মানুষদের সুখ-দুঃখের অরূপকথার খবর রাখতেন। মুক্তিযুদ্ধে বাবাকে হারানোর পর, সরকারি নিপীড়নে বাড়ি হারানোর পর তিনি নিজেও কাটিয়েছিলেন অসম্ভব দারিদ্র্যের এক যৌবন। তিনিই এঁদের মনের কথা লিখেছিলেন, ‘আজন্ম সলজ্জ সাধ, একদিন আকাশে কিছু ফানুস উড়াই’। কথাটা কাব্যিক শোনায়, কিন্তু এই কাব্যের জন্ম দুঃখের নন্দিত নরকে, শঙ্খনীল কারাগারে। এমন কোনো স্বপ্ন যুবকেরা হৃদয়ে পুষে রাখবেই। রাজীবেরও হয়তো এমন কোনো সলজ্জ সাধ ছিল। তাই হাসপাতালের বিছানায় শুয়ে অভিমান করে কাটা হাতের দিকে তাকিয়ে বলেছিল, ‘রাজীব কে? রাজীব মারা গেছে!’ হাতের সঙ্গে সঙ্গে স্বপ্নটা মরে গিয়েছিল বলে হয়তো ভেঙে পড়েছিল সে। তারপর সাত দিন অচেতন থেকে চলে গেল।

এখন বাসমালিকেরা নিদ্রা দিন। পরিবহনের চালক-মালিক সমিতির নেতারা মুখ ঘুরিয়ে থাকুন। যোগাযোগমন্ত্রী দিনরাত বিরোধীদের নিয়ে মশকরা করে যান। কেবল গরু-ছাগল চেনার যোগ্যতায় গাড়ি চালানোর লাইসেন্স দিতে চাওয়া নৌ পরিবহনমন্ত্রী জীবন উপভোগ করুন। মাঝেমধ্যে তিনি শিল্পচর্চা করেন, নায়িকাদের শুটিং সেটে হুট করে যে হাজির হন, সেসব করে যান মনের আনন্দে। কাউকে দায় নিতে হবে না? টনক নামের বস্তুটা যখন নেই, তখন সেটা নাড়ানোর দাবি করাও বৃথা।

এদিকে দৃশ্যের পর দৃশ্যের জন্ম হতে থাক। 


রাজীবের ঘটনার পরপরই রুনী নামের বিশ্ববিদ্যালয়পড়ুয়া আরেকটি মেয়ের পা বাসের চাপায় থেঁতলে যাওয়ার দৃশ্য আমরা দেখেছি  । আমরা দেখেছি, কোটার বৈষম্যের বিরুদ্ধে আন্দোলন করতে গিয়ে ‘অজ্ঞাতনামা’ গুলি খেয়ে পড়ে যাচ্ছে আশিকুর নামের এক ছাত্র। আমরা দেখেছি, পরীক্ষার দাবি জানানোর ‘অপরাধে’ সিদ্দিকুরের চোখ পুলিশের ছররা গুলিতে অন্ধ হয়ে গেল  । দেখেছি, হলে দুই পক্ষের মারামারির মধ্যে গুলি খেয়ে মরে গেল ঢাকা বিশ্ববিদ্যালয়ের দরিদ্র পরিবারের ছাত্র আবু বকর । আমরা দেখেছি, সে সময়ের স্বরাষ্ট্রমন্ত্রী সাহারা খাতুন এই মাসুম ছাত্রের মৃত্যুতে বলেছেন, ‘এটা ঘটতেই পারে’  । গত ৬ ফেব্রুয়ারি রাত দুইটা থেকে পরদিন বেলা আড়াইটা পর্যন্ত ছাত্রলীগের নির্যাতনে চোখ হারাতে বসা এহসান রফিককেও দেখছি, মাথার যন্ত্রণা তার শেষ হচ্ছে না, ভুলতে পারছে না শিক্ষাজীবন থেকে একটি বছর হারিয়ে ফেলার ক্ষতি  ।

এসব ঘটেই চলুক আর নিরাপদ থাকুক রাজধানীসহ সারা দেশের পরিবহন ব্যবসার মাফিয়াতন্ত্র । একে আগলে রাখুক মন্ত্রী ও প্রশাসন। আর আমরা দেখে দেখে বিজ্ঞ হয়ে উঠি। যারা পারি, বিদেশ চলে যাই। যারা পারি, নিজেরাও ওপর তলায় উঠে অন্যদের ঘাড়ে পাড়া দিয়ে প্রতিষ্ঠা অর্জন করি। এবং ভুলে যাই এই তরুণদের নিজের পায়ে দাঁড়ানোর কঠিন সংগ্রামের গল্পটা। 

দারিদ্র্য নামক এক অতলগহ্বর থেকে মাটি কামড়ে ধরে হাঁচড়ে-পাঁচড়ে উঠে আসছিল তারা। যৌবনের দুয়ার পর্যন্ত আসতে তাকে কত শ্রম, কত রাত জাগা, কত অনাহার, কত অপমান সইতে হয়েছিল, তার কোনো পরিমাপ কেউ করতে পারবে না। আবু বকরের বাবা ছিলেন দিনমজুর, বড় ভাই মুদি দোকানদার, মা খরচ কমাতে বহু বছর চুলে তেল দেন না।

আবু বকরের মতোই জীবন কাটাচ্ছে কোটাবৈষম্য-বিরোধী আন্দোলনের অন্যতম যুগ্ম আহ্বায়ক রাশেদ খান। তারও বাবা দিনমজুর। তার গল্পটাও আবু বকরদের মতো। কিন্তু এ জীবন তার মতো ছেলেমেয়েরা চায়নি বলে সারল্যের জেদ নিয়ে আন্দোলনে এসেছে। আজ তাকে তার দুই সহযোদ্ধাসহ চোখ বেঁধে তুলে নিয়ে যাওয়া হয়, তার বাবাকে থানায় আটকে যা-তা বলা হয়।

রাজীবদের মতো, আবু বকরদের মতো, রাশেদ খানদের মতো ছেলেমেয়েরা অনেক বছরের শ্রম, ঘাম আর অশ্রুতে গড়া মাটির মানুষ। মাটি পুড়ে পুড়ে ঝামা হয়, দুঃখ আর বঞ্চনার আগুনে পুড়ে হৃদয় তামা তামা হয়। দেশে তো কোনো যুদ্ধ চলছে না, তাহলে এত রকমের অপঘাতে কেন রক্ত ঝরবে, কেন তাজা প্রাণ লুটিয়ে পড়বে, কেন রাজীব মারা যাবে, কেন আশিকুরের বুকের গুলি লাগবে আর সেই গুলিটা আর বের করা যাবে না? জীবন কি এত অবলীলায়-আধলীলায় খরচযোগ্য? বঙ্গবন্ধুর অমর ভাষণের মতো এদের বুকেও কি গুমগুম করে হুংকার বাজে—আর দাবায়ে রাখবার পারবা না!
\end{document}